\chapter{Joining Our Community}\label{s:joining}

We hope you will choose to help us make this book better.
If you are new to working in this collaborative way,
please see \appref{s:conduct} for our code of conduct,
and then:

\begin{description}

\item[Start small.]
  Fix a typo,
  clarify the wording of an exercise,
  correct or update a citation,
  or suggest a better example or analogy to illustrate some point.

\item[Join the conversation.]
  Have a look at the issues and proposed changes that other people have already filed
  and add your comments to them.
  It's often possible to improve improvements,
  and it's a good way to introduce yourself to the community and make new friends.

\item[Discuss, then edit.]
  If you want to propose a large change,
  such as reorganizing or splitting an entire chapter,
  please file an issue that outlines your proposal and your reasoning and tag it with ``Proposal.''
  We encourage everyone to add comments to these issues
  so that the whole discussion of what and why is in the open and can be archived.
  If the proposal is accepted,
  the actual work may then be broken down into several smaller issues or changes
  that can be tackled independently.

\end{description}

\seclbl{Using This Material}{s:joining-using}

As \chapref{s:intro} stated,
all of this material can be freely distributed and re-used
under the Creative Commons Attribution-NonCommercial 4.0 license
(\appref{s:license}).
You can use the online version at \url{http://teachtogether.tech/} in any class (free or paid),
and can quote short excerpts under \hreffoot{https://en.wikipedia.org/wiki/Fair\_use}{fair use} provisions,
but cannot republish large parts in commercial works without prior permission.

This material has been used in many ways,
from a multi-week online class to an intensive in-person workshop.
It's usually possible to cover large parts of \chapref{s:models} to \chapref{s:process},
\chapref{s:performance},
and \chapref{s:motivation} in two long days.

\subsection*{In Person}

This is the most effective way to deliver this training,
but also the most demanding.
Participants are physically together.
When they need to practice teaching in small groups,
some or all of them go to nearby breakout spaces.
Participants use their own tablets or laptops to view online material during the class
and for shared note-taking (\secref{s:classroom-notetaking}),
and use pen and paper or whiteboards for other exercises.
Questions and discussion are done aloud.

If you are teaching in this format,
you should use sticky notes as status flags
so that you can see who needs help,
who has questions,
and who's ready to move on (\secref{s:classroom-sticky-notes}).
You should also use them to distribute attention
so that everyone gets a fair share of the teacher's time,
and as minute cards to encourage learners to reflect on what they've just learned
and to give you actionable feedback while you still have time to act on it.

\subsection*{Online in Groups}

In this format,
10--40 learners are together in 2--6 groups of 4--12,
but those groups are geographically distributed.
Each group uses one camera and microphone to connect to the video call,
rather than each person being on the call separately.
Good audio matters more than good video in both directions:
a voice with no images (like radio)
is much easier to understand than images with no narrative,
and instructors do not have to be able to see people to answer questions
so long as those questions can be heard clearly.
That said,
if a lesson isn't accessible then it's not useful (\secref{s:motivation-accessibility}):
providing descriptive text helps everyone when audio quality is poor,
and everyone with hearing challenges even when it's not.

The entire class does shared note-taking together,
and also uses the shared notes for asking and answering questions.
Having several dozen people try to talk on a call works poorly,
so in most sessions,
the teacher does the talking and learners respond through the note-taking tool's chat.

\subsection*{Online as Individuals}

The natural extension of being online in groups is to be online as individuals.
As with online groups,
the teacher will do most of the talking and learners will mostly participate via text chat.
Good audio is once again more important than good video,
and participants should use text chat to signal that they want to speak next (\appref{s:meetings}).

Having participants online individually makes it more difficult to draw and share concept maps (\secref{s:memory-exercises})
or give feedback on teaching (\secref{s:performance-exercises}).
Teachers should therefore rely more on exercises with written results that can be put in the shared notes,
such as giving feedback on stock videos of people teaching.

\subsection*{Multi-Week Online}

The class meets every week for an hour via video conferencing.
Each meeting may be held twice to accommodate learners' time zones and schedules.
Participants use shared note-taking as described above for online group classes,
post homework online between classes,
and comment on each other's work.
In practice,
comments are relatively rare:
people strongly prefer to discuss material in the weekly meetings.

This was the first format used,
and I no longer recommend it:
while spreading the class out gives people time to reflect and tackle larger exercises,
it also greatly increases the odds that they'll have to drop out because of other demands on their time.

\seclbl{Contributing and Maintaining}{s:joining-contributing}

Contributions of all kinds are welcome,
from suggestions for improvements to errata and new material.
All contributors must abide by our Code of Conduct (\appref{s:conduct});
by submitting your work,
you are agreeing that it may incorporated in either original or edited form
and released under the same license as the rest of this material (\appref{s:license}).
If your material is incorporated,
we will add you to the acknowledgments (\secref{s:intro-acknowledgments}) unless you request otherwise.

The source for this book is stored on GitHub at:

\begin{center}
  \url{https://github.com/gvwilson/teachtogether.tech/}
\end{center}

\noindent
If you know how to use Git and GitHub and would like to change, fix, or add something,
please submit a \gref{g:pull-request}{pull request} that modifies the LaTeX source.
If you would like to preview your changes,
please run \texttt{make~pdf} or \texttt{make~html} on the command line.

If you want to report an error,
ask a question,
or make a suggestion,
please file an issue in the repository.
You need to have a GitHub account in order to do this,
but do not need to know how to use Git.

If you do not wish to create a GitHub account,
please email your contribution to \texttt{gvwilson@third-bit.com}
with either ``T3'' or ``Teaching Tech Together'' somewhere in the subject line.
We will try to respond within a week.

Finally,
we always enjoy hearing how people have used this material,
and are \emph{always} grateful for more diagrams.
