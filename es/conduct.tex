\chapter{Código de conducta}\label{s:conduct} \index{Código de Conducta}
Con el objetivo de fomentar un ambiente abierto y amigable, 
las personas encargadas del proyecto, colaboradoras/es y personas de soporte, 
nos comprometemos a hacer de la participación en nuestro proyecto
y en nuestra comunidad una experiencia libre de acoso para todas/os,
independientemente de su edad, tamaño corporal, discapacidad, etnia,
identidad y expresión de género, nivel de experiencia, educación,
nivel socioeconómico, nacionalidad, apariencia personal, raza,
religión o identidad y orientación sexual.

\section*{Nuestros estándares}

Ejemplos de comportamientos que contribuyen a crear un ambiente positivo
para nuestra comunidad:

\begin{itemize}
\item
  utilizar un lenguaje amigable e inclusivo,
\item
  respetar diferentes puntos de vista y experiencias,
\item
  aceptar adecuadamente la crítica constructiva,
\item
 enfocarse en lo que es mejor para la comunidad y
\item
 mostrar empatía hacia otras/os participantes de la comunidad.
\end{itemize}

\noindent
Ejemplos de comportamiento inaceptable:

\begin{itemize}
\item
  el uso de lenguaje o imágenes sexualizadas así como  
  dar atención o generar avances sexuales no deseados,
\item
  ofender o provocar de modo malintencionado (\emph{trolling}), comentarios despectivos, insultantes y ataques personales o políticos,
\item
 acoso  público o privado,
\item
  publicar información privada de otras personas, tales como direcciones
  físicas o de correo electrónico, sin su permiso explícito, y
\item
  otras conductas que puedan ser razonablemente consideradas
  como inapropiadas en un entorno profesional.
\end{itemize}

\section*{Nuestras responsabilidades}

Las personas encargadas del proyecto somos responsables de aclarar los estándares de
comportamiento aceptable y se espera que tomemos medidas de acción correctivas
apropiadas y justas en respuesta a cualquier caso de comportamiento inaceptable.

Las personas encargadas del proyecto tenemos el derecho y la responsabilidad de
eliminar, editar o rechazar comentarios, \emph{commits}, código, ediciones en la wiki, \emph{issues} y otras
contribuciones que no estén alineadas con este Código de Conducta. También pueden
prohibir la participación temporal o permanente de cualquier persona por comportamientos
que sean considerados inapropiados, amenazantes, ofensivos o dañinos.

\section*{Alcance}

Este Código de Conducta aplica tanto a espacios dentro del proyecto
como en espacios públicos, mientras una persona represente al proyecto o a
la comunidad. Ejemplos de representación del proyecto o la comunidad incluyen
el uso de una dirección de correo electrónico oficial del proyecto,
realizar publicaciones a través de una cuenta oficial en redes sociales
o actuar como representante designada/o en cualquier evento presencial o en línea.
La representación del proyecto puede ser aclarada y definida con más
detalle por las personas encargadas.

\section*{Aplicación}

Los casos de comportamiento abusivo, acosador o inaceptable
pueden ser denunciados enviando un correo electrónico a la persona encargada del proyecto a la dirección \texttt{gvwilson@third-bit.com}.
Todas las quejas serán revisadas e investigadas y darán como resultado
una respuesta que se considere necesaria y apropiada a las circunstancias.
El equipo encargado del proyecto está obligado a mantener la confidencialidad de quien reporte un incidente.
Se pueden publicar por separado más detalles
de políticas de aplicación específicas.

Aquellas personas encargadas del proyecto que no cumplan o apliquen 
este código de conducta de buena fe pueden enfrentar repercusiones
temporales o permanentes determinadas por el resto del equipo a cargo
del proyecto.

\section*{Atribución}

Este código de conducta es una adaptación del
\hreffoot{https://www.contributor-covenant.org}{Contributor Covenant} version 1.4.
