\chapter{Construyendo una comunidad de práctica}\label{s:community}

No tienes que arreglar todos los males de la sociedad para enseñar programación,
pero \emph{sí debes} involucrarte en
lo que sucede fuera de tu clase si quieres que las personas aprendan.
Esto se aplica tanto a las personas que enseñan como a las que aprenden:
gran parte de las/los docentes free-range deben balancear  sus clases con muchos otros compromisos
porque su trabajo es voluntario o de medio tiempo.
Lo que sucede fuera del aula es tan importante para su éxito
como lo es para el de sus estudiantes,
así que la mejor manera de ayudar a ambas partes es fomentar una comunidad de enseñanza.


\begin{aside}{Finlandia y por qué no}
Las escuelas de Finlandia se encuentran entre las más exitosas del mundo
pero, tal como lo señaló Anu Partanen\index{Partanen, Anu}\hreffoot{https://www.theatlantic.com/national/archive/2011/12/what-americans-keep-ignoring-about-finlands-school-success/250564/},
su éxito no se explica de manera aislada.
  Los intentos de otros países por adoptar métodos de enseñanza finlandeses están condenados al fracaso
  a menos que esos países también garanticen que la niñez (además de sus madres y padres) tengan seguridad, alimentación saludable y reciban un trato adecuado en la administración de justicia ~\cite{Sahl2015, Wilk2011}.
  Esto no es ninguna sorpresa considerando lo que sabemos sobre la importancia de la motivación para el aprendizaje (\chapref{s:motivation}):
  a las personas les va peor si creen que el sistema es impredecible, injusto o indiferente.
\end{aside}

Un marco para pensar en enseñar a comunidades de práctica es el \gref{g:situado-learning}{aprendizaje situado},
el cual se centra en cómo la \gref{g:legitimate-peripheral-participation}{participación periférica legítima}
motiva a las personas a convertirse en miembros de
una \gref{g:community-of-practice}{comunidad de práctica}~\cite{Weng2015}.
Desglosando estos términos,
una comunidad de práctica se refiere a  un grupo de personas unidas por su interés en alguna actividad,
como tejer o la física de partículas.
Una  participación periférica legítima implica realizar aquellas tareas simples y de bajo riesgo
que la comunidad reconoce como contribuciones válidas:
tejer tu primera bufanda,
llenar sobres durante una campaña electoral,
o revisar documentación de software de código abierto.

El aprendizaje situado se centra en la transición entre ser una persona recién llegada
hasta la aceptación como colega por quienes ya son parte de la comunidad.
Esto generalmente significa comenzar con tareas y herramientas simples,
para después realizar tareas similares con herramientas más complejas
y finalmente abordar el mismo trabajo que practicantes avanzadas/os.
Por ejemplo,
las personas que aprenden música comienzan tocando canciones infantiles con flauta o ukelele, para
después tocar otras canciones simples en trompeta o saxofón junto a una banda,
y recién entonces comienzan a explorar sus propios gustos musicales.
Algunas formas comunes de apoyar esta evolución incluyen:

\newpage
\begin{description}

\item[Resolución de problemas:]
  ``No puedo avanzar --- ¿podemos trabajar en el diseño de esta lección conjuntamente?''

\item[Pedidos de información:]
  ``¿Cuál es la contraseña para administrar la lista de correo?''

\item[Búsqueda de experiencia:]
  ``¿Alguien ha tenido una/un estudiante con discapacidad para leer?''

\item[Compartir recursos:]
  ``El año pasado armé un sitio web para una clase que puedes usar como punto de partida''.
  
\item[Coordinación:]
  ``¿Podemos combinar nuestros pedidos de camisetas para obtener un descuento?''

\item[Construir un argumento:]
  ``Será más fácil convencer a mi jefe de que haga cambios si sé cómo hacen esto en otros talleres''.

\item[Documentar proyectos:]
  ``Ya hemos tenido este problema cinco veces. Vamos a escribirlo de una vez por todas''.

\item[Mapeo de conocimientos:]
  ``¿Qué otros grupos están haciendo algo similar en vecindarios o ciudades cercanas?''

\item[Visitas:]
  ``¿Podemos visitar su programa extracurricular? Necesitamos establecer uno en nuestra ciudad''.

\end{description}

\hreffoot{https://www.feverbee.com/types-of-community-and-activity-within-the-community/}{En términos generales},
una comunidad de práctica puede ser:

\begin{description}

\item[Comunidad de acción:]
  personas enfocadas en un objetivo compartido,
  como conseguir que un candidato sea elegido.

\item[Comunidad de cuidado:]
 en la que las personas integrantes de la comunidad se unen alrededor de un problema compartido, como tratar una enfermedad a largo plazo.

\item[Comunidad de interés:]
  enfocada en el amor compartido por algo, por ejemplo el backgammon o tejer.

\item[Comunidad de lugar:]
  personas que viven o trabajan juntas.

\end{description}
 
La mayoría de las comunidades son mezclas de estos diferentes tipos,
como son las personas en Toronto a las que les gusta enseñar tecnología.
El enfoque de una comunidad también puede cambiar con el tiempo:
por ejemplo,
un grupo de apoyo para personas que padecen depresión (comunidad de cuidado)
puede decidir recaudar fondos para mantener una línea de ayuda (comunidad de acción).
A partir de entonces, mantener el funcionamiento de la línea de ayuda puede convertirse en el enfoque del grupo (comunidad de interés).

\begin{aside}{Sopa, luego himnos}
  Los manifiestos son divertidos de escribir,
  pero la mayoría de las personas se une a una comunidad para ayudar y recibir ayuda, no para discutir cómo redactar una declaración de visión\footnote{Las personas que prefieren lo último a menudo están \emph{solo} interesadas en discutir{\ldots}}.
  Por lo tanto, debes centrarte en
  qué personas pueden crear lo que el resto de la comunidad usará de inmediato.
  Una vez que tu organización demuestre que puede lograr cosas pequeñas,
  la gente estará más segura de que vale la pena ayudarte con proyectos más grandes.
  Ese es el momento de preocuparse por definir los valores que guiarán a tus miembros.
\end{aside}

\seclbl{Aprende, luego hazlo}{s:community-learn-then-do}

El primer paso para construir una comunidad es decidir si deberías construirla
o si sería más efectivo unirte a una organización existente.
Miles de grupos ya están enseñando habilidades tecnológicas,
desde el \hreffoot{http://www.4-h-canada.ca/}{4-H Club}
y \hreffoot{https://www.frontiercollege.ca/}{programas de alfabetización}
hasta organizaciones de programación sin fines de lucro como
\hreffoot{http://www.blackgirlscode.com/}{Black Girls Code}
y \hreffoot{http://bridgeschool.io/}{Bridge}.
Unirse a un grupo existente te dará ventajas en la enseñanza,
un conjunto inmediato de colegas
y la oportunidad de aprender más sobre cómo manejar las cosas;
Con suerte,
aprender esas habilidades a la par que haces una contribución inmediata
será más importante que poder decir que
eres la persona que fundó algo nuevo.

Ya sea que te unas a un grupo existente o inicies uno propio,
tu efectividad será mayor si investigas un poco sobre organización de comunidades.
\cite{Alin1989,Lake2018} son probablemente los trabajos más conocidos sobre organización de grupos de base,
mientras que ~\cite{Brow2007,Midw2010,Lake2018} son manuales prácticos basados en décadas de experiencia.
Si quieres leer más profundamente,
\cite{Adam1975} es la historia de la Highlander Folk School,
cuyo enfoque ha sido emulado por muchos grupos exitosos,
mientras que ~\cite{Spal2014} es una guía para enseñar a personas adultas
escrita por alguien con profundas raíces personales en la organización de grupos de base
y \hreffoot{https://www.nonprofitready.org/}{NonprofitReady.org}
ofrece capacitación profesional gratuita.

\seclbl{Cuatro pasos}{s:community-four-steps}

Todas las personas que se involucren con tu organización
(incluyéndote)
pasan por cuatro fases:
reclutamiento, incorporación, retención y retiro.
No necesitas preocuparte por este ciclo cuando estés gestando una comunidad,
pero sí cuando llegues al punto en que se involucren más de un puñado de personas no fundadoras.

El primer paso es reclutar voluntarias/os.\index{reclutamiento (de miembros)}
Tu estrategia de marketing debería ayudarte con esto, al garantizar que tu organización sea localizable\index{localizable!organización}
y al lograr que la misión y valor sean claros
para las personas que quieran involucrarse (\chapref{s:outreach}).
Comparte historias que ejemplifiquen el tipo de ayuda que buscas 
así como historias sobre las personas a las que estás ayudando,
y deja en claro que hay muchas maneras de involucrarse.
(Discutiremos esto con más detalle en la siguiente sección).

La mejor fuente de potenciales reclutas son tus propias clases:
el método ``verlo, practicarlo, enseñarlo'' ha funcionado bien para organizaciones voluntarias
desde que las organizaciones voluntarias existen.
Asegúrate de que cada clase u otro tipo de encuentro
terminen informando a las personas cómo pueden ayudar y que su ayuda será bienvenida.
De esta manera, las personas que se acerquen a ti sabrán lo que haces
y habrán tenido la experiencia reciente de ser receptores de lo que ofreces,
lo que contribuye a que tu organización evite el punto ciego de la persona experta a nivel de la comunidad (\chapref{s:memory}).

\begin{aside}{Empieza pequeño}
  Pedirle a una persona que haga algo pequeño por ti
  es un buen paso para lograr que haga algo más grande. Esto se basa en el \hreffoot{https://en.wikipedia.org/wiki/Ben\_Franklin\_effect}{efecto Ben Franklin}:
  quien ya le ha hecho un favor a alguien es más propensa/o
  a hacerle /emph{otro} favor a la misma persona,
  (en comparación con la disposisión que tiene quien ha /emph{recibido} un favor).
  En el marco de una comunidad de enseñanza, 
  puedes solicitar correcciones de redacción o de errores ortográficos en los materiales de tus lecciones,
  o sugerencias de nuevos ejercicios o ejemplos.
  Escribir tus materiales de una manera mantenible (\secref{s:process-maintainability}),
  le da a tu comunidad la oportunidad de practicar algunas habilidades útiles
  y te permite a ti comenzar una conversación
  que podría conducir a sumar una nueva persona.
\end{aside}


La mitad del ciclo de vida de una persona voluntaria es la incorporación y la retención,
que cubriremos en las Secciones ~\ref{s:community-onboarding} y ~\ref{s:community-retention}.
El último paso es cuando alguien deja de ser parte de la organización:\index{retiro (de miembros)}
eventualmente, el resto de las personas siguen adelante,
y las organizaciones saludables se preparan para ese momento.
Algunas cosas simples pueden generar una sensación positiva ante el cambio, tanto para la persona que se va como para todas las que se quedan:


\begin{description}

\item[Pide a las personas ser explícitas sobre su partida.]
  para que el resto sepa que realmente se han ido.

\item[Asegúrate de que no se sientan con vergüenza por irse]
  o por cualquier otro motivo.

\item[Dales la oportunidad de transmitir sus conocimientos.]
  Por ejemplo,
  puedes pedirles que asesoren a alguien durante algunas semanas como su última contribución,
  o que sean entrevistadas/os por alguien que se queda en la organización para recopilar cualquier historia que valga la pena volver a contar.

\item[Asegúrate de que entreguen las llaves.]
  Es incómodo descubrir, seis meses después de que alguien se fue,
  que esa era la única persona que sabía cómo reservar un lugar para el picnic anual.

\item[Haz un seguimiento 2 a 3 meses después de que se vayan]
  para ver si tienen más ideas sobre lo que funcionó y lo que no funcionó mientras estuvieron contigo,
  o algún consejo para ofrecer que tampoco pensaron dar
  o que se sentían incómodos de dar mientras se iban.

\item[Agradéceles,]
  tanto cuando se van como la próxima vez que tu grupo se reúna.

\end{description}

\begin{aside}{El manual que falta}
  Se han escrito miles de libros sobre cómo iniciar una empresa.
  Solo unos pocos describen cómo cerrar una o cómo dejarla sin problemas,
  a pesar de que existe un final para cada comienzo.
  Si alguna vez escribes uno,
  por favor házmelo saber.
\end{aside}

\seclbl{Incorporación}{s:community-onboarding}

Después de decidir formar parte de un grupo,
la gente necesita ponerse al día,
y \cite{Shol2019} resume lo que sabemos al respecto.\index{incorporación (de miembros)}
La primera regla es tener y hacer cumplir un código de conducta (\secref{s:classroom-coc})\index{Código de Conducta}
y encontrar alguien independiente a tu comunidad que acepte recibir y revisar informes de comportamiento inapropiado.
Alguien fuera de la organización tendrá la objetividad de la que las/los integrantes pueden carecer.
Además, una persona externa puede proteger a quienes dudan si denunciar incidentes relacionados a las personas
a cargo de proyectos por temor a represalias o daños a su reputación.
El equipo que lidera el proyecto debe hacer públicas las decisiones donde se aplique el código de conducta
para que la comunidad reconozca que se le da relevancia al código.

La siguiente regla más importante es ser amigable.
Como dijo Fogel ~\cite{Foge2005},
``Si un proyecto no hace una buena primera impresión,
quienes recién llegan van a esperar mucho tiempo antes de darle una segunda oportunidad.''
Otras investigaciones han confirmado empíricamente la importancia de entornos sociales, amables y educados
en proyectos abiertos~\cite{Sing2012,Stei2013,Stei2018}:

\begin{description}

\item[Publica un mensaje de bienvenida]
  en las páginas de redes sociales, canales de comunicación de tu comunidad, foros o listas de correo electrónico del proyecto.
  Los proyectos podrían considerar mantener un canal o lista de ``Bienvenida'',
  donde alguna de las personas que lideran el proyecto o gestiona la comunidad escribe una breve publicación pidiendo que quienes recién llegan se presenten.

\item[Ayuda a las personas a hacer una contribución inicial,]
por ejemplo, puedes etiquetar lecciones particulares o talleres que necesitan trabajo como ``adecuados para las personas recién llegadas''
  y pedir a miembros ya establecidos que no las arreglen. De esta forma se aseguran
  lugares adecuados para que las personas recién llegadas comiencen a trabajar.

\item[Dirige a personas recién llegadas hacia miembros similares a ellas]
  para demostrarles que pertenecen a tu comunidad.

\item[Comparte los recursos esenciales del proyecto con las personas recién llegadas],
  tal como las pautas de contribución.

\item[Designa una o dos personas del proyecto como contacto]
  para cada persona nueva.
  Esto puede hacer que quienes recién llegan sean menos reticentes a formular preguntas.

\end{description}

Una tercera regla que ayuda a todas las personas (no solo a quienes recién llegan)
es hacer que el conocimiento esté actualizado y a disposición.
Las personas nuevas son como exploradoras/es que deben orientarse dentro de un paisaje desconocido~\cite{Dage2010}.
La información dispersa hace que las nuevas personas se sientan perdidas y desorientadas.
Considerando las diferentes opciones de lugares para mantener información, 
(por ejemplo, wikis, archivos en un repositorio con control de versiones, documentos compartidos, tweets antiguos, mensajes en un chat grupal o archivos de correo electrónico)
es importante mantener la información sobre un tema específico consolidada en un único lugar,
de modo que las personas nuevas no naveguen por múltiples fuentes de datos para encontrar lo que necesitan.
Organizar la información hace que las personas recién llegadas tengan más confianza y mejor orientación~\cite{Stei2016}.

Finalmente,
reconoce las primeras contribuciones de quienes recién comienzan
y piensa dónde y cómo podrían ayudar a largo plazo.
Una vez realizada su primera contribución,
es probable que tengan una mejor idea de lo que pueden ofrecer
y cómo el proyecto puede ayudarles.
Ayuda a las personas nuevas a encontrar el siguiente problema en el que tal vez quieran trabajar
o guiales hacia el siguiente tema que podrían disfrutar leyendo.
En particular,
animarles a ayudar a la próxima ola de nuevas personas
es una buena manera de reconocer lo que han aprendido
y una forma efectiva de transmitirlo.

\seclbl{Retención}{s:community-retention}
\index{retención (de participantes)}

\begin{quote}

 Si tu gente no la está pasando bien, algo está muy mal.\\
  --- Saul Alinsky\index{Alinsky, Saul}

\end{quote}

Quienes participan de la comunidad no deberían esperar disfrutar cada momento de su trabajo con tu organización,
pero si no disfrutan ninguno
no se quedarán.
El disfrute no necesariamente significa tener una fiesta anual:
la gente puede disfrutar cocinar,
entrenar a otras personas
o simplemente trabajar en silencio junto al resto del grupo.
Hay varias cosas que toda organización debe hacer para garantizar
que las personas tengan algo que valoran de su trabajo:

\begin{description}

\item[Pregunta a las personas qué quieren en vez de adivinar.]
  Así como no eres tus estudiantes (\secref{s:process-personas}),
  probablemente seas diferente a otras personas de tu organización.
  Pregúnta a las personas qué quieren hacer,
  qué se sienten cómodas haciendo (que puede no ser lo mismo),
  y qué limitaciones de tiempo tienen.
  Pueden decir: ``Cualquier cosa''.
  pero incluso una breve conversación probablemente ayude a descubrir que
  les gusta interactuar con las personas pero preferirían no administrar las finanzas del grupo,
  o viceversa.

\item[Proporciona muchas formas de contribuir.]
    Cuantas más formas haya para que las personas ayuden, más gente podrá hacerlo.
  Alguien a quien no le gusta estar frente a público
  puede mantener el sitio web de su organización,
  manejar sus cuentas
  o corregir las lecciones.

\item[Reconoce las contribuciones.]
  A todos y todas nos gusta que nos aprecien,
  así que las comunidades deben reconocer las contribuciones
  de sus miembros, tanto en público como en privado,
  mencionándoles en presentaciones,
  poniéndolas en el sitio web, etc.
  Cada hora que alguien le haya dado a tu proyecto
  puede ser una hora restada de su vida personal o de su empleo oficial;
  reconoce ese hecho
  y deja en claro que, si bien más horas serían bienvenidas,
  no esperas que hagan sacrificios insostenibles.

\item[Haz espacio.]
  Crees que estás siendo útil,
  pero intervenir en cada decisión priva de autonomía a las demás personas,
  lo que reduce su motivación (\secref{s:motivation}).
  En particular, si siempre eres quien responde primero a correos electrónicos o mensajes de chat,
  las personas tienen menos oportunidades de crecer como miembros
  y crear colaboraciones horizontales.
  Como resultado,
  la comunidad continuará centrada en una o dos personas
  en lugar de convertirse en una red altamente conectada
  en la que otras/os se sientan cómodas/os participando.

\end{description}

Otra forma de recompensar la participación es ofrecer capacitación.
Las organizaciones necesitan presupuestos, propuestas de subsidios y resolución de disputas.
A la mayoría de las personas no se les enseña cómo hacer estas tareas más de lo que se les enseña a enseñar,
así que la oportunidad de adquirir habilidades transferibles
es una razón poderosa para que las personas se involucren y se mantengan involucradas.
Si vas a hacer esto, no intentes proporcionar la capacitación por tu cuenta
a menos que sea en lo que te especializas.
Muchos grupos cívicos y comunitarios tienen programas de este tipo
y probablemente puedas llegar a un acuerdo con alguno de ellos.

Finalmente,
si bien  las personas voluntarias pueden hacer mucho,
tareas como la administración del sistema y la contabilidad eventualmente necesitan personal remunerado.
Cuando llegues a este punto, o no pagas nada o pagas un salario adecuado.
Si no les paga nada, su verdadera recompensa es la satisfacción de hacer el bien;
por otro lado, si les pagas una cantidad simbólica, le quitas esa satisfacción sin darles la posibilidad de ganarse la vida.

\seclbl{Gobernanza}{s:community-governance}
\index{gobernanza}

Cada organización tiene una estructura de poder:
la única pregunta es si es formal y deriva en responsabilidades, o informal y, por lo tanto, no hay responsabilidades explícitas~\cite{Free1972}.
La estructura informal funciona bastante bien para grupos de hasta media docena de personas
en los que todas las personas se conocen.
Si el grupo es más range,
necesitas reglas para explicar
quién tiene la autoridad para tomar qué decisiones
y cómo lograr consensos (\secref{s:meetings-marthas-rules}).

El modelo de gobernanza que prefiero se denomina \gref{g:commons}{Los comunes} (\emph{commons}, en inglés). Los comunes es una manera de gestión conjunta realizada por una comunidad
de acuerdo a las reglas que ella misma ha desarrollado y adoptado~\cite{Ostr2015}.
Como subraya ~\cite{Boll2014}, las tres partes de esta definición son esenciales:
los comunes no es solo una propiedad compartida o un conjunto de bibliotecas de software,
sino que también incluye a la comunidad que los comparte y a las reglas que usan para hacerlo.

Los modelos más populares son los de corporaciones con fines de lucro y los de organizaciones sin fines de lucro;
la mecánica varía de una jurisdicción a otra,
por lo que debes buscar asesoramiento antes de elegir\footnote{
  Este es uno de los momentos
  en que vale la pena tener vínculos con el gobierno local u otras organizaciones afines.}.
Ambos tipos de organización otorgan la máxima autoridad a su directorio.
En términos generales, se trata de un \gref{g:service-board}{directorio de servicio}
con miembros que también asumen otras funciones en la organización
o un \gref{g:governance-board}{directorio de gobernanza} o simplemente directorio cuya 
responsabilidad principal es contratar, supervisar
y, si es necesario, despedir quien dirige la organización.
Las/los integrantes del directorio pueden ser elegidos por la comunidad o nombrados;
en cualquier caso, es importante priorizar la capacidad sobre la pasión
(la última es más importante para las bases)
y tratar de reclutar habilidades particulares como contabilidad, marketing, etc.

\begin{aside}{Elige la democracia}
  Cuando llegue el momento,
  haz de tu organización una democracia:
  tarde o temprano (generalmente más temprano que tarde),
  cada directorio o junta designada se convierte en una sociedad de mutuo acuerdo.
  Darle poder a tus miembros es complicado,
  pero es la única forma inventada hasta ahora para garantizar
  que las organizaciones continúen satisfaciendo las necesidades reales de las personas.
\end{aside}

\seclbl{Cuidarte y cuidar a tu comunidad}{s:community-care}

El síndrome de desgaste ocupacional (emph{burnout} en inglés) es un riesgo crónico en cualquier actividad comunitaria~\cite{Pign2016},\index{burnout}
\index{desgaste ocupacional}
así que aprende a decir \emph{no} más seguido de lo que dices \emph{sí}.
Si no te cuidas,
no podrás cuidar a tu comunidad.

\begin{aside}{Quedándose sin ``No''}
  Investigaciones en la década de 1990 parecían mostrar que nuestra capacidad de ejercer fuerza de voluntad es finita:
  si tenemos que resistirnos a comer el último bombón de la caja cuando tenemos hambre,
  somos menos propensas/os a doblar la ropa, y viceversa.
  Este fenómeno se llama \gref{g:ego-depletion}{agotamiento del ego}
  y si bien estudios recientes no han podido replicar esos primeros resultados~\cite{Hagg2016},
  decir ``sí'' cuando estamos demasiado cansados para decir ``no''
  es una trampa en la que caen muchos organizadores.
\end{aside}

Una forma de asegurarte de cumplir con tu ``no''
es escribir una lista de no-tareas, en la que anotes cosas que vale la pena hacer
pero que \emph{no} vas a hacer.
Al momento de escribir este libro, mi lista incluye cuatro libros,
dos proyectos de software,
el rediseño de mi sitio web personal
y aprender a tocar la flauta irlandesa. 

Finalmente,
recuerda de vez en cuando que
eventualmente toda organización necesita ideas frescas y nuevo liderazgo.
Cuando llegue ese momento,
entrena a tus sucesoras/es y sigue adelante con la mayor gracia posible.
Indudablemente tomarán decisiones que tú no harías,
pero pocas cosas en la vida son tan satisfactorias como 
ver que algo que ayudaste a construir cobra vida propia.
Celebra eso --- no tendrás ningún problema para encontrar otra actividad que te mantenga ocupada/o.

\seclbl{Ejercicios}{s:community-exercises}

Varios de estos ejercicios se toman de~\cite{Brow2007}.

\exercise{¿Qué tipo de comunidad?}{individual}{15'}

Vuelve a leer la descripción de los cuatro tipos de comunidades
y decide con cuál o cuáles se identifican o a cuales aspiran.

\exercise{Personas que puedes conocer}{grupos pequeños}{30'}

Al organizar una comunidad,
parte de tu trabajo es ayudar a las personas a encontrar una manera de contribuir a pesar de sí mismas.
En grupos pequeños,
elige tres de las personas descriptas a continuación
y discute cómo les ayudarías a convertirse en mejores colaboradoras/es para tu organización.

\begin{description}

\item[Ana]
  sabe más sobre cada tema que todas las demás personas juntas --- al menos,
  ella cree que lo hace.
  No importa lo que digas,
  ella te corregirá;
  no importa lo que sepas, ella lo sabe mejor.
    
\item[Catalina]
  tiene tan poca confianza en su propia habilidad
  que no tomará ninguna decisión,
  sin importar qué tan pequeña sea,
  hasta haber consultado con alguien más.

\item[Fernando]
  disfruta de saber cosas que otras personas no saben.
  Puede hacer milagros,
  pero cuando se le pregunta cómo lo hizo,
  sonreirá y dirá:
  ``Oh, estoy seguro de que puedes resolverlo''.

\item[Andrea]
  es tranquila.
  Nunca habla en las reuniones,
  incluso cuando sabe que otras personas están equivocadas.
  Podría contribuir a la lista de correo,
  pero es muy sensible a las críticas
  y siempre retrocede en lugar de defender su punto.

\item[Cristian]
  aprovecha el hecho de que la mayoría de las personas preferirían hacer su parte del trabajo antes 
  que quejarse de él.
  Lo frustrante es que él es muy convincente cuando alguien finalmente lo confronta.
  `Todas las partes han cometido errores,''
  dice él,
  o ``Bueno, creo que estás siendo un poco quisquillosa/o.''.

\item[Melisa]
  tiene buenas intenciones
  pero, por alguna razón, siempre surge algo
  y termina sus tareas a último momento.
  Por supuesto,
  eso significa que quienes dependen de ella no pueden hacer su trabajo
  hasta \emph{después} de ese último momento {\ldots}
 
\item[Juan José]
 es grosero.
  ``Esa es la forma en que hablo'', dice.
  ``Si no puedes solucionarlo, ve a buscar otro equipo''.
  Su frase favorita es: ``Eso es estúpido''.
  y dice una obscenidad cada segunda oración.

\end{description}

\exercise{Valores}{grupos pequeños}{45'}

Responde estas preguntas por tu cuenta y
luego compara tus respuestas con las del resto del grupo.

\begin{enumerate}

\item
  ¿Cuáles son los valores que expresa tu organización?

\item
  ¿Son estos los valores que deseas que la organización exprese?

\item
  Si tu respuesta es no, ¿qué valores te gustaría expresar?

\item
  ¿Cuáles son los comportamientos específicos que demuestran esos valores?

\item
  ¿Qué comportamientos demostrarían lo contrario de esos valores?
\end{enumerate}

\exercise{Procedimientos para reuniones}{grupos pequeños}{30'}

Responde estas preguntas por tu cuenta y
luego compara tus respuestas con las del resto del grupo.

\begin{enumerate}

\item
  ¿Cómo se manejan las reuniones?

\item
  ¿Es así como quieres que se realicen tus reuniones?

\item
  ¿Las reglas para conducir reuniones son explícitas o simplemente se asumen?

\item
  ¿Estas son las reglas que quieres tener?

\item
  ¿Quién tiene derecho a votar o tomar decisiones?

\item
  ¿Son estas las personas a quienes quisieras otorgar autoridad para tomar decisiones?

\item
  ¿Utilizan la regla de la mayoría, toman decisiones por consenso o usan otro mecanismo?

\item
  ¿Es así como quieres tomar decisiones?

\item
  ¿Cómo se enteran las personas en una reunión que se ha tomado una decisión?

\item
  ¿Cómo saben las personas que no estuvieron en una reunión qué decisiones se tomaron?

\item
  ¿Funciona esto para tu grupo?

\end{enumerate}

\exercise{Tamaño}{grupos pequeños}{20'}

Responde estas preguntas por tu cuenta y
luego compara tus respuestas con las del resto del grupo.


\begin{enumerate}

\item
¿Qué tan grande es tu grupo?

\item
  ¿Es este el tamaño que deseas para su organización?

\item
  Si tu respuesta es no, ¿de qué tamaño te gustaría que fuera?

\item
  ¿Tienes algún límite en cuanto a la cantidad de miembros?

\item
  ¿Te beneficiarías de establecer ese límite?

\end{enumerate}

\exercise{Convertirse en miembro}{grupos pequeños}{45}

Responde estas preguntas por tu cuenta y
luego compara tus respuestas con las del resto del grupo.


\begin{enumerate}

\item
  ¿Cómo se incorpora alguien a tu grupo?

\item
  ¿Qué tan bien funciona este proceso?

\item
    ¿Hay cuotas de membresía?

\item
  ¿Se requiere que las personas estén de acuerdo con alguna regla de comportamiento al unirse?

\item
  ¿Son estas las reglas de comportamiento que quieres?

\item
  ¿Cómo descubren las personas recién llegadas lo que hay que hacer?

\item
  ¿Qué tan bien funciona este proceso?
 
\end{enumerate}

\exercise{Contrataciones}{grupos pequeños}{30'}

Responde estas preguntas por tu cuenta y
luego compara tus respuestas con las del resto del grupo.


\begin{enumerate}

\item
  ¿Tienes personal asalariado en tu organización o 
  o son todas/os voluntarias/os?

\item
  ¿Deberías tener personal asalariado?

\item
  ¿Quieres / necesitas más o menos personal?

\item
  ¿Qué hace cada integrante de tu personal? ¿A qué se dedica?

\item
  ¿Son estos los roles y funciones principales que necesitas que el personal desempeñe?

\item
  ¿Quién supervisa a tu personal?

\item
  ¿Es este el proceso de supervisión que quieres para tu grupo?

\item
  ¿Cuánto le pagan a tu personal?

\item
  ¿Es este el salario adecuado para realizar el trabajo necesario?

\end{enumerate}

\exercise{Dinero}{grupos pequeños}{30'}

Responde estas preguntas por tu cuenta y
luego compara tus respuestas con las del resto del grupo.

\begin{enumerate}

\item
  ¿Quién paga qué?

\item
  ¿Esa es la persona que quisieras que pague? 
  
\item
  ¿De dónde consiguen el dinero?

\item
  ¿Es así como quieres conseguirlo?

\item
  Si no, ¿tienes algún plan para conseguirlo de otra manera?

\item
  Si es así, ¿cuáles son esos planes?

\item
  ¿Quién da seguimiento a estos planes para asegurarse de que sucedan?

\item
  ¿Cuánto dinero tienen?

\item
  ¿Cuánto dinero necesitan?

\item
  ¿En qué gastan la mayor parte del dinero?

\item
  ¿Es así como quieres gastar el dinero?

\end{enumerate}

\exercise{Tomando ideas prestadas}{toda la clase}{15'}

Muchas de mis ideas sobre cómo construir una comunidad
han sido moldeadas por mi experiencia en el desarrollo de software de código abierto.
\cite{Foge2005} (que está \hreffoot {http://producingoss.com/}{disponible en línea})
es una buena guía de lo que ha funcionado y lo que no ha funcionado para esas comunidades,
y el \hreffoot{https://opensource.guide/es} {sitio de Guías de Código Abierto}
también tiene una gran cantidad de información útil.
Elige una sección de este último recurso, puede ser  ``Encontrando Usuarios para Tu Proyecto'',
o ``Liderazgo y Gobierno''
y presenta al grupo, en dos minutos, una idea
que encontraste útil o una con la que estuviste muy en desacuerdo.

\exercise{¿Quién eres tú?}{grupos pequeños}{20'}

La Administración Nacional Oceánica y Atmosférica estadounidense (\emph{NOAA} por sus siglas en inglés) tiene una guía breve, útil y divertida para
\hreffoot{https://coast.noaa.gov/ddb/}{lidiar con comportamientos disruptivos}.
La guía clasifica esos comportamientos bajo etiquetas como ``habladora/hablador'', ``indecisa/o'' y ``tímida/o''
y describe estrategias para manejar cada uno.
En grupos de 3 a 6 personas,
lean la guía (está disponible solo en inglés) y decidan cuál de las descripciones les queda mejor.
¿Crees que las estrategias descriptas para manejar personas como tú son efectivas?
¿Son otras estrategias igualmente efectivas o más?

\exercise{Creando lecciones en comunidad}{grupos pequeños}{30'}

Una de las claves del éxito de \hreffoot{http://carpentries.org}{Carpentries}
es su énfasis en construir y mantener lecciones en forma colaborativa~\cite{Wils2016,Deve2018}.
Trabajando en grupos de 3--4:

\begin{enumerate}

\item
  Elige una lección breve que todo el grupo haya usado.

\item
  Haz una revisión cuidadosa para crear una única lista con sugerencias de mejoras.

\item
  Ofrece esas sugerencias al autor de la lección.

\end{enumerate}

\exercise{¿Estás crujiente?}{individual}{10'}

\hreffoot{https://mailchi.mp/d54702d0a790/take-my-horse-to-the-sand-hill-road}{Johnathan Nightingale escribió}:

\begin{quote}
  Cuando trabajaba en Mozilla,
  utilizábamos el término ``crujiente'' (\emph{crispy} en inglés) para referirnos al estado justo antes de llegar al síndrome de desgaste ocupacional.
  Las personas que son crujientes no son divertidas.
  Son bruscas.
  Están ansiosas por una pelea que pueden ganar.
  Lloran sin mucha advertencia.
  {\ldots} reconocíamos lo ``crujiente'' en  nuestros colegas y nos cuidábamos  mutuamente
  [pero] es una cosa tan fea, que vimos tanto, que teníamos un proceso cultural completo a su alrededor.
\end{quote}

\noindent
Responde ``sí'' o ``no'' a cada una de las siguientes preguntas.
¿Qué tan cerca estás de tener síndrome de desgaste ocupacional?

\begin{itemize}
\item ¿Te has vuelto cínica/o o crítica/o en el trabajo?
\item ¿Tienes que arrastrarte al trabajo o tienes problemas para comenzar a trabajar?
\item ¿Te has vuelto irritable o impaciente con tus compañeros de trabajo?
\item ¿Te resulta difícil concentrarte?
\item ¿No logras satisfacción con tus logros?
\item ¿Estás usando comida, drogas o alcohol para sentirte mejor o simplemente no sentir?
\end{itemize}
