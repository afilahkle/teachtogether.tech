\chapter{Ejemplos de mapas conceptuales}\label{s:conceptmaps}

\begin{reviewer}
{Yanina Bellini Saibene}
{Laura Acion y Natalia Morandeira}
\end{reviewer}

Estos mapas conceptuales fueron creados por Amy Hodge de la Universidad de Stanford
y se reutilizan con su permiso.

\figpdfhere{figures/library-patron-concept-map.pdf}{Mapa conceptual desde el punto de vista los/las socios/as de la biblioteca}{f:conceptmaps-library-patron}{Mapa conceptual desde el punto de vista de los/las socios/as de la biblioteca: los/as socios/as de la biblioteca llevan niños/as a talleres de cuentos que se realizan en el edificio de la biblioteca, el cual está lleno de libros y materiales. Los/as socios/as también buscan libros y hacen preguntas al personal de la biblioteca. Este personal se encarga de comprar libros y materiales, ayuda a los/las socios/as y evalúa si debe aplicar multas a algunos socios/as.}

\figpdfhere{figures/library-director-concept-map.pdf}{Mapa conceptual desde el punto de vista de la dirección de la biblitoteca}{f:conceptmaps-library-director}{Mapa conceptual desde el punto de vista de la dirección de la biblioteca: el presupuesto de la bilbioteca es controlado por el gobierno de la ciudad. Con este presupuesto se paga por el personal, las instalaciones, el material y los programas de la biblioteca. Los/las socios/as usan los materiales, participan en los programas y visitan las instalaciones donde trabaja el personal. Este personal es gobernado por los sindicatos que pueden influenciar al gobierno de la ciudad.}

\figpdfhere{figures/library-friends-concept-map.pdf}{Mapa conceptual desde el punto de vista de los/las amigos/as de la biblioteca}{f:conceptmaps-library-friends}{Mapa conceptual desde el punto de vista de los/las amigos/as de la biblioteca: Los/las amigos/as de la biblioteca realizan las ventas de libros que generan fondos. Estos fondos se usan para pagar el mantenimiento y mejoras de las instalaciones y para pagar la realización de programas que apoyan a los socios/as y a la comunidad. Algunos/as de estos/as socios/as son parte de los/as amigos/as de la biblitoeca y también participan en las ventas de libros. El personal de la biblioteca es consultado por los socios/as sobre las instalaciones y los programas disponibles en la biblioteca.}
