\chapter{Un proceso para diseñar lecciones}\label{s:process}

La mayoría de las personas diseñan lecciones de esta manera:
\begin{enumerate}

 \item
Alguien te pide que enseñes algo que sabes muy poco o que no has pensado en años.

 \item
Empiezas escribiendo diapositivas para explicar lo que sabes sobre el tema.

 \item
Después de 2 o 3 semanas,
preparas una tarea basada en lo que has enseñado hasta ahora
 \item
Repites el paso 3 varias veces.
 \item
Permaneces sin dormir hasta altas horas de la mañana
para crear un examen final
y te prometes que serás más organizada/o la próxima vez.
\end{enumerate}

Un método más efectivo es similar en esencia a una práctica de programación llamada \gref{g:test-driven-development}{desarrollo-impulsado-por pruebas}(TDD por sus siglas en inglés).
Las programadoras y los programadores que usan TDD no escriben software
y luego testean/prueban que esté funcionando correctamente.
En su lugar,
escriben la prueba primero,
luego escriben suficiente software nuevo para que esas pruebas pasen.
 
TDD funciona porque escribir pruebas obliga a quienes programan a ser más precisas/os acerca de lo que están intentando lograr y cómo se ve ``hecho''.
TDD también evita el pulido sin fin:
cuando pasan las pruebas, dejas de codificar.
Finalmente,
esto reduce el riesgo de sesgo de confirmación:
alguien que aún no ha escrito un fragmento de software
será más objetivo que alguien que acaba de dedicar varias horas al trabajo duro
y realmente, realmente quiere terminar.
 
Un método similar denominado \gref{g:backward-design}{Reingeniería} funciona muy bien para el diseño de lecciones.
Este método fue desarrollado en forma independiente en ~\cite{Wigg2005,Bigg2011,Fink2013} y está resumido en~\cite{McTi2013}
En forma simplificada, sus pasos son:
 
\begin{enumerate}
 
\item
Crear o reciclar estudiante tipo (discutido en la siguiente sección)
para imaginar a quién estás intentando ayudar y qué les atraerá.

 \item
Haz una lluvia de ideas para tener una idea aproximada de lo que quieres cubrir,
cómo lo vas a hacer,
qué problemas o conceptos erróneos esperas encontrar,
qué \emph{no} no se va a incluir,
y etc.
Dibujar mapas conceptuales puede ayudar mucho en esta etapa (\secref{s:memory-concept-maps }).\index{concept map}
 
\item
Crea una evaluación sumativa (\secref{s:models-formative-
assessment})\index{summative assessment}
para definir tu objetivo general.
Esto puede ser el examen final para un curso
o el proyecto final para un taller de un día;
independientemente de su forma o tamaño,
muestra lo lejos que esperas llegar
más claramente que una lista puntual de objetivos.
 
\item
Crea evaluaciones formativas\index{formative assessment}
eso le dará a las personas una oportunidad para practicar las cosas que están aprendiendo.
Estas también te dirán a ti (y a las personas) si están progresando
y dónde deben centrar su atención.
El mejor camino para hacer esto es detallar los conocimientos y las habilidades 
utilizadas en la evaluación sumativa que desarrollaste en el paso anterior y 
luego crear al menos una evaluación sumativa para cada uno.
 
\item
Ordena las evaluaciones formativas para crear un esquema del curso
en función de su complejidad,
sus dependencias,
y cuán bien los temas motivarán a tus aprendices (\secref{s:motivation-authentic}).
 
\item
Escribe material para conseguir que los/las alumnos/as pasen de una evaluación formativa a la siguiente.
Cada hora de instrucción debe constar de tres a cinco episodios.
 
\item
Escribe una descripción resumida del curso
para ayudar a tu audiencia objetivo a encontrarlo
y averiguar si es adecuado para ella.
 \end{enumerate}
 
Este método ayuda a mantener la enseñanza enfocada en sus objetivos.
También asegura que las/los estudiantes no se enfrenten a nada para lo que no están preparadas/os al final del curso.
 
\begin{aside}{Incentivos Perversos}
La reingeniería \emph{no} es lo mismo que gref{g:teaching-to-the-test}{enseñar para el examen}.
Cuando se usa la reingeniería,
el conjunto de docentes establece objetivos para ayudar en el diseño de las lecciones;
es posible que ellas/ellos nunca den el examen final que escribieron.
En muchos sistemas escolares,
por otro lado,
una autoridad externa define los criterios de evaluación para todas/todos las/los estudiantes,
independientemente de sus situaciones individuales.
Los resultados de esas evaluaciones sumativas afectan directamente el salario y promoción de las y los profesores,
lo que significa que estos tienen un incentivo para enfocarse en que las/los estudiantes pasen las pruebas en lugar de ayudarles a aprender.
  
\cite{Gree2014} argumenta que enfocarse en las pruebas y la medición hace un llamado a quienes tienen el poder de establecer las  pruebas, pero es poco probable mejorar los resultados
a menos que esto se acompañe con apoyo para que las/los profesores realicen mejoras basadas en los resultados de las pruebas.
Esto último a menudo falta porque
las grandes organizaciones usualmente valoran uniformidad por sobre la productividad ~\cite{Scot1998}.
\end{aside}
 
El diseño inverso se describe como una secuencia,
pero casi nunca se hace de esa manera.
Podemos,
por ejemplo, cambiar nuestra opinión acerca de lo que queremos enseñar
en base a algo que se nos ocurre mientras estamos escribiendo un MCQ,
o re-evaluar a quién estamos intentando ayudar una vez que tengamos un resumen de la lección.
Sin embargo,
las notas que dejamos atrás deben presentar cosas en el orden descrito anteriormente para que quien tenga que usar o mantener la lección después de nosotras/nosotros pueda seguir sobre nuestro pensamiento
~\cite{Parn1986}.
  
\seclbl{Estudiante tipo}{s:process-personas}
 
El primer paso en el proceso de diseño inverso es averiguar quién es tu audiencia.
Una manera para hacer esto es escribir dos o tres
\grefdex{g:learner-persona}{learner personas}{estudiante tipo}
como los de \secref{s:intro-audience}.
Esta técnica es tomada de diseñadores de experiencia de usuario,
quienes crean perfiles breves de usuarios típicos
para ayudarles a pensar en su audiencia.
Una/un estudiante tipo consiste en:
 
\begin{enumerate}
 
\item
antecedentes generales de la persona
 
\item
lo que ya saben;
 
\item
lo que quieren hacer;
y
 
\item
cualquier necesidad especial que tengan.
 
\end{enumerate}
 Las personas en \secref{s:intro-audience} tienen los cuatro puntos listados anteriormente,
junto con un breve resumen de cómo este libro les ayudará.
Una/un estudiante tipo para un grupo de voluntarios que realiza talleres de Python los fines de semana sería:
 
\begin{enumerate}
 
\item
 Jorge se acaba de mudar de Costa Rica a Canadá para estudiar ingeniería agrícola.
El se ha unido al equipo de fútbol universitario
y espera aprender a jugar hockey sobre hielo.
 
\item
Aparte de usar Excel, Word y el internet,
la experiencia previa más significativa de Jorge con computadoras
es ayudar a su hermana a construir un sitio en WordPress
para el negocio familiar en casa.
 
 
\item
  Jorge quiere medir las propiedades del suelo en granjas cercanas
usando un dispositivo de mano que envía datos a su computadora.
Ahora mismo él tiene que abrir cada archivo de datos en Excel,
eliminar la primera y la última columna,
y calcular algunas estadísticas sobre lo que queda.
El tiene que recopilar al menos 600 mediciones en los próximos meses
y realmente no quiere tener que hacer estos pasos a mano para cada uno.
 
\item
Jorge puede leer Inglés bien,
pero algunas veces le cuesta sostener una conversación hablada que involucre mucha jerga.
 
\end{enumerate}
 
En lugar de escribir nuevos tipos para cada lección o curso,
las/los profesores usualmente crean y comparten media docena
que cubren a todas las personas a las que probablemente enseñan,
luego eligen algunos de ese conjunto para describir a la audiencia un material en particular.
Los tipos que se usan de esta manera se convierten en un conveniente atajo para los problemas de diseño:
Al hablar entre nosotras/os,
las/los profesores pueden decir,
``¿Jorge entendería por qué estamos haciendo esto?''
o
``¿Qué problemas de instalación enfrentaría Jorge?''
 
\begin{aside}{Sus metas, no las tuyas}
Los tipo deberían siempre describir lo que la/el estudiante quiere hacer
en lugar de lo que creen que necesitan.
Pregúntate lo que ellas/ellos están buscando en línea;
probablemente no incluirá jerga que no conocen aún,
así que parte de lo que tienes que hacer como diseñadora/diseñador instruccional es
descubrir cómo hacer tu lección encontrable (visible). index{findability!of lessons}
\end{aside}


 \seclbl{Objetivos de aprendizaje}{s:process-objectives}
 
Evaluaciones formativas y sumativas ayudan a las/los profesores a descubrir lo que van a enseñar,
pero para comunicar eso a las/los estudiantes y otro conjunto de docentes,
debe tener también una descripción del curso.
\grefdex{g:learning-objective}{learning objectives}{objetivo de aprendizaje}.
Estos ayudan a asegurar que
todos tengan el mismo entendimiento de lo que se supone que una lección debe lograr.
Por ejemplo,
una declaración como ``entendiendo Git'' podría significar cualquiera de los siguientes ítemes:
 
\begin{itemize}
 
\item
  Las/los estudiantes pueden describir tres maneras
  en la cual los sistemas de versión de control como Git  son mejores que herramientas para compartir archivos como Dropbox
  y dos formas que son las peores.
 
\item
Las/los estudiantes pueden hacer commit a un archivo modificado en un repositorio de Git
usando una herramienta GUI de escritorio.
 
\item
  Las/los estudiantes pueden explicar qué es un HEAD por separado
 y recuperarlo usando operaciones de línea de comandos.
 
\end{itemize}
 

\begin{aside}{Objetivos vs. Resultados}
 Un objetivo de aprendizaje es lo que una lección se esmera por lograr.
 Un \gref{g:learning-outcome}{resultado de aprendizaje} es lo que realmente se logra,
 es decir, lo que las/los estudiantes realmente se llevan.
El rol de la evaluación sumativa es por lo tanto
para comparar resultados de aprendizajes con objetivos de aprendizajes.
\end{aside}
 
Un objetivo de aprendizaje describe cómo la/el estudiante demostrará lo que ha aprendido
una vez que ha completado exitosamente una lección.
Más específicamente,
este tiene un \emph{verbo medible o verificable} que establece lo que la/el estudiante hará
y especifica los \emph{criterios aceptables de rendimiento}.
Escribirlos puede inicialmente parecer restrictivo,
pero ellos te harán a ti,
a tus compañeras/compañeros docentes,
y a tus estudiantes más felices a largo plazo:
terminarás con guías claras tanto para su enseñanza como para su evaluación,
y tus estudiantes apreciarán tener expectativas claras.

Una forma de comprender lo que constituye un buen objetivo de aprendizaje
es ver cómo se puede mejorar uno pobre:

 \begin{itemize}
 
\item
  \emph{La/el estudiante tendrá oportunidad para aprender buenas prácticas de programación.}\\
Esto describe el contenido de la lección,
no los atributos de éxito de las/los estudiantes.\\
 
\item
  \emph{La/el estudiante tendrá una mejor apreciación
de las buenas prácticas de programación.}\\
 Esto no empieza con un verbo activo ni define el nivel de aprendizaje,
y el tema de aprendizaje no tiene contexto y no es específico.\\
 
\item
  \emph{La/el estudiante comprenderá cómo programar en R.}\\
  Si bien esto comienza con un verbo activo,
   no define el nivel aprendizaje, 
  y el tema de aprendizaje es todavía demasiado vago para evaluarlo.\\
 
\item
  \emph{La/el estudiante escribirá scripts de análisis de datos para leer, filtrar y resumir datos tabulares usando R.}\\
 Esto comienza con un verbo activo,
define el nivel de aprendizaje
y provee contexto para asegurar que los resultados puedan evaluarse.
\end{itemize}
 
Cuando se trata de elegir verbos,
la mayoría de docentes usan la gref{g:blooms-taxonomía}{Taxonomía de Bloom}.
Publicado por primera vez en 1956 y actualizado a principios de siglo ~\cite{Ande2001},
es un marco ampliamente usado para discutir los niveles de comprensión.
Su forma más reciente tiene 6 categorías;
la lista a continuación da algunos de los verbos típicamente usados en los objetivos de aprendizaje escritos para cada uno:


\begin{description}
 
\item[Recordar:]
  Demostrar memoria del  material previamente aprendido
recordando hechos, términos,
conceptos básicos y respuestas.
  \emph{(reconocer, listar, describir, nombrar, encontrar.)}
 
\item[Comprender:]
  Demostrar comprensión de los hechos e ideas
organizando, comparando, traduciendo, interpretando, dando descripciones y estableciendo ideas principales.
  \emph{(interpretar, resumir, parafrasear, clasificar, explicar)}
 
\item[Aplicar:]
  Resolver problemas nuevos aplicando los conocimientos,
hechos, técnicas y reglas adquiridos de una forma diferente
  \emph{(construir, identificar, usar, planificar, seleccionar)}
 
\item[Analizar:]
  Examinar y dividir la información en partes identificando motivos o causas,
  hacer inferencias y encontrar evidencia para apoyar generalizaciones.
  \emph{(comparar, contrastar, simplificar)}
 
\item[Evaluar:]
  Presentar y defender opiniones emitiendo juicios sobre información,
  validez de las ideas,
o calidad del trabajo basada en un conjunto de criterios.
  \emph{(comprobar, elegir, criticar, probar, calificar)}
 
\item[Crear:]
 Recopilar información de forma diferente
combinando elementos en un nuevo patrón o proponiendo soluciones alternativas.
  \emph{(diseñar, construir, mejorar, adaptar, maximizar, resolver)}
 
\end{description}
 
La Taxonomía de Bloom aparece en casi todos los libros de texto sobre educación,
pero ~\cite{Masa2018} encontró que
Incluso los educadores experimentados tienen problemas para ponerse de acuerdo
sobre cómo clasificar cosas específicas.
Los verbos siguen siendo útiles,
aunque,
al igual que la noción de construir la comprensión en pasos:
como Daniel Willingham ha dicho, \index{Willingham, Daniel}
la gente no puede pensar sin algo en qué pensar ~\cite{Will2010},
y esta taxonomía puede ayudar a las/los docentes a asegurarse
que las/los estudiantes tengan esas cosas cuando las necesiten.

Otra manera de pensar acerca de los objetivos de aprendizaje proviene de ~\cite{Fink2013},
el cual define aprendizaje en términos del cambio que se supone que debe producirse en la/el estudiante.
La \gref{g:finks-taxonomy}{Taxonomía de Fink} también tiene seis categorías,
pero a diferencia de las de Bloom ellas son complementarias en lugar de jerárquicas:
 
\begin{description}
 
\item[Conocimiento fundamental:]
  Comprender y recordar información e ideas.
  \emph{(recordar, comprender, identificar)}
 
\item[Aplicación:]
  habilidades, pensamiento crítico, gestión de proyectos.
  \emph{(usar, resolver, calcular, crear)}
 
\item[Integración:]
  conectar ideas, experiencias de aprendizaje y vida real
  \emph{(conectar, relatar, comparar)}
 
\item[Dimensión Humana:]
  Aprender sobre sí misma/mismo y otras/otros.
  \emph{(llegar a verse a sí misma/mismo, entender a las/los demás en términos de, decidir ser)}
 
\item[Cuidando:]
  Desarrollar nuevos sentimientos, intereses y valores
  \emph{(emocionarse, estar preparada/preparado para, valorar)}
 
\item[Aprendiendo a aprender:]
  Convertirse en una/un mejor estudiante.
  \emph{(identificar la fuente de información para, enmarcar preguntas útiles sobre)}
 
\end{description}
 
Un conjunto de objetivos de aprendizaje basados en esta taxonomía para un curso introductorio sobre HTML y CSS sería:
 
\begin{itemize}
 
\item
  Explicar qué son las propiedades de CSS y cómo funcionan los selectores de CSS.
 
\item
  Diseñar una página web usando etiquetas comunes y propiedades CSS.
 
\item
 Comparar y contrastar la escritura HTML y CSS
para escribir con herramientas de edición de escritorio.
  
\item
  Identificar y corregir problemas en páginas web de muestra
  que dificultarían la interacción de las personas con discapacidad visual.
 
\item
Describir las características de los sitios web favoritos
cuyo diseño te atraiga de forma particular
y explica el por qué.
 
\item
  Describir tus dos fuentes de información favoritas en línea acerca
de CSS y explica qué te gusta de ellas.
 
\end{itemize}
 
\seclbl{Mantenimiento}{s:process-maintainability}
 
Una vez que una lección ha sido creada alguien debe mantenerla,
y hacerlo es mucho más fácil si se ha construido de manera que se pueda mantener.
¿Pero qué significa exactamente ``mantenible´´? \index{ maintainability (of lessons)}
La respuesta corta es que una lección es mantenible
si es más barato actualizarla que reemplazarla.
Esta ecuación depende de 4 factores:
 
\begin{description}
 
\item[¿Qué tan bien documentado está el diseño del curso?]
  Si la persona que realiza el mantenimiento no conoce (o no recuerda)
  lo que se supone la lección debe lograr,
  o por qué los temas son presentados en un orden en particular,
  le llevará más tiempo actualizarla.
  Una razón para usar el diseño inverso
  es captar decisiones sobre por qué cada curso es como es.
 
\item[¿Qué tan fácil es para los colaboradores ayudar?]
  Las/los docentes suelen compartir material enviándose por correo archivos de PowerPoint entre ellas/ellos o poniéndolos en una unidad compartida.
  Herramientas de escritura colaborativa como \hreffoot{http://docs.google.com}{\emph{Google Docs}} and wikis
  son una gran mejora,
  ya que permiten que muchas personas actualicen el mismo documento
  y comenten las actualizaciones de otras personas.
  El sistema de control de versiones usado por programadores,
  tales como \hreffoot{http://github.com}{\emph{GitHub}},
  son otro enfoque.
  Permiten que cualquier número de personas trabajen de forma independiente
  y luego unir sus cambios en forma controlada y revisable.
  Desafortunadamente,
  los sistemas de control de versión tienen una curva de aprendizaje pronunciada
   y no manejan formatos de documentos de oficina comunes.
 
\item[Qué tan dispuestas están las personas a colaborar.]
  Las herramientas necesarias para construir un Wikipedia para lecciones
  existen hacen veinte años,
  pero la mayoría de docentes no escriben ni comparten sus lecciones
  de la misma manera en que escriben y comparten las entradas de enciclopedias.
 
\item[Qué tan útil es compartir en realidad.]
  La \gref{g:reusability-paradox}{Paradoja de la Reusabilidad} establece que
  cuanto más reutilizable es un objetivo de aprendizaje,
  menos efectivo pedagógicamente es ~\cite{Wile2002}.
  La razón es que una buena lección se parece más a una novela que a un programa:
  sus partes están estrechamente acopladas en lugar de ser cajas negras independientes.
  Por lo tanto, la reutilización directa puede ser el objetivo equivocado de las lecciones;
  podríamos llegar más lejos tratando de hacerlas más fáciles de mezclar.
 \end{description}

Si la paradoja de Reusabilidad es cierta,
la colaboración será más probable
si las cosas en las que se colabora son pequeñas.
Esto se ajusta a la teoría de Mike Caulfield \index{Caulfield, Mike}
\hreffoot{https://hapgood.us/2016/05/13/choral-explanations/}{\emph{choral explanations}}(explicaciones corales),
que sostiene que sitios como
\hreffoot{https://stackoverflow.com/}{\emph{Stack Overflow}} tienen éxito porque
proporcionan un coro de respuestas para cada pregunta,
cada una de las cuales es más adecuada para
una persona que pregunta y que es ligeramente diferente a las demás que preguntan.
Si esto es correcto
las lecciones de mañana puedes ser visitas guiadas de repositorios
de preguntas y respuestas seleccionadas por la comunidad
diseñadas para estudiantes en niveles muy diferentes.
 
 
\seclbl{Ejercicios}{s:process-exercises}
 
\exercise{Crear estudiantes tipo}{grupos pequeños}{30'}
 
Trabajando en grupos pequeños,
crea un tipo de 4 puntos que describa a una/uno de sus estudiantes estándar.
 
\exercise{Clasificar Objetivos de Aprendizaje}{pares}{10'}
 
Mira el ejemplo de objetivos de aprendizaje
para un curso introductorio sobre HTML y CSS
en \secref{s:process-objectives}
y clasifica cada uno de acuerdo a la Taxonomía de Bloom.
Compara tus respuestas con las de tu pareja.
¿Dónde estuvieron de acuerdo y en desacuerdo?
 
\exercise{Escribir Objetivos de Aprendizaje}{pares}{20'}
 
Escribe uno o más objetivos de aprendizaje
para algo que actualmente enseñas o planeas enseñar
utilizando la Taxonomía de Bloom.
Trabajando con una pareja,
critica y mejora los objetivos.
¿Cada uno tiene un verbo verificable y establece claramente
los criterios  para un desempeño aceptable?
 
\exercise{Escribir más Objetivos de Aprendizaje}{pares}{20'}
 
Escribe uno o más objetivos de aprendizaje
para algo que actualmente enseñas o planeas enseñar
utilizando la Taxonomía de Fink.
Trabajando con una pareja,
critica y mejora los objetivos.
 
\exercise{Ayúdame a hacerlo sola/solo}{grupos pequeños }{15'}
 
El teórico de la educación Lev Vygotsky introdujo la noción de una 
\gref{g:zpd}{Zona de Desarrollo Proximal} (ZPD por sus siglas en inglés),
la cual incluye los problemas que las personas no pueden resolver aún por sí mismas
pero que son capaces de resolver con la ayuda de una mentora/un mentor.
Estos son los problemas que resultan más fructíferos abordar a continuación,
ya que están fuera de tu alcance pero son alcanzables.
 
Trabajando en grupos pequeños,
escoge una/un estudiante tipo que hayas desarrollado
y describe dos o tres problemas que se encuentran en la ZPD de esa/ese estudiante.
 
\exercise{Construyendo lecciones, restando complejidad}{individual}{20'}
 
Una forma para construir una lección de programación
es escribir el programa que deseas que las/los estudiantes terminen,
luego elimina la parte más compleja que deseas que escriban
y conviértela en el último ejercicio.
Tú puedes luego remover la siguiente parte más compleja que deseas que escriban
y conviértela en el penúltimo ejercicio, etc.
Todo lo que quede después de haber retirado los ejercicios,
como cargar librerías o leer datos,
se convierte en el código al inicio les das.
 Elige un programa o página web que desees que
tus estudiantes puedan crear,
y trabaja hacia atrás para dividirlo en partes que se asimilen.
¿Cuántos hay?
¿Qué idea clave introduce cada uno?
 
\exercise{Rareza no esencial}{individual}{15'}
 
Betsy Leondar-Wright acuñó la frase
``\hreffoot{http://www.classmatters.org/2006\_07/its-not-them.php}{\emph{inessential weirdness }}''(rareza no esencial)
para describir cosas que hacen los grupos
que no son realmente necesarias,
pero que alienan a personas que aún no son miembros de ese grupo
Sumana Harihareswara luego usó esta noción
como base para una charla sobre
\hreffoot{https://www.harihareswara.net/sumana/2016/05/21/0}{\emph{inessential weirdnesses in open source software}}(rarezas no esenciales en el software de código abierto),
que incluye códigos como el uso de herramientas de línea de comandos con nombres crípticos.
Toma unos minutos para leer estos artículos,
luego has una lista de las rarezas no esenciales que crees
que tus estudiantes podrían encontrar
Cuando les enseñes por primera vez
¿Cuántas de estas puedes evitar?
 
\exercise{PETE}{individual}{15'}
\index{PETE (lesson pattern)}
\index{lesson pattern!PETE}

Un patrón que trabaja bien para lecciones de programación es PETE:
Introduce el  \textbf{P}roblema,
trabaja a través de un \textbf{E}jemplo,
explica la \textbf{T}eoría,
y luego \textbf{E}labora un Segundo ejemplo
para que las/los estudiantes puedan ver qué es específico en cada caso
y qué se aplica a todos los casos.
Elige algo que ya hayas enseñado o les hayan enseñado.
Y delinea una pequeña lección que siga estos cinco pasos.
.
\exercise{PRIMM}{individual}{15'}
\index{PRIMM (lesson pattern)}
\index{lesson pattern!PRIMM}

Otro patrón de lección es PRIMM~\cite{Sent2019}:
\textbf{P}redecir el comportamiento o salida de un programa,
Correr, del inglés \textbf{R}un, el programa para ver lo que realmente hace,
\textbf{I}nvestigar por qué lo hace pasando a través del mismo en un depurador o dibujando el flujo de control.
\textbf{M}odificar el programa (o sus entradas),
y luego hacer, del inglés \textbf{M}ake, algo similar desde cero.
Elige algo que hayas enseñado o te hayan enseñado recientemente
y bosqueja una breve lección que siga los siguientes cinco pasos.
 
\exercise{Concreto-Figurativo-Abstracto}{pares}{15'}
\index{CRA (lesson pattern)}
\index{lesson pattern!CRA}
 
 
\hreffoot{https://makingeducationfun.wordpress.com/2012/04/29/concrete-representational-abstract-cra/}{ \emph{Concrete-Representational-Abstract }} (Concreto-Representativo-Abstracto) (CRA), por sus siglas en inglés, es un enfoque para introducir nuevas ideas
que es usado principalmente con estudiantes más jóvenes:
manipular físicamente un objeto \textbf{C}oncreto,
\textbf{R}epresentar el objeto con una imagen,
luego realizar las mismas operaciones
usando números, símbolos o algo \textbf{A}bstracto.
 \begin{enumerate}
 
\item
  Escribe cada uno de los números 2, 7, 5, 10, 6 en una nota adhesiva.
 
\item
  Simula un bucle que encuentre el valor más grande buscando cada uno por turno (concreto).
 
\item
 Dibuja un diagrama del proceso que usaste etiquetando cada paso (representativo)
 
\item
  Escribe instrucciones que alguien más podría seguir por medio de los pasos (abstracto)
\end{enumerate}
 
 
Compara tus materiales representativo y abstracto con tus compañeras/compañeros.
 
\exercise{Evaluación de un repositorio de lecciones}{grupos pequeños}{10'}
 
\cite{Leak2017} explora por qué las/los docentes de ciencias computacionales
no utilizan sitios para compartir lecciones y recomienda formas para hacerlas más atractivas:
 
\begin{enumerate}
 
\item
  La página de destino debe permitir a los visitantes del sitio
identificar sus antecedentes y sus intereses al visitarlo.
Los sitios deben hacer dos preguntas:
``¿Cuál es su función actual? '' y
``¿En qué curso y grado estás interesada/interesado? ''
 
\item
  Los sitios deben mostrar todos los recursos de aprendizaje
en el contexto del curso completo para que los usuarios potenciales puedan comprender
su contexto de uso previsto.
 
\item
  A muchas/muchos docentes les preocupa que sus pares
juzguen su (falta de) conocimiento si publican en los foros de discusión
de los sitios.
Por tanto, estos foros deberían permitir la publicación anónima.
\end{enumerate}

En grupos pequeños,
Discute si estas tres características serían
suficientes para convencerte de usar un sitio para compartir lecciones,
y si no,
lo que haría.

 
\section*{Revisión}
 
\figpdfhere{figures/conceptmap-personas.pdf}{Concepto: Estudiantes}
