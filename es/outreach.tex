\chapter{Difusión}\label{s:outreach}

Está de moda en los círculos tecnológicos menospreciar a las universidades y las
instituciones gubernamentales como si fueran dinosaurios lentos, pero en mi experiencia no son peores que empresas de tamaño similar.
Tanto la junta del consejo escolar (NOTA:sé que existe algo como school board pero no se si cumple el mismo rol que el consejo escolar), 
la biblioteca o la oficina del concejal de la ciudad puede llegar a ofrecer espacio, fondos, publicidad, conexiones con otros grupos que 
todavía no hayas conocido y una gran cantidad de cosas útiles; conocerlas puede ayudar
a resolver o evitar problemas en el corto plazo y generar beneficios en el futuro.

\seclbl{Marketing}{s:outreach-marketing}

Las personas con conocimientos académicos y técnicos muchas veces piensan que
el  \gref{g:marketing}{marketing} trata sobre confundir y engañar. 
En la realidad, trata sobre ver cosas desde la perspectiva de otras personas,
comprendiendo sus deseos y necesidades, y explicando cómo puedes ayudarles---en pocas palabras, cómo enseñarles.
Este capítulo analizará cómo usar ideas de los capítulos anteriores
para lograr que las personas entiendan y apoyen lo que estás haciendo.

El primer paso es averiguar qué es lo que le ofreces a quién, es decir, 
lo que realmente atrae a los voluntarios, a los fondos 
y otro tipo de apoyo que puedas necesitar para continuar.
La respuesta generalmente es contraintuitiva.
Por ejemplo, la mayoría de los científicos creen que sus productos son sus artículos,
cuando en realidad son sus propuestas de subsidios, 
ya que son éstos los que atraen el dinero del subsidio~\cite{Kuch2011}.
Los artículos son la publicidad que convence a otras personas para otorgarle fondos a las propuestas, 
así como ahora los álbumes son los que convencen a la gente a comprar tickets para el show y remeras del músico que van a ver.

Suponiendo que tu grupo ofrece talleres de programación de fin de semana
a personas que están re-insertándose en la fuerza de trabajo después de haber estado alejadas por varios años.
Si los asistentes pueden pagar lo suficiente para cubrir tus costos,
entonces son tus clientes y el taller es tu producto. 
Si por otro lado, los talleres son gratuitos o los estudiantes solo pagan un monto simbólico para reducir la tasa de ausencias,
entonces tu producto real puede ser una mezcla de:

\begin{itemize}

\item
   tus proyectos de subsidio;

\item
 los antiguos alumnos de tus talleres
a los que las empresas  que te patrocinaron quisieran contratar;

\item
el resumen de media página de tus talleres en el balance anual de tu intendente/alcalde al concejo deliberante
que muestra cómo ella apoya el sector tecnológico local;

\item
     la satisfacción personal que obtienen los voluntarios cuando enseñan

\end{itemize}

Tal como con el diseño de  lecciones (\chapref{s:process}),
los primeros pasos en marketing son crear
el equivalente a estudiantes tipo\index{learner persona},
gente que podría estar interesada en lo que estás haciendo 
y averiguar cuáles de sus necesidades puedes satisfacer.
Una manera de resumir lo último es escribir \grefdex{g:elevator-pitch}{discursos de presentación}{discurso de presentación} 
dirigido a diferentes personas.
Una plantilla muy usada para esto es:


\newpage
\begin{longtable}{ll}
  Para        & \emph{ objetivo de audiencia} \\
  quien        & \emph{ insatisfacción con lo que se encuentra actualmente disponible} \\
  nosotros        & \emph{categoría} \\
  provee    & \emph{beneficio clave}. \\
  a diferencia de    & \emph{alternativas} \\
  nuestro programa    & \emph{característica distintiva clave}

\end{longtable}

\noindent
Continuando con el ejemplo del taller de fin de semana,
se podría usar este discurso para los participantes


\begin{quote}

quienes \emph{tienen todavía responsabilidades familiares },
nuestros \emph{talleres introductorios de programación}
provee \emph{clases en fin de semana con guardería incluida}.
A diferencia de emph{clases en línea},
nuestro programa \emph{ le da a la gente la oportunidad de conocer
a otras personas en el misma etapa de la vida}.


\end{quote}

\noindent
y ésta otra para los tomadores de decisiones en las empresas que podrían patrocinar los talleres:

\begin{quote}

 Para\emph{empresas que quieren reclutar desarrolladores de software de nivel básico}
  que \emph{ tienen dificultades para  encontrar candidatos con suficiente madurez  de diversas formaciones}
  nuestros \emph{ talleres introductorios de programación}
  proveen de\emph{potenciales reclutas}.
  A diferencia de \emph{ feria de reclutamiento universitario},
  nuestro programa \emph{ conecta empresas con una gran variedad de candidato/ass}.


\end{quote}

Si no sabes por qué diferentes potenciales accionistas pueden estar interesados en los que haces,
preguntales.
Si lo sabes,
preguntales igual:
las respuestas cambian con el tiempo,
y pueden descubrir cosas que no habías notado antes

Una vez que tienes estas discursos,
estos deberían llevarlos a lo que se publique en el sitio web y el material de difusión
para ayudar a la gente para ayudar a las personas a descubrir lo más rápido posible
si tu y ellos tienen algo de qué hablar
( \emph{No deberías} copiarlos textualmente,
aunque:
muchas personas en tecnología han visto esta plantilla tan seguido que 
sus ojos se podrán vidriosos si la vuelven a encontrar.)



Mientras escribes estos discursos
recuerda que hay varias razones para aprender cómo programar
(\secref{s:intro-exercises}).
Una sensación de logro,
control sobre sus propias vidas,
y ser parte de una comunidad puede motivar a las personas más que el dinero
(\chapref{s:motivation}).
Podrían ofrecerse como voluntarios para enseñar contigo  porque sus amigos lo están haciendo;
 de igual manera,
 una empresa puede decir que está patrocinando clases para estudiantes de secundaria que estén desfavorecidos económicamente
 porque quieren tener un grupo más grande de empleados potenciales en el futuro, 
 pero el CEO podría realmente estar haciéndolo simplemente porque es lo correcto.

\seclbl{Branding y Posicionamiento}{s:outreach-branding}

Una \gref{g:brand}{marca} es la primera reacción de una persona a la mención de un producto;
Si ante la primera reacción de la mención de un producto;
la reacción es “¿Que es eso?”,
uno (todavía) no tiene una marca.
“ Branding” es importante porque
la gente no va a ayudar en algo que no conocen o no le interesa/importa.

La mayor parte de la discusión actual sobre “branding” se enfoca 
en como crear conciencia en línea.
Listas de correo,
blogs,
y Twitter, todos generan maneras de llegar a la gente,
pero así como el volumen de desinformación aumenta,
la gente le presta menos atención a cada interrupción individual.

Esto hace que el \gref{g:positioning}{posicionamiento} sea más importante.
Algunas veces llamado “diferenciación”,
es lo que distingue a tu oferta de las otras, la sección “a diferencia de” de tu “discurso de presentación”.
Cuando te comunicas con personas que están familiarizadas en tu campo,
debes enfatizar o hacer hincapié en tu posicionamiento,
ya que es eso lo que va a llamar su atención.


Existen otras cosas que puedes hacer para construir tu “marca”
Una de ellas es usar accesorios como un robot que uno de los estudiantes hizo a partir de restos que encontró en su casa~\cite{Schw2013}
o el sitio web que otro estudiante hizo para el de geriatrico sus padres.

Otra opción es hacer un video corto-- de no más de un par de minutos de duración--
que resalte los antecedentes y logros de tus estudiantes.
El objetivo es tanto contar una historia:
mientras que la gente siempre pide datos,
creen y recuerdan historias.


\begin{aside}{Mitos Fundacionales}
 Una de las historias más convincentes que una persona o grupo puede contar es
por qué y cómo comenzaron.
¿Estás enseñando algo que quisieras que alguien te hubiera enseñado pero no lo hizo?
¿Había una persona en particular a la que quisieras ayudar,
y eso abrió las compuertas?

Si no hay una sección en tu sitio web que comience con “Había una vez,”
piensa en agregar una.

\end{aside}

Un paso crucial es lograr que tu organización pueda ser encontrada en las búsquedas en línea .\index{findability!of organizations}
\cite{DiSa2014b} descubrió que
los términos de búsqueda que los padres usan para las clases de computación fuera de la escuela de sus hijos
en realidad no encontraban esas clases,
y muchos otros grupos se enfrentan a desafíos similares.
Hay mucho “folklore “ sobre cómo hacer que las cosas puedan ser halladas en internet 
( también conocido como\gref{g:seo}{Motor de optimización de posicionamiento en buscadores} or SEO);
dados los  poderes cuasi-monopólicos y falta de transparencia de Google,
la mayor parte de esto se reduce a tratar de estar un paso adelante de los 
algoritmos diseñados para alejar a las personas clasifiquen en rankings basados en juegos.

A menos que estés muy bien financiado,
los mejor que puedes hacer es buscarte y buscar a tu organización frecuentemente
y ver que surge,
lee entonces \hreffoot{https://moz.com/learn/seo/on-page-factors}{estas guías}
y haz lo que puedas para mejorar tu sitio
A menos que tengas muchos fondos,
Ten en mente\hreffoot{https://xkcd.com/773/}{esta viñeta de XKCD} :
la gente no quiere saber sobre el organigrama o un paseo virtual por el sitio-- ellos necesitan tu dirección,
información sobre donde hay estacionamiento cerca, 
y alguna idea sobre que enseñas,
cuando lo enseñas,
y cómo va a cambiar sus vidas.


\begin{aside}{No todo el mundo vive en línea}
Estos ejemplos asumen que la gente tiene acceso a internet 
y que los grupos tienen dinero, materiales, tiempo libre, y/o  habilidades técnicas.
La mayoría no tiene-- de hecho,
aquellos que trabajan con grupos económicamente desfavorecidos muy probablemente no los tienen.
 (Como Rosario Robinson dice, ``Gratis funciona para aquellos para quien pueden darse lujo de que sea gratuito.'')\index{Robinson, Rosario}
Historias son más importantes que el programa del curso en estas situaciones
porque son más fáciles de volver a contar.
 De manera similar,
si las personas que deseas o esperas alcanzar no están tan en línea como tú,
debes entonces prestar atención a los avisos en las carteleras de las escuelas,
en bibliotecas locales,
centro de acogida,
y almacenes puede ser el camino más efectivo para alcanzarlos.

\end{aside}

\seclbl{El arte de las llamadas en frío}{s:outreach-cold-call}

Crear un sitio web y desear que las personas lo encuentren es fácil;
llamar por teléfono o golpear en las puertas de sus casa sin ningún tipo de introducción previa
es más difícil.
Al igual que pararse y enseñar,
sin embargo, es un oficio que puede aprenderse.
 Aquí hay diez reglas simples para convencer a las personas:


\begin{description}

\item[1: No o no lo hagas]

Si tienes que convencer a alguien de algo,
lo más probable es que realmente no quieran hacerlo.
Respeta que:
es casi siempre mejor en el largo plazo dejar alguna cosa en particular sin terminar
que usar culpa u otro truco psicológico inescrupuloso que solo generaría resentimiento.


\item[2: Se amable.]
No se si realmente existe un libro llamado
 \emph{Trucos secretos de los maestros de ventas Ninja},
pero si existe,
probablemente le dice a sus lectores que hacer algo por un potencial cliente
crea un sentido de obligación,
lo que a su vez aumenta las probabilidades de una venta.
Esto puede funcionar, pero solo funciona una vez y es una cosa medio extraña para hacer.
Por otra parte,
si eres genuinamente amable
y ayudas a otras personas porque eso es lo que las buenas personas hacen,
puedes tal vez inspirarlos/as a ser buenas personas también.


\item[3: Apela al bien mayor.]
Si les hablas abiertamente sobre que hay disponible ellos/as,
estás indicando que deberían pensar en su interacción contigo
como un  si fuera un intercambio comercial de valor para negociar que debe negociarse.
En su lugar,
empieza por explicar cómo lo que sea que queremos que nos ayude puede hacer del mundo un lugar mejor, y en dilo en serio.
Si lo que estás proponiendo no va a hacer del mundo un lugar mejor,
propone algo mejor.

\item[4: Comienza desde algo pequeño.]
Es comprensible que la mayoría de las personas se muestren reacias a sumergirse de lleno en las cosas, 
así que debes darle  la oportunidad de probar las aguas 
y conocerte a ti y a todos los demás involucrados
 en lo que sea en  que necesites ayuda.
No te sorprendas o decepciones si es así como las cosas terminaron:
todo el mundo está ocupado o cansado o tiene proyectos propios,
o tal vez tienen un modelo mental de cómo las colaboraciones deberían funcionar.
Recuerda la regla  90-9-1-- el 90\% va a mirar,
el 9\% va a hablar 
y el 1\% realmente va hacer cosas-- ajusta tus expectativas de modo acorde.

\item[5: No crea un proyecto: crea una comunidad.]
Solía pertenecer a un equipo de baseball que nunca realmente jugaba al baseball:
nuestros “ juegos o partidos” eran solo una excusa para pasar tiempo juntos y disfrutar la compañía del otro.
Probablemente no quieres llegar tan lejos,
pero compartir una taza de té con alguien o celebrar el cumpleaños de su primer/a nieto/a
pueden darte cosas que ninguna cantidad de dinero pueden dar

\item[6: Establece un punto de conexión.]
“Estaba hablando con x” o “Nos conocimos en Y” les da contexto,
lo que a su vez los hace sentir más cómodos.
Esto debe ser específico:
quienes envían correo basura y empresas que llaman por teléfono constantemente
nos han entrenado para ignorar cualquier cosa que comience con la frase
“ Hace poco tiempo encontré tu sitio web {\ldots}''
\item[7: Sé específico sobre lo que estás pidiendo.]

Las personas necesitan saber esto
para que puedan determinar si el tiempo y las habilidades que tienen
coinciden con lo que necesitas.
Ser realista desde el principio también es una señal de respeto:
 si le dices a la gente que necesitas una mano para  mover algunas cajas
 cuando en realidad estás mudando una casa entera,
 probablemente no te ayudarán por segunda vez.

\item[8: Establece tu credibilidad.]
Menciona a tus patrocinadores,
tu tamaño,
cuanto tiempo tu grupo ha existido, o algo que hayas logrado en el pasado
para que ello/as crea que vale la pena tomarte en serio


\item[9: Crea una ligera sensación de urgencia.]
“Esperamos lanzar esto en la primavera” es mucho más probable que genere una respuesta positiva que “Eventualmente queremos lanzar esto.”
Sin embargo la palabra “ligera” es importante:
si tu pedido es urgente, 
la mayoría de las personas asumen que eres una persona desorganizadas o que algo ha salido mal
y pueden errar por ser prudentes.


\item[10: Entiende la indirecta.]
Si la primera persona a la que le pides ayuda dice no,
pregúntale a otra.
Si la quinta o décima persona dice no,
debes preguntarte si los que estás tratando de hacer tiene sentido y vale la pena hacerse

\end{description}

Esta plantilla de correo electrónico sigue todas estas reglas.
Ha funcionado bastante bien:
hallamos que cerca de la mitad de los correos son respondidos,
y aproximadamente la mitad de estos querían hablar más,
y la mitad des estos últimos condujeron a talleres,
lo que significa que 10-15\% de los correos electrónicos objetivos resultaron en talleres.
Esto puede ser bastante desmoralizante si no te encuentras acostumbrado a esto, 
pero es mucho mejor que la tasa de respuesta de entre  2--3\% que la mayoría de las organizaciones esperan con llamadas imprevistas.

\begin{quote}

  \noindent
    Hola NOMBRE
 
 Espero que no te moleste que escriba repentinamente,
 pero quería continuar con nuestra conversación en LUGAR DE REUNIÓN
 para ver si estarían interesados en que nosotros/as hiciéramos un taller para entrenamiento de docentes-- estamos programando  la próxima tanda durante las próximas dos semanas.

 Este taller de un día le enseñará a tus voluntarios
 una serie de prácticas útiles de enseñanza basadas en evidencia.
 Se ha impartido más de cien veces de diversas maneras en seis continentes
 para organizaciones sin fines de lucro, bibliotecas y empresas,
 y  todo el material disponible gratuitamente en línea en http://teachtogether.tech.
 
El temario incluye:

  \begin{itemize}
  \item estudiantes tipo
  \item  diferencias entre diferentes tipos de estudiantes 
  \item uso de evaluaciones formativas para diagnosticar malentendidos
  \item teaching as a performance art enseñanza como un arte performativa
  \item que motiva y desmotiva a estudiantes adultos
  \item la importancia de la inclusividad y como ser un buen aliado
  \end{itemize}
  
 Si esto te resulta interesante,
por favor avísame-- Sería muy bienvenida la oportunidad de hablar de modos y medios para hacerlo.
Gracias,
  NOMBRE

\end{quote}

\begin{aside}{Referencias}
Construir alianzas con otros grupos que hacen cosas relacionadas a tu trabajo
vale la pena de muchas maneras.
Una de ellas son las referencias:
si alguien se te aproxima en busca de ayuda sería mejor atendido por alguna otra organización,
tomate un momento para hacer una introducción
Si ya has hecho esto varias veces
agrega algo a tu sitio web que pueda ayudar a la próxima persona a encontrar lo que necesita.
Las organizaciones a las que estás ayudando pronto empezaran a ayudarte a cambio.
\end{aside}

Todo el mundo tiene miedo a lo desconocido y a pasar vergüenza frente a otros/as
En consecuencia,
la mayoría de la gente prefiere fracasar que cambiar.
Por ejemplo,
Lauren Herckis investigó \index{Herckis, Lauren}
\hreffoot{https://www.insidehighered.com/news/2017/07/06/anthropologist-studies-why-professors-dont-adopt-innovative-teaching-methods}{ por que el profesorado universitario no adopta mejores metodos de enseñanza}.
Ella halló que la razón principal es el miedo a parecer estúpido/a frente a los estudiantes;
las razones secundarias fueron
preocupación porque los inevitables contrastes que haya en el cambio de los métodos de enseñanza puedan afectar las evaluaciones del curso
(que en consecuencia afectan las promoción o los cargos estables/titulares)
y el deseo de la gente de seguir imitando a los profesores/maestros que los han inspirado.

No tiene sentido discutir si estos problemas son “reales” o no:
el profesorado cree que son reales,
así que cualquier plan para trabajar con el profesorado necesita referirse a ellos\footnote{
Y la prevalencia de mentalidades fijas en el profesorados a lo que se refiere a la enseñanza, es decir la creencia de que algunas personas son “solo mejores profesores” }.

Ellos/as preguntaron y respondieron tres preguntas claves:

\begin{description}

\item[¿Como el profesorado se entera sobre nuevas prácticas de enseñanza?]

Buscan intencionalmente nuevas prácticas
porque están motivados a resolver un problema ( en particular, la participación de los estudiantes),
se tornan conscientes a través de iniciativas deliberadas por parte de sus instituciones,
las copian o replican de sus colegas,
o las obtienen por interacciones esperadas  \emph{e inesperadas} en conferencias
(relacionadas a la enseñanza o de otro tipo).

\item[¿Por qué las prueban?]

Algunas veces por incentivos institucionales
( por ejemplo innovan para mejorar sus chances de promoción),
pero hay a veces tensión en instituciones de investigación
donde la retórica sobre la importancia de la enseñanza tiene poca credibilidad
Otra razón importante es su propio análisis costo/beneficio:
¿La innovación es la que les va a ahorrar tiempo?
Una tercera razón es que se inspiran en modelos a seguir-- otra vez,
esto afecta en gran medida las innovaciones que tienen como objetivo mejorar la motivación y participación más que los resultados en el aprendizaje
-- y un cuarto factor son fuentes confiables o de confianza,
por ejemplo personas que han  conocido en congresos o conferencias que se encuentran en la misma situación que ellos/as 
y reportaron aprobación exitosa.
Pero el profesorado tiene preocupaciones que no siempre son abordadas por el grupo de personas que abogan por modificaciones.
La primera era la ley de Glass:
cualquier nueva herramienta o práctica inicialmente te ralentiza o  te vuelve más lento,
entonces mientras que las nuevas prácticas pueden hacer la enseñanza más efectivo en el largo plazo, son costosas en el corto plazo.
Otro es que la distribución física de las aulas torna difíciles a muchas nuevas prácticas:
por ejemplo,
los grupos de discusión no funcionan bien en modo de asientos estilo teatro.

Pero el resultado más revelador fue éste:
`` A pesar de que ellos mismo son investigadores,
el profesorado en ciencias de la computación con el que hablamos en su mayoría no creía
que resultados sobre estudios educacionales fueran razones creíbles suficiente para probar prácticas de enseñanza.”
Esto es consistente con otros hallazgos:
incluso personas cuyas carreras están dedicadas  a la investigación a menudo ignoran investigaciones en educación. 


\item[¿Por qué las siguen usando?]
  Como~\cite{Bark2015} dice, ``Las devoluciones de los estudiantes son vitales,''
y son normalmente las razones más fuertes para continuar usando una práctica,
(aunque la asistencia a clases es un buen indicador de participación).
aunque sabemos que las auto-evaluaciones no correlacionan fuertemente con los resultados del aprendizaje~\cite{Star2014,Uttl2017}
Otro motivo para retener alguna  práctica es requerimiento institucional,
aunque si esta es la única motivación,
las personas abandonaran la práctica 
cuando el incentivo explícito o el  monitoreo desaparecen.


\end{description}

La buena noticia es que puedes abordar estos problemas sistemáticamente.
\cite{Baue2015} observó  la adopción de nuevas tecnicas medicas dentro de la Administración de Veteranos de Estados Unidos.
 Hallaron que prácticas basadas en evidencia en medicina
They found that evidence-based practices in medicine
toman en promedio 17 años en ser incorporadas en prácticas generales de rutina,
y que solo la mitad de estas prácticas llegan a ser ampliamente adoptadas.
Este deprimente hallazgo y otros han estimulado el crecimiento de
\gref{g:implementation-science}{implementation science},
que es el estudio de cómo lograr que la gente adopte mejores prácticas.

Como el \chapref{s:community} decía,
el punto de partida es hallar qué es lo que creen que necesitan las personas que quieres ayudar.
Por ejemplo,
\cite{Yada2016} resumen los comentarios de los maestros de  escolarizaciones primaria y secundaria en la preparación y apoyo que quieren.
Aunque puede no ser aplicable a todos los entornos,
tomar una taza de té con unas pocas personas y escucharlas antes de hablar
hace un mundo de diferencia en su voluntad de intentar algo nuevo. 

Una vez que sabes que es lo que la gente necesita,
el siguiente paso es hacer cambios de manera incremental,
dentro de los propios esquemas o entornos de las instituciones.
\cite{Nara2018} describe un programa intensivo de tres años de bachillerato/licenciatura
basado en cohortes muy unidas y apoyo administrativo
que triplicó las tasas de graduación,
mientras que~\cite{Hu2017} describe el impacto de implementar un programa de certificaciones de seis meses
para profesores de secundaria que quieran enseñar computación.
El  número de maestros de computación se ha mantenido estable entre 2007 to 2013,
pero se cuadruplicó después de la introducción de un nuevo programa de certificación 
sin disminuir la calidad:
los maestros que eran novatos en impartir computación parecían ser tan efectivos en el curso introductorio como maestros con más entrenamiento en computación. 


De modo más amplio,
\cite{Borr2014} categoriza maneras para lograr que ocurran cambios en educación superior.
Las categorías están definidas por si el cambio es individual o sistémico y si está prescripto (de arriba hacia abajo) o emergente(de abajo hacia arriba)
La persona que trata de hacer los cambios ( y hacer que duren)
tiene un rol distinto en cada situación,
y de manera acorde debe seguir diferentes estrategias.
El artículo continúa explicando en detalle cada uno de los métodos,
mientras que~\cite{Hend2015a,Hend2015b} presenta las mismas ideas en una forma más 
procesable.


Desde el exterior o viniendo desde afuera,
probablemente en principio caigas en alguna de las categorías  Individuo/Emergente,
dado que te aproximarás a los maestros uno a uno
y tratar de lograr que los cambios ocurran de abajo hacia arriba.
Si este es el caso,
las estrategias Borrego y Henderson recomiendan centrar alrededor
de tener maestros que reflexionan en su enseñanza de manera individual o en grupos.
Hacer live coding para mostrarles lo que haces o los ejemplos que usas,
y deja que tengan su turno para hacer programación en vivo
para mostrar cómo usarían esas ideas y técnicas en su escenario,
les da a todos/as la oportunidad de  captar cosas que les seran útiles en su contexto.


\seclbl{Docentes de rango libre}{s:outreach-free-range}

Las escuelas y las universidades no son los únicos lugares en donde la gente las personas pueden ir a aprender programación;
en los últimos años, un número creciente a turned a talleres de rango libre y programas intensivos.
Estos últimos típicamente duran uno a seis meses,
run by grupos de voluntarios o por empresas con fines de lucro,
y su objetivo son personas que se están re-entrenando para entrar en tecnología.
Algunos son de muy alta calidad,
pero otros existen primariamente para separar personas de su dinero~\cite{McMi2017}.


\cite{Thay2017} entrevistó a 26 graduados de estos entrenamientos intensivos
que proveen de una segunda oportunidad para aquellos que no tuvieron antes oportunidades de educación en computación
( aunque expresarlo de este modo realizar ciertas grandes suposiciones 
cuando se refiere a personas de grupos poco representados).
Los participantes de los entrenamientos intensivos enfrentan grandes costos y riesgos personales:
deben pasar una cantidad significativa de tiempo, dinero y esfuerzo antes durante y después de los entrenamientos intensivos, y cambiar de carreras puede tomar un año o más.
Varios de los entrevistados sienten que sus certificados fueron mal vistos por sus empleadores;
como dicen algunos,
obtener un trabajo significa aprobar una entrevista,
pero dado que los entrevistadores muchas veces no comparten sus motivos para rechazar,
es difícil saber que arreglar o que más aprender.
Muchos/as han tenido que recurrir a pasantías (pagas o de otro tipo)
y pasan mucho tiempo construyendo sus portfolios y haciendo networking.
Las tres barreras informales que más fácilmente identificables son jerga,
síndrome del impostor, y una sensación de no encajar.


\cite{Burk2018} profundizó en esto
comparando las habilidades y credenciales que los reclutadores de la industria tecnológica buscan 
entre  entrenamientos intensivos y grados/diplomas/carreras de cuatro años.
Basándose en entrevistas con 15 gerentes de contratación de empresas de varios tamaños y algunos grupos focales,
encontraron que los reclutadores enfatizaban uniformemente en habilidades “blandas”
(especialmente trabajo en equipo, comunicación y la habilidad para continuar aprendiendo)
Muchas compañías requieren un título de cuatro años
(aunque no necesariamente en informática),
pero muchos también elogiaron a los graduados de  entrenamientos intensivos por ser mayores en edad o más maduros
y tener un conocimiento más actualizado.


Si te está aproximas a un  entrenamiento intensivo existente,
tu mejor estrategia podría ser enfatizar lo que sabes sobre enseñanza
en lugar de lo que sabes sobre tecnología,
dado que muchos de sus fundadores y personal tienen experiencia en programación
pero poca o ninguna capacitación en educación.
Los primeros capítulos de este libro en el pasado han servido bien con esta audiencia,
y ~\cite{Lang2016} describe
prácticas de enseñanza basadas en evidencia que pueden implementarse
con mínimo esfuerzo y a bajo costo.
éstas tal vez no tengan el mayor impacto, 
pero lograr algunas victorias tempranas ayuda a generar apoyo para esfuerzos más grandes.



\seclbl{Reflexiones Finales }{s:outreach-final}
Es imposible cambiar grandes instituciones por tu propia cuenta:
necesitas aliados
y para conseguir aliados,
necesitas tácticas.
La guía más útil que he encontrado es~\cite{Mann2015}, 
que cataloga más de cuatro docenas de estas tácticas
y las organiza de acuerdo a si se implementan mejor temprano,
luego,
a lo largo del ciclo de cambio,
o cuando encuentras resistencia.
Algunos de sus patrones incluyen:

\begin{description}

\item[En tu espacio:]
Mantén la nueva idea visible
ubicando recordatorios a lo largo de la organización.

\item[Símbolo o recuerdo:] 
Para mantener viva una nueva idea en la memoria de una persona,
entregue un recuerdo (COMENTARIO esto es token, no es un souvenir)   que puedan identificarse con el tema que se está introduciendo.

\item[Campeón escéptico:]
Pregunte a los líderes con opiniones  fuertes que sean escépticos de la nueva idea.
para desempeñar el papel/rol de ``escéptico oficial ''.
Usa sus comentarios para mejorar tu esfuerzo,
incluso si no logras cambiar su opinión.


\item[Compromiso Futuro:]
 
 Si puedes anticipar algunas de sus necesidades,
puedes pedir un compromiso futuro a las personas más ocupadas.
Si se les da un tiempo de entrega,
pueden estar más dispuestos a ayudar.

  
\end{description}

La estrategia más importante es
estar dispuesto a cambiar tus metas
según lo que aprendas de las personas a las que intentas ayudar.
Tutoriales que les muestran cómo usar una hoja de cálculo
podría ayudarlos de manera más rápida y confiable que
una introducción a JavaScript.
A menudo he cometido el error de confundir cosas que me apasionaban
con cosas que las otras personas deberían saber;
si realmente quieres ser quien acompañe,
recuerda siempre que el aprendizaje y el cambio tienen que ir en ambos sentidos.

La parte más difícil de construir relaciones es comenzarlas.
Reserva una o dos horas cada mes
para encontrar aliados y mantener tus relaciones con ellos.
Una forma de hacer esto es pedirles consejo:
¿Cómo creen que deberías crear conciencia de lo que están haciendo?
¿Dónde han encontrado espacio para dar clases?
¿Qué necesidades creen que no se están cumpliendo
y serías capaz de cumplir?
Cualquier grupo que haya existido durante algunos años tendrá consejos útiles;
también se sentirán halagados de que se les haya consultado,
y sabrán quién eres la próxima vez que llames.


Y como~\cite{Kuch2011} decía,
si no puedes ser el primero/a en una categoría,
intenta crear una nueva categoría en la que sí  puedas ser el primero/a.
Si no puedes hacer eso,
únete a un grupo existente o piensa en hacer algo completamente diferente.
Esto no es derrotista:
si alguien más ya está haciendo lo que tienes en mente,
deberías incorporarte o abordar una de las otras cosas igualmente útiles
que podrías estar haciendo en su lugar.

\seclbl{Ejercicios}{s:outreach-exercises}

\exercise{Discurso de presentación para un/a concejal}{individual}{10}

Este capítulo describe una organización
que ofrece talleres de programación de fin de semana para personas que re-ingresan a la fuerza laboral.
Escribir un discurso de presentación para esa organización
dirigido a un concejal de la ciudad cuyo apoyo las organización necesita.


\exercise{Presenta tu Organización}{individual}{30}

Identifica dos grupos de personas de la que tu organización necesite apoyo
y escribe un discurso de presentación dirigido a cada uno/a.


\exercise{Adjuntos de correo electrónico}{pares/parejas}{10}

Escriban las líneas de asunto (y solo las líneas de asunto) para tres mensajes de correo electrónico:
uno anunciando un nuevo curso,
uno anunciando un nuevo patrocinador,
y uno que anuncia un cambio en el liderazgo del proyecto.
Compare sus líneas de asunto con las de un compañero/a
y vea si pueden combinar las mejores características de cada una mientras que también las acortan.

\exercise{Manejando la Resistencia Pasiva}{grupos pequeños}{30}

Las personas que no quieren cambios a veces lo dicen en voz alta:
pero a menudo también pueden usar varias formas de resistencia pasiva,
como simplemente no lidiar con ello una y otra vez,
o planteando un posible problema tras otro 
para hacer que el cambio parezca más arriesgado y más costoso de lo que probablemente es
\ cite {Scot1987}.
Trabajando en grupos pequeños,
enumere tres o cuatro razones por las cuales las personas podrían no querer que su iniciativa de enseñanza siga adelante,
y explique qué puede hacer con el tiempo y los recursos que tiene para contrarrestar cada una de esas razones.


\exercise{Por que/para que aprender a programar?}{individual}{15}

Revise el ejercicio ``¿Por qué aprender a programar?'' En \ secref {s: intro-exercise}.
¿Dónde se alinean sus razones para enseñar y las razones de sus alumnos para aprender?
¿y donde no?
¿Cómo afecta eso a su comercialización?

\exercise{Conversational Programmers}{pensar en parejas y compartir}{15}

Un/a \gref{g:conversational-programmer}{programador/a conversacional}
es alguien que necesita saber lo suficiente sobre informática
para tener una conversación valiosa con un programador,
pero ellos mismos no van a programar.

\cite{Wang2018} descubrió que la mayoría de los recursos de aprendizaje no abordan las necesidades de este grupo.
Trabajando en parejas/pares,
escriban un discurso para un taller de medio día destinado a ayudar a las personas que se ajustan a esta descripción
y luego comparte el discurso de tu pareja con el resto de la clase.


\exercise{Colaboraciones}{grupos pequeños}{30}
Responda por su cuenta las siguientes preguntas,
luego compare sus respuestas con las dadas por otros miembros de su grupo.


\begin{enumerate}

\item
¿Tiene algún acuerdo o relación con otros grupos?

\item
¿Quieres tener relaciones con algún otro grupo?
\item
¿Cómo tener (o no tener) colaboraciones
podría ayudar a alcanzar sus objetivos?
\item
¿Cuáles son sus relaciones colaborativas clave?
\item
¿Son estos los colaboradores adecuados o indicados para alcanzar sus objetivos?
\item
¿Qué grupos o entidades quisieras que tu organización
tenga acuerdos o lazos ?


\end{enumerate}

\exercise{Educacionalización}{toda la clase}{10}

\cite{Laba2008} explora porque en los Estados Unidos y en otros países
siguen empujando la solución de problemas sociales hacia las instituciones educativas 
y eso sigue sin funcionar.
él remarca,
``[Educación] ha hecho muy poco para promover igualdad de raza, clase, y género;
para mejorar la salud pública, la productividad económica y buena ciudadanía;
o reducir el sexo en adolescente, las muestras por accidentes de tránsito, obesidad y la destrucción ambiental.
De hecho,
de muchas maneras ha tenido un efecto negativo en estos problemas
sacando dinero y energía de las reformas sociales que podrían tener un impacto más substancial.”
Él continúa escribiendo: 


\begin{quote}

  Entonces, ¿cómo debemos entender el éxito de esta institución?
a la luz de su no hacer lo que le pedimos?
Una forma de pensar en esto es que
la educación puede no estar haciendo lo que pedimos,
pero está haciendo lo que queremos.
Queremos una institución que persiga nuestros objetivos sociales.
de una manera que está en línea con el individualismo en el corazón del ideal liberal,
con el objetivo de resolver problemas sociales
buscando cambiar los corazones, las mentes y las capacidades de cada estudiante.
Otra forma de decir esto es que
queremos una institución a través de la cual podamos expresar nuestros objetivos sociales
sin violar el principio de elección individual
que se encuentra en el centro de la estructura social,
incluso si esto tiene el costo de no lograr estos objetivos.
Entonces la educación puede servir como un punto de orgullo cívico,
un lugar de muestra para nuestros ideales,
y un medio para participar en disputas edificantes pero  que en última instancia son intrascendentes
sobre visiones alternativas de la buena vida.
Al mismo tiempo,
también puede servir como un conveniente chivo expiatorio
al que podemos culpar por su fracaso en lograr nuestras más altas aspiraciones para nosotros mismos como sociedad.


\end{quote}

¿Cómo encajar en este marco los esfuerzos que se hacen para enseñar pensamiento computacional y ciudadanía digital en las escuelas?
¿Los entrenamientos intensivos evitan estas trampas o simplemente las entregan con una nueva apariencia?

\exercise{Adopción Institucional}{ clase completa}{15}

Re-leer la lista de motivaciones para adoptar nuevas prácticas
dadas en \secref{s:outreach-schools}.
¿Cuáles de estos se aplican a tí y tus colegas?
¿Cuáles son irrelevantes en tu contexto?
¿Cual enfatizamos
cuando y si interactúas con personas que trabajan en instituciones educativas formales?


\exercise{Si al principio no tienes éxito}{grupos pequeños}{15}

W.C.~Fields probablemente nunca dijo,
``Si al principio no tienes éxito, inténtalo, inténtalo de nuevo.
Entonces déjalo, no sirve de nada ser un tonto al respecto''
Sigue siendo un buen consejo:
si las personas con las que intenta comunicarse no responden,
podría ser que nunca los convencerás.
En grupos de 3 a 4,
hagan una breve lista de señales de que se debe dejar de intentar hacer algo en lo que cree.
¿Cuántos de ellos ya son verdaderos?


\exercise{Logrando/Haciendo que falle}{individual}{15}
\cite{Farm2006} presenta algunas reglas ironicas para lograr que nuevas herramientas \emph{no} sean adoptadas,
todas de las cuales aplican a nuevas prácticas de enseñanza:

\begin{enumerate}

\item
  Hacerlo opcional.

\item
Economizar en entrenamiento. 

\item
  No usarlas en un proyecto real.

\item
  Nunca integrarlas.

\item
Usarlas esporádicamente.

\item
  Hacerlas parte de una iniciativa de calidad

\item
  Marginalizar al campeón.

\item
Capitalizar en los primeros errores.

\item
Hacer una inversión pequeña

\item
 Explotar miedo, incertidumbre, duda, pereza e inercia.


\end{enumerate}

¿Cuál de estas has visto  hechas recientemente?
¿Cuales has hecho tú mismo?
¿Qué forma tuvieron?


\exercise{Mentoring}{whole class}{15}

\exercise{Mentoreo}{todo la clase/todo el grupo}{15}

The \hreffoot{http://www.iaamcs.org/}{Institute for African-American Mentoring in Computer Science}
ha publicado \hreffoot{http://iaamcs.org/guidelines}{ guías para mentorear estudiantes de doctorado}.
Lea individualmente
luego discutan como una clase
y califica los esfuerzos para tu propio grupo como +1 (definitivamente haciendo),
0 (no estoy seguro o no es aplicable),
o -1 (definitivamente no se está haciendo).


