\chapter{Difusión y vinculación}\label{s:outreach}


Está de moda en los círculos tecnológicos 
menospreciar a las universidades y las instituciones gubernamentales como si fueran dinosaurios lentos, 
pero en mi experiencia no son peores que empresas de tamaño similar.
El consejo de la escuela local, la biblioteca o la alcaldía o municipio pueden ofrecer espacio, 
financiación, publicidad, conexiones con otros grupos que puede que aún no hayas conocido 
y una serie de otras cosas útiles. 
Conocer estas posibilidades puede ayudar a resolver o evitar problemas a corto plazo 
y generar beneficios en el futuro.

\seclbl{Marketing}{s:outreach-marketing}

Las personas con antecedentes académicos y técnicos a menudo piensan que 
el \gref{g:marketing}{\emph{marketing}} se trata sobre confundir y engañar. 
En realidad, trata sobre ver cosas desde la perspectiva de otras personas, 
comprendiendo sus deseos y necesidades, y explicando cómo puedes ayudarlas---en pocas palabras, 
cómo enseñarles.
Este capítulo analizará cómo usar ideas de los capítulos anteriores 
para hacer que las personas entiendan y apoyen lo que estás haciendo.

El primer paso es averiguar qué le ofreces a cada persona, es decir, 
lo que realmente atrae a los/las voluntarios/as, a los fondos y a otro tipo de apoyo que puedas necesitar para continuar.
La respuesta es contra-intuitiva.
Por ejemplo, la mayoría de los/las científicos/as creen 
que los artículos científicos (conocidos como \emph{papers}) 
que publican en revistas académicas son sus productos, 
pero en realidad los productos son los proyectos que presentan a convocatorias de subsidios: 
son estos proyectos los que atraen el dinero~\cite{Kuch2011}.
Los artículos científicos son la publicidad que convence a otras personas a financiar esas propuestas, 
así como los álbumes son los que convencen a la gente de comprar entradas a conciertos y camisetas de bandas musicales.

Imagina que tu grupo ofrece talleres de programación de fin de semana 
a personas que están reinsertándose a la actividad laboral después de haber estado inactivas por varios años.
Si los participantes del taller pueden pagar lo suficiente para cubrir tus costos, 
entonces son tus clientes y el taller es tu producto. Si por otro lado, los talleres son gratuitos 
o los/las estudiantes sólo están pagando un monto simbólico para reducir la tasa de ausencias, 
entonces tu producto real puede ser una mezcla de: 

\begin{itemize}

\item
	tus propuestas de proyectos para solicitar subsidios;

\item
 
	los/las personas que terminan exitosamente tus talleres
	a los que las empresas que te patrocinaron les gustaría contratar;

\item
   el resumen de media página de tus talleres que quien gobierna incluye 
   en el balance o resumen anual presentado al concejo deliberante,  
   que muestra cómo apoya al sector tecnológico local;
o

\item
   la satisfacción personal que obtienen los/las voluntarios/as cuando enseñan.
\end{itemize}

Al igual que el diseño de lecciones (\chapref{s:process}),
los primeros pasos en marketing son crear
las personas tipo \index{personas tipo}
de la gente que podría estar interesada en lo que estás haciendo, 
y averiguar cuáles de sus necesidades puedes cumplir.
Una manera de resumir lo último es escribir \grefdex{g:elevator-pitch}{discursos de presentación}{discurso de presentación} 
dirigidos a diferentes personas tipo.
Una plantilla ampliamente utilizada para este objetivo es:

\begin{longtable}{ll}
  Para        & \emph{ objetivo de audiencia} \\
  quién        & \emph{ insatisfacción con lo que está disponible actualmente} \\
  nuestros/as        & \emph{categoría} \\
  proveen    & \emph{beneficio clave}. \\
  A diferencia de    & \emph{alternativas} \\
  nuestro programa    & \emph{característica distintiva clave.}
\end{longtable}

\noindent
En el ejemplo del taller de fin de semana,
podríamos usar este tono para los participantes

:
\begin{quote}

Para \emph{personas que regresan a la actividad laboral después de estar inactivas por varios años}

quiénes \emph{tienen todavía responsabilidades familiares },
nuestros \emph{talleres introductorios de programación}
proveen \emph{clases los fines de semana con guardería incluida}.
A diferencia de las \emph{clases en línea},
nuestro programa \emph{le da a la gente la oportunidad de conocer a otras personas en la misma etapa de la vida}.

\end{quote}

\noindent
y este otro discurso de presentación para quienes toman decisiones en las empresas que podrían patrocinar los talleres:

\begin{quote}

  Para \emph{empresas que quieren reclutar desarrolladores/as de software de nivel básico}
  quiénes \emph{tienen dificultades para encontrar candidatos/as con suficiente madurez en diversas formaciones}
  nuestros \emph{talleres introductorios de programación}
  proveen \emph{potenciales candidatos/as}.
  A diferencia de \emph{una feria de reclutamiento universitario},
  nuestro programa \emph{conecta empresas con una gran variedad de candidatos/as}.
\end{quote}

Si no sabes por qué diferentes patrocinadores/as potenciales podrían estar interesados/as en lo que haces,
pregúntales.
Si lo sabes,
pregúntales igual:
las respuestas pueden cambian con el tiempo,
y puedes descubrir cosas que no habías notado antes.

Estos discursos de presentación, una vez elaborados,
deberían guiar lo que publicas en tu sitio web y en el material de difusión,
para ayudar a la gente a descubrir lo más rápido posible
si tú y ellos tienen algo de qué hablar.
(\emph{No deberías} copiar los discursos textualmente,
porque
muchas personas en el área tecnológica han visto esta plantilla tantas veces que 
perderán el interés si vuelven a leerla.) 

Mientras escribes estos discursos,
recuerda que hay varias razones para aprender cómo programar (\secref{s:intro-exercises}).
Una sensación de logro,
control sobre sus propias vidas,
y ser parte de una comunidad puede motivar a las personas más que el dinero
(\chapref{s:motivation}).

Algunas personas podrían ofrecerse a enseñar contigo de forma voluntaria  
porque sus amistades lo están haciendo. Igualmente, una empresa puede decir que 
está patrocinando clases para estudiantes de secundaria en condiciones económicas desfavorables 
porque quieren tener un grupo más grande de empleados potenciales en el futuro, 
aunque quizás quien dirige la empresa lo podría estar haciendo, simplemente porque es lo correcto.

\seclbl{Marcas y posicionamiento}{s:outreach-branding}

Una \gref{g:brand}{marca} es la primera reacción de una persona a la mención de un producto;
Si la reacción es ``¿Qué es eso?'',
todavía no tienes una marca.
La marca es importante porque
la gente no va a ayudar a algo que no conoce o no le importa.

La mayor parte de la discusión actual sobre las marcas se enfoca 
en cómo crear reconocimiento en línea.
Las listas de correo,
los blogs
y Twitter te dan maneras de llegar a la gente,
pero a medida que aumenta el volumen de desinformación,
la gente presta menos atención a cada interrupción individual.

Esto hace que el \gref{g:positioning}{posicionamiento} sea cada vez más importante.
A veces llamado ``diferenciación'',
es lo que distingue tu oferta de las otras, la sección ``a diferencia de'' de tu discurso de presentación.
Cuando te comunicas con personas que están familiarizadas con tu campo,
debes enfatizar el posicionamiento,
ya que es eso lo que llamará su atención.

Hay otras cosas que puedes hacer para ayudar a construir tu marca.
Una de ellas es usar ejemplos de éxito como un robot que una de tus estudiantes 
hizo a partir de piezas que encontró en su casa~\cite{Schw2013}
o el sitio web que otro estudiante hizo para el geriátrico sus padres.

Otra opción es hacer un video corto---no más de unos pocos minutos de duración---
que resalte los antecedentes y logros de tus estudiantes.
El objetivo de ambos ítems anteriores es contar una historia:
mientras que la gente siempre pide datos,
lo que creen y recuerdan son las historias.

\begin{aside}{Mitos fundacionales}
Una de las historias más convincentes que una persona o grupo puede contar es
por qué y cómo comenzaron.
¿Estás enseñando algo que desearías que alguien te hubiera enseñado pero no lo hizo?
¿Había una persona en particular a la que querías ayudar
y eso abrió las puertas?

Si no hay una sección en tu sitio web que comience con ``Había una vez'',
piensa en agregar una.
\end{aside}

Un paso crucial es lograr que tu organización sea encontrada en las búsquedas en internet. \index{findability!of organizations}
\cite{DiSa2014b} descubrió que
los términos de búsqueda que los padres y las madres usaban para las clases de computación fuera de la escuela no encontraron esas clases, 
y muchos otros grupos se enfrentan a desafíos similares.
Hay mucho mito sobre técnicas para hacer que las cosas puedan ser encontradas en internet (lo que a veces se refiere como \gref{g:seo}{motor de optimización de posicionamiento en buscadores} o \emph{SEO} por su sigla en inglés (\emph{``Search Engine Optimization''})). 
Dado el poder casi-monopólico y la falta de transparencia de Google,
la mayor parte de estas estrategias se reducen a estar un paso por delante de los
algoritmos diseñados para prevenir a las personas sobre rankings manipulados, sesgados o poco realistas.


A menos que tengas muy buena financiación,
lo mejor que puedes hacer es hacer búsquedas regulares de tu organización y de ti en internet 
para ver qué encuentras.
Con esa inforamción puedes leer \hreffoot{https://moz.com/learn/seo/on-page-factors}{estas guías (en inglés)}
y hacer lo que esté a tu alcance para mejorar tu sitio.
Tener en mente \hreffoot{https://xkcd.com/773/}{esta viñeta de XKCD (en inglés)}:
la gente no quiere saber sobre tu organigrama u obtener un recorrido virtual de su sitio-- ellos quieren tu dirección,
información sobre estacionamiento cerca 
y alguna idea de lo que enseñas,
cuándo lo enseñas
y cómo va a cambiar sus vidas.


\begin{aside}{No todo el mundo vive en línea}
Estos ejemplos asumen que la gente tiene acceso a internet y que los grupos 
tienen dinero, materiales, tiempo libre y/o habilidades técnicas.
La mayoría no tiene estos recursos---de hecho,
aquellos que trabajan con grupos económicamente desfavorecidos muy probablemente no los tengan.
(Como Rosario Robinson dice, ``Gratis funciona para aquellas personas que pueden permitirse lo gratuito.'')\index{Robinson, Rosario}
Las historias son más importantes que el programa del curso en esas situaciones,
porque son más fáciles de volver a contar.
Igualmente,
si las personas a las que esperas llegar no están en línea tan a menudo como tú,
entonces publicar avisos en las carteleras de las escuelas,
en bibliotecas locales,
en centros de ayuda,
y en los mercados y tiendas puede ser la forma más efectiva de llegar a tu público.
\end{aside}


\seclbl{El arte de la llamada en frío}{s:outreach-cold-call}

Crear un sitio web y esperar que las personas lo encuentren es fácil;
llamar por teléfono o golpear en las puertas de sus casas sin ningún tipo de introducción previa es mucho más difícil.
Al igual que ponerse de pie y enseñar,
sin embargo, es un oficio que puede aprenderse.
 Aquí hay diez reglas simples para convencer a las personas:

\begin{description}
\item[1: No lo hagas]

Si tienes que convencer a alguien de hacer algo,
lo más probable es que realmente no quieran hacerlo.
Respeta eso:
casi siempre es mejor a largo plazo dejar algo en particular sin hacer 
que usar la culpa o cualquier truco psicológico encubierto que sólo generará resentimiento.


\item[2: Sé amable.]
No sé si existe un libro llamado
 \emph{Trucos secretos de los maestros ninja de ventas},
pero si existe,
probablemente le dice a los lectores que hacer algo por un cliente potencial 
crea un sentido de obligación,
lo que a su vez aumenta las probabilidades de una venta.
Esto puede funcionar, pero solo funciona una vez y no es una práctica recomendable.
Por otro lado,
si eres genuinamente amable
y ayudas a otras personas porque eso es lo que las buenas personas hacen,
sólo podrías inspirarlas a ser buenas personas también.

\item[3: Apela al bien mayor.]
Si comienzas hablándole a las personas sobre lo que hay disponible para ellas,
les estás señalando que deberían pensar en su interacción contigo
como si se tratara de  un intercambio comercial que se puede negociar.
En cambio,
comienza explicando cómo su aporte y ayuda puede hacer del mundo un lugar mejor 
y \emph{dilo en serio}.
Si lo que estás proponiendo no va a hacer del mundo un lugar mejor,
debes mejorar tu propuesta.


\item[4: Comienza desde algo pequeño.]
La mayoría de las personas son comprensiblemente reacias a sumergirse de lleno en las cosas, 
así que debes darles  la oportunidad de conocer el terreno 
y conocerte a ti y a las demás personas involucradas
 en lo que sea que necesites ayuda.
No te sorprendas o decepciones si ahí es donde terminan las cosas:
todo el mundo está ocupado o cansado o tiene proyectos propios,
o tal vez simplemente tiene un modelo mental diferente de cómo se supone que funcionan las colaboraciones.
Recuerda la regla  90-9-1-- el 90\% va a mirar, el 9\% va a hablar y el 1\% 
realmente va a hacer cosas---ajusta tus expectativas de modo acorde.

\item[5: No construyas un proyecto: construye una comunidad.]
Solía pertenecer a un equipo de béisbol que nunca jugaba realmente al béisbol:
nuestros ``juegos`` eran sólo una excusa para que estuviéramos juntos y disfrutemos de la compañía de las demás personas.
Probablemente no quieras llegar tan lejos,
pero compartir una taza de té con alguien o celebrar el cumpleaños de su primer/a nieto/a
pueden darte cosas que no podrías obtener con ninguna cantidad razonable de dinero.

\item[6: Establece un punto de conexión.]
``Estaba hablando con X'' o ``Nos conocimos en Y'' les da contexto,
lo que a su vez hace sentir más cómodas a las personas.
Esto debe ser específico:
quienes envían correo basura y quienes hacen llamadas en frío
nos han entrenado para ignorar cualquier cosa que comience con la frase
``Hace poco tiempo encontré tu sitio web {\ldots}''

\item[7: Sé específico/a sobre lo que estás pidiendo.]
Este detalle es necesario para que las personas
puedan determinar si el tiempo y las habilidades que tienen
coinciden con lo que necesitas.
Ser realista desde el principio también es una señal de respeto:
 si le dices a alguien que necesitas una mano para mover algunas cajas 
 cuando en realidad estás mudando una casa entera,
 probablemente no te ayudarán por segunda vez.

\item[8: Establece tu credibilidad.]
Menciona a quienes te patrocinan,
tu tamaño,
cuánto tiempo hace que existe tu grupo o algo que hayas logrado en el pasado
para que las personas crean que vale la pena tomarte en serio.

\item[9: Crea una ligera sensación de urgencia.]
``Esperamos lanzar esto en primavera'' es mucho más probable que genere una respuesta positiva 
que ``Con el tiempo queremos lanzar esto.''
Sin embargo la palabra ``ligera'' es importante:
si tu pedido es urgente, 
la mayoría de las personas pueden asumir que eres una persona desorganizada 
o que algo ha salido mal
y por lo tanto pueden pecar de prudencia.

\item[10: Entiende la indirecta.]
Si la primera persona a la que le pides ayuda dice no,
pregúntale a otra.
Si la quinta o décima persona dice no,
pregúntate si lo que estás tratando de hacer tiene sentido y vale la pena hacerlo.
\end{description}

La siguiente plantilla de correo electrónico sigue todas estas reglas.
Ha funcionado bastante bien:
descubrimos que alrededor de la mitad de los correos fueron respondidos, en
aproximadamente  la mitad de estas respuestas querían hablar más,
y la mitad de estos últimos condujeron a talleres,
lo que significa que entre el 10 y 15\% de las cuentas de correo a las que nos dirigimos resultaron en talleres.
Esto puede ser bastante desmoralizador si no estás acostumbrado/a, 
pero es mucho mejor que la tasa de respuesta del 2 a 3\% que la mayoría de las organizaciones esperan con llamadas en frío.

\begin{quote}

  \noindent
  Hola NOMBRE
 
 Espero que no te resulte inoportuno recibir este correo,
 pero quería continuar con nuestra conversación en LUGAR DE REUNIÓN
 para ver si tendrías interés en que realicemos un taller de entrenamiento para docentes---estamos programando  la próxima tanda para las próximas dos semanas.

 Este taller de un día les enseñará a tus voluntarios/as
 una serie de prácticas de enseñanza útiles y basadas en evidencias.
El taller se ha impartido más de cien veces de diversas maneras en seis continentes
 para organizaciones sin fines de lucro, bibliotecas y empresas,
 y todo el material está disponible gratuitamente en línea en http://teachtogether.tech.
 
El temario incluye:

  \begin{itemize}
  \item estudiantes tipo
  \item  diferencias entre diferentes tipos de estudiantes 
  \item uso de evaluaciones formativas para diagnosticar malentendidos
  \item la enseñanza como un arte performativo
  \item que motiva y desmotiva a estudiantes adultos/as
  \item la importancia de la inclusión y como ser un buen aliado/a
  \end{itemize}

Si esto te resulta interesante,
por favor avísame---Agradecería la oportunidad de hablar de los modos 
y por qué medios sería adecuado hacerlo.

Gracias,

  NOMBRE

\end{quote}

\begin{aside}{Referencias}
Construir alianzas con otros grupos que hacen cosas relacionadas a tu trabajo
vale la pena por muchas razones.
Una de ellas son las referencias:
si la persona que se aproxima en busca de tu ayuda podría ser mejor atendida por alguna otra organización,
tómate un momento para presentarlos.
Si ya has hecho esto varias veces,
agrega información a tu sitio web para ayudar a la próxima persona a encontrar lo que necesita.
En retribución, las organizaciones a las que estás ayudando pronto empezarán a ayudarte.
\end{aside}

\seclbl{Cambio académico}{s:outreach-schools}

Todo el mundo tiene miedo a lo desconocido y a pasar vergüenza. En consecuencia,
la mayoría de la gente prefiere fracasar que cambiar.
Por ejemplo,
Lauren Herckis investigó \index{Herckis, Lauren}
\hreffoot{https://www.insidehighered.com/news/2017/07/06/anthropologist-studies-why-professors-dont-adopt-innovative-teaching-methods}
{por qué los/las docentes de nivel universitario no adoptan mejores metodos de enseñanza}.
Lauren halló que el principal motivo es el miedo a parecer estúpido/a delante de  los estudiantes; 
Las otras razones fueron:
preocupación que el cambio de los métodos de enseñanza pudieran afectar las evaluaciones de los cursos de forma negativa
(que  a su vez podría afectar promociones y cargos) 
y el deseo de la gente de seguir emulando a los/las docentes que les inspiraron. 

No tiene sentido discutir si estas cuestiones son ``reales'' o no:
los/las docentes creen que son reales,
por lo que cualquier plan para trabajar con el personal docente de las universidades necesita tenerlas en cuenta\footnote{
Así como la prevalencia de la mentalidad fija entre los/las docentes cuando se trata de, 
la creencia de que algunas personas son ``solo mejores enseñando'' }.

\cite{Bark2015} realizaron un estudio de dos etapas sobre cómo quienes 
son docentes de ciencias de la computación adoptan nuevas prácticas de enseñanza, 
ya sea individualmente, como organización o en la sociedad en su conjunto. En este estudio se preguntaron 
y respondieron tres preguntas claves:

\begin{description}


\item[¿Cómo se enteran de nuevas prácticas de enseñanza?]

Buscan intencionalmente nuevas prácticas
porque su motivación es  resolver un problema (en particular, la participación de sus estudiantes),
son conscientes de las nuevas prácticas a través de iniciativas deliberadas por parte de sus instituciones,
las replican de colegas,
o las obtienen por interacciones esperadas  \emph{e inesperadas} en conferencias
(relacionadas a la enseñanza o de otro tipo).

\item[¿Por qué las prueban?]

A veces, debido a los incentivos institucionales
(por ejemplo, innovan para mejorar sus chances de promoción),
pero a menudo hay tensión en las instituciones de investigación,
donde la retórica sobre la importancia de la enseñanza tiene poca credibilidad.
Otra razón importante es su propio análisis costo/beneficio:
¿les va a ahorrar tiempo esa innovación?
Una tercera razón es que se inspiran en modelos de roles a seguir---otra vez,
esto afecta en gran medida a las innovaciones destinadas a  mejorar la motivación y participación más que los resultados del aprendizaje.
Finalmente, el cuarto factor son fuentes de confianza,
por ejemplo, personas que han conocido en congresos o conferencias que se encuentran en la misma situación que ellos/as 
y han tenido experiencias exitosas al adoptar las nuevas prácticas.
Pero los/las dicentes tienen preocupaciones que no siempre son abordadas por el grupo de personas que abogan por modificaciones.
La primera era la ley de Glass:
cualquier nueva herramienta o práctica inicialmente te relentiza,
entonces mientras que las nuevas prácticas pueden hacer la enseñanza más efectiva a largo plazo, no pueden permitirse en el corto plazo.
Otra preocupación es que el diseño físico de las aulas dificulta muchas prácticas nuevas:
por ejemplo,
los grupos de discusión no funcionan bien en asientos de estilo teatro.

Pero el resultado más revelador fue el siguiente:
``A pesar de ser los propios investigadores/as quienes enseñan,
la mayor parte de los/las docentes de ciencias de computación con quienes hablamos
no creía que los resultados de estudios sobre educación fueran razones creíbles para probar prácticas de enseñanza.''
Esto es consistente con otros hallazgos:
incluso personas cuyas carreras están dedicadas a la investigación a menudo ignoran investigaciones sobre educación. 

\item[¿Por qué las siguen usando?]

Como dice ~\cite{Bark2015}, ``Las devoluciones de los estudiantes son vitales,''
y son normalmente la razón más fuerte para continuar usando una práctica,
a pesar de que sabemos que las auto-evaluaciones realizadas por estudiantes 
no se correlacionan fuertemente con los resultados del aprendizaje~\cite{Star2014,Uttl2017}
(aunque la asistencia a clases sí es un buen indicador del compromiso de los/las estudiantes).
Otro motivo para retener una práctica son los requisitos institucionales,
aunque si esta es la única motivación,
las personas abandonarán la práctica 
cuando el incentivo explícito o el monitoreo desaparezcan.

\end{description}

La buena noticia es que puedes abordar estos problemas sistemáticamente.
\cite{Baue2015} observó  la adopción de nuevas técnicas médicas dentro de la Administración de Veteranos de los Estados Unidos. 
Hallaron que las prácticas médicas basadas en evidencia
toman en promedio 17 años en ser incorporadas a las prácticas generales de rutina,
y que solo cerca de la mitad de estas prácticas llegan a ser ampliamente adoptadas.
Este deprimente hallazgo junto con otros, han estimulado el crecimiento de la
\gref{g:implementation-science}{ciencia de la implementación},
que es el estudio de cómo lograr que la gente adopte mejores prácticas.

Como decía el \chapref{s:community},
el punto de partida es descifrar qué es lo que creen que necesitan las personas que quieres ayudar.
Por ejemplo, \cite{Yada2016} resume las devoluciones de docentes de nivel primario y secundario sobre la preparación y el apoyo que desean.
Aunque esto puede no ser aplicable a todos los entornos,
tomar una taza de té con unas pocas personas y escucharlas antes de hablar
hace un mundo de diferencia en su voluntad de intentar algo nuevo. 

Una vez que sepas lo que la gente necesita,
el siguiente paso es hacer cambios de manera incremental,
dentro de los propios esquemas de las instituciones.
\cite{Nara2018} describe un programa intensivo de tres años
basado en cohortes muy unidas y apoyo administrativo
que triplicó las tasas de graduación,
mientras que~\cite{Hu2017} describe el impacto de implementar un programa de certificaciones de seis meses para docentes de secundaria que quieren enseñar computación.
El  número de docentes de computación se ha mantenido estable entre 2007 y 2013,
pero se cuadruplicó --sin disminuir la calidad-- después de la introducción de un nuevo programa de certificación:
los/las docentes que eran personas novatas en computación parecían ser tan eficaces en el curso introductorio como los/las docentes con más entrenamiento o formación informática en la enseñanza.


En un sentido más amplio,
\cite{Borr2014} presenta categorías para lograr que ocurran cambios en la educación superior.
Las categorías están definidas de acuerdo a si el cambio es individual o 
sistémico y si se prescribe (de arriba hacia abajo) o es un cambio emergente (de abajo hacia arriba). 
La persona que trata de hacer los cambios (y hacer que duren)
tiene un rol distinto en cada una de estas situaciones,
y consecuentemente debe seguir diferentes estrategias.
El artículo continúa explicando en detalle cada uno de los métodos,
mientras que~\cite{Hend2015a,Hend2015b} presentan las mismas ideas en una forma más 
procesable.

Al llegar a una institución,
probablemente en principio caigas en la categoría de cambio individual/emergente,
dado que te aproximas a los/las docentes uno a uno
y tratarás de lograr que los cambios ocurran de abajo hacia arriba.
Si este es el caso,
las estrategias que recomiendan Borrego y Henderson se enfocan
en que los/las docentes reflexionen sobre sus prácticas de enseñanza de manera individual o en grupos.
Haz programación en vivo para mostrarles lo que haces o los ejemplos que usas,
luego haz que tus estudiantes programen en vivo en turnos 
para mostrar cómo usarían esas ideas y técnicas en su desempeño
así les darás a todos/as la oportunidad de elegir cosas que les serán útiles en su contexto.

\seclbl{Docentes \emph{free-range}}{s:outreach-free-range}

Las escuelas y las universidades no son los únicos lugares en donde la gente va a aprender programación;
en los últimos años, un número creciente de personas ha acudido a talleres \emph{free-range} y programas de entrenamiento intensivo.
Estos últimos suelen tener entre uno y seis meses de duración,
son llevados a cabo por grupos de voluntarios/as o por empresas con fines de lucro
y su objetivo/apuntan a personas que se están re-entrenando para entrar al mundo de la tecnología.
Algunos programas son de muy alta calidad,
pero otros existen primariamente para sacarle dinero del bolsillo a la gente~\cite{McMi2017}.
 
\cite{Thay2017} entrevistó a 26 graduados de estos entrenamientos intensivos
que les dan una segunda oportunidad a aquellos que han perdido oportunidades previas de educación en computación
(aunque expresarlo de este modo implica realizar grandes suposiciones 
en lo que respecta a  personas de grupos subrepresentados).
Las personas que participan de los entrenamientos intensivos enfrentan grandes costos y riesgos personales:
deben gastar una cantidad significativa de tiempo, dinero y esfuerzo antes, 
durante y después de los entrenamientos intensivos, y cambiar de carrera puede tomar un año o más.
Varias de las personas entrevistadas sienten que sus certificados fueron subestimados por sus empleadores;
como dijeron algunas de ellas:
obtener un trabajo significa aprobar una entrevista,
pero dado que quienes te entrevistan muchas veces no comparten sus motivos de rechazo,
es difícil saber qué arreglar o qué más aprender.
Muchas personas han tenido que recurrir a pasantías (pagas o de otro tipo)
y pasan mucho tiempo construyendo sus portfolios y haciendo networking.
Las tres barreras informales que más fácilmente identificaron las personas entrevistadas fueron la jerga,
el síndrome del impostor/a, y una sensación de no encajar.

\cite{Burk2018} profundizó en este tema
al comparar las habilidades y credenciales 
que la industria tecnológica busca con aquellas habilidades provistas por carreras de cuatro años y programas de entrenamiento intensivo.
A partir de entrevistas con 15 gerentes de contratación de empresas de varios tamaños y algunos grupos focales,
encontraron que los/las reclutadores/as enfatizaban uniformemente en habilidades blandas
(especialmente trabajo en equipo, comunicación y la habilidad para continuar aprendiendo).
Muchas compañías requerían un título de cuatro años
(aunque no necesariamente en ciencias de la computación),
pero muchas también elogiaron a egresados/as de programas de entrenamiento intensivo por ser mayores en edad o tener más madurez
y por poseer conocimientos más actualizados.

Si te estás aproximando a un programa de entrenamiento intensivo que ya existe,
tu mejor estrategia podría ser enfatizar lo que sabes sobre enseñanza,
en lugar de lo que sabes sobre tecnología,
ya que gran parte del personal y fundadores tiene experiencia en programación
pero poca o ninguna capacitación en educación.
Los primeros capítulos de este libro han servido con este público en el pasado, 
y ~\cite{Lang2016} describe
prácticas de enseñanza basadas en evidencia que pueden implementarse
con un esfuerzo mínimo y a bajo costo.
Estas prácticas tal vez no tengan el mayor impacto, 
pero lograr algunas victorias tempranas ayuda a generar apoyo para esfuerzos más grandes.


\seclbl{Reflexiones finales }{s:outreach-final}
Es imposible cambiar grandes instituciones por tu cuenta:
necesitas aliados/as
y para conseguir aliados/as,
necesitas tácticas.
La guía más útil que he encontrado es~\cite{Mann2015}, 
que cataloga más de cuatro docenas de estas tácticas
y las organiza de acuerdo a si se implementan mejor temprano,
más adelante,
a lo largo del ciclo de cambio,
o cuando encuentras resistencia.
Algunos de sus patrones incluyen:

\begin{description}

\item[En tu espacio:]
Mantén la nueva idea visible
colocando recordatorios en toda la organización.

\item[Símbolo:] 
Para una nueva idea se mantenga viva en la memoria de una persona,
entrega un objeto simbólico  que pueda identificarse con el tema que se está presentando.

\item[El rol del escepticismo:]
Identifica a los/las líderes de opinión fuertes que son escépticos de tu nueva 
idea para desempeñar el papel de ``persona escéptica oficial''.
Usa sus comentarios para mejorar tus esfuerzos,
incluso si no logras hacer que cambien de opinión.

\item[Compromiso futuro:]
Si puedes anticipar algunas de sus necesidades,
podrás pedir (y tener éxito) en obtener un compromiso 
futuro de las personas más ocupadas.
Si se les da algún tiempo de anticipación, 
es posible que tengan mayor disposición a ayudar
porque permites que se organicen.

\end{description}

La estrategia más importante es
el deseo de cambiar tus metas
basado en lo que aprendes de las personas a las que estás tratando de ayudar.
Por ejemplo, enseñarles tutoriales que muestran cómo usar una hoja de cálculo
podría ser una ayuda más rápida y confiable que 
una introducción a JavaScript.
A menudo he cometido el error de confundir cosas que me apasionan
con cosas que las otras personas deberían saber;
si realmente quieres ser quien las acompañe,
recuerda siempre que el aprendizaje y el cambio tienen que ir en ambos sentidos.

La parte más difícil de construir relaciones es comenzarlas.
Reserva una o dos horas cada mes
para encontrar aliados/as y mantener tus relaciones con ellos/as.
Una forma de hacer esto es pedirles consejo:
¿cómo creen que deberías hacer para que lo que están haciendo sea más conocido?
¿Dónde han encontrado espacio para dar clases?
¿Qué necesidades creen que no son cubiertas
y tú serías capaz de lograr?
Cualquier grupo que haya existido durante algunos años tendrá consejos útiles;
también se sentirán halagados/as de que se les haya consultado,
y sabrán quién eres la próxima vez que llames.


Y como dice~\cite{Kuch2011},
si no puedes ser el/la primero/a en una categoria,
tratar de crear una nueva categoría en la que sí puedas.
Si no puedes hacerlo,
únete a un grupo existente o piensa en hacer algo completamente diferente.
Esto no es derrotista:
si alguien más ya está haciendo lo que tienes en mente,
deberías incorporarte y contrinuir a ese grupo o 
abordar alguna de las otras cosas igualmente útiles
en las que podrías estar trabajando.

\seclbl{Ejercicios}{s:outreach-exercises}

\exercise{Discurso de presentación para un/a integrante del concejo de la ciudad}{individual}{10'}

Este capítulo describe una organización
que ofrece talleres de programación de fin de semana para las personas que re-ingresan a la actividad laboral.
Escribe un discurso de presentación para esa organización,
dirigido a un/a integrante del consejo de la ciudad cuyo apoyo la organización necesita.

\exercise{Presenta tu organización}{individual}{30'}

Identifica dos grupos de personas de los que tu organización necesite apoyo
y escribe un discurso de presentación dirigido a cada uno de estos grupos.

\exercise{Asuntos de correo electrónico}{parejas}{10'}

Escribe las líneas de asunto (y solo las líneas de asunto) para tres mensajes de correo electrónico:
uno anunciando un nuevo curso,
uno anunciando un nuevo patrocinio,
y uno anunciando un cambio en el liderazgo del proyecto.
Compara tus líneas de asunto con las de tu pareja.
Analicen  si se pueden combinar las mejores características de cada asunto a la par que los acortan.

\exercise{Manejando la resistencia pasiva}{grupos pequeños}{30'}

Las personas que no quieren cambios a veces lo dirán en voz alta,
pero a menudo pueden utilizar otras formas de resistencia pasiva,
como simplemente no lidiar con ello,
o plantear un posible problema después de otro 
para hacer que el cambio parezca más arriesgado 
y más costoso de lo que en realidad es probable que sea
\cite {Scot1987}.
En grupos pequeños,
enumeren tres o cuatro razones por las que las personas podrían no querer 
que tu iniciativa de enseñanza siga adelante,
y expliquen qué pueden hacer con el tiempo y los recursos que tienen 
para contrarrestar cada una de esas razones.

\exercise{¿Por qué aprender a programar?}{individual}{15'}

Revisa el ejercicio ``¿Por qué aprender a programar?'' en la \secref {s:intro-exercises}.
¿Dónde conciden tus razones para enseñar y las razones de tus estudiantes para aprender?
¿Dónde no coinciden?
¿Cómo afecta eso a tu marketing?

\exercise{Programadores/as conversacionales}{pensar-parejas-compartir}{15'}

Un/una \gref{g:conversational-programmer}{programador/a conversacional}
es alguien que necesita saber lo suficiente sobre computación
para tener una conversación valiosa con otro/a programador/a,
pero no va a programar por su cuenta.
\cite{Wang2018} descubrió que la mayoría de los recursos de aprendizaje no abordan las necesidades de este grupo.
En parejas,
escriban un discurso para un taller de medio día destinado a ayudar a las personas que se ajustan a esta descripción.
Luego, comparte el discurso que armaron con tu pareja con el resto de la clase.

\exercise{Colaboraciones}{grupos pequeños}{30'}
Responde por tu cuenta las siguientes preguntas,
luego compara tus respuestas con las dadas por otros miembros de tu grupo.

\begin{enumerate}

\item
¿Tienes algún acuerdo o relación con otros grupos?

\item
¿Quieres generar lazos con algún otro grupo?

\item
El hecho de tener (o no tener) colaboraciones, 
¿podría ayudarte a alcanzar tus metas?

\item
¿Cuáles son tus colaboraciones clave?

\item
¿Son estos los colaboradores adecuados o indicados para alcanzar tus metas u objetivos?

\item
¿Con qué grupos o entidades te gustaría que tu organización
tuviera acuerdos o lazos?



\end{enumerate}

\exercise{Educación}{toda la clase}{10'}

\cite{Laba2008} explora por qué los Estados Unidos y otros países
siguen trasladando la solución a los problemas sociales hacia las instituciones educativas 
y por qué eso sigue sin funcionar.
Tal como él remarca,
``[La educación] ha hecho muy poco para promover la igualdad de raza, clase y género;
para mejorar la salud pública, la productividad económica y la buena ciudadanía;
o reducir el sexo adolecente, las muertes por accidentes de tránsito, la obesidad y la destrucción ambiental.
De hecho,
en muchos sentidos la educación ha tenido un efecto negativo sobre estos problemas,
sacando dinero y energía de las reformas sociales que podrían haber tenido un impacto más substancial.''
El autor continúa escribiendo: 

\begin{quote}

Entonces, ¿cómo debemos entender el éxito de esta institución,
a la luz de su fracaso en lo que le pedimos?
Una forma de pensar en esto es que
la educación puede no estar haciendo lo que pedimos,
pero está haciendo lo que queremos.
Queremos una institución que persiga nuestros objetivos sociales
de una manera que esté en línea con el individualismo en el corazón del ideal liberal,
con el objetivo de resolver problemas sociales
buscando cambiar los corazones, mentes y capacidades de cada estudiante individual.
Otra forma de decir esto es que
queremos una institución a través de la cual podamos expresar nuestros objetivos sociales
sin violar el principio de elección individual
que se encuentra en el centro de la estructura social,
incluso si esto tiene el costo de no lograr estos objetivos.
Entonces la educación puede servir como un punto de orgullo cívico,
un lugar para exponer nuestros ideales,
y un medio para participar en disputas edificantes pero  
que en última instancia intrascendentes
sobre visiones alternativas de la buena vida.
Al mismo tiempo,
la educación también puede servir como un conveniente chivo expiatorio
al que podemos culpar por su fracaso en lograr nuestras más altas aspiraciones como sociedad.

\end{quote}

¿Cómo los esfuerzos para enseñar pensamiento computacional y la ciudadanía digital en las escuelas se adaptan a este marco?
¿Los programas de entrenamiento intensivo evitan estas trampas o simplemente las entregan con una nueva apariencia?

\exercise{Adopción institucional}{clase completa}{15'}

Relean la lista de motivaciones para adoptar nuevas prácticas
presentada en la \secref{s:outreach-schools}.

¿Cuáles de estas motivaciones se aplican a tí y a tus colegas?
¿Cuáles son irrelevantes en tu contexto?
¿En cuáles de estas motivaciones haces hincapié
en los casos en que interactúas con personas que trabajan en instituciones educativas formales?


\exercise{Si al principio no tienes éxito}{grupos pequeños}{15'}

W.C.~Fields probablemente nunca dijo
``Si al principio no tienes éxito, inténtalo, inténtalo de nuevo.
Luego abandona---no sirve de nada ser un maldito tonto al respecto.''
Sin embargo, sigue siendo un buen consejo:
si las personas con las que intentas comunicarte no están respondiendo,
podría ser que nunca los convenzas.
En grupos de 3 a 4 personas,
hagan una breve lista de señales que indiquen que debes dejar de intentar hacer algo en lo que crees.
¿Cuántas de estas señales ya son verdaderas?


\exercise{Haz que falle}{individual}{15'}
\cite{Farm2006} presenta algunas reglas irónicas para lograr que nuevas herramientas \emph{no sean} adoptadas,
todas de las cuales aplican a nuevas prácticas de enseñanza:

\begin{enumerate}

\item
  Hazlo opcional.

\item
Economiza en entrenamiento. 

\item
  No las uses en un proyecto real.

\item
  Nunca la integres.

\item
Úsala esporádicamente.

\item
  Hazla parte de una iniciativa de calidad.

\item
  Marginaliza al campeón/a.

\item
Capitaliza los primeros errores.

\item
Haz una inversión pequeña.

\item
 Explota el miedo, la incertidumbre, la duda, la pereza y la inercia.

\end{enumerate}
¿Cuál de estas reglas has visto implementarse recientemente?
¿Cuáles has utilizado tú mismo/a?
¿De qué forma se usaron esas reglas?

\exercise{Mentoreo}{todo la clase}{15'}

El \hreffoot{http://www.iaamcs.org/}{\emph{Institute for African-American Mentoring in Computer Science}} 
(Instituto de Mentoría Afroamericana en Ciencias de la Computación)
ha publicado \hreffoot{http://iaamcs.org/guidelines}{guías para mentorear estudiantes de doctorado}.
Lee estas guías individualmente,
luego discutan en la clase
y califiquen los esfuerzos para tu propio grupo como: +1 (definitivamente lo haría),
0 (no estoy seguro o no es aplicable),
o -1 (definitivamente no lo haría).
