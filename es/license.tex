\chapter{Licencia}\label{s:licencia}

{\setlength{\parindent}{0em}

\emph{
  Este es un resumen de lectura sencilla para personas (y no un sustituto) de la licencia.
  Por favor mira \url{https://creativecommons.org/licenses/by-nc/4.0/legalcode} para el texto legal completo.
}

\vspace{\baselineskip}

\noindent
Este trabajo se licencia bajo
\hreffoot{https://creativecommons.org/licenses/by-nc/4.0/}{Creative Commons Atribución -- No Comercial 4.0} 
(CC-BY-NC-4.0).\\

\noindent
\textbf{Eres libre de:}

\begin{itemize}
\item
  \textbf{Compartir}---copiar y redistribuir el material en cualquier medio o
  formato
\item
  \textbf{Adaptar}---reacomodar, transformar y construir sobre el material.
\end{itemize}

El/la licenciante no puede revocar estas libertades mientras sigas los
términos de la licencia.

\vspace{\baselineskip}

\textbf{Bajo los siguientes términos:}

\begin{itemize}
\item
  \textbf{Atribución}---Debes dar el crédito apropiado, proporcionar un enlace
  a la licencia e indicar si se hicieron cambios. Puedes hacerlo de cualquier manera
  razonable, pero siempre que no sugiriera que el/la licenciante te respalda 
  a ti o al uso que le das al material.\\
\item
  \textbf{No Comercial}---No puedes utilizar el material con fines comerciales.
\end{itemize}

\textbf{Sin restricciones adicionales}---No puedes aplicar términos legales o
medidas tecnológicas que restrinjan legalmente a otros/as de hacer cualquier cosa
que la licencia permita.

\vspace{\baselineskip}

\textbf{Avisos:}

\begin{itemize}

\item
  No tienes que cumplir con la licencia para aquellos elementos del
  material que son de dominio público o cuando su uso esté permitido
  por una excepción o limitación aplicable. 

\item
  No se otorgan garantías. Es posible que la licencia no otorgue
  todos los permisos necesarios para el uso que se pretende dar al material. 
  Por ejemplo, derechos relacionados a la publicidad, privacidad o derechos morales 
  pueden limitar la forma en la que puedes usar el material.

\end{itemize}
}
