\chapter{Enseñar Online}\label{s:online}

\begin{quote}

 Si usas robots para enseñar, les enseñas a las personas a ser robots. \\
  --- atribuido a varias personas

\end{quote}

La tecnología ha cambiado la enseñanza y el aprendizaje muchas veces.
Antes de que se introdujeran los pizarrones en las escuelas a principios del siglo XIX,
no había forma de que los profesores compartieran un ejemplo improvisado,
un diagrama,
o hacer ejercicio con toda una clase a la vez.
Barato,
de confianza,
fácil de usar,
y flexible,
los pizarrones permitieron a los docentes hacer cosas rápidamente y a gran escala
que antes solo habían podido hacer lentamente y poco a poco.
De forma similar,
las cámaras de video portátiles revolucionaron el entrenamiento atlético;
al igual que las grabadoras revolucionaron la enseñanza musical una década antes.

Muchas de las personas que introducen Internet en las aulas no conocen esta historia,
y no se dan cuenta de que el suyo es solo el último de  
\hreffoot{http://teachingmachin.es/timeline.html}{una larga serie de intentos}
de utilizar máquinas para enseñar~\cite{Watt2014}.
Desde la imprenta pasando por la radio y la televisión
a computadoras de escritorio y dispositivos móviles,
cada nueva forma de compartir conocimientos ha producido una ola de optimistas agresivos
que creen que la educación no funciona y que la tecnología puede arreglarlo.

Sin embargo,
los defensores más acérrimos de la tecnología de la educación a menudo han sabido menos sobre ``educación'' que sobre ``tecnología'',
y detrás de su retórica,
muchos han sido impulsados más por la perspectiva de ganancias
que por el deseo de empoderar a los alumnos.

El debate actual a menudo se enturbia al confundir ``en línea'' con ``automatizado''.
Corre bien
una docena de personas que resuelven un problema en un chat de video
se siente como cualquier otra discusión en grupos pequeños.
Por el contrario,
un escuadrón de ayudantes de enseñanza que califican cientos de trabajos con una rúbrica inflexible
bien podría ser una colección de scripts de Perl.\index{Perl (referencia despectiva a)}
Por lo tanto, este capítulo comienza con la instrucción en línea totalmente automatizada,
usando videos grabados y ejercicios calificados automáticamente,
luego explora algunos modelos híbridos alternativos.

\seclbl{MOOCs}{s:online-moocs}

El esfuerzo de más alto perfil para reinventar la educación usando Internet
son los \gref{g:mooc}{Cursos masivos en línea (en inglés \emph{Massive Open Online Course}}), o \emph{MOOC}, por sus siglas en Inglés.
El término fue inventado por David Cormier en 2008\index{Cormier, David}
para describir un curso organizado por George Siemens\index{Siemens, George}
y Stephen Downes.\index{Downes, Stephen}
Ese curso se basó en una visión \grefdex{g:connectivism}{conectivista}{connectivism} del aprendizaje,
que sostiene que el conocimiento se distribuye
y el aprendizaje es el proceso de encontrar, crear y podar conexiones.

El término ``MOOC'' fue rápidamente co-optado por los creadores de
cursos que se parecían más al modelo de disertación de un aula tradicional,
con el maestro como el centro definiendo los objetivos
y los y las estudiantes vistos como recipientes o replicadores de conocimientos.
Las clases que utilizan el modelo conectivista original se suelen denominar ``cMOOCs,''
mientras que las clases que centralizan el control se llaman ``xMOOCs.''
(A este último también se le llama a veces un `` MESS '' (la palabra \emph{mess} significa \emph{lío} en Inglés),
por las siglas Sabio Masivamente Realzado en el Escenario (\emph{Massively Enhanced Sage on the Stage}, por sus siglas en Inglés.))

Cinco años atrás,
no se podía caminar por los campus de las universidades más grandes
sin escuchar a alguien hablando sobre como los MOOCs revolucionarian la educación,
la destruirían, 
o posiblemente ambas cosas.
Los MOOCs le darían a los y las estudiantes acceso a un gran abanico de cursos
y les permitirían trabajar cuando les fuera conveniente
en lugar de acomodar su aprendizaje a la agenda de otra persona.

Pero los MOOC no han sido tan efectivos
como predijeron sus defensores más entusiastas~\cite{Ubel2017}.
Una razón es que
el contenido grabado es ineficaz para muchos novatos
porque no puede aclarar sus conceptos erróneos individuales(\chapref{s:models}):
si no entienden una explicación la primera vez,
por lo general, no g¿hay otra diferente para ofrecer.
Otra razón es que la evaluación automatizada necesaria para lograr lo ``masivo'' en MOOC
solo funciona bien en los niveles más bajos de la taxonomía de Bloom(\secref{s:process-objectives}).\index{Bloom's Taxonomy}
Ahora también está claro que
los y las estudiantes tienen que soportar mucho más la carga de mantenerse concentrados en un MOOC,
que la impersonalidad de trabajar en línea puede fomentar un comportamiento descortés y desmotivar a las personas,
y que ``disponible para todos'' significa en realidad
``disponible para todo el mundo lo suficientemente pudiente como para tener Internet de alta velocidad y mucho tiempo libre''.

\cite{Marg2015} examinó 76 MOOCs sobre varios temas y descubrió que
si bien la organización y presentación del material fue buena,
la calidad del diseño de las lecciones fue deficiente.
Mas cerca de casa,
\cite{Kim2017} estudió treinta tutoriales on-line y populares sobre programación,
y descubrió que en gran medida enseñaban el mismo contenido de la misma manera:
de abajo hacia arriba,
comenzando con conceptos de programación de bajo nivel y avanzando hacia metas de alto nivel.
La mayoría requirió que los alumnos escribieran programas y proporcionaron algún tipo de retroalimentación inmediata,
pero esta retroalimentación fue típicamente muy superficial.
Pocos explican cuándo y por qué los conceptos son útiles
(es decir, no mostraron cómo transferir conocimientos),
o proporcionaron orientación para errores comunes
y aparte de una diferenciación rudimentaria basada en la edad,
ninguna lección se personalizaba basada en la experiencia previa en programación o en los objetivos del/la estudiante.

\begin{aside}{Aprendizaje personalizado}
  Pocos términos han sido utilizados y abusados ​​de tantas formas 
  como \gref{g:personalized-learning}{aprendizaje personalizado}.
  Para la mayoría de los defensores de la educacuón con tecnología,
  significa ajustar dinámicamente el ritmo de las lecciones en función del rendimiento del alumno,
  de modo que si alguien responde varias preguntas seguidas correctamente,
  la computadora omitirá algunas de las preguntas siguientes.

  Hacer esto puede producir
  \hreffoot{https://www.rand.org/pubs/research\_briefs/RB9994.html}{modestas mejoras},
  pero se puede hacer mejor.
  Por ejemplo,
  si muchos/as estudiantes encuentran difícil un tema en particular,
  el/la docente puede preparar múltiples explicaciones alternativas de ese punto
  en lugar de acelerar un solo camino.
  De esa manera,
  si una explicación no resuena,
  otras están disponibles.
  Sin embargo,
  esto requiere mucho más trabajo de diseño por parte del/la docente,
  que puede ser la razón por la que no ha demostrado su popularidad.
  Incluso si funciona,
  es probable que los efectos sean mucho menores de lo que creen algunos de sus defensores.
  Un/a buen/a docente hace una diferencia de 0.1 a 0.15 desviaciones estándar en el desempeño de fin de año en la escuela primaria~\cite{Chet2014}
  (ver \hreffoot{http://educationnext.org/in-schools-teacher-quality-matters-most-coleman/}{este artículo} para un breve resumen).
  No es realista creer que cualquier tipo de automatización pueda superar esto en el corto plazo.
\end{aside}

Entonces, ¿cómo \emph{debe} ser usada la internet para enseñar y aprender habilidades tecnológicas?
Sus pros y contras son:\index{online learning!pros and cons of}

\begin{description}

\item[Los y las estudiantes pueden acceder a más lecciones y más rapido que nunca antes]
  Previsto,
  por supuesto,
  que un motor de búsqueda considere que vale la pena indexar esas lecciones,
  que su proveedor de servicios de Internet y el gobierno no lo bloqueen,
  y que la verdad no se ahoga en un mar de desinformación que agota la atención.

\item[Los y las estudiantes pueden acceder a \emph{mejores} lecciones que nunca antes,]
  a menos que estén siendo dirigidos/as hacia material de segunda categoría
  para redistribuir la riqueza de las personas que no tienen a las personas que si tienen~\cite{McMi2017}.
  También vale la pena recordar que la escasez aumenta el valor percibido,
  para que la educación en línea sea más barata
  se convertirá cada vez más en lo que todo el mundo quiere para los/as hijos/as de otra persona.

\item[Los estudiantes también pueden acceder a muchos más contactos que nunca.]
  Pero solo si esos/as estudiantes realmente tienen acceso a la tecnología requerida,
  puede permitirse usarlo,
  y no están fuera de línea por acoso o marginados/as
  porque no se ajustan a las normas sociales de cualquier grupo que lleve la voz cantante.
  En la práctica,
  la mayoría de los usuarios de MOOC provienen de entornos seguros y acomodados~\cite{Hans2015}.

\item[Los profesores pueden obtener información mucho más detallada sobre cómo trabajan los/as estudiantes.]
  Siempre que los/as estudiantes hagan cosas que sean susceptibles de análisis automatizado a gran escala
  y no se opongan a la vigilancia en el aula
  o no son lo suficientemente poderosos como para que sus objeciones importen.

\end{description}

\cite{Marg2015,Mill2016a,Nils2017} describe formas de acentuar los aspectos positivos en la lista anterior
evitando los negativos:

\begin{description}

\item[Hacer que los plazos sean frecuentes y bien publicitados,]
  y aplíquelos para que los/as estudiantes entren en ritmo de trabajo.

\item[Mantener al mínimo las actividades sincronizadas de todas las clases, como conferencias en vivo]
  para que las personas no se pierdan cosas debido a conflictos de agenda.

\item[Hacer que los/as estudiantes contribuyan al conocimiento colectivo,]
  ej.\ tomar notas compartidas (\secref{s:classroom-notetaking}),
  servir como escribas en el aula,
  o contribuir con problemas a conjuntos de problemas compartidos (\secref{s:individual-peer}).

\item[Animar o solicita a los/as estudiantes que realicen parte de su trabajo en grupos pequeños]
  que \emph{si} tienen actividades sincrónicas en linea
  como una discusión semanal.
  Esto ayuda a los/as estudiantes a mantenerse comprometidos/as y motivados/as sin crear demasiados problemas de agenda.
  (Ver \appref{s:meetings} para obtener algunos consejos sobre cómo hacer que estas discusiones sean justas y productivas.)

\item[Crear, publicitar y hacer cumplir un código de conducta.]
  para que todos puedan participar en los debates en línea (\secref{s:classroom-coc}).

\item[Utilizar muchas lecciones breves en lugar de pocos fragmentos largos al estilo conferencias.]
  para minimizar la carga cognitiva
  y brindar muchas oportunidades para la evaluación formativa.
  Esto también ayuda con el mantenimiento:
  si todos tus videos son cortos,
  simplemente puedes volver a grabar cualquiera que necesite actualización,
  lo que a menudo es más barato que intentar arreglar los más largos.

\item[Utilizar el video para fomentar la participación en lugar de instruir.]
  Dejando de lado las discapacidades (\secref{s:motivation-accessibility}),
  los/as estudiantes pueden leer más rápido de lo que tú puedes hablar.
  La excepción a esta regla es que
  el video es la mejor manera de enseñar verbos(acciones):
  videos de pantallas cortos que muestran cómo usar un editor,
  cómo recorrer el código en un depurador,
  y así sucesivamente, son más eficaces que las capturas de pantalla con texto.

\item[Identificar y aclarar tempranamente conceptos erróneos.]
  Si los datos muestran que los/as estudiantes tienen dificultades con algunas partes de una lección,
  crear explicaciones alternativas de esos partes
  y ejercicios adicionales para que practiquen.

\end{description}

Todo esto tiene que ser implementado de alguna manera,\index{online learning!implementation of}
lo que significa que necesitas alguna clase de plataforma de enseñanza.
Puedes utilizar tanto un \gref{g:lms}{sistema de gestión del aprendizaje (\emph{learning management system}, en inglés)} todo en uno
como \hreffoot{http://moodle.org}{Moodle} o \hreffoot{https://www.sakaiproject.org/}{Sakai},
o generar uno por ti mismo/a 
usando \hreffoot{http://slack.com}{Slack} o \hreffoot{https://zulipchat.com/}{Zulip} para el chat,
\hreffoot{http://hangouts.google.com}{Google Hangouts}
o \hreffoot{https://appear.in/}{appear.in} para videoconferencias,
y \hreffoot{https://wordpress.org/}{WordPress},
\hreffoot{http://docs.google.com}{Google Docs},
o cualquier número de sistemas wiki para la autoría colaborativa.
Si recién estás comenzando,
elije lo que sea más fácil de configurar y administrar
y lo que sea más familiar para tus estudiantes.
Si te enfrentas a una elección,
la segunda consideración es más importante que la primera:
esperas que las personas aprendan mucho en tu clase,
por lo que es justo que tu aprendas a manejar las herramientas con las que se sientan más cómodas.

Montar una plataforma para el aprendizaje es necesario pero no suficiente:
si quieres que tus estudiantes prosperen,
necesitas crear una comunidad.
Cientos de libros y presentaciones hablan sobre cómo hacer esto,
pero la mayoría se basan en las experiencias personales de sus autores.
\cite{Krau2016} es una excepción bienvenida:
si bien es anterior al descenso acelerado de Twitter y Facebook hacia el abuso y la desinformación,
la mayoría de sus hallazgos siguen siendo relevantes.
\cite{Foge2005} también está lleno de consejo útiles
sobre las comunidadesd de práctica a las que los/as estudiantes pueden esperar unirse;
exploramos algunas de sus ideas en \chapref{s:community}.

\begin{aside}{Libertad para, libertad de}
  El ensayo de 1958 de Isaiah Berlin\index{Berlin, Isaiah}
  ``\hreffoot{https://en.wikipedia.org/wiki/Two\_Concepts\_of\_Liberty}{Dos conceptos de libertad}''
  hizo una distinción entre libertad positiva,
  que es la capacidad de hacer algo,
  y libertad negativa,
  que es la ausencia de reglas que digan que no puedes hacerlo.
  Las discusiones en línea generalmente ofrecen libertad negativa
  (nadie te impide decir lo que piensas)
  pero no libertad positiva
  (muchas personas no pueden ser escuchadas, en realidad).
  Una forma de abordar esto es introducir algún tipo de limitación,
  como permitir que cada estudiante contribuya con un mensaje por hilo de discusión al día.
  Hacer esto les da a aquellos/as que tienen algo que decir la oportunidad de decirlo,
  mientras deja espacio para que otros/as también digan cosas.
\end{aside}

Otra preocupación que se tiene sobre la enseñanza en línea es la posibilidad de que los/as estudiantes hagan trampa.
La deshonestidad del día a día no es más común en las clases en línea que en los entornos presenciales~\cite{Beck2014},
pero la tentación de que otra persona escriba el examen final,
y la dificultad de comprobar si esto realmente sucedió,
es una de las razones por las que las instituciones educativas se han mostrado reacias a ofrecer créditos para las clases solamente en línea.
Es posible supervisar el examen a distancia,
pero antes de invertir en esto,
lee~\cite{Lang2013}:
explora por qué y cómo los alumnos hacen trampa,
y cómo se pueden estructurar los cursos para evitar darles una razón para hacerlo.

\seclbl{Video}{s:online-video}

Una característica destacada de la mayoría de los MOOC es el uso de clases grabadas en video.
Estos videos pueden ser efectivos:
como se menciona en \chapref{s:performance},
una técnica de enseñanza llamada instrucción directa\index{Direct Instruction}
basado en la entrega precisa de un guión bien diseñado ha demostrado repetidamente su eficacia~\cite{Stoc2018}.
Sin embargo,
deben diseñarse guiones para la instrucción directa,
probando,
y refinado con mucho cuidado,
lo que es una inversión que muchos MOOC no han querido o no han podido hacer.
Hacer un pequeño cambio en una página web o en una plataforma de diapositivas solo toma unos minutos;
hacer incluso un pequeño cambio en un video corto lleva una hora o más,
por lo que el costo de actuar sobre la base de los comentarios puede ser insoportable para el docente.
E incluso cuando están bien hechos
los videos deben combinarse con actividades para que sean beneficiosos:
\cite{Koed2015} estima,
``{\ldots}el aprendizaje el beneficio de hacer {\ldots} es
más que seis veces que mirar o leer.''

Si estás enseñanzo programación,
puede usar grabaciones de pantallas en lugar de diapositivas,\index{screencasts}
ya que ofrecen algunas de las mismas ventajas que la codificación en vivo (\secref{s:performance-live}).
\cite{Chen2009} ofrece consejos útiles para crear y criticar grabaciones de pantallas y otros videos;
\figref{f:online-screencasting} (de \cite{Chen2009}) reproduce los patrones que presenta el papel
y las relaciones entre ellos.
(También es un buen ejemplo de mapa conceptual (\secref{s:memory-concept-maps}).)

\figpdf{figures/screencast.pdf}{Patrones para grabaciones de pantalla}{f:online-screencasting}

Entonces, ¿qué hace que un video instructivo sea efectivo?
\cite{Guo2014} midió el compromiso, observando cuánto tiempo los alumnos vieron los videos de MOOCs,
y encontró que:

\begin{itemize}

\item
  Los videos más cortos son mucho más atractivos; los videos no deben durar más de seis minutos.

\item
  Una cabeza parlante superpuesta a las diapositivas es más atractiva que una sola voz en off.

\item
  Los videos que se sienten personales pueden ser más atractivos que las grabaciones de estudio de alta calidad,
  por lo que filmar en entornos informales podría funcionar mejor que filmar en un estudio profesional por un costo menor.

\item
  Dibujar en una tableta es más atractivo que las diapositivas de PowerPoint o las grabaciones de pantalla con código,
  aunque no está claro si esto se debe al movimiento y la informalidad
  o porque reduce la cantidad de texto en la pantalla.

\item
  Está bien que los profesores hablen bastante rápido siempre que estén entusiasmados.

\end{itemize}

Una cosa~\cite{Guo2014} que no se abordó es el problema del huevo y la gallina:
¿Los/as estudiantes encuentran cierto tipo de video atractivo porque están acostumbrados?
Entonces, ¿producir más videos de ese tipo aumentará la participación simplemente debido a un ciclo de retroalimentación?
¿O estas recomendaciones reflejan algunos procesos cognitivos más profundos?
Otra cosa que este documento no analizó son los resultados del aprendizaje:
sabemos que las evaluaciones de los/as estudiantes de los cursos no se correlacionan con el aprendizaje~\cite{Star2014,Uttl2017},
y si bien es plausible que los alumnos no aprendan de las cosas que no ven,
queda por demostrar que \emph{aprenden} de las cosas que \emph{ven}.

\begin{aside}{Estoy un poco incómodo/a}
  La investigación de \cite{Guo2014}'s fue aprobada por una junta universitaria de ética en investigación,
  los/as estudiantes cuyos hábitos de visualización fueron monitoreados casi con certeza hicieron clic en ``aceptar''
  en un acuerdo de términos de servicio en algún momento,
  y me alegra tener estos nuevos conocimientos.
  Por otra parte,
  la palabra ``privacidad'' no apareció en el título o en el resumen
  de \emph{ninguno} de las decenas de artículos o posters en la conferencia donde se presentaron estos resultados.
  Si puedo elegir,
  prefiero no saber qué tan comprometidos están los estudiantes
  que fomentar la vigilancia ubicua en el aula.
\end{aside}

Hay muchas formas diferentes de grabar lecciones en video;
para saber cuáles son más eficaces,
\cite{Mull2007a} asignó 364 estudiantes de física de primer año
a los tratamientos multimedia en línea de la Primera y Segunda Ley de Newton en uno de cuatro estilos:

\begin{description}

\item[Exposición:]
  presentación concisa al estilo de una clase magistral.

\item[Exposición extendida:]
  igual que la anterior, pero con información adicional interesante.

\item[Refutación:]
  Exposición con conceptos erróneos comunes explícitamente declarados y refutados.

\item[Diálogo:]
  Discusión estudiante-docente del mismo material que en la Refutación.

\end{description}

Refutación y diálogo produjeron los mayores beneficios de aprendizaje en comparación con la exposición;
los/as estudiantes con pocos conocimientos previos se beneficiaron más,
y aquellos/as con un alto conocimiento previo no estaban en desventaja.
De nuevo,
esto destaca la importancia de abordar directamente los conceptos erróneos de los/as estudiantes.
No sólo, le digas a las personas lo que \emph{es}:
díles también qué \emph{no es} y por qué no.

\seclbl{Modelos híbridos}{s:online-hybrid}
\index{hybrid teaching}

La enseñanza totalmente automatizada es solo una forma de utilizar la web en la enseñanza.
En la práctica,
casi todo el aprendizaje en las sociedades prósperas tiene actualmente un componente en línea,
ya sea oficialmente
o a través de canales de comunicación persona a persona y búsquedas subrepticias de respuestas a preguntas sobre la tarea.
La combinación de instrucción en vivo y automatizada permite a los maestros usar las fortalezas de ambos.
En un aula tradicional,
el/la docente puede responder preguntas de inmediato,
pero los/as estudiantes necesitan días o semanas para recibir comentarios sobre sus ejercicios de programación.
En línea,
un/a estudiante puede tardar más en obtener una respuesta,
pero pueden obtener comentarios inmediatos sobre el código que programó
(al menos para ese tipo de ejercicios podemos calificar automáticamente).

Otra diferencia es que
los ejercicios en línea deben ser más detallados
porque tienen que anticiparse a las preguntas de los/as estudiantes.
Encuentro que las lecciones en persona comienzan con la intersección de lo que todos necesitan saber y se expanden a pedido,
mientras que las lecciones en línea deben incluir la unión de lo que todas las personas necesitan saber
porque el/la docente no está ahí para hacer esta expansión.

En realidad,
la distinción entre online y presencial ahora es menos importante para la mayoría de las personas
que la distinción entre síncrono y asíncrono:
¿Los/as docentes y los/as estudiantes interactúan en tiempo real?
¿O su comunicación se extiende e intercala con otras actividades?
En persona casi siempre será sincrónico,
pero en línea es, cada vez más, una mezcla de ambos:

\begin{quote}

  Creo que nuestros nietos probablemente considerarán, la distinción que hacemos
  entre lo que llamamos el mundo real y lo que ellos consideran simplemente el mundo,
  como la cosa más pintoresca e incomprensible sobre nosotros. \\
  --- William Gibson\index{Gibson, William}

\end{quote}

La implementación más popular de este futuro combinado hoy
es el \gref{g:flipped-classroom}{aula invertida},
en el que los/as estudiantes ven lecciones grabadas por su cuenta
y el tiempo de la clase se utiliza para discutir y resolver conjuntos de problemas.
Descrito originalmente en~\cite{King1993},
la idea se popularizó como parte de la instrucción entre pares (\secref{s:classroom-peer})
y se ha estudiado intensamente durante la última década.
Por ejemplo,
\cite{Camp2016} comparó a los/as estudiantes que tomaron una clase de introducción a la informática en línea
con los que la tomaron en un aula invertida.
La finalización de ejercicios de práctica (sin marcar) se correlacionó con los puntajes de los exámenes para ambos, 
pero la tasa de finalización de los ejercicios de ensayo por parte de los estudiantes en línea
fue significativamente más baja que las tasas de asistencia a clase para los estudiantes presenciales.

Pero si hay grabaciones disponibles,
¿Seguirán asisitiendo los/as estudiantes a las clases para hacer ejercicios de práctica?
\cite{Nord2017} examinó el impacto de las grabaciones en la asistencia a las clases
y al desempeño de los/as estudiantes en diferentes niveles.
En la mayoría de los casos, el estudio no encontró consecuencias negativas al hecho de hacer disponibles las grabaciones;
en particular,
los/as estudiantes no se saltaron las clases cuando las grabaciones estaban disponibles
(al menos, no más de lo que suelen faltar).
Los beneficios de proporcionar grabaciones fueron mayores para los/as estudiantes al principio de sus carreras,
pero disminuye a medida que los/as estudiantes maduran.

Otro modelo híbrido lleva la vida en línea al aula.
Tomar notas juntos es un primer paso (\secref{s:classroom-notetaking});
agrupando respuestas a preguntas de opción múltiple en tiempo real
usando herramientas como \hreffoot{https://www.peardeck.com/}{Pear~Deck}
y \hreffoot{https://socrative.com/}{Socrative} es otra.
Si la clase es pequeña --- digamos, de una docena a quince personas --- también puedes
hacer que todos/as los/as estudiantes se unan a una videoconferencia
para que puedan compartir la pantalla con el/la docente.
Esto les permite mostrar su trabajo (o sus problemas) a toda la clase.
sin tener que conectar su computadora portátil al proyector.
Los/as estudiantes también pueden usar el chat en la videollamada para hacer preguntas para el/la docente;
en mi experiencia,
la mayoría de las preguntas serán respondidas por sus compañeros/as,
y el/la docente puede encargarse del resto cuando lleguen a un momento natural de descanso.
Este modelo ayuda a nivelar ''la cancha´´ para los/as estudiantes remotos:
si alguien no puede asistir a clase por razones de salud
o por compromisos familiares o laborales,
todavía puede participar casi en pie de igualdad,
si todo el mundo está acostumbrado a colaborar online y en tiempo real.

También he impartido clases utilizando instrucción remota en tiempo real,
en el que los/as estudiantes comparten la ubicación en 2 a 6 sitios difrentes, con ayudantes presentes
mientras enseñaba vía videoconferencia (\secref{s:joining-using}).
Esto escala bien,
ahorra en gastos de viaje,
y permite el uso de técnicas como la programación por pares (\secref{s:classroom-pair}).
Lo que \emph{no} funciona, es tener un grupo en persona y uno o más grupos de forma remota:
aún con la mejor voluntad del mundo,
los/as participantes en persona reciben mucha más atención.

\seclbl{Participación on-line}{s:online-engagement}

\cite{Nuth2007} descubrió que hay tres mundos superpuestos en cada aula:
lo público (lo que dice y hace el/la docente),
lo social (interacciones entre pares, entre los estudiantes),
y lo privado (dentro de la cabeza de cada estudiante).
De estos,
el más importante suele ser el social:
los/as estudiantes captan tanto a través de las señales de sus compañeros/as como de la instrucción formal.

Por lo tanto, la clave para hacer efectiva cualquier forma de enseñanza en línea es
facilitar las interacciones entre pares.
Para ayudar a lograr esto,
los cursos casi siempre tienen algún tipo de foro de discusión.
\cite{Mill2016a} observó que los/as estudiantes los utilizan de formas muy diferentes:

\begin{quote}

  {\ldots}es muy poco probable que las personas procrastinadoras participen en foros de discusión en línea,
  y esta participación reducida,
  a su vez,
  se correlaciona con peores calificaciones.
  Una posible explicación de esta correlación es que
  las personas procrastinadoras son especialmente reacias a unirse, una vez que la discusión está en curso,
  quizás porque les preocupa ser percibidas como recién llegadas en una conversación establecida.
  Esta aversión a participar tarde
  provoca que se pierdan los beneficios importantes de aprendizaje y motivación de la interacción entre pares.

\end{quote}

\cite{Vell2017} analiza publicaciones en foros de discusión de 395 estudiantes de CS2 en dos universidades
dividiéndolos en cuatro categorías:

\begin{description}

\item[Activos/as:]
  solicitud de ayuda que no muestra razonamiento
  y no muestra lo que el/la estudiante ya ha probado o ya sabe.

\item[Constructivo/a:]
  refleja el razonamiento de los/as estudiantes
  o intenta construir una solución al problema.

\item[Logístico/a:]
  políticas del curso, horarios, envío de tareas, etc.
  
\item[Aclaración de contenido:]
  solicitud de información adicional
  que no revela el propio pensamiento del/la estudiante.

\end{description}

Descubrieron que dominaban las preguntas constructivas y logísticas,
y que las preguntas constructivas se correlacionan con las calificaciones.
También encontraron que los/as estudiantes rara vez hacen más de una pregunta activa en un curso,
y que estas \emph{no} se correlacionan con las calificaciones.
Si bien esto es decepcionante,
saberlo ayuda a establecer las expectativas de los/as docentes:
si bien es posible que todos deseemos que nuestros cursos tengan comunidades en línea animadas,
tenemos que aceptar que la mayoría no lo hará,
o que la mayor parte de la discusión de estudiante a estudiante se llevará a cabo
a través de canales que ya están usando
de la que no podemos ser parte.

\begin{aside}{Cooperación}
  \cite{Gull2004} describe un concurso de codificación en línea que combina colaboración y competencia.
  El concurso comienza cuando se publica una descripción del problema junto con una solución correcta pero ineficiente.
  Cuando acaba,
  la persona ganadora es quien ha hecho la mayor contribución global
  para mejorar el rendimiento de la solución global.
  Todas las presentaciones están abiertas,
  para que los participantes puedan ver el trabajo de los demás y tomar prestadas ideas entre ellos.
  Como muestra el trabajo,
  la solución final es casi siempre un híbrido que utiliza ideas de muchas personas.

  \cite{Batt2018} describió una variación a pequeña escala de esto en una clase de introducción a la informática.
  En la etapa uno,
  cada estudiante presentó un proyecto de programación de forma individual.
  En la etapa dos,
  los/as estudiantes trabajaron en parejas para crear una solución mejorada al mismo problema.
  La evaluación indica que los proyectos de dos etapas tienden a mejorar la comprensión de los/as estudiantes
  y que disfrutaron del proceso.
  Proyectos como estos, no solo mejoran la participación,
  también brindan a los/as estudiantes más experiencia sobre la base del código de otra persona.
\end{aside}

La discusión no es la única forma de lograr que los alumnos trabajen juntos en línea.
\cite{Pare2008} y~\cite{Kulk2013} reportan experimentos
en el que los alumnos califican el trabajo de los demás,
y las calificaciones que asignan se comparan con
las calificaciones otorgadas por profesores/as, asistentes de posgrado u otros expertos/as.
Ambos trabajos encontraron que las calificaciones asignadas por los/as estudiantes coincidían con las calificaciones asignadas por los/as expertos/as
con tanta frecuencia como las calificaciones de los/as expertos/as coincidían entre sí,
y que unos sencillos pasos
(como filtrar respuestas obviamente no consideradas o estructurar rúbricas)
disminuyó aún más el desacuerdo.
Como se discute en \secref{s:individual-peer},
la colusión y el sesgo \emph{no} son factores importantes en la calificación de pares.

\begin{aside}{Confía, pero educa}
  The most common way to measure the validity of feedback
  is to compare learners' grades to experts' grades,
  but calibrated peer review (\secref{s:individual-peer}) can be equally effective.
  \index{calibrated peer review}
  Before asking learners to grade each others' work,
  they are asked to grade samples and compare their results with the grades assigned by the teacher.
  Once the two align,
  the learner is allowed to start giving grades to peers.
  Given that critical reading is an effective way to learn,
  this result may point to a future in which learners use technology to make judgments,
  rather than being judged by technology.
\end{aside}

One technique we will definitely see more of in coming years is
online streaming of live coding sessions~\cite{Raj2018,Haar2017}.
This has most of the benefits discussed in \secref{s:performance-live},
and when combined with collaborative note-taking (\secref{s:classroom-notetaking})
it can be a close approximation to an in-class experience.

Looking even further ahead,
\cite{Ijss2000} identified four levels of online presence,
from realism (we can't tell the difference)
through immersion (we forget the difference)
and involvement (we're engaged but aware of the difference)
to suspension of disbelief (we are doing most of the work).
Crucially,
they distinguish physical presence,
which is the sense of actually being somewhere,
and social presence, which is the sense of being with others.
The latter is more important in most learning situations,
and again,
we can foster it by using learners' everyday technology in the classroom.
For example,
\cite{Deb2018} found that real-time feedback on in-class exercises
using learners' own mobile devices
improved concept retention and learner engagement while reducing failure rates.

Online and asynchronous teaching are both still in their infancy.
Centralized MOOCs may prove to be an evolutionary dead end,
but there are still many other promising models to explore.
In particular,
\cite{Broo2016} describes fifty ways that groups can discuss things productively,
only a handful of which are widely known or implemented online.
If we go where our learners are technologically
rather than requiring them to come to us,
we may wind up learning as much as they do.

\seclbl{Exercises}{s:online-exercises}

\exercise{Two-Way Video}{pairs}{10}

Record a 2--3 minute video of yourself doing something,
then swap machines with a partner
so that each of you can watch the other's video at 4x speed.
How easy is it to follow what's going on?
What if anything did you miss?

\exercise{Viewpoints}{individual}{10}

According to~\cite{Irib2009},
different disciplines focus on different factors
affecting the success or otherwise of online communities:

\begin{description}

\item[Business:]
  customer loyalty, brand management, extrinsic motivation.

\item[Psychology:]
  sense of community, intrinsic motivation.

\item[Sociology:]
  group identity, physical community, social capital, collective action.

\item[Computer Science:]
  technological implementation.

\end{description}

Which of these perspectives most closely corresponds to your own?
Which are you least aligned with?

\exercise{Helping or Harming}{small groups}{30}

\hreffoot{https://www.nytimes.com/2018/01/19/business/online-courses-are-harming-the-students-who-need-the-most-help.html}{Susan Dynarski's article in the \emph{New York Times}}
explains how and why schools are putting students who fail in-person courses into online courses,
and how this sets them up for even further failure.
Read the article and then:

\begin{enumerate}

\item
  In small groups,
  come up with 2--3 things that schools could do to compensate for these negative effects
  and create rough estimates of their per-learner costs.

\item
  Compare your suggestions and costs with those of other groups.
  How many full-time teaching positions do you think would have to be cut
  in order to free up resources to implement the most popular ideas for a hundred learners?

\item
  As a class,
  do you think that would be a net benefit for the learners or not?

\end{enumerate}

Budgeting exercises like this are a good way to tell who's serious about educational change.
Everyone can think of things they'd like to do;
far fewer are willing to talk about the tradeoffs needed to make change happen.
