\chapter{Conocimiento de la pedagogía del contenido}\label{s:pck}

Cada docente necesita tres cosas:

\begin{description}

\item[\gref{g:content-knowledge}{conocimiento del contenido}],
  por ejemplo, como programar;

\item[\gref{g:general-pedagogical-knowledge}{conocimiento pedagógico general}],
  por ejemplo, comprensión de la psicología del aprendizaje;
  y

\item[\gref{g:pedagogical-content-knowledge}{conocimiento de la pedagogía del contenido}],
  que es el conocimiento específico acerca de cómo enseñar un concepto particular a un público en particular.
  En informática,
  el conocimiento de la pedagogía del contenido incluye qué ejemplos usar cuando se enseña, cómo se incluyen parámetros en una función o qué conceptos erróneos sobre etiquetas HTML anidadas son los más comunes.
\end{description}

Podemos agregar conocimiento técnico a este conjunto~\cite{Koeh2013},
pero eso no cambia el punto clave: no es suficiente saber sobre el tema y cómo enseñar---tienes que saber cómo enseñar ese tema en particular~\cite{Maye2004}.
Este capítulo resume algunos resultados de investigaciones sobre enseñanza de informática para añadir a tu colección sobre el conocimiento de la pedagogía del contenido.

Como con toda investigación,
se requiere cierta precaución al interpretar los resultados:

\begin{description}

\item[Las teorías cambian a medida que se obtienen más datos.]
  La investigación sobre educación en informática es una disciplina nueva:\index{investigación sobre educación en computación}
  la Sociedad Americana de Educación en Ingeniería fue fundada en 1893 y el Consejo Nacional de Docentes de Matemática en 1920, pero la Asociación de Docentes de Informática no se creó hasta 2005.
  Mientras que existe un flujo constante de nuevo conocimiento en conferencias como \hreffoot{http://sigcse.org/}{SIGCSE},
  \hreffoot{http://iticse.acm.org/}{ITiCSE}
  e \hreffoot{https://icer.hosting.acm.org}{ICER},
  simplemente no sabemos tanto sobre cómo aprender a programar como sí sabemos sobre aprender a leer, jugar un deporte o resolver cálculos simples.
 
\item[La mayoría de las personas en estos estudios
  viven en sociedades occidentales, democráticas, industrializadas y con alto nivel de riqueza y educación] y se los denomina WEIRD\index{WEIRD} por \emph{Western, Education, Industrialized, Rich, and Democratic en inglés}, tal como señala ~\cite{Henr2010}.
  Además,
  son principalmente jóvenes que estudian en instituciones educativas, ya que es la población a la que la mayoría de las personas que investigan tienen fácil acceso.
  Sabemos mucho menos sobre adultos, grupos marginados y estudiantes en ambientes educativos flexibles (estudiantes free-range), así como sobre \grefdex{g:end-user-programmer}{usuarias/os finales programadoras/es}{usuario/a final programadora/or},
  aún cuando son la mayoría.

\end{description}

Si esto fuera un ensayo académico, empezaría la mayoría de oraciones con frases como,
``Algunas investigaciones parece indicar que{\ldots}''
Pero dado que las/los docentes reales que enseñan en aulas reales tienen que tomar decisiones independientemente de si las investigaciones tienen respuestas claras o no, este capítulo presenta las mejores conjeturas prácticas en lugar de sutiles posibilidades.

\begin{aside}{Jerga}
  Como cualquier especialidad,
  la investigación sobre educación en informática tiene jerga.
  \gref{g:cs1}{CS1} refiere a un curso introductorio de un semestre de duración, donde los estudiantes aprenden variables, bucles y funciones por primera vez, mientras que \gref{g:cs2}{CS2} refiere a un segundo curso que cubre las estructuras de datos básicas como pilas y colas, 
  y \gref{g:cs0}{CS0} se refiere a un curso introductorio para personas sin experiencia previa y que no tienen intención de continuar con computación de inmediato.
  Las definiciones completas de estos términos se encuentran en los \hreffoot{https://www.acm.org/education/curricula-recommendations}{lineamientos del programa \emph{ACM} (
Association for Computing Machinery, por sus siglas en inglés)}.
\end{aside}

\seclbl{¿Qué les estamos enseñando ahora?}{s:pck-now}

Se sabe muy poco sobre qué se enseña en entrenamientos de programación intensivo e iniciativas free-range, en parte porque muchas personas son reticentes a compartir los programas.
Sabemos más sobre lo que  se enseña en instituciones~\cite{Luxt2017}:

\begin{longtable}{llll}
\textbf{Temas}            & \textbf{\%}    & \textbf{Temas}        & \textbf{\%} \\
Proceso de programación         & 87\%        & Tipos de datos                & 23\% \\
Pensamiento abstracto para programación    & 63\%        & Entrada/Salida                  & 17\% \\
Estructuras de datos              & 40\%        & Librerías                     & 15\% \\
Conceptos orientados a objetos        & 36\%        & Variables y asignación       & 14\% \\
Estructuras de control              & 33\%        & Recursión             & 10\% \\
Operaciones y funciones         & 26\%        & Punteros y administración de memoria    &  5\%
\end{longtable}

Títulos de temas de alto nivel como estos pueden esconder una gran cantidad de fallas.
Un resultado más tangible surge de \cite{Rich2017},
quienes revisaron cien artículos y encontraron trayectorias de aprendizaje para clases de computación en escuelas primarias y secundarias.
Sus resultados para la secuenciación, la repetición y los condicionales son esencialmente mapas conceptuales colectivos
que combinan y racionalizan el pensamiento implícito y explícito de gran cantidad de docentes
(\figref{f:pck-trajectory}).

\newpage

\figpdfhere{figures/conditionals.pdf}{Trayectoria de aprendizaje para condiciones (extraído  de~\cite{Rich2017})}{f:pck-trajectory}

\seclbl{¿Cuánto están aprendiendo?}{s:pck-learning}

Puede haber un mundo de distancia entre lo que enseñan las/los docentes y cuánto aprenden sus estudiantes.
Para explorar cuánto se aprende, debemos usar otras medidas o hacer estudios directos.
Tomando el enfoque tradicional, aproximadamente dos tercios de las/los estudiantes de nivel superior aprueban su primer curso de informática,
con algunas variaciones dependiendo del tamaño de la clase,
pero sin diferencias significativas a lo largo del tiempo o basadas en el lenguaje~\cite{Benn2007a,Wats2014}.

¿Cómo afecta la experiencia previa a estos resultados?\index{efecto de la experiencia previa}
Para responder esto,
\cite{Wilc2018} compararon el desempeño y la confianza de personas novatas
con y sin experiencia previa en programación en CS1 y CS2 (ver más abajo).
Encontraron que personas novatas con experiencia previa superaron a personas sin experiencia en un 10\% en CS1,
pero esas diferencias desaparecieron hacia el final de CS2.
También encontraron que las mujeres con experiencia previa superaron a sus pares masculinos en todas las áreas,
pero siempre tenían menos confianza en sus habilidades (\secref{s:motivation-inclusivity}).

En cuanto a estudios sobre cuánto aprenden las personas novatas,
\cite{McCr2001} presentaron un estudio internacional en múltiples espacios que luego fue replicado por~\cite{Utti2013}.
De acuerdo al primer estudio,
``{\ldots}los decepcionantes resultados sugieren que
muchas/os estudiantes no saben cómo programar al final de los cursos introductorios''.
Más específicamente,
``Para una muestra combinada de 216 estudiantes de cuatro universidades,
la puntuación media fue de 22,89 sobre 110 puntos en los criterios generales de evaluación desarrollados para este estudio.''
Este resultado puede hablar tanto de las expectativas de docentes como de la habilidad de las/los estudiantes,
pero de cualquier manera,
nuestra primera recomendación es \recommendation{mide y haz un seguimiento de los resultados}
de tal manera que se puedan comparar a través del tiempo para que puedas saber si tus lecciones se están volviendo más o menos efectivas.

\seclbl{¿Qué conceptos erróneos tienen las personas novatas?}{s:pck-misunderstand}

El \chapref{s:models} explicó por qué aclarar los conceptos erróneos a las personas novatas es tan importante como enseñarles
estrategias para resolver problemas.
La mayor confusión de las personas novatas ---a veces llamada el \emph{``superbug''} o ``supererror'' en programación ---es\index{superbug} \index{supererror}
la creencia de que las computadoras entienden lo que las personas quieren decir de la misma manera que cualquier ser humano lo haría~\cite{Pea1986}.
Nuestra segunda recomendación es entonces \recommendation{enseña a las personas novatas que las computadoras no entienden los programas},
es decir, que llamar a una variable ``precio'' no garantiza que su valor sea realmente un precio.

\cite{Sorv2018} muestra más de cuarenta conceptos erróneos que las/los docentes también pueden intentar aclarar,
muchos de las cuales también se discuten en el estudio de~\cite{Qian2017}.
Una es la creencia de que las variables en los programas funcionan de la misma manera que en planillas de cálculo,
es decir, que luego de ejecutar:

\begin{minted}{text}
nota = 65
total = nota + 10
nota = 80
print(total)
\end{minted}

\noindent
el valor de \texttt{total} será 90 en vez de 75~\cite{Kohn2017}.
Este es un ejemplo de la forma en que las personas novatas construyen un modelo
mental plausible pero erróneo haciendo analogías. Otras confusiones incluyen:


\begin{itemize}

\item
  Una variable guarda la historia de los valores que le fueron asignados,
  es decir, recuerda cuál solía ser su valor.

\item
  Está garantizado que dos objetos con el mismo valor para sus atributos \texttt{nombre}
  o  identificación \texttt{id} son el mismo objeto.   
 

\item
  Las funciones son ejecutadas cuando se las define,
  o son ejecutadas en el orden en que son definidas.

\item
  La condición de un bucle \texttt{while} se evalúa constantemente,
  y el bucle se detiene tan pronto como se vuelve falso.
  Por el contrario,
  la condición de las sentencias\texttt{if} es constantemente evaluada,
  y sus declaraciones son ejecutadas tan pronto como la condición se vuelve verdadera,
 independientemente de dónde se encuentre el flujo de control en ese momento.

\item
  Las asignaciones modifican valores,
  es decir, después de \texttt{a\ =\ b}, la variable \texttt{b} queda vacía.

\end{itemize}

\seclbl{¿Qué errores cometen las personas novatas?}{s:pck-mistakes}

Los errores que cometen las personas novatas nos indican qué priorizar cuando enseñamos,
pero resulta que la mayoría de las personas que enseñan no saben cuán comunes son los diferentes tipos de errores.
El estudio más importante es el de \cite{Brow2017},
que encontró que la falta de paréntesis y comillas son el tipo de error más común en programas Java escritos por personas novatas,
además de tratarse el error más sencillo de resolver. Por otro lado,
algunos errores (como poner la condición de un \texttt{if} en \texttt{\{\}} en vez de \texttt{()})
se cometen solo una vez.
No extraña que los errores que producen problemas de compilación son resueltos mucho más rápido
que aquellos que no lo hacen.
Algunos errores, en cambio, se repiten muchas veces, como llamar métodos con los argumentos incorrectos
(p. ej.\ pasar una cadena  de caracteres en vez de un número entero).

\begin{aside}{No es correcto versus No está resuelto}
  Una dificultad en una investigación como esta es distinguir los errores del trabajo en proceso.
  Por ejemplo,
  una estructura \texttt{if} vacía o un método que se define pero aún no se ha usado
  puede ser señal de que el código está incompleto más que señal de un error.
\end{aside}

\cite{Brow2017} también comparó los errores que las personas novatas realmente cometen
con los que sus docentes pensaron que cometieron.
Descubrieron que,
``{\ldots}las/los docentes llegaron a un escaso consenso sobre cuáles son los errores más frecuentes.
Sus calificaciones tenían solo una correspondencia moderada con la de las/los estudiantes en los{\ldots}datos
y la experiencia de las/los docentes no tuvo ningún efecto en este nivel de acuerdo.''
Por ejemplo,
confundir \texttt{=} (asignación) con \texttt{==} (igualdad)
no eran tan común como la mayoría de las/los docentes creían.

\begin{aside}{No solo por el código}
  \cite{Park2015} recopiló datos de un editor HTML en línea durante un curso introductorio de desarrollo web
  y encontró que la mayoría de las/los estudiantes cometieron errores de sintaxis que permanecieron sin ser resueltos por semanas durante el curso.
  El 20\% de esos errores estaban relacionados con reglas relativamente complejas
  que indican \emph{cuándo} es válido que los elementos HTML estén anidados entre sí,
  pero el 35\% estaba relacionado a sintaxis de etiquetas más simples que determinan \emph{cómo} los elementos HTML están anidados.
  La tendencia de muchas/os docentes a decir
  ``Pero las reglas son simples,''
  es un buen ejemplo del punto ciego de las personas expertas que se analiza en el \chapref{s:memory}{\ldots}
\end{aside}

\seclbl{¿Cómo programan las personas novatas?}{s:pck-programming}

\cite{Solo1984,Solo1986} son trabajos pioneros en la exploración de las estrategias de programación de personas novatas y expertas.
El hallazgo clave es que las personas expertas saben al mismo tiempo el ``qué'' y el ``cómo,''
es decir, entienden qué poner en los programas
\emph{y} tienen un conjunto de patrones o plan para guiar su construcción.\index{patrones de programa}
Las personas principiantes no tienen ninguna de las dos cosas,
pero la mayoría de las/los docentes se enfocan únicamente en lo primero,
a pesar de que los errores son usualmente causados por no tener una estrategia para resolver el problema,
en lugar de falta de conocimiento sobre el lenguaje.
Un trabajo reciente mostró la efectividad de enseñar cuatro habilidades distintas en un orden específico~\cite{Xie2019}:

\begin{longtable}{lll}
			& \textbf{semántica del código}				& \textbf{plantillas asociadas a objetivos} \\
\textbf{leyendo}	& 1.\ leer el código y predecir su comportamiento	& 3.\ reconocer plantillas y sus usos \\
\textbf{escribiendo}	& 2.\ escribir la sintaxis correcta			& 4.\ usar las plantillas para alcanzar objetivos
\end{longtable}

Por lo tanto, nuestras siguientes recomendaciones son:
\recommendation{haz que tus estudiantes lean código, luego lo modifiquen, luego lo escriban} y además  \recommendation{preséntales explícitamente patrones comunes y haz que practiquen usándolos}.
\cite{Mull2007b} es uno de los tantos estudios que muestran los beneficios de enseñar patrones comunes de manera explícita. Descomponer los problemas
en patrones comunes crea oportunidades naturales para crear y etiquetar sub-objetivos~\cite{Marg2012,Marg2016}.\index{labeled subgoals}

\seclbl{¿Cómo identifican y resuelven errores las personas novatas??}{s:pck-debug}

Una década atrás,
\cite{McCa2008} escribieron:
``Es sorprendente el poco espacio que se dedica a los errores y cómo resolverlos
en la mayoría de los libros introductorios de programación.''
Poco ha cambiado desde entonces:
hay cientos de libros sobre compiladores y sistemas operativos,
pero solo unos pocos sobre depuración de errores \index{depuración de errores}
y nunca he visto un curso de pregrado dedicado a este tema.

\cite{List2004,List2009} encontraron que a muchas personas novatas les cuesta predecir el resultado de pocas líneas de código
y seleccionar la salida correcta del código a partir de un conjunto de posibilidades
cuando les decían lo que se suponía que el código debía hacer. Más recientemente,
\cite{Harr2018} encontraron que la brecha entre poder entender el código y poder escribirlo se ha cerrado en gran medida gracias a CS2,
pero que las personas novatas que aún tienen esa brecha es probable que les vaya mal.

Nuestra quinta recomendación es entonces \recommendation{enséñales explícitamente a las personas principiantes cómo depurar errores}.
\cite{Fitz2008,Murp2008} encontraron que las personas que pueden resolver errores son buenas programando,
pero no todas las personas que son buenas programando son buenas resolviendo errores.
Aquellas personas que usaron un depurador de errores simbólico para recorrer sus programas,
rastrearon su ejecución manualmente,
escribieron pruebas
y releyeron las especificaciones frecuentemente,
todos hábitos que se pueden enseñar.
Sin embargo,
rastrear la ejecución paso a paso a veces se usa de manera ineficaz:
por ejemplo,
una persona novata podría poner la misma declaración \texttt{print} en ambas partes de una estructura \texttt{if}-\texttt{else}.
Las personas novatas también comentarían líneas de código que en realidad eran correctas al tratar de aislar un problema.
Las/los docentes pueden cometer estos dos errores deliberadamente,
señalarlos
y corregirlos para ayudar a las personas novatas a aprender cómo resolverlos.

Enseñar a principiantes cómo depurar errores también puede ayudar a que las clases sean más simples de manejar.
\cite{Alqa2017} encontró que estudiantes con mayor experiencia resolvieron problemas de depuración de errores significativamente más rápido,
pero que los tiempos variaron ampliamente:
4--10 minutos fue el rango típico para ejercicios individuales,
lo cual significa que algunas/os estudiantes necesitan entre 2 y 3 veces más tiempo que otras/os para resolver el mismo ejercicio.
Para ayudar a que el progreso de todo el grupo sea más uniforme,
es conveniente enseñarles a quienes resuelven los ejercicios más lentamente qué están haciendo las personas más rápidas.

La depuración de errores depende de ser capaz de leer el código,
algo que múltiples estudios han demostrado que es la forma más efectiva de encontrar errores~\cite{Basi1987,Keme2009,Bacc2013}.
La rúbrica de calidad de código desarrollada por~\cite{Steg2014,Steg2016a}
es una buena lista de verificación de elementos a revisar,
aunque se presenta mejor de a partes  en lugar  de toda junta.

Hacer que las/los estudiantes lean código y resuman su comportamiento es un buen ejercicio (\secref{s:individual-strategies}),
pero frecuentemente lleva mucho tiempo practicarlo en clase.
Hacer que las/los estudiantes predigan la salida de un programa justo antes de que sea ejecutado,
por otro lado,
ayuda a reforzar el aprendizaje (\secref{s:classroom-practices})
y además les da la oportunidad ideal para hacer el tipo de preguntas ``qué pasaría si''.
Las/los docentes o estudiantes pueden además rastrear los cambios en las variables a medida que avanzan,
algo que \cite{Cunn2017} encontró efectivo (\secref{s:exercises-tracing}).

\seclbl{¿Qué pasa con las pruebas (\emph{tests})?}{s:pck-testing}
\index{testing (software)}
\index{pruebas (software)}


Las personas novatas en programación parecen tan reacias a testear software como aquellas personas profesionales.
No hay duda de que hacerlo es valioso---\cite{Cart2017} encontró que
las personas novatas con un alto rendimiento dedican mucho tiempo a testear el código,
mientras que las personas novatas con un bajo rendimiento dedican mucho más tiempo a trabajar en el código con errores---y muchas/os docentes
requieren que las/los estudiantes escriban tests en las actividades.
¿Pero qué tan bien lo hacen?
Una posible respuesta proviene de~\cite{Bria2015},
quienes calificaron el código de las/los estudiantes según la cantidad de situaciones, definidas por la/el docente, que pasaban correctamente las pruebas de los programas.
A la inversa, también calificaron
los casos de prueba escritos por estudiantes de acuerdo con la cantidad de errores detectados que habían sido sembrados deliberadamente.
Los autores encontraron que las pruebas de personas novatas usualmente tienen una baja cobertura (es decir, no testean gran parte del código)
y que usualmente testean muchas cosas al mismo tiempo, lo que dificulta determinar las causas de los errores.

Otra respuesta proviene de ~\cite{Edwa2014b},
quienes examinaron todos los errores en los envíos de códigos realizados por personas novatas
y a su vez identificaron aquellos errores detectados por el conjunto de pruebas de personas novatas.
En este trabajo se encontró que las pruebas de personas novatas solo detectaban un promedio de 13,6\% de los errores presentes en la totalidad de los programas.
Además,
el 90\% de las pruebas de las personas novatas fueron muy parecidas,
lo que indica que las personas novatas escriben pruebas principalmente para confirmar que el código está haciendo lo que se supone que debe hacer
en lugar de encontrar situaciones en las que no ocurre esto.

Un enfoque para enseñar mejores prácticas para generar pruebas es
definir un problema de programación proporcionando un conjunto de pruebas que deben ser pasadas
en lugar de una descripción (\secref{s:exercises-classics}).
Sin embargo,
antes de hacerlo
tómate un momento para ver cuántas pruebas has escrito en tu propio código recientemente
y luego decide si estás enseñando lo que crees que las personas deberían hacer,
 o lo que ellos (y tú) realmente hacen.

\seclbl{¿Importa el lenguaje?}{s:pck-language}
\index{programación basada en bloques}

La respuesta corta es ``sí'':
las personas novatas aprenden a programar más rápido y aprenden más
usando herramientas basadas en bloques como Scratch (\figref{f:pck-scratch})~\cite{Wein2017}.\index{Scratch}
Una razón es que esos sistemas basados en bloques reducen la carga cognitiva al eliminar la posibilidad de cometer errores de sintaxis.
Otra razón es que esas interfaces de bloques promueven la exploración de una manera que el texto no hace:
como todas las buenas herramientas,
Scratch se puede aprender accidentalmente ~\cite{Malo2010}.

Pero, ¿qué sucede \emph{luego} de los bloques?
\cite{Chen2018} encontró que las personas cuyo primer lenguaje de programación fue gráfico,
tenían calificaciones más altas en cursos introductorios de programación
en comparación con las personas cuyo primer lenguaje era textual
cuando los lenguajes se aprendieron durante o antes de los primeros años de adolescencia.
Nuestra sexta recomendación es
\recommendation{preséntales interfaces basadas bloques a niñas/os y adolescentes}
antes de avanzar a sistemas basados en texto.
La calificación de edad corresponde a que Scratch deliberadamente parece destinado a usuarias/os jóvenes,
y puede ser difícil convencer a personas adultas de que lo tomen en serio.

\figimg{figures/scratch-program.png}{Scratch}{f:pck-scratch}

Scratch probablemente ha sido estudiado más que cualquier otra herramienta de programación.
Un ejemplo es ~\cite{Aiva2016},
quienes analizaron más de 250.000 proyectos de Scratch
y encontraron (entre otras cosas) que alrededor del 28\% de esos proyectos tenían algunos bloques que nunca eran usados o activados.
Como en la sección anterior sobre los programas en Java incompletos versus incorrectos,
las autoras plantean la hipótesis de que las/los usuarias/os pueden estar utilizando estos bloques como borrador
para hacer un seguimiento de las partes del código que no quieren eliminar (todavía).
Otros ejemplos son~\cite{Grov2017,Mlad2017},
que estudiaron a personas novatas aprendiendo sobre bucles en Scratch, Logo y Python.\index{Scratch}\index{Python}
Encontraron que los conceptos erróneos sobre bucles se minimizan cuando se usa un lenguaje basado en bloques
en lugar de un lenguaje basado en texto.
Además,
a medida que las tareas se vuelven más complejas (como el uso de bucles anidados)
las diferencias son mayores.

\begin{aside}{Más difícil de lo necesario}
  Las personas que crean lenguajes de programación hacen que esos lenguajes sean más difíciles de aprender al no realizar pruebas básicas de usabilidad.
  Por ejemplo,
  \cite{Stef2013} encontraron que,
  ``{\ldots}las tres palabras más comunes para generar bucles en ciencias de la computación,
  \texttt{for}, \texttt{while} y \texttt{foreach},
  fueron calificadas como las tres opciones más anti-intuitivas por personas que no programan.''
  Su trabajo muestra que sintaxis al estilo de C (como se usa en Java y Perl)
  son tanto más difíciles de aprender para personas novatas como una sintaxis diseñada de manera aleatoria,
  pero que la sintaxis de lenguajes como Python y Ruby\index{Python}\index{Ruby}
  es significativamente más fácil de aprender,
  y la sintaxis de un lenguaje cuyas características son probadas antes de incorporarse es aún más fácil.
  \cite{Stef2017} es un útil y breve resumen de lo que realmente sabemos sobre el diseño de lenguajes de programación
  y por qué creemos que es cierto,
  mientras que \cite{Guzd2016} expone cinco principios que los lenguajes de programación para estudiantes deberían seguir.
\end{aside}

\subsection*{Programación orientada a objetos}
\index{object-oriented programming}

Los objetos y clases son herramientas poderosas para personas con experiencia en programación,
y muchas/os docentes promueven un enfoque de \gref{g:objects-first}{primero objetos} para enseñar a programar
(aunque a veces no concuerdan exactamente con lo que eso significa~\cite{Benn2007b}).
\cite{Sorv2014} describen y motivan este enfoque,
y~\cite{Koll2015} describe tres generaciones de herramientas
designadas para apoyar a personas novatas para que programen en ambientes orientados a objetos.

Introducir objetos temprano tiene algunos desafíos.
\cite{Mill2016b} encontró que la mayoría de las personas novatas que usan Python\index{Python}
tenían problemas para entender \texttt{self}
(que refiere al objeto actual):
las personas lo omitieron en las definiciones de métodos,
no lo usaron cuando hacían referencia a los atributos del objeto,
o tuvieron ambas dificultades.
\cite{Rago2017} encontró algo similar en estudiantes de nivel secundario,
y además encontró que docentes de secundaria usualmente tampoco tenían claro el concepto.
En resumen,
recomendamos que, como docente, \recommendation{comiences con funciones en vez de objetos},
es decir, que no les enseñes a tus estudiantes cómo definir clases
hasta que comprendan las estructuras básicas de control y los tipos de datos.

\subsection*{Declaración de tipos}
\index{declaración de tipos}

Las personas que programan han discutido durante décadas acerca de si los tipos de datos de variables deberían ser declarados o no,
usualmente basándose en su experiencia personal como profesionales
en lugar de tener en cuenta el tipo de datos.
\cite{Endr2014,Fisc2015} encontraron que pedir a personas novatas que declaren los tipos de variables suma cierta complejidad a los programas,
pero se compensa rápidamente al actuar como documentación del uso de los métodos ---en particular,
evita preguntas sobre qué métodos están disponibles y cómo usarlos.

\subsection*{Nombrando variables}
\index{nombrar variables}

\cite{Kern1999} escribieron:
``A menudo se alienta a quienes programan a usar nombres de variables largos independientemente del contexto.
Esto es un error: la claridad a menudo se logra siendo breves.''
Muchas/os programadoras/es creen esto,
pero \cite{Hofm2017} encontraron que usar palabras completas en nombres de variables
 condujo, en promedio, a una comprensión un 19\% más rápida en comparación con letras y abreviaturas.
En contraste,
\cite{Beni2017} encontraron que el uso de nombres de variables de una sola letra no afectaba la capacidad de las personas principiantes de modificar el código.
Esto puede deberse a que sus programas son más cortos que los de profesionales
o porque algunos nombres de variables de una sola letra tienen tipos y significados implícitos.
Por ejemplo,
la mayoría de las personas que programan asumen que \texttt{i}, \texttt{j} y \texttt{n} son enteros
y que \texttt{s} es una cadena de caracteres,
mientras que \texttt{x}, \texttt{y} y \texttt{z} son números de punto flotante o enteros
más o menos equivalentemente.

¿Qué tan importante es esto?
\cite{Bink2012} reportaron que leer y comprender el código es fundamentalmente distinto de leer prosa:
``{\ldots}la estructura más formal y la sintaxis del código fuente
permite a quienes programan asimilar y comprender partes del código rápidamente independientemente del estilo.
En particular{\ldots}las guías y los planes de programación juegan un papel importante en la comprensión.''
También encontró que a las/los desarrolladoras/es con experiencia no les afecta el estilo del identificador,
entonces nuestra recomendación es utilizar un estilo coherente en todos los ejemplos.
Dado que la mayoría de los lenguajes tienen guías de estilo
(por ejemplo,\ \hreffoot{https://www.python.org/dev/peps/pep-0008/}{PEP 8} para Python)
y herramientas para verificar que el código siga esas pautas,
nuestra recomendación completa es
\recommendation{utiliza herramientas para garantizar que todos los ejemplos de código se adhieran a un estilo consistente}.

\seclbl{¿Ayudan mejores mensajes de error?}{s:pck-error}
\index{mensajes de error}

Los mensajes de error incomprensibles son una fuente importante de frustración para personas novatas
(y también para personas experimentadas ).
Por lo tanto, en varias investigaciones se exploró si mejores mensajes de error ayudan a evitar esto.
Por ejemplo,
\cite{Beck2016} reescribieron algunos de los mensajes del compilador de Java para que en lugar de:

\begin{minted}{text}
C:\stj\Hola.java:2: error: cannot find symbol
        public static void main(string[ ] args)
^
1 error
Process terminated ... there were problems.
\end{minted}

\noindent
las personas vean:

\begin{minted}{text}
Looks like a problem on line number 2.
If ``string'' refers to a datatype, capitalize the 's'!
\end{minted}

\noindent
Efectivamente,
las personas novatas que recibieron estos mensajes repitieron menos errores y cometieron menos errores en general.

\cite{Bari2017} fueron más allá y usaron seguimiento ocular para mostrar que
a pesar de las quejas de las personas que escriben los compiladores,
las personas realmente leen los mensajes de error---de hecho, pasan del 13 al 25\% de su tiempo haciendo eso.
Sin embargo,
leer mensajes de error resulta ser tan difícil como leer el código fuente,
y la dificultad en leer los mensajes de error predice fuertemente el desempeño de la tarea.
Por lo tanto, al enseñar
\recommendation{muéstrales a tus estudiantes cómo leer e interpretar mensajes de error}.
\cite{Marc2011} tiene una rúbrica con respuestas a los mensajes de error que puede ser útil para calificar ese tipo de ejercicios.

\subsection*{¿Ayuda la visualización?}
\index{visualizar un programa}

Visualizar la ejecución de un programa es una idea siempre popular,
y herramientas como el tutor de Python en línea~\cite{Guo2013}
y \hreffoot{http://latentflip.com/loupe/}{Loupe}
(que muestra cómo funciona un bucle de eventos de JavaScript)
son útiles para enseñar.
Sin embargo,
las personas aprenden más al construir visualizaciones
que al ver visualizaciones construidas por otras personas \cite{Stas1998,Ceti2016},
entonces, ¿las visualizaciones realmente ayudan al aprendizaje?

Para responder esto,
\cite{Cunn2017} replicaron un estudio previo sobre los tipos de esquemas que realizan las/los estudiantes cuando rastrean la ejecución del código.
Encontraron que no hacer esquemas se correlaciona con un menor éxito,
mientras que el seguimiento de los cambios de valor de una variable escribiendo los nuevos valores cerca de su nombre
a medida que cambian fue la estrategia más efectiva.

Un posible elemento de confusión que revisaron es el tiempo:
dado que quienes hacen esquemas llevan significativamente más tiempo resolver los problemas,
¿lo hacen mejor solo porque piensan por más tiempo?
La respuesta es no:
no encontraron correlación entre el tiempo que se tomaron y el puntaje alcanzado.
Nuestra recomendación es, por lo tanto, \recommendation{enseña a rastrear los valores que toman las variables al depurar errores}.

\begin{aside}{Diagramas de flujo}\index{diagramas de flujo}
  Un hallazgo que a menudo se pasa por alto sobre la visualización es que
  las personas entienden mejor los diagramas de flujo que el pseudocódigo \emph{si ambos están igualmente bien estructurados}~\cite{Scan1989}.
  Los estudios previos que señalaban que el pseudocódigo superaba a los diagramas de flujo usaron pseudocódigo estructurado pero diagramas de flujo desordenados:
  cuando se niveló el campo de juego,
  se observó que a las personas novatas les fue mejor con la representación gráfica.
\end{aside}

\seclbl{¿Qué más podemos hacer para ayudar?}{s:pck-help}

\cite{Viha2014} examinó la mejora promedio en las tasas de aprobación de varios tipos de intervenciones en clases de programación.
Señalan que hay muchas razones para tomar sus conclusiones con cautela:
las prácticas de enseñanza previas al cambio rara vez se establecen claramente,
la calidad del cambio no es juzgada,
y solo el 8.3\% de los estudios reportaron resultados negativos. 
Entonces, o hay un sesgo al reportar resultados,
o la forma en la que enseñamos en este momento es la peor posible y cualquier cosa sería una mejora.
Como muchos otros estudios discutidos en este capítulo,
en este trabajo sólo estaban observando clases universitarias,
por lo que sus hallazgos pueden no ser generalizables a otros grupos.

Con esas advertencias en mente,
los autores encontraron diez cosas que las/dos docentes puede hacer para mejorar los resultados (\figref{f:pck-interventions}):

\begin{description}

\item[Colaboración:]
  Actividades que fomenten la colaboración entre las/los estudiantes, ya sea en aulas o en laboratorios.

\item[Cambio del contenido:]
  Se modificaron o actualizaron partes del material.

\item[Contextualización:]
  El contenido y las actividades del curso se alinearon con un contexto específico, como juegos o medios.

\item[CS0:]
  Creación de un curso preliminar a tomar antes del curso introductorio de programación;
  podría organizarse solo para algunas personas (por ejemplo, si están en riesgo).

\item[Temática de juego:]
  Una componente de juego fue introducida en el curso.

\item[Esquema de calificación:]
  Un cambio en el sistema de calificación,
  como aumentar la importancia de las actividades de programación y reducir la del examen.

\item[Trabajo en equipos:]
  Actividades con un mayor compromiso de trabajo en grupo, como el aprendizaje en equipo y cooperativo.

\item[Recursos multimedia:]
  Actividades que usen explícitamente el uso de recursos multimedia (\chapref{s:motivation}).

\item[Apoyo de colegas:]
  El apoyo de pares en forma de parejas, grupos, mentores contratados o tutorías.

\item[Otro apoyo:]
  Un término general para todas las actividades de apoyo,
  por ejemplo, mayor cantidad de horas docentes, canales de ayuda adicionales, etc.

\end{description}

\figimg{figures/interventions-scaled.png}{Efectividad de las intervenciones}{f:pck-interventions}

Esta lista destaca la importancia del aprendizaje cooperativo.
\cite{Beck2013} analizaron esto específicamente durante tres años académicos en cursos a cargo dos docentes diferentes
y encontraron beneficios significativos en general y para muchos subgrupos.
Las personas que cooperaron obtuvieron calificaciones más altas
y dejaron menos preguntas en blanco en el examen final,
lo que indica una mayor autoeficacia y disposición para depurar cosas.

\begin{aside}{Computación sin código}
  Escribir código no es la única forma de enseñar a las personas a programar:
  hacer que las personas novatas trabajen en ejercicios de creatividad computacional mejora sus calificaciones en varios niveles~\cite{Shel2017}.
  Un ejercicio típico es describir un objeto cotidiano (como un clip o un cepillo de dientes)
  en términos de sus entradas, salidas y funciones.
  Este tipo de enseñanza a veces es llamada \grefdex{g:cs-unplugged}{CS desconectado}{ciencias de la computación desconectada};
  La página web \hreffoot{https://csunplugged.org/es/}{CS~Unplugged} tiene lecciones y ejercicios para esto.
\end{aside}

\seclbl{¿Qué sigue?}{s:pck-final}

Para aquellas personas que quieran profundizar,
\cite{Finc2019} es un completo resumen de investigación sobre educación en informática y
\cite{Ihan2016} resume los métodos que estos estudios usan más frecuentemente.
Espero que algún día tengamos catálogos como~\cite{Ojos2015}
y más materiales para capacitar docentes como~\cite{Hazz2014,Guzd2015a,Sent2018}
para ayudarnos a enseñar mejor.

La mayor parte de la investigación mencionada en este capítulo fue financiada con fondos públicos
pero tenemos que pagar para leerla:
rompí la ley aproximadamente 250 veces
para descargar trabajos científicos de páginas web como \hreffoot{https://en.wikipedia.org/wiki/Sci-Hub}{Sci-Hub}
mientras escribía este libro.\index{descargas ilegales (minería)}
Espero que llegue el día en que nadie tenga que hacer esto;
si investigas,
por favor ayuda a que ese día se acerque publicando tu trabajo en lugares de acceso abierto.

\seclbl{Ejercicios}{s:pck-exercises}

\exercise{Conceptos erróneos de tus estudiantes}{grupos pequeños}{15'}

Trabajando en grupos pequeños,
vuelvan a leer la \secref{s:pck-misunderstand} y escriban una lista de conceptos erróneos que piensan que pueden tener sus estudiantes.
¿Qué tan específicos son?
¿Cómo verificarían que tan precisa es su lista?

\exercise{Revisando errores comunes}{individual}{20'}

Estos errores comunes corresponden a una lista más larga en ~\cite{Sirk2012}:

\begin{description}

\item[Asignación invertida:]
  La/el estudiante asigna el valor del lado izquierdo de la variable al lado derecho de la variable
  en vez de hacerlo al revés.

\item[Rama equivocada:]
  La/el estudiante piensa que el código del cuerpo de un \texttt{if} corre,
  aun si la condición es falsa.

\item[Ejecutar una función en vez de definirla:]
  Las/los estudiantes creen que una función se ejecuta a medida que es definida.

\end{description}

\noindent
Escribe un ejercicio para chequear que tus estudiantes \emph{no} están cometiendo cada uno de estos errores.

\exercise{Código desarmado}{parejas}{15'}

\cite{Chen2017} describen ejercicios en los que las/los estudiantes reconstruyen el código que ha sido desarmado
al eliminar comentarios,
borrar o reemplazar líneas de código,
al mover líneas,
etc.
El rendimiento en estos ejercicios se correlaciona fuertemente con el rendimiento en las evaluaciones
en las que las personas deben escribir código,
pero estas preguntas requieren menos trabajo para evaluar.
Toma la solución de un ejercicio de programación que hayas creado en el pasado y desármalo de dos maneras distintas.
Intercambia el ejercicio con tu pareja.
Analicen cuánto tiempo les lleva responder correctamente los problemas elaborados por la otra persona.

\exercise{El problema de la lluvia}{parejas}{10'}

\cite{Solo1986} presentó el problema de la lluvia,
que se ha usado en muchos estudios posteriores sobre programación~\cite{Fisl2014,Simo2013,Sepp2015}.
Escribe un programa que repetidamente lea números enteros positivos hasta que llega a 99999.
Después de llegar a 99999,
el programa imprime el promedio de esos números leídos.

\begin{enumerate}

\item
  Resuelve el problema de la lluvia en el lenguaje de programación que prefieras.

\item
  Compara tu solución con la de tu pareja.
  ¿Qué hace tu programa que no hace el de tu colega y viceversa?

\end{enumerate}

\exercise{Roles de variables}{parejas}{15'}

\cite{Kuit2004,Byck2005,Saja2006} presentaron un conjunto de patrones de una sola variable
que encuentro muy útil en la enseñanza de personas novatas:

\begin{description}

\item[Valor fijo:]
  Un elemento que no toma un nuevo valor después de ser inicializado.

\item[Paso a paso:]
  Un elemento que pasa por una sucesión sistemática y predecible de valores.

\item[Caminante:]
  Un elemento que atraviesa una estructura de datos.

\item[Elemento más reciente:]
  Un elemento que guarda el último valor encontrado
  al pasar por una sucesión de valores.

\item[Elemento más buscado:]
  Un elemento que guarda el mejor o el valor más apropiado encontrado hasta el momento.

\item[Recolector:]
  Un elemento que acumula el efecto de valores individuales.

\item[Seguidor:]
  Un elemento que siempre obtiene su nuevo valor a partir del valor anterior de algún otro elemento.

\item[Marca unidireccional:]
  Un elemento de datos de dos valores que no puede obtener su valor inicial una vez que su valor se ha cambiado.

\item[Temporal:]
  Un elemento que contiene un valor solamente por poco tiempo.

\item[Organizador:]
  Una estructura de datos que guarda elementos que pueden ser reordenados.

\item[Contenedor:]
  Una estructura de datos que guarda elementos que pueden ser eliminados.

\end{description}

Elige un programa de 5 a 15 líneas y clasifica sus variables usando estas categorías.
Compara tus clasificaciones con tu pareja.
En los casos en los que no estuvieron de acuerdo,
¿entendieron el punto de vista de la otra persona?

\exercise{¿Qué estás enseñando?}{individual}{10'}

Compara los temas que enseñas con la lista desarrollada en~\cite{Luxt2017} (\secref{s:pck-now}).
¿Qué temas tratas?
¿Qué temas \emph{no} incluyes?
¿Qué temas adicionales incluyes pero no están en la lista?

\exercise{Actividades beneficiosas}{individual}{10'}

Mira la lista de intervenciones desarrolladas por~\cite{Viha2014} (\secref{s:pck-help}).
¿Cuáles de estas cosas ya haces en tus clases?
¿Cuáles podrías agregar fácilmente?
¿Cuáles son irrelevantes?

\exercise{Conceptos erróneos y desafíos}{grupos pequeños}{15'}

El sitio \hreffoot{http://www.pd4cs.org/}{Desarrollo Profesional para la Enseñanza de Principios  de CS}
incluye \hreffoot{http://www.pd4cs.org/mc-index/}{una lista detallada de conceptos erróneos de estudiantes y ejercicios}.
Trabajando en grupos pequeños,
elige una sección (por ejemplo, estructura de datos o funciones) y revisa la lista.
¿Cuáles de estos conceptos erróneos recuerdas haber tenido cuando eras estudiante?
¿Cuáles tienes todavía?
¿Cuáles has visto en tus estudiantes?

\exercise{¿Qué es lo que más te importa?}{toda la clase}{15'}

\cite{Denn2019} les pidieron a varias personas que propongan y califiquen preguntas de investigación sobre educación en informática.
Encontraron que no había superposición entre las preguntas que más les importaban a las/los investigadoras/es
y las que les interesaban a quienes no investigaban.
Las preguntas favoritas de las/los investigadoras/es fueron:

\begin{enumerate}

\item                              
  ¿Qué conceptos fundamentales de programación son los más desafiantes para las/los estudiantes?

\item
  ¿Qué estrategias de enseñanza son más efectivas cuando se trata de una amplia gama de experiencias previas en clases introductorias de programación?

\item
  ¿Qué afecta la capacidad de las/los estudiantes para generalizar a partir de ejemplos de programación simples?

\item
  ¿Qué prácticas de enseñanza son más efectivas para enseñar computación a niñas/os?

\item
  ¿Qué tipo de problemas de una clase de computación resultan más interesantes para las personas?    

\item
  ¿Cuáles son las formas más efectivas de enseñar programación a varios grupos?

\item
  ¿Cuáles son las formas más efectivas de escalar la educación informática para llegar a la población general de estudiantes?

\end{enumerate}

\noindent
Mientras que las preguntas más importantes para las personas que no investigan fueron:

\begin{enumerate}

\item
  ¿Cómo y cuándo es mejor dar retroalimentación a las/los estudiantes sobre su código para mejorar el aprendizaje?

\item
  ¿Qué tipo de ejercicios de programación son más efectivos al enseñar a estudiantes de Ciencias de la Computación?

\item
  ¿Cuáles son los méritos relativos del aprendizaje basado en proyectos, las clases y el aprendizaje activo, para estudiantes de computación?

\item
  ¿Cuál es la forma más efectiva de hacer comentarios a las/los estudiantes en las clases de programación?

\item
  Al dividir los problemas en tareas más pequeñas mientras programan, ¿qué les resulta más difícil de aprender a las personas?

\item
  ¿Cuáles son los conceptos clave que las/los estudiantes deben entender en las clases introductorias de computación?

\item
  ¿Cuáles son las formas más efectivas de desarrollar competencia en estudiantes de disciplinas no informáticas?

\item
  ¿Cuál es el mejor orden para enseñar conceptos y habilidades básicas en informática?

\end{enumerate}

Haz que cada persona de la clase, independientemente, le otorgue un punto a las
las ocho preguntas que más les interese, de cualquiera de las dos listas.
Luego, calcula el puntaje promedio para cada pregunta.
¿Cuáles son las preguntas más populares en tu clase?
¿En qué grupo  están las preguntas más populares?
