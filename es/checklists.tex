\chapter{Listas de verificación y plantillas}\label{s:checklist}

\cite{Gawa2007} hizo popular la idea de que usar listas de verificación puede salvar vidas
y estudios más recientes apoyan su efectividad~\cite{Avel2013,Urba2014,Rams2019}.
En particular encontramos útiles las listas de verificación,
cuando hay docentes que recién se incorporan al equipo.
Los ejemplos que siguen pueden servirte como material inicial a partir del cual desarrollar tus propias listas de verificación.

\seclbl{Enseñando a evaluar}{s:checklists-teacheval}

Esta rúbrica fue diseñada para evaluar lo que se enseñó durante 5 a 10 minutos
con diapositivas, programación en vivo o una combinación de ambas estrategias.
Valora cada ítem como ``Sí,'' ``Más o menos,'' ``No,'' o ``No corresponde (NC).''

\noindent
\begin{longtable}{p{.25\textwidth}p{.75\textwidth}}

  Inicio
  & Presente (usa NC para otras las respuestas si no hay un inicio) \\
  & Adecuada duración (10 a 30 segundos) \\
  & Se presenta \\
  & Presenta el tema que se trabajará \\
  & Describe los requisitos \\
  \\ [-1.5ex] \hline \\ [-1.5ex]

  Contenido
  & Objetivos claros/narrativa fluida \\
  & Lenguaje inclusivo \\
  & Ejemplos y tareas reales \\
  & Enseña buenas prácticas/utiliza código idiomático\\
  & Señala un camino intermedio entre la Escila de la jerga y la Caribdis de la sobresimplificación \\
  \\ [-1.5ex] \hline \\ [-1.5ex]

  Dando la lección
  & Voz clara y entendible (usa ``Más o menos'' o ``No'' para acentos muy marcados) \\
  & Ritmo: ni muy rápido ni muy lento, no realiza pausas largas o se interrumpe, no aparenta estar leyendo sus notas \\
  & Seguridad: no se pierde en el pozo de alquitrán de la incertidumbre ni tampoco en las colinas de estiércol de la condescendencia \\
  \\ [-1.5ex] \hline \\ [-1.5ex]

  Diapositivas
  & Usa diapositivas (completa con N/A el resto de las respuestas si no usa diapositivas) \\
  & Diapositivas y discurso se complementan uno al otro (programación dual) \\
  & Fuentes y colores legibles/sin bloques de texto abrumadores por su tamaño\\
  & Pantalla: cambia frecuentemente (algo cada 30 segundos) \\
  & Adecuado uso de figuras \\
  \\ [-1.5ex] \hline \\ [-1.5ex]

  Programación en vivo
  & Usa programación en vivo (completa con N/A el resto de las respuestas si no usa programación en vivo) \\
  & Código y discurso se complementan uno al otro\\
  & Fuentes y colores legibles/adecuada cantidad de código en pantalla \\
  & Uso de herramientas de forma adecuada \\
  & Resalta elementos clave del código \\
  & Analiza los errores \\
  \\ [-1.5ex] \hline \\ [-1.5ex]

  Cierre
  & Presente (valora NC para otras respuestas si el cierre no está presente) \\
  & Adecuada duración (10 a 30 segundos) \\
  & Resume puntos clave \\
  & Presenta un esquema general de los próximos pasos \\
  \\ [-1.5ex] \hline \\ [-1.5ex]

  En general
  & Puntos claramente conectados/flujo lógico \\
  & Hace que el tema sea interesante (i.e.\ no aburrido) \\
  & Comprende el tema \\

\end{longtable}

\seclbl{Evaluación del grupo docente}{s:checklists-teameval}

Esta rúbrica fue diseñada para evaluar el desempeño de individuos dentro de un grupo. 
Los ejemplos a continuación pueden servirte como material inicial a partir del cual desarrollar tus propias rúbricas. 
Valora cada ítem como ``Sí,'' ``Más o menos,'' ``No,'' o ``No corresponde (NC).''

\noindent
\begin{longtable}{p{.25\textwidth}p{.75\textwidth}}

  Comunicación
  & Escucha atentamente y sin interrumpir \\
  & Aclara lo que se ha dicho para asegurar la comprensión \\
  & Articula ideas en forma clara y concisa \\
  & Argumenta adecuadamente sus ideas \\
  & Obtiene el apoyo de otras/os integrantes del equipo \\
  \\ [-1.5ex] \hline \\ [-1.5ex]

  Toma de decisiones
  & Analiza los problemas desde diferentes puntos de vista \\
  & Aplica lógica para resolver problemas \\
  & Propone soluciones basadas en hechos y no en ``corazonadas'' o intuición \\
  & Invita a las/los integrantes del equipo a proponer nuevas ideas \\
  & Genera nuevas ideas \\
  & Acepta cambios \\
  \\ [-1.5ex] \hline \\ [-1.5ex]

  Colaboración
  & Reconoce los problemas que el equipo necesita enfrentar y resolver \\
  & Trabaja para hallar soluciones que sean aceptables para todas las partes involucradas \\
  & Comparte el crédito del éxito con otras/os integrantes del equipo \\
  & Promueve la participación entre todos las/los integrantes del equipo \\
  & Acepta la crítica abiertamente y sin ``ponerse a la defensiva'' \\
  & Coopera con el equipo \\
  \\ [-1.5ex] \hline \\ [-1.5ex]

  Autogestión
  & Monitorea sus avances para asegurar que se alcancen los objetivos \\
  & Le da máxima prioridad a obtener resultados \\
  & Define tareas prioritarias para los encuentros de trabajo \\
  & Promueve que otras/os integrantes del equipo manifiesten sus opiniones, incluso si no coinciden con las propias \\
  & Mantiene la atención durante la reunión \\
  & Usa eficientemente el tiempo de reunión \\
  & Sugiere formas de trabajar en las reuniones \\

\end{longtable}

\seclbl{Organización de eventos}{s:checklists-events}

Las listas de verificación que se presentan a continuación pueden usarse antes, durante y después de un evento.

\subsection*{Programar el evento}

\begin{itemize}

\item
  Decidir si será presencial,
  virtual para un lugar,
  o virtual para más de un lugar.

\item
  Conversar con la/el disertante sobre sus expectativas
  y asegurarse que están de acuerdo en cuanto a quién cubrirá los costos de traslado.

\item
  Definir quiénes podrán participar:
  ¿será el evento abierto a todas las personas?
  ¿restringido a integrantes de una organización?
  ¿una situación intermedia?

\item
  Organizar quiénes serán docentes.

\item
  Organizar el espacio, incluyendo salas para grupos pequeños si fuera necesario.

\item
  Definir la fecha.
  Si fuera presencia, reservar lo relativo al viaje.

\item
  Conseguir nombres y direcciones de e-mail de participantes a través de la/el disertante.

\item
  Asegurarse de que la totalidad de las/los participantes esté registrada.

\end{itemize}

\subsection*{Construir el evento}

\begin{itemize}

\item
  Crea una página web con los detalles del taller,
  que incluya fecha,
  lugar,
  y lo que las/los participantes deben traer consigo.

\item
  Confirma las necesidades especiales de las/los participantes.

\item
  Si el evento es virtual prueba el modo de videoconferencia, dos veces.

\item
  Asegúrate que las/los participantes tengan acceso a internet.

\item
  Crea un espacio para compartir apuntes y soluciones a los ejercicios (p.ej.\ un documento Google Doc (\secref{s:classroom-notetaking})).

\item
  Establece contacto con las/los asistentes por correo electrónico con un mensaje de bienvenida que contenga
  el link a la página del taller,
  lecturas sobre la temática,
  la descripción de la configuración que deban hacer en su computadora,
  una lista de los elementos requeridos para el taller,
  y un mecanismo para establecer contacto con la/el disertante o docente durante el día.

\end{itemize}

\subsection*{Al comienzo del evento}

\begin{itemize}

\item
  Recuerda a las/los asistentes el código de conducta.

\item
  Toma lista
  y crea una lista de nombres para pegar en el documento compartido para tomar notas.

\item
  Reparte pequeñas notas adhesivas.

\item
  Asegúrate que tengan acceso a internet.
  
\item
  Asegúrate que pueden acceder al documento compartido.

\item
  Registra información relevante sobre la identificación de las/los asistentes en sus perfiles online.

\end{itemize}

\subsection*{Al finalizar el evento}

\begin{itemize}

\item
  Actualiza la lista de participantes.

\item
  Lleva un registro de la retroalimentación brindada por las/los participantes.

\item
  Haz una copia del documento compartido.

\end{itemize}

\subsection*{Equipo de viaje}

Aquí algunas cosas que las/los docentes llevan consigo a los talleres:

\begin{longtable}{p{0.45\textwidth}p{0.45\textwidth}}

notas adhesivas y caramelos para suavizar la garganta \\
zapatos cómodos y una pequeña libreta de notas \\
adaptador de corriente eléctrica de repuesto y camisa de repuesto \\
desodorante y adaptadores para video \\
pegatinas (*stickers*) para computadoras y tus notas (impresas o en una tableta) \\
barrita de cereal o similar y antiácido (problema de comer al paso) \\
tarjeta de presentación y anteojos/lentes de contacto de repuesto \\
libreta y bolígrafo, y puntero láser \\
vaso térmico para té/café y marcadores de pizarra adicionales \\
cepillo de dientes o enjuague bucal y toallitas húmedas descartables (puede volcarse algo encima de tu ropa) \\

\end{longtable}

Al viajar
muchas/os docentes llevan además zapatos deportivos, traje de baño, mat de yoga
o el material que necesiten para hacer actividad física.
También una conexión WiFi portátil por si la de la habitación no funciona,
y alguna memoria USB con los instaladores del software que las/los estudiantes 
necesitarán para el curso.

\seclbl{Diseño de lecciones}{s:checklists-design}

Esta sección resume el diseño de lecciones por el método de reingeniería,
que fue desarrollado independientemente por ~\cite{Wigg2005,Bigg2011,Fink2013}.
Propone una progresión paso a paso
para ayudarte a pensar en qué hacer en cada uno de los pasos y en el orden adecuado
y proporciona ejercicios breves espaciados
para que puedas reorientar o redirigir tu esfuerzo sin demasiadas sorpresas desagradables.

Todo lo que hagas del paso 2 en adelante será parte de tu lección final
por lo que no estarás perdiendo en tiempo:
como se describió en el \chapref{s:process},
construir ejercicios de práctica desde el comienzo te ayuda a asegurarte que
todo lo que preguntes a las/los estudiantes contribuirá a los objetivos de la lección
y que todo lo que necesitan saber está cubierto.

Los pasos se describen en orden creciente de detalle
pero el proceso en sí es siempre iterativo.
Con frecuencia y a medida que resuelvas preguntas más avanzadas,  volverás a revisar tus respuestas en trabajos anteriores, 
y te darás cuenta que tu plan inicial no iba a funcionar como pensaste originalmente.

\subsection*{¿Para quién es esta lección?}

Crea algunas personas tipo (\secref{s:process-personas})
o (mejor aún) elige entre los que tú y tus colegas han creado para uso general.
Cada estudiante tipo debe tener:

\begin{enumerate}

\item
  un contexto general,

\item
  lo que ya sabe,

\item
  lo que cree que quiere saber y

\item
  qué necesidades especiales tiene.

\end{enumerate}

~\\
\noindent
\textbf{Ejercicio breve:} resumen breve de a quién estás intentando ayudar.

\subsection*{¿Cuál es la idea principal?}

Responde tres o cuatro de las siguientes preguntas sólo enumerando elementos
para ayudarte a descifrar el enfoque de la lección.
No necesitas responder todas las preguntas,
y puedes plantear y responder otras preguntas si crees que ayudarán,
pero debes incluir sí o sí un par de respuestas a la primera pregunta.
Además, en esta etapa puedes crear un mapa conceptual (\secref{s:memory-concept-maps}).

\begin{itemize}

\item
  ¿Qué problemas aprenderán a resolver?

\item
  ¿Cuáles conceptos y técnicas aprenderán?

\item
  ¿Cuáles herramientas tecnológicas, paquetes y funciones usarán?

\item
  ¿Qué términos de jerga definirás?

\item
  ¿Qué analogías usarás para explicar conceptos?

\item
  ¿Qué errores o conceptos erróneos esperas encontrar?

\item
  ¿Qué conjuntos de datos utilizarás?

\end{itemize}

~\\
\noindent
\textbf{Ejercicio breve}
enfoque general y sin detalles de la lección.
Compártelo con una/un colega --- una breve devolución en esta instancia
puede ahorrar horas de esfuerzo más tarde.

\subsection*{¿Qué harán las/los estudiantes durante la lección?}

Establece los objetivos del paso 2 escribiendo descripciones detalladas de
algunos ejercicios que las/los estudiantes serán capaces de resolver al final de la lección.
Hacer esto es análogo a \hreffoot{https://es.wikipedia.org/wiki/Desarrollo_guiado_por_pruebas}{desarrollo impulsado por pruebas}:
en vez de trabajar en función de un conjunto de objetivos de aprendizaje (probablemente ambiguos),
hazlo ``hacia atrás'': elabora ejemplos concretos que quieres que puedan resolver tus estudiantes.
Esto además permite dejar en evidencia requisitos técnicos necesarios
que de otro modo podrían no descubrirse hasta que fuera demasiado tarde.

Para complementar la descripción detallada de los ejercicios
escribe la descripción de uno o dos ejercicios para cada hora de lección como una lista de conceptos breve
para mostrar qué tan rápido esperas que las/los estudiantes avancen.
De nuevo, 
esto permitirá tener una visión realista sobre lo que asumiste de las/los estudiantes
y ayudará a hacer evidentes los requisitos técnicos necesarios para resolver el ejercicio.
Una manera de elaborar estos ejercicios adicionales
es hacer una lista con las habilidades que necesitan para resolver los ejercicios principales
y crear un ejercicio que aborde cada una.

~\\
\noindent
\textbf{Ejercicio breve:} 1 a 2 ejercicios explicados de principio a fin
que usen las habilidades que tus estudiantes van a aprender,
y una media docena de ejercicios con su solución esquematizada.
Incluye soluciones completas
para que puedas asegurarte que el programa que usen funciona.

\subsection*{¿Cómo están conectados los conceptos}

Coloca los ejercicios que creaste en un orden lógico
y a partir de ellos deriva el esquema general de una lección.
El esquema debe tener entre 3 a 4 ítems por hora de clase
con una evaluación formativa para cada uno.
En esta etapa es común que modifiques las evaluaciones
de forma que puedan basarse sobre las anteriores.

~\\
\noindent
\textbf{Ejercicio breve:} el esquema de una lección.
Es muy probable que te encuentres con que te habías olvidado de algunos elementos y que no están incluidos en tu trabajo hasta aquí,
así que no te sorprendas si debes ir y venir varias veces.

\subsection*{Descripción general de la lección}

Ahora puedes escribir la descripción general de la lección que incluya:

\begin{itemize}

\item
  un párrafo de descripción (i.e.\ un discurso de venta para tus estudiantes),

\item
  media docena de objetivos de aprendizaje y

\item
  un resumen de los requisitos.

\end{itemize}

Hacer esto antes suele ser un esfuerzo inútil
ya que el material que compone la lección aumenta, se recorta o cambia de lugar en las etapas anteriores.

~\\
\noindent
\textbf{Ejercicio breve:}
descripción del curso,
objetivos de aprendizaje
y requisitos.

\seclbl{Cuestionario pre-evaluación}{s:checklists-preassess}

Este cuestionario ayuda a las/los docentes a estimar el conocimiento previo sobre programación
de las/los participantes de un taller introductorio a JavaScript.
Las preguntas y respuestas son concretas
y el cuestionario es corto, para que no resulte intimidante.

\begin{enumerate}

\item
  ¿Cuál de estas opciones describe mejor
  tu experiencia con la programación en general?

  \begin{itemize}
    
  \item
    No tengo ninguna experiencia.
    
  \item
    He escrito unas pocas líneas de código alguna vez.
    
  \item
    He escrito programas para uso personal de un par de páginas de extensión.
    
  \item
    He escrito y mantenido porciones grandes de programas.\\
    
  \end{itemize}

\item
  ¿Cuál de estas opciones describe mejor
  tu experiencia con la programación en JavaScript?

  \begin{itemize}
    
  \item
    No tengo ninguna experiencia.
    
  \item
    He escrito unas pocas líneas de código alguna vez.
    
  \item
    He escrito programas para uso personal de un par de páginas de extensión.
    
  \item
    He escrito y mantenido porciones grandes de programas.\\
    
  \end{itemize}

\item
  ¿Cuál de estas opciones describe mejor cuán fácil te resultaría escribir un programa
  en el lenguaje de programación que prefieras
  para hallar el número más alto en una lista?

  \begin{itemize}
    
  \item
    No sabría por dónde comenzar.
    
  \item
    Podría resolverlo con prueba y error y realizando bastantes búsquedas en internet.
    
  \item
    Lo resolvería rápido con poco o nada de ayuda externa.\\
    
  \end{itemize}

\item
  ¿Cuál de estas opciones describe mejor
  cuán fácil te resultaría escribir un programa en JavaScript
  para hallar y cambiar a mayúscula todos los títulos de una página web?

  \begin{itemize}
    
  \item
    No sabría por dónde comenzar.
    
  \item
    Podría resolverlo con prueba y error y realizando bastantes búsquedas en internet.
    
  \item
    Lo resolvería rápido con poco o nada de ayuda externa.\\
    
  \end{itemize}

\item
  ¿Qué te gustaría saber o poder hacer al finalizar esta clase
  que no sabes o puedes hacer ahora?

\end{enumerate}
