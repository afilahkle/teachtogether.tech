\chapter{Glosario}\label{s:gloss}

\begin{description}

\gitem{g:absolute-beginner}{Principiante absoluto} Alguien que nunca se ha
encontrado con los conceptos o material antes. El término se usa en distinción para
\grefcross{g:false-beginner}{falso principiante}.

\gitem{g:active-learning}{Aprendizaje activo} Un enfoque de la enseñanza en el que
los estudiantes se involucran con el material a través de la discusión, resolución de problemas, estudios de casos,
y otras actividades que requieren que reflexionen y usen nueva información en
tiempo real. Ver también \grefcross{g:passive-learning}{aprendizaje pasivo}.

\gitem{g:active-teaching}{Enseñanza activa} Un enfoque de la instrucción en el que el
maestro actúa sobre la nueva información adquirida de los alumnos mientras enseña (ej.\ cambiando
dinámicamente un ejemplo o reorganizando el orden previsto del contenido).
Ver también \grefcross{g:passive-teaching}{enseñanza pasiva}.

\gitem{g:authentic-task}{Tarea auténtica} Una tarea que contiene elementos importantes 
de cosas que los alumnos harían en situaciones reales (fuera del aula). Para 
ser auténtica, una tarea debe requerir que los alumnos construyan sus propias respuestas
en lugar de elegir entre las respuestas proporcionadas, y trabajar con las mismas 
herramientas y datos que usarían en la vida real.

\gitem{g:automaticity}{Automaticidad} La capacidad de hacer una tarea sin 
concentrarse en sus detalles de bajo nivel.

\gitem{g:backward-design}{Reingeniería} Un método de diseño instruccional que
trabaja hacia atrás desde una evaluación sumativa hasta evaluaciones formativas y desde allí
al contenido de la lección.

\gitem{g:behaviorism}{Conductivismo} Una teoría del aprendizaje cuyo principio central
es estímulo y respuesta, y cuyo objetivo es explicar el comportamiento sin recurrir
a estados mentales internos u otros inobservables. Ver
además \grefcross{g:cognitivism}{cognitivismo}.

\gitem{g:blooms-taxonomy}{Taxonomía de Bloom} Una clasificación jerarquíca, ampliamente adoptada, 
de seis estapas de comprensión y cuyos niveles son \emph{conocimiento},
\emph{comprensión}, \emph{aplicación}, \emph{análisis}, \emph{síntesis} y
\emph{evaluación}. 
Ver también \grefcross{g:finks-taxonomy}{Taxonomía de Fink}.

\gitem{g:brand}{Marca} Las asociaciones que las personas tienen con el nombre de un producto o
identidad.

\gitem{g:calibrated-peer-review}{Revisión por pares calibrada} Hacer que los alumnos comparen sus 
revisiones del trabajo de ejemplo con las de un maestro antes de que 
se les permita revisar el trabajo de sus pares.

\gitem{g:chunking}{Fragmentación} El acto de agrupar conceptos relacionados juntos
que pueden almacenarse y procesarse como una sola unidad.

\gitem{g:co-teaching}{Co-enseñanza} Enseñar con otro docente en el
salón de clases.

\gitem{g:cognitive-apprenticeship}{Aprendizaje cognitivo} Una teoría de
aprendizaje que enfatiza el proceso del docente que transmite habilidades e ideas
situacionalmente al estudiante.

\gitem{g:cognitive-load}{Carga cognitiva} El esfuerzo mental necesario para resolver un problema.
La teoría de la carga cognitiva lo divide en
 \grefcross{g:intrinsic-load}{carga intrínseca},
\grefcross{g:germane-load}{carga pertinente}
y \grefcross{g:extraneous-load}{carga extrínseca}.
Sostiene que las personas aprenden más rápido cuando se reduce la carga pertinente y extraña.

\gitem{g:cognitivism}{Cognitivismo} Una teoría del aprendizaje que sostiene que los estados 
y procesos mentales pueden y deben incluirse en modelos de aprendizaje. Ver también
\grefcross{g:behaviorism}{conductivismo}.

\gitem{g:commons}{Commons} algo gestionado conjuntamente por una comunidad 
de acuerdo con las reglas que la misma comunidad ha desarrollado y adoptado.

\gitem{g:community-of-practice}{Comunidad de práctica} Un grupo de personas que se perpetúan a sí mismas 
y comparten y desarrollan un oficio como tejedoras/es, músicas/os o programadoras/es. Ver también
\grefcross{g:legitimate-peripheral-participation}{participación inicial legítima}.

\gitem{g:community-representation}{Representación de la comunidad} Usar el capital cultural 
para resaltar las identidades sociales, las historias y las redes comunitarias de 
las/os estudiantes en las actividades de aprendizaje.

\gitem{g:computational-integration}{Integración Computacional} Usar la informática 
para volver a implementar artefactos culturales preexistentes, por ejemplo, 
\ crear variantes de diseños tradicionales usando herramientas de dibujo por computadora.


\gitem{g:competent-practitioner}{Practicante competente} Alguien que puede 
realizar tareas normales con un esfuerzo normal en circunstancias normales. Ver también
\grefcross{g:novice}{principiante} and \grefcross{g:expert}{experta/o}.

\gitem{g:computational-thinking}{Pensamiento computacional} Pensar la
resolución de problemas en formas inspiradas en la programación (aunque el término se usa de muchas
otras maneras).

\gitem{g:concept-inventory}{Inventario de conceptos} Una prueba diseñada para determinar 
qué tan bien un alumno comprende un dominio. A diferencia de la mayoría de las pruebas realizadas por instructores, 
los inventarios de conceptos se basan en una extensa investigación y validación.

\gitem{g:concept-map}{Mapa conceptual} Una imagen de un modelo mental en el que 
los conceptos son nodos en un gráfico y las relaciones entre esos conceptos son arcos (etiquetados).

\gitem{g:connectivism}{Conectivismo} Una teoría del aprendizaje que sostiene que el conocimiento se distribuye, 
que el aprendizaje es el proceso de navegación, crecimiento y poda de conexiones, y que enfatiza los aspectos 
sociales del aprendizaje hechos posibles por Internet.

\gitem{g:constructivism}{Constructivismo} Una teoría del aprendizaje que considera a 
las/os estudiantes construyendo activamente el conocimiento.

\gitem{g:content-knowledge}{Conocimiento del contenido} La comprensión de una 
persona de un tema. Ver también
\grefcross{g:general-pedagogical-knowledge}{conocimiento pedagógico general}
y \grefcross{g:pedagogical-content-knowledge}{conocimiento de contenido pedagógico}.

\gitem{g:contributing-student}{Contribuyendo a la pedagogía estudiantil} Tener a las/os estudiantes
produciendo artefactos para contribuir al aprendizaje de otros.

\gitem{g:conversational-programmer}{Programadora conversacional} Alguien que necesita saber
lo suficiente sobre computación para tener una conversación significativa con un programador, 
pero no que va a programar por sí mismo.

\gitem{g:cs-unplugged}{Ciencias de la computación acústico} Un estilo de enseñanza que introduce 
conceptos informáticos utilizando ejemplos y artefactos que no son de programación.

\gitem{g:cs0}{CS0. Introducción a las ciencias de la computación} Un curso introductorio de nivel universitario sobre computación 
dirigido a estudiantes no avanzados con poca o ninguna experiencia previa en programación.

\gitem{g:cs1}{CS1. ciencias de la computación I} Un curso introductorio de ciencias de la computación a nivel universitario, 
generalmente de un semestre, que se enfoca en variables, bucles, funciones y otras mecánicas básicas.

\gitem{g:cs2}{CS2. ciencias de la computación II} Un segundo curso de ciencias de la computación de nivel universitario 
que generalmente presenta estructuras de datos básicas como pilas, colas y diccionarios.

\gitem{g:deficit-model}{Modelo deficitario} La idea de que algunos grupos 
están subrepresentados en informática (o algún otro campo) porque sus miembros 
carecen de algún atributo o calidad.

\gitem{g:deliberate-practice}{Practica deliberada} El acto de observar el desempeño 
de una tarea mientras se realiza para mejorar la capacidad.

\gitem{g:demonstration-lesson}{Leccion de demostracion} Una lección dictada por un/a docente a estudiantes reales 
mientras otras/os docentes observan para aprender nuevas técnicas de enseñanza.

\gitem{g:diagnostic-power}{Poder de diagnóstico} El grado en que una respuesta incorrecta 
a una pregunta o ejercicio le dice al docente qué conceptos erróneos tiene un/a estudiante en particular.

\gitem{g:direct-instruction}{Instrucción directa} Un método de enseñanza centrado 
en un diseño curricular meticuloso dictado a través de guiones pre-escritos.

\gitem{g:dunning-kruger-effect}{Efecto Dunning-Kruger} La tendencia de las personas que solo saben 
un poco sobre un tema a estimar incorrectamente su comprensión del mismo.

\gitem{g:educational-psychology}{Psicología Educacional} El estudio de cómo
la gente aprende. Ver también \grefcross{g:instructional-design}{diseño instruccional}.

\gitem{g:ego-depletion}{Agotamiento del ego} El deterioro del autocontrol debido del uso prolongado o intensivo. 
Trabajo recientes no han podido corroborar su existencia.

\gitem{g:elevator-pitch}{Discurso de presentación} Una breve descripción de una idea, 
proyecto, producto o persona que se puede dar y comprender en solo unos segundos.

\gitem{g:end-user-programmer}{Usuario final programador} Alguien que no se considera 
un programador, pero que, sin embargo, escribe y depura software, como por ejemplo, un artista que 
crea macros complejas para una herramienta de dibujo.

\gitem{g:end-user-teacher}{Usuario final docente} Por analogía con
\grefcross{g:end-user-programmer}{usuario final programador},
alguien que enseña con frecuencia, pero cuya ocupación principal no es la enseñanza, 
que tiene poca o ninguna experiencia en pedagogía y que puede trabajar fuera de las aulas institucionales.

\gitem{g:expert}{Experta/o} Alguien que puede diagnosticar y manejar situaciones inusuales, 
sabe cuándo no se aplican las reglas habituales y tiende a reconocer soluciones en lugar de razonarlas. 
Ver también \grefcross{g:competent-practitioner}{practicante competente}
y \grefcross{g:novice}{novata/o}.

\gitem{g:expert-blind-spot}{Punto ciego del experto} La incapacidad de las personas expertas 
para empatizar con las personas novatas que se encuentran por primera 
vez con conceptos o prácticas.

\gitem{g:expertise-reversal}{Efecto inverso de la experiencia} La forma en que 
la instrucción que es efectiva para los novatos se vuelve ineficaz para 
los profesionales competentes o expertos.

\gitem{g:externalized-cognition}{Cognición externalizada} El uso de ayuda gráfica, 
física o verbal para aumentar el pensamiento.

\gitem{g:extraneous-load}{Carga extrínseca} Cualquier \grefcross{g:cognitive-load}{carga cognitiva}
que distrae del aprendizaje.

\gitem{g:extrinsic-motivation}{Motivación extrínseca} Ser impulsado por 
recompensas externas como el pago o el miedo al castigo. Ver
también \grefcross{g:intrinsic-motivation}{motivación intrínseca}.

\gitem{g:faded-example}{Ejemplos desvanecidos} Una serie de ejemplos en los que 
se borra un número cada vez mayor de pasos clave. Ver
también \grefcross{g:scaffolding}{andamiaje}.

\gitem{g:false-beginner}{Falso principiante} Alguien que ha estudiado un idioma antes pero lo está aprendiendo nuevamente. Los falsos principiantes comienzan en el mismo punto que los principiantes verdaderos (es decir, en una evaluación inicial mostrarán el mismo nivel de competencia) pero pueden avanzar mucho más rápidamente.


\gitem{g:far-transfer}{Transferencia lejana} La \grefcross{g:transfer-of-learning}{transferencia de aprendizaje} entre dominios ampliamente separados, por ejemplo, mejora en las habilidades matemáticas como resultado de jugar al ajedrez.

\gitem{g:finks-taxonomy}{Taxonomía de Fink} Una clasificación de comprensión no jerárquica 
de seis partes, propuesta por primera vez en~\cite{Fink2013} cuyas categorías son \emph{conocimiento fundamental}, \emph{aplicación}, \emph{integración}, \emph{dimensión humana}, \emph{cuidado} y 
\emph{aprender a aprender}. Ver también \grefcross{g:blooms-taxonomy}{Taxonomía de Bloom}.

\gitem{g:fixed-mindset}{Mentalidad fija} La creencia de que una habilidad es innata y que 
el fracaso se debe a la falta de algún atributo necesario. Ver también
\grefcross{g:growth-mindset}{mentalidad de crecimiento}.

\gitem{g:flipped-classroom}{Aula invertida} Una clase en la que los alumnos 
ven lecciones grabadas en su propio tiempo, mientras que el tiempo de clase 
se utiliza para resolver conjuntos de problemas y responder preguntas.

\gitem{g:flow}{Flujo} La sensación de estar completamente inmerso en una actividad, 
frecuentemente asociada con una alta productividad.

\gitem{g:fluid-representation}{Representación fluida} La capacidad de moverse
rápidamente entre diferentes modelos de un problema.

\gitem{g:formative-assessment}{Evaluación formativa} Evaluación que se lleva a cabo 
durante una clase para dar retroalimentación tanto al alumno como al profesor 
sobre la comprensión real. Ver
también \grefcross{g:summative-assessment}{evaluación sumativa}.

\gitem{g:free-range-learner}{Estudiante free-range. Estudiante de rango libre.} Alguien que aprende fuera de un aula institucional con un plan de estudios y tareas obligatorias. (Quienes usan este término, ocasionalmente se refieren a los estudiantes en las aulas como "estudiantes battery-farmed", pero nosotros no lo hacemos porque sería grosero).

\gitem{g:fuzz-testing}{Fuzz testing} Una técnica de prueba de software 
basada en generar y enviar datos aleatorios.

\gitem{g:general-pedagogical-knowledge}{Conocimiento pedagógico general} La 
comprensión de una persona de los principios generales de la enseñanza. Ver también
\grefcross{g:content-knowledge}{conocimiento del contenido}
y \grefcross{g:pedagogical-content-knowledge}{conocimiento de contenido pedagógico}.

\gitem{g:germane-load}{Carga pertinente} La \grefcross{g:cognitive-load}{carga cognitiva}
requerida para vincular la nueva información con la antigua.

\gitem{g:governance-board}{Directorio. Junta} Una junta cuya responsabilidad principal es
contratar, supervisar y, si es necesario, despedir al director.

\gitem{g:growth-mindset}{Mentalidad de crecimiento} La creencia de que la habilidad 
viene con la práctica. Ver también \grefcross{g:fixed-mindset}{mentalidad fija}.

\gitem{g:guided-notes}{Notas guiadas} Notas preparadas por la/el docente que indican a las/os 
estudiantes que respondan a la información clave en una conferencia o discusión.

\gitem{g:hashing}{Hashing} Generar una clave digital pseudoaleatoria condensada a partir de datos; cualquier entrada específica produce la misma salida, pero es muy probable que diferentes entradas produzcan diferentes salidas.

\gitem{g:hypercorrection}{Efecto de hipercorrección} Cuanto más creé alguien 
que su respuesta en un exámen era correcta, más probabilidades hay de que no 
repitan el error una vez que descubre que, de hecho, estaban equivocados.

\gitem{g:implementation-science}{Ciencia de implementación} El estudio de cómo traducir 
los hallazgos de la investigación a la práctica clínica diaria.

\gitem{g:impostor-syndrome}{Síndrome del impostor} Un sentimiento de inseguridad sobre 
los logros propios, que se manifiesta como un miedo a ser expuesto como un fraude.

\gitem{g:inclusivity}{Inclusión} Trabajar activamente para incluir 
personas con diversos antecedentes y necesidades.

\gitem{g:inquiry-based-learning}{Aprendizaje basado en la indagación} La práctica de permitir que las/os estudiantes hagan sus propias preguntas, establezcan sus propios objetivos y encuentren su propio camino a través de un tema.

\gitem{g:instructional-design}{Diseño instruccional} El arte de crear y evaluar 
lecciones específicas para audiencias específicas. Ver también
\grefcross{g:educational-psychology}{psicología educacional}.

\gitem{g:intrinsic-load}{Carga instrínseca} La \grefcross{g:cognitive-load}{carga cognitiva}
requerida para absorber nueva información.

\gitem{g:intrinsic-motivation}{Motivación intrínseca} Ser impulsada/o por el disfrute o 
la satisfacción de hacer una tarea como fin en si mismo.  Ver también
\grefcross{g:extrinsic-motivation}{motivación extrínseca}.

\gitem{g:intuition}{Intuición} La capacidad de comprender algo de inmediato, 
sin necesidad aparente de razonamiento consciente.

\gitem{g:jugyokenkyu}{Jugyokenkyu} Literalmente "estudio de lección", un conjunto de prácticas que incluye hacer que las/os docentes se observen rutinariamente entre sí y discutan las lecciones que imparten para compartir conocimientos y mejorar habilidades.

\gitem{g:learned-helplessness}{Impotencia aprendida} Una situación en la que las personas que son sometidas repetidamente a comentarios negativos de que no tienen forma de escapar aprenden a ni siquiera intentar escapar cuando pueden.

\gitem{g:learner-persona}{Estudiante tipo} Una breve descripción de un/a estudiante objetivo tipo
para una lección que incluye: sus antecedentes generales, lo que ya sabe, lo que quiere hacer, cómo la lección le ayudará y cualquier necesidad especial que puedan tener.

\gitem{g:lms}{Sistema de gestión de aprendizaje} (\emph{LMS}, por sus siglas en Inglés): Una aplicación para registrar la inscripción a cursos, presentaciones de ejercicios, calificaciones y otros aspectos burocráticos del aprendizaje formal en el aula.

\gitem{g:learning-objective}{Objetivo de aprendizaje} Qué está intentando lograr la lección.

\gitem{g:learning-outcome}{Resultado de aprendizaje} Qué es lo que la lección realmente logra.

\gitem{g:legitimate-peripheral-participation}{Participación inicial legítima} La participación de las/os recién llegadas/os en tareas simples y de bajo riesgo que una  
\grefcross{g:community-of-practice}{comunidad de práctica} reconoce como contribuciones válidas.

\gitem{g:live-coding}{Programación en vivo} El acto de enseñar programación 
escribiendo software frente a los alumnos a medida que avanza la lección.

\gitem{g:long-term-memory}{Memoria de largo plazo} La parte de la memoria que 
almacena información durante largos períodos de tiempo. La memoria a largo plazo es muy grande, 
pero lenta. Ver también \grefcross{g:short-term-memory}{memoria de corto plazo}.

\gitem{g:manual}{Manual} Material de referencia, destinado a ayudar a alguien que ya comprende un tema, a completar (o recordar) detalles.

\gitem{g:marketing}{Marketing} El arte de ver las cosas desde la perspectiva 
de otras personas, comprender sus deseos y necesidades y encontrar formas de satisfacerlas.


\gitem{g:mooc}{Cursos on-line masivos} (\emph{MOOC}, por sus siglas en Inglés) Un curso en línea diseñado para la inscripción masiva y el estudio asincrónico, que generalmente usa videos grabados y calificaciones automáticas.

\gitem{g:mental-model}{Modelo mental} Una representación simplificada 
de los elementos clave y las relaciones de algunos dominios de problemas que es 
lo suficientemente buena como para apoyar la resolución de problemas.

\gitem{g:metacognition}{Metacognición} Pensar sobre pensar.

\gitem{g:minimal-manual}{Manual mínimo} Un enfoque de capacitación que divide 
cada tarea en instrucciones de una sola página que también 
explican cómo diagnosticar y corregir errores comunes.

\gitem{g:minute-cards}{Tarjetas de minutos} Una técnica de retroalimentación en la 
que las/os estudiantes pasan un minuto escribiendo una cosa positiva sobre una lección 
(por ejemplo, una cosa que han aprendido) y una cosa negativa (por ejemplo, una pregunta que aún no ha sido respondida).

\gitem{g:near-transfer}{Transferencia cercana} \grefcross{g:transfer-of-learning}{Transferencia de aprendizaje} entre dominios estrechamente relacionados, por ejemplo, mejora en la comprensión de decimales como resultado de hacer ejercicios con fracciones.

\gitem{g:notional-machine}{Máquina nocional} Un modelo general simplificado de cómo se ejecuta una familia particular de programas.

\gitem{g:novice}{Novata/o. Persona novata. Principiante} Alguien que aún no ha construido un modelo mental utilizable de un dominio. Ver también \grefcross{g:competent-practitioner}{Practicantes competentes} y \grefcross{g:expert}{experta/o}.

\gitem{g:objects-first}{Primero los objetos} Un enfoque para enseñar programación donde 
objetos y clases se introducen temprano.

\gitem{g:pair-programming}{Programación en pareja} Una práctica de desarrollo de software en la que dos programadores comparten una computadora. Un programador (el piloto) escribe, mientras que el otro (el navegante) ofrece comentarios y sugerencias en tiempo real. La programación en pareja a menudo se usa como práctica docente en las clases de programación.


\gitem{g:parsons-problem}{Problemas de Parson} Una técnica de evaluación desarrollada 
por Dale Parsons y otros en la que los alumnos reorganizan el material dado 
para construir una respuesta correcta a una pregunta.~\cite{Pars2006}.

\gitem{g:passive-learning}{Aprendizaje pasivo} Un enfoque de la enseñanza en el que las/os estudiantes leen, escuchan o miran sin utilizar inmediatamente nuevos conocimientos. El aprendizaje pasivo es menos efectivo que el \grefcross{g:active-learning}{aprendizaje activo}.

\gitem{g:passive-teaching}{Enseñanza pasiva} Un enfoque a la enseñanza en el que la/el docente no ajusta el ritmo o los ejemplos, o no actúa de acuerdo con los comentarios de las/os estudiantes, durante la lección.  Ver también \grefcross{g:active-teaching}{enseñanza activa}.

\gitem{g:pedagogical-content-knowledge}{Conocimiento de contenido pedagógico} (\emph{PCK}, por sus siglas en Inglés)
The understanding of how to teach a particular subject, i.e.\ the best order in
which to introduce topics and what examples to use. Ver también
\grefcross{g:content-knowledge}{conocimiento del contenido}
y \grefcross{g:general-pedagogical-knowledge}{conocimiento pedagógico general}.

\gitem{g:peer-instruction}{Instrucción por pares} A teaching method in which an
teacher poses a question and then learners commit to a first answer, discuss
answers with their peers, and commit to a (revised) answer.

\gitem{g:persistent-memory}{Memoria persistente} Ver \grefcross{g:long-term-memory}{memoria de largo plazo}.

\gitem{g:personalized-learning}{Aprendizaje personalizado} Automatically tailoring
lessons to meet the needs of individual learners.

\gitem{g:plausible-distractor}{Distractor pausible} A wrong or less-than-best
answer to a multiple-choice question that looks like it could be right. Ver también
\grefcross{g:diagnostic-power}{Poder de diagnóstico}.

\gitem{g:positioning}{Posicionamiento} What sets one brand apart from other,
similar brands.

\gitem{g:preparatory-privilege}{Privilegio preparatorio} The advantage of coming
from a background that provides more preparation for a particular learning task
than others.

\gitem{g:productive-failure}{Falla productiva} A situation in which learners
are deliberately given problems that can't be solved with the knowledge they
have and must go out and acquire new information in order to make progress.
Ver también \grefcross{g:zpd}{Zona de Desarrollo Proximal}.

\gitem{g:pull-request}{Pull request} A set of proposed changes to a GitHub
repository that can be reviewed, updated, and eventually merged.

\gitem{g:read-cover-retrieve}{Leer-cubrir-recuperar} A study practice in which
the learner covers up key facts or terms during a first pass through material,
then checks their recall on a second pass.

\gitem{g:reflective-practice}{Práctica reflexiva}
Ver \grefcross{g:deliberate-practice}{practica deliberada}.

\gitem{g:reusability-paradox}{Paradoja de la reusabilidad} Holds that the more reusable
part of a lesson is, the less pedagogically effective it is.

\gitem{g:scaffolding}{Andamiaje} Extra material provided to early-stage
learners to help them solve problems.

\gitem{g:seo}{Motor de optimización de posicionamiento en buscadores} (\emph{SEO}, por sus siglas en Inglés)  Increasing the quantity and quality
of web site traffic by making pages more easily found by, or seem more important
to, search engines.

\gitem{g:service-board}{Directorio o junta de servicio} A board whose members take on working roles
in the organization.

\gitem{g:short-term-memory}{Memoria de corto plazo} The part of memory that briefly
stores information that can be directly accessed by consciousness.

\gitem{g:situated-learning}{Aprendizaje situado} A model of learning that focuses
on people's transition from being newcomers to be accepted members of a
\grefcross{g:community-of-practice}{comunidad de práctica}.

\gitem{g:split-attention-effect}{Efecto de atención dividida} The decrease in
learning that occurs when learners must divide their attention between multiple
concurrent presentations of the same information (e.g.\ captions and a
voiceover).

\gitem{g:stereotype-threat}{Amenaza del estereotipo} A situation in which people feel
that they are at risk of being held to stereotypes of their social group.

\gitem{g:subgoal-labeling}{Etiquetado de submetas} Giving names to the steps in a
step-by-step description of a problem-solving process.

\gitem{g:summative-assessment}{Evaluación sumativa} Assessment that takes
place at the end of a lesson to tell whether the desired learning has taken
place.

\gitem{g:tangible-artifact}{Artefacto tangible} Something a learner can work on
whose state gives feedback about the learner's progress and helps the learner
diagnose mistakes.

\gitem{g:teaching-to-the-test}{Enseñando para el exámen} Any method of ``education''
that focuses on preparing students to pass standardized tests, rather than on
actual learning.

\gitem{g:test-driven-development}{Desarrollo basado en test} A software
development practice in which programmers write tests first in order to give
themselves concrete goals and clarify their understanding of what ``done'' looks
like.

\gitem{g:think-pair-share}{Piensa-trabaja en pareja-comparte} A collaboration method in which
each person thinks individually about a question or problem, then pairs with a
partner to pool ideas, and then have one person from each pair present to the
whole group.

\gitem{g:transfer-of-learning}{Transferencia de aprendizaje} Applying knowledge learned
in one context to problems in another context.  Ver también
\grefcross{g:near-transfer}{transferencia cercana} and \grefcross{g:far-transfer}{transferencia lejana}.

\gitem{g:transfer-appropriate-processing}{Transferencia apropiada de procesamiento} The
improvement in recall that occurs when practice uses activities similar to those
used in testing.

\gitem{g:tutorial}{Tutorial} A lesson intended to help someone improve their
general understanding of a subject.

\gitem{g:twitch-coding}{Twitch coding} Having a group of people decide moment
by moment or line by line what to add to a program next.

\gitem{g:working-memory}{Memoria de trabajo} see \grefcross{g:short-term-memory}{Memoria de corto plazo}.

\gitem{g:zpd}{Zona de Desarrollo Proximal} (\emph{ZPD}, por sus siglas en Inglés) Includes the set of problems that
people cannot yet solve on their own but are able to solve with assistance from
a more experienced mentor.  Ver también \grefcross{g:productive-failure}{falla productiva}.

\end{description}
