\chapter*{Sobre la traducción}

Este es el sitio web de la versión en español, \textbf{aún en proceso de traducción}, de \emph{Teaching Tech Together} de Greg Wilson.
La traducción de \emph{Enseñar Tecnología en Comunidad} es un proyecto colaborativo
de la comunidad de \hreffoot{https://rladies.org/}{R-Ladies}
y de \hreffoot{https://www.metadocencia.org/}{MetaDocencia} en Latinoamérica,
que tiene por objetivo traducir al español material actualizado y de calidad para hacerlo accesible a hispanohablantes.
Iniciamos la traducción en Marzo del año 2020.

Quienes trabajamos en este proyecto somos (en orden alfabético):
\hreffoot{https://twitter.com/_lacion_}{Laura Acion},
\hreffoot{https://twitter.com/MonicaLA2000}{Mónica Alonso},
\hreffoot{https://twitter.com/Zjbb}{Zulemma Bazurto},
\hreffoot{https://twitter.com/AlejaBellini}{Alejandra Bellini},
\hreffoot{https://twitter.com/yabellini}{Yanina Bellini Saibene},
\hreffoot{https://twitter.com/July_Benitezs}{Juliana Benitez Saldivar},
\hreffoot{https://twitter.com/ruthy_root}{Ruth Chirinos},
\hreffoot{https://twitter.com/PaobCorrales}{Paola Corrales},
\hreffoot{https://twitter.com/anadiedrichs}{Ana Laura Diedrich},
\hreffoot{https://twitter.com/patriloto}{Patricia Loto},
\hreffoot{https://twitter.com/pmnatural}{Priscilla Minotti},
\hreffoot{https://twitter.com/Nat_Mora_}{Natalia Morandeira},
\hreffoot{https://twitter.com/_luciarp_}{Lucía Rodríguez Planes},
\hreffoot{https://twitter.com/palolili23}{Paloma Rojas},
\hreffoot{https://twitter.com/GabySandovalM}{Gabriela Sandoval},
\hreffoot{https://twitter.com/YkSosaP}{Yuriko Sosa}
y \hreffoot{https://twitter.com/_yarena}{Yara Terrazas-Carafa}.
La coordinación del trabajo está a cargo de Yanina Bellini Saibene y la edición final a cargo de Yanina Bellini Saibene y Laura Ación.

Malena Zabalegui nos aconsejó sobre el uso de lenguaje no sexista e inclusivo para la realización de esta traducción.

También generamos un
\hreffoot{https://yabellini.shinyapps.io/T3Glossary/}{glosario y diccionario bilingüe de términos de educación y tecnología}
a partir del glosario del libro y del listado de términos a traducir (o no) del libro.
El desarrollo de este glosario está a cargo de Yanina Bellini Saibene.

Todos los detalles del proceso de traducción se pueden consultar
\hreffoot{https://github.com/gvwilson/teachtogether.tech/blob/master/es/README.md}{en la documentación del proyecto}.
