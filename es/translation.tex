\chapter*{Sobre la traducción}

Este libro es la versión en español de \emph{Teaching Tech Together} de Greg Wilson.
La traducción de \emph{Enseñar Tecnología en Comunidad} es un proyecto colaborativo
de la comunidad de \hreffoot{https://rladies.org/}{R-Ladies}
y de \hreffoot{https://www.metadocencia.org/}{MetaDocencia} en Latinoamérica,
que tiene por objetivo traducir al español material actualizado y de calidad para hacerlo accesible a hispanohablantes.
Iniciamos el proceso de traducción en marzo del año 2020 y lo completamos en marzo del 2021.

El trabajo se organizó de manera que cada capítulo tuvo una persona asignada a cargo de la traducción 
y dos personas que realizaron las revisiones de cada capítulo.  
Se buscó que el idioma de las traductoras y revisoras tuviera origen en diferentes países para
 poder considerar las diferentes y hermosas formas en que hablamos español en todo el mundo.
Al finalizar todo el proceso se realizó una edición final de todo el libro en su conjunto.

Tomamos algunas decisiones para el proceso de traducción basado en experiencias previas
del equipo y otras guías de traducciones colaborativas al español como {R para Ciencia de Datos}{}
y {The Carpentries}{}:

La variedad dialéctica del español (castellano) utilizada en la traducción corresponde 
a Latinoamérica y se utilizó una voz conversacional en lugar de una voz formal o académica.

Decidimos intentar ajustar la redacción para evitar la marca de género, pero
en caso de no poder evitar su uso, decidimos utilizar lenguaje no sexista  
que implica el uso del femenino y masculino privilegiando la agilidad y fluidez del texto, 
que el mismo se entienda y que sea claro el mensaje. Para que haya coherencia 
a lo largo del texto y mostrar que no hay una determinada jerarquía 
alternamos el uso del femenino/masculino o masculino/femenino entre capítulos 
y el uso fue consistente durante todo el capítulo. 

También decidimos buscar las versiones al español de referencias como 
entradas en Wikipedia y lecciones de The Carpentries.  En caso que no exitieran 
se dejaron las versiones en inglés.

Finalmente, se decidió cambiar algunos ejemplos a realidades más regionales, 
para que sean más cercanos a la región de donde son la mayoría de las
traductoras.

Quienes trabajamos en este proyecto somos (en orden alfabético):
Laura Acion,
Mónica Alonso,
Zulemma Bazurto,
Alejandra Bellini,
Yanina Bellini Saibene,
Juliana Benitez Saldivar,
Lupe Canaviri Maydana,
Silvia Canelón,
Ruth Chirinos,
Paola Corrales,
María Dermit,
Ana Laura Diedrich,
Patricia Loto,
Priscilla Minotti,
Natalia Morandeira,
Lucía Rodríguez Planes,
Paloma Rojas Saunero,
Yuriko Sosa,
Natalie Stroud,
y Yara Terrazas-Carafa.

En cada capítulo encontrarás el detalle de las personas que estuvieron a cargo de traducirlo
y revisarlo.

La coordinación del trabajo estuvo a cargo de Yanina Bellini Saibene y 
la edición final a cargo de Yanina Bellini Saibene y Natalia Morandeira.

Malena Zabalegui nos aconsejó sobre el uso de lenguaje no sexista e 
inclusivo para la realización de esta traducción y Francisco Etchart 
diseñó el hex sticker.

También generamos un
\hreffoot{https://yabellini.shinyapps.io/T3Glossary/}{glosario y 
diccionario bilingüe de términos de educación y tecnología}
a partir del glosario del libro y del listado de términos 
a traducir (o no) del libro.
El desarrollo de este glosario estuvo a cargo de Yanina Bellini Saibene utilizando
{glosario}{}.

Todos los detalles del proceso de traducción se pueden consultar
\hreffoot{https://github.com/gvwilson/teachtogether.tech/blob/master/es/README.md}{en la documentación del proyecto}.
