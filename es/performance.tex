\chapter{Enseñar como un arte performativo}\label{s:performance}

En \emph{Darwin entre las máquinas},
George Dyson\index{Dyson, George} escribió,
``En el juego de la vida y la evolución hay 3 jugadores en la mesa:
los humanos, la naturaleza y las máquinas.
Estoy firmemente del lado de la naturaleza.
Pero la naturaleza, sospecho, está del lado de las máquinas{\ldots}''
De manera similar, ahora hay 3 jugadores en el juego de la educación:
los libros de texto y otros materiales de lectura,
las clases en vivo,
y las clases en línea automatizadas.
Podrías darle a tus estudiantes clases escritas y alguna combinación
de videos grabados y ejercicios para que realicen a su propio ritmo,
pero si vas a enseñar en persona tienes
que ofrecer algo diferente (y con suerte mejor que) cualquiera de los anteriores.
Por lo tanto, este capítulo se enfoca en cómo enseñar a programar programando.

\seclbl{Programar en vivo}{s:performance-live}

\begin{quote}

  La enseñanza es teatro, no cine. \\
  --- Neal Davis\index{Davis, Neal}

\end{quote}

La manera más efectiva de enseñar a programar es \gref{g:live-coding}{programando en vivo}~\cite{Rubi2013,Haar2017,Raj2018}.
En vez de presentar material previamente escrito,
quien enseña escribe el código en frente de la clase
mientras que los/las estudiantes lo siguen a la par,
escribiendo y ejecutándo el código a medida que avanzan.
Programar en vivo funciona mejor que las presentaciones por varias razones:

\begin{itemize}

\item
  Permite la \gref{g:active-teaching}{enseñanza activa}
  al permitir a quienes están enseñando responder a los intereses y preguntas de los/las estudiantes en el momento.
  Una presentación de diapositivas es como una vía de ferrocarril:
  podrá ser un viaje suave,
  pero tienes que decidir hacia donde vas con anticipación.
  Programar en vivo es cómo manejar un vehículo todo terreno:
  podrá ser más accidentado,
  pero es mucho más fácil cambiar de dirección e ir hacia donde la gente quiere.

\item
  Mirar como se va escribiendo un programa es más motivador
  que mirar a alquien pasar diapositivas.

\item
  Facilita la transferencia de conocimiento de manera involuntaria:\index{unintended knowledge transfer}
  las personas aprenden más de lo que enseñamos conscientemente
  al observar \emph{cómo} hacemos las cosas.

\item
  Disminuye la velocidad de la persona que está enseñando:
  si tiene que escribir el programa medida que avanza,
  entonces solo puede ir el doble de rápido que sus estudiantes
  en vez de 10 veces más rápido como lo harían usando diapositivas.

\item
  Ayuda a reducir la carga en la memoria de corto plazo
  porque hace que quien esté enseñando se más consciente de cuanto 
  le está mostrando a sus estudiantes.

\item
  Los/Las estudiantes pueden ver cómo diagnosticar y corregir errores.
  Van a dedicar mucho tiempo a esto;
  a menos que puedan tipear de manera perfecta,
  programar en vivo asegura que puedan ver como hacer esto.

\item
  Ver a quienes enseñan cometer errorer muestra a sus estudiantes que está bien que cometan errores.
  Si el/la docente no se avergüenza al cometer errores y habla sobre ellos,
  sus estudiantes también se sentirán más cómodos/as hacíendolo.

\end{itemize}

Otro beneficio de la programación en vivo es que demuestra el orden en que se deben escribir los programas.
Cuando observaron cómo las personas resuelven problemas de Parson,\index{Parsons Problem}
\cite{Ihan2011} encontraron que las personas con experiencia programando usualmente ubican la identificación del método al principio,
luego agregan la mayor parte del control de flujo (es decir, bucles y condiciones),
y solo luego de eso, agregan detalles como la inicialización de variables y el manejo de casos especiales.
Es método "fuera de orden" es ajeno para las personas novatas,
que leen y escriben código en el orden en que se presenta en la página;
ver el código les ayuda a descomponer los problemas en submetas que pueden abordar una a la vez.
La programación en vivo además les da quienes están enseñando la chance de enfatizar la importancia de los pequeños pasos con comentarios frecuentes~\cite{Blik2014}
y la importancia de definir un plan en vez de hacer 
cambios más o menos aleatorios y esperar que las cosas mejoren~\cite{Spoh1985}.

Sentise cómodo/a al hablar mientras se escribe código en 
frente de una audiencia, requiere práctica,
pero la mayoría de las personas indican que rápidamente se vuelve igual que difícil que hablar alrededor de una presentación de diapositivas.
Las secciones que siguen ofrecen consejos sobre cómo mejorar la manera de programar en vivo.

\subsection*{Aprovecha tus errores}

\begin{quote}

  Los errores de tipeo son la pedagogía. \\
  --- Emily Jane McTavish\index{McTavish, Emily Jane}

\end{quote}

La regla más importante de la programación en vivo es aprovechar tus errores.\index{mistakes (importance of embracing)}
No importa que tan bien te prepares, 
cometerás algunos errores;
cuando lo hagas,
piensa sobre ellos con tu audiencia.
Si bien obtener los datos es difícil,
programadores/as profesionales dedican del 25\% al 60\% de su tiempo identificando y resolviendo errores;
las personas novatas dedican le dedicanmucho más (\secref{s:pck-debug}),
pero la mayoría de los libros de texto y tutoriales dedican poco tiempo a diagnosticar y corregir problemas.
Si hablas en voz alta mientras intentas identificar que escribiste mal
o dónde tomaste el camino equivocado,
y explicas cómo lo corriges,
les darás a tus estudiantes un conjunto de herramientras que pueden usar cuando comentan sus propios errores.

\begin{aside}{Tropiezos deliberados}
  Una vez que hayas enseñado una lección varias veces,
  es poco probable que cometas nada más que errores básicos de tipeo
  (que de todas maneras pueden ser informativos).
  Puede intentar recordar errores pasados y cometerlos deliberadamente,
  pero usualmente eso se siente forzado.
  Un enfoque alternativo es \gref{g:twitch-coding}{sacudir la programación}:
  pide a tus estudiantes, uno a uno que te indiquen que escribir a continnuación.
  Esto practicamente garantiza que te encuentres en algún tipo de problemas.
\end{aside}

\subsection*{Pregunta por predicciones}

Una manera de mantener a tus estudiantes motivados/as mientras estas programando en vivo
es pedirles que hagan predicciones sobre que hará el código que ven en la pantalla.
Luego, puedes escribir las primeras sugerencias que hagan,
hacer que toda la clase vote sobre cúal piensan que es la opción más probable,
y finalmente ejecutar el código.
Esta es una forma simple de instrucción entre pares,\index{peer instruction}
que discutiremos en la sección \secref{s:classroom-peer};
además de mantener su atención en la actividad,
les permite practicar cómo razona sobre el comportamiento del código.

\subsection*{Tomalo con calma}

Cada vez que escribas un comando,
agregues una lidea de código a un programa,
o selecciones un elemento de un menú,
di que estas haciendo en voz alta,
luego señala lo que haz hecho y su resultado en la pantalla
y repasalo una segunda vez.
Esto ayuda a tus estudiantes a ponerse al día
y a revisar que lo que acaban de hacer es correcto.
Esto es particularmente importante cuando algunos de tus estudiantes tienen dificultades para ver o escuchar o no dominan el idioma en el que estás enseñando.

Hagas lo que hagas,
\emph{no} copies y pegues códgio:
hacer eso practicamente garantiza que iras mucho más rápido que tus estudiantes.
Y si usas la tecla tab para autocompletar lo que estás escribiendo,
decilo en vos alta para que tus estudiantes entiendan lo que estás haciendo:
``Usemos turtle dot `r' `i' y tab para completar con `right'.''

Si la salida de tu comando o código hace que lo que acabas de escribir desaparezca de la vista,
volvé arriba para que tus estudiantes puedan verlo de nuevo.
Si eso no es posible,
ejecuta el mismo comando una segunda vez
o copia y pega el último comando o comandos en las notas compartidas del taller.

\subsection*{Ser visto/a y escuchado/a}

Cuando te sientas,
es más probable que mires tu pantalla en vez de mirar a tu audiencia
y puedes quedar fuera de la vista de tus estudiantes en las últimas filas del aula.
Si eres físicamente capaz de pararte durante un par de horas,
debes hacerlo mientras enseñas.
Planifíscalo y asegúrate de tener una mesa elevada, 
un escritorio de pie,
o un atril
para tu computadora portátil
para que no tengas que inclinarte al escribir.

Independientemente de si estás de pie o sentado/a,
asegurate de moverte lo más que puedas:
acercate a la pantalla para señalar algo,
dibuja algo en la pizarra,
o simplemente alejate de la computadora por un momento y hablale directamente a tu audiencia.
Hacer esto aleja la atención de tus estudiantes de sus pantallas
y les proporciona un mmomento natural para hacer preguntas.

Si vas a enseñar por más de un par de horas,
vale la pena usar un micrófono incluso si la habitación es pequeña.
Tu garganta se cansa tanto como cualquier otra parte de tu cuerpo;
usar un micrófono no es diferente de usar zapatos cómodos 
(algo que también deberías usar).
También puede marcar una gran diferencia para las personas que tienen discapacidad auditiva.

\subsection*{Copia la pantalla de tu estudiante}

Es posible que hayas personalizado tu entorno de trabajo con una terminal de Unix shell elegante,
un esquema de colores personalizado,
o una gran cantidad de atajos de teclado.
Tu estudiantes no tendrán nada de eso,
así que intenta crear un entorno de trabajo que refleje lo que \emph{sí} tienen.
Algunos/as docentes crean un usuario distinto con configuración básica en sus computadoras
o una cuenta específica para enseñar 
si están usando algún servicio online como Scratch o GitHub.
Hacer esto también puede ayudar a evitar que los paquetes que instalaste ayer para trabajar 
rompan la lección que se supone que enseñes hoy.

\subsection*{Usa la pantalla de menera sabia}

Por lo general, necesitaras agrandar el tamañana de la letra considerablemente
para que las personas en el fondo de la sala puedan leer. 
Esto significa que podrás colocar muchas menos cosas en la pantalla de las que estás acostumbrado/a.
En muchos casos, se reducirá a 60--70 columnas y 20--30 filas,
por lo que estarás usando una super computadora del siglo 21
como si fuera una sencilla terminal de principios de la década de 1980.

Para organizar esto,
maximiza la ventana que estás usando para enseñar
y luego preguntale a todos si pueden leer lo que están en la pantalla o no.
Usa una fuente de color negro sobre un fondo ligeramente coloreado en vez de una fuente de color claro sobre un fondo oscuro---el tono claro deslubrará 
menos que el blanco puro.

Presta atención a la iluminación de la sala:
no debe estar completamente a oscuras, y no debe haber luces directamente 
o por encima de la pantalla de protección.
Dedica algunos minitos para que tus estudiantes puedan reacomodar sus mesas
para ver con claridad.

Cuando la parte inferior de la proyección de la pantalla está a la misma altura que las cabezas de tus estudiantes,
las personas en el fondo no podrán ver lo que ocurre en esa sección de la pantalla.
Puede elevar la parte inferior de la ventana para compesar esto,
pero eso generará que tengas aún menos espacio para escribir.

Si puedes acceder a un segundo proyector y pantalla,
usalos:
el espacio adicional te permitirá mostrar el código de un lado
y su resultado o comportamiento del otro lado.
Si la segunda pantalla requiere su propia computadora,
pidele a un/a ayudante que la controle
en lugar de ir y venir entre los dos teclados.

Finalmente,
si estás enseñando algo como la terminal de Unix shell en una consola,
es importante decirle a las personas usas un editor de texto en la consola
y cuando regresas a la consola propiamente dicha.
La mayoría de las personas novatas no han visto nunca una ventana asumir multiples personalidades de esta manera,
y pueden confundirse rápidamente
cuando estás interactuando en la terminal,
cuando estás escribiendo en un editor de texto,
y cuando estás trabajando de manera interactiva con Python u otro lenguaje.
Puedes evitar este problema usando ventanas separadas para usar el editor de texto;
si haces esto,
siempre avísales a tus estudiantes cuando estás cambiando de una ventana a la otra.

\begin{aside}{Las herramientas de accesibilidad ayudan a todas las personas}
  Las herramientas como \hreffoot{https://boinx.com/mousepose/overview/}{Mouseposé} (para Mac)
  y \hreffoot{http://www.pointerfocus.com/}{PointerFocus} (para Windows)
  resaltan la posición del cursor del mouse en la pantalla,
  y las herramientras de grabación de pantalla como \hreffoot{https://www.techsmith.com/video-editor.html}{Camtasia}
  y aplicaciones independientes como \hreffoot{https://github.com/keycastr/keycastr}{KeyCastr}
  muestran teclas invisibles como tab y Control-J a medida que las usas.
  Esto puede ser un poco molesto al comienzo,
  pero ayuda a tus estudiantes a descubrir lo que estás haciendo.
\end{aside}

\subsection*{Dos dispositivos}

Algunas personas usan dos dispositivos cuando enseñan:
una computadora portatil conectada al proyector que los/as estudiantes vean
y una tablet para que puedan ver sus propias notas y las notas que los/as estudiantes están tomando (\secref{s:classroom-notetaking}).
Esto es más confiable que pasar de un escritorio virtual al otro, 
aunque imprimir la lección sigue siendo la técnología de respaldo más confiable.

\subsection*{Dibuja temprano, dibuja seguido}

Los diagramas son siempre una buena idea.
A veces tengo una presentación de diapostivas llena de diagramas preparada de antemano,
pero construir los diagramas paso a paso ayuda a retenerlos más (\secref{s:architecture-brain})
y te permite improvisar.

\subsection*{Evita las distracciones}

Desactiva las notificaciones que usas en tu computadora,
espcialmente las de redes sociales.
Ver mensajes parpadeando en la pantalla te distrae a vos y a tus estudiantes,
y puede ser incómodo si aparece un mensaje que no te gustaría que otras personas vean.
De nuevo,
es posible que quieras crear una segunda cuenta en tu computadora que no tenga correo electrónico u otras herramientas configuradas.

\subsection*{Improvisa---luego de haber aprendido el material}

No te alejes de la lección que planificaste o pediste prestada la primera vez que la enseñes.
Puede ser tentandor desviarse del material
porque te gustaría mostrar un lindo truco o demostrar otra manera de hacer algo,
pero existe la posibilidad de que te encuentres con algo inesperado 
que te lleve más tiempo del que tenés.

Sin embargo, una vez que el material te resulte más familiar,
puedes y debes comenzar a improvisar en base a los antecedentes de tus estudiantes,
sus preguntas durante la clase,
y lo que personalmente te parezca más interesante.
Esto es como tocar una nueva canción:
sigues la partitura las primeras veces,
pero después de que te sientes comodo/a con los cambios de melodía y acordes,
puedes comenzar a ponerle tu propio sello.

Cuando quieras usar algo nuevo,
revísalo de antemano
\emph{usando la misma computadora que usarás cuando des la clase}:
instalar cientos de megabytes de programas a través del WiFi de la escuela secundaria
en frente de joneves de 16 años aburridos no es algo por lo que alguna vez quieras pasar.

\begin{aside}{Enseñanza directa}
  La \gref{g:direct-instruction}{Enseñanza directa} (ED) es un método de enseñanza
  centrando en el diseño meticuloso de la currícula dictado usando un guío predefinido.
  Es más como un actor recitando lineas que como el enfoque de inprovisación que recomendamos.
  \cite{Stoc2018} encontró que la ED tiene un efecto estadísticamente significativo positivo 
  a pesar de que a veces pueda ser muy repetitivo.
  Yo prefiero improvisar porque la ED requiere más preparación inicial que lo que la mayoría 
  de los grupos de estudiantes free-range pueden permitirse.
\end{aside}

\subsection*{Mira a la pantalla---de vez en cuando}

Está bien enfrentar la pantalla donde estás proyectando ocacionalmente
cuando estás mostrando una sección de código o dibujando un esquema:
\emph{no} mirar a la sala llena de personas que te están mirando a vos
puede ayudarte a reducir tu nivel de ansiedad y darte un momento para pensar qué decir a continuación.

Sin embargo, no deberías hacerto por más de unos segundos.
Una buena regla general es tratar a la pantalla como a uno/a de tus estudiantes:
si mirar a una persona durante el tiempo que miras a la pantalla te resulta incómodo, es hora de darte la vuelta y mirar a la clase nuevamente.

\subsection*{Incovenientes}

Porgramar en vivo tiene algunos inconvenientes,
pero pueden evitarse o solucionarse con un poco de práctica.
Si descrubres que están cometiendo demasiados errores de tipeo,
reserva 5 minutos por día para practicar escribir con el teclado:
también te ayudará en tu trabajo diario.
Si crees que dependes demasiado de las notas de la clase,
divídelas en partes más pequeñas 
para que solo tengas que pensar en un pequeño paso a la vez.

Y si sientes que estás pasando demasiado tiempo escribiendo código para importar librerías, encabezados de clases y código repetitivo, 
genera un esqueleto de código para que vos y tus estudiantes usen como punto de partida (\secref{s:classroom-blank}).
Hacer esto también reducirá su carga cognitiva,
dado que centrarán su atención donde vos quieres.

\seclbl{Estudiar la lección}{s:performance-jugyokenkyu}

Desde políticos hasta investigador/as y docentes,
quines reforman la educación han diseñado sistemas 
para encontrar y apoyar a personas que pueden enseñar bien
y eliminar a las personas no lo hacen.
Pero la suposición de que algunas personas nacen como docentes es errónea:
en cambio,
como cualquier otra representación artística,
las claves para enseñar mejor son práctica y colaboración.
Cómo explica~\cite{Gree2014},
En japonés este enfoque se llama \gref{g:jugyokenkyu}{jugyokenkyu},
que signofica ``estudiar la lección'':

\begin{quote}

  Para graduarse,
  los especialistas en educación [japoneses] no solo tenían que ver como trabaja el docente que le asignan,
  tenían que reemplazarlo efectivamente,
  participando en su aula primero como observadores/as y luego,
  a la tercera semana,
  como una aproximación{\ldots}titubeante del propio maestro.
  Funcionó como una especie de relevo de docentes.
  Cada estudiante eligió una asignatura,
  preparando clases para 5 días{\ldots} [y luego] cada uno enseñó un día.
  Para pasar la batuta,
  tenía que enseñar una clase de un día en cada asignatura:
  la que tenían planeada y las 4 que no{\ldots}
  y tenían que hacerlo delante de su maestro.
  Después, todos---el docente, los estudiantes para ser docentes,
  y a veces, uncluso un observador externo---se sentaban alrededor de una mesa
  para hablar de lo que observaron.

\end{quote}

Poner el trabajo bajo un microscopio para mejorarlo es común
en áreas tan diversas como la \hreffoot{https://en.wikipedia.org/wiki/W.\_Edwards\_Deming}{fabricación} y la múscia.
Un/a músico/o profesional,
por ejemplo,
analizará media docena de grabaciones de ``Body and Soul'' o ``Smells Like Teen Spirit'' antes de interpretarlas.
También recibirán comentarios de colegas musicos durante la práctica y después de las actuaciones.

Pero la retroalimentación continua no es parte de la cultura de la enseñanzan en la mayoría de los paises de habla inglesa.
Allí,
lo que sucede en el aula se queda en el aula:
quienes enseñan no miran las clases de sus colegas de manera regular,
por lo que no pueden tomar prestadas las buenas ideas de las demás personas.
Los/as docentes podrán acceder a los planes de clases y tareas de otros colegas,
la junta escolar o una editorial de libros de texto,
o revisar MOOCs en Internet,
pero cada persona tiene que descubrir
como dar las clases específicas en aulas específicas para estudiantes específicos/as.
Esto es particularmente cierto para personas voluntarias y docentes free-range
que participan en talleres y actividades fuera de la escuela.

Writing up new techniques
and giving \grefdex{g:demonstration-lesson}{demonstration lessons}{demonstration lesson}
(in which one person teaches actual learners while other teachers observe)
are not solutions.
For example,
\cite{Finc2007,Finc2012} found that of the 99 change stories analyzed,
teachers only searched actively for new practices or materials in three cases,
and only consulted published material in eight.
Most changes occurred locally,
without input from outside sources,
or involved only personal interaction with other educators.
\cite{Bark2015} found something similar:

\begin{quote}

  Adoption is not a ``rational action''{\ldots}but
  an iterative series of decisions made in a social context,
  relying on normative traditions, social cueing,
  and emotional or intuitive processes{\ldots}
  Faculty are not likely to use educational research findings
  as the basis for adoption decisions{\ldots}
  Positive student feedback is taken as strong evidence by faculty
  that they should continue a practice.

\end{quote}

\emph{Jugyokenkyu} works because it maximizes the opportunity for unplanned knowledge transfer between teachers:
someone sets out to demonstrate X,
but while watching them,
their audience actually learns Y as well (or instead).
For example,
a teacher might intend to show learners how to search for email addresses in a text file,
but what their audience might take away is some new keyboard shortcuts.

\seclbl{Giving and Getting Feedback on Teaching}{s:performance-feedback}

Observing someone helps you,
and giving them feedback helps them,
but it can be hard to receive feedback,
especially when it's negative (\figref{f:performance-feedback-feelings}).

\figimg{figures/deathbulge-jerk.jpg}{Feedback feelings (copyright © Deathbulge 2013)}{f:performance-feedback-feelings}

Feedback is easier to give and receive when both parties share expectations
about what is and isn't in scope
and about how comments ought to be phrased.
If you are the person asking for feedback:
\index{feedback!getting}

\begin{description}

\item[Initiate feedback.]
  It's better to ask for feedback than to receive it unwillingly.

\item[Choose your own questions,]
  i.e.\ ask for specific feedback.
  It's a lot harder for someone to answer,
  ``What do you think?''
  than to answer either,
  ``Was I speaking too quickly?''
  or ,
  ``What is one thing from this lesson I should keep doing?''
  Directing feedback like this is also more helpful to you.
  It's always better to try to fix one thing at once
  than to change everything and hope it's for the better.
  Directing feedback at something you have chosen to work on helps you stay focused,
  which in turn increases the odds that you'll see progress.

\item[Use a feedback translator.]
  Have someone else read over all the feedback and give you a summary.
  It can be easier to hear,
  ``Several people think you could speed up a little,''
  than to read several notes all saying, ``This is too slow''
  or, ``This is boring.''

\item[Be kind to yourself.]
  Many of us are very critical of ourselves,
  so it's always helpful to jot down what we thought of ourselves
  \emph{before} getting feedback from others.
  That allows us to compare what we think of our performance
  with what others think,
  which in turn allows us to scale the former more accurately.
  For example,
  it's very common for people to think that they're saying ``um'' and ``err'' too often
  when their audience doesn't notice it.
  Getting that feedback once allows teachers to adjust their assessment of themselves
  the next time they feel that way.

\end{description}

\noindent
You can give feedback to others more effectively as well:
\index{feedback!giving}

\begin{description}

\item[Interact.]
  Staring at someone is a good way to make them feel uncomfortable,
  so if you want to give feedback on how someone normally teaches,
  you need to set them at ease.
  Interacting with them the way that a real learner would is a good way to do this,
  so ask questions or (pretend to) type along with their example.
  If you are part of a group,
  have one or two people play the role of learner
  while the others take notes.

\item[Balance positive and negative feedback.]
  The ``compliment sandwich'' made up of one positive comment,
  one negative,
  and a second positive
  becomes tiresome pretty quickly,
  but it's important to tell people what they should keep doing
  as well as what they should change\footnote{
    For a while,
    I was so worried about playing in tune that I completely lost my sense of timing.
  }.

\item[Take notes.]
  You won't remember everything you noticed
  if the presentation lasts longer than a few seconds,
  and you definitely won't recall how often you noticed them.
  Make a note the first time something happens
  and then add a tick mark when it happens again
  so that you can sort your feedback by frequency.

\end{description}

Taking notes is more efficient when you have some kind of rubric
so that you're not scrambling to write your observations
while the person you're observing is still talking.
The simplest rubric for free-form comments from a group
is a 2x2 grid whose vertical axis is labeled ``what went well'' and ``what can be improved'',
and whose horizontal axis is labeled ``content'' (what was said)
and ``presentation'' (how it was said).
Observers write their comments on sticky notes as they watch the demonstration,
then post those in the quadrants of a grid drawn on a whiteboard
(\figref{f:performance-rubric}).

\figpdf{figures/2x2-rubric.pdf}{Teaching rubric}{f:performance-rubric}

\begin{aside}{Rubrics and Question Budgets}
  \secref{s:checklists-teacheval} contains a sample rubric
  for assessing 5--10 minutes of programming instruction.
  It presents items in more or less the order that they're likely to come up,
  e.g.\ questions about the introduction come before questions about the conclusion.

  Rubrics like this one
  tend to grow over time as people think of things they'd like to add.
  A good way to keep them manageable is to insist that
  the total length stays constant:
  if someone wants to add a question,
  they have to identify one that's less important and can be removed.
\end{aside}

If you are interested in giving and getting feedback,
\cite{Gorm2014} has good advice
that you can use to make peer-to-peer feedback a routine part of your teaching,
while~\cite{Gawa2011} looks at the value of having a coach.

\begin{aside}{Studio Classes}
  Architecture schools often include studio classes\index{studio class}
  in which students solve small design problems
  and get feedback from their peers right then and there.
  These classes are most effective when the teacher critiques the peer critiques
  so that participants learn not only how to make buildings
  but how to give and get feedback~\cite{Scho1984}.
  Master classes in music serve a similar purpose,
  and I have found that giving feedback on feedback
  helps people improve their teaching as well.
\end{aside}

\seclbl{How to Practice Performance}{s:performance-practice}

The best way to improve your in-person lesson delivery
is to watch yourself do it:

\begin{itemize}

\item
  Work in groups of three.

\item
  Each person rotates through the roles of teacher, audience, and videographer.
  The teacher has 2 minutes to explain something.
  The person pretending to be the audience is there to be attentive,
  while the videographer records the session using a cellphone or other handheld device.

\item
  After everyone has finished teaching,
  the whole group watches the videos together.
  Everyone gives feedback on all three videos,
  i.e.\ people give feedback on themselves as well as on others.

\item
  After the videos have been discussed,
  they are deleted.
  (Many people are justifiably uncomfortable about images of themselves appearing online.)

\item
  Finally,
  the whole class reconvenes
  and adds all the feedback to a shared 2x2 grid of the kind described above
  \emph{without} saying who each item of feedback is about.

\end{itemize}

In order for this exercise to work well:

\begin{itemize}

\item
  Record all three videos and then watch all three.
  If the cycle is teach-review-teach-review,
  the last person to teach invariably runs short of time
  (sometimes on purpose).
  Doing all the reviewing after all the teaching
  also helps put a bit of distance between the two,
  which makes the exercise slightly less excruciating.

\item
  Let people know at the start of the class that they will be asked to teach something
  so that they have time to choose a topic.
  Telling them this too far in advance can be counter-productive,
  since some people will fret over how much they should prepare.

\item
  Groups must be physically separated to reduce audio cross-talk between their recordings.
  In practice,
  this means 2--3 groups in a normal-sized classroom,
  with the rest using nearby breakout spaces, coffee lounges, offices,
  or (on one occasion) a janitor's storage closet.

\item
  People must give feedback on themselves as well as on each other
  so that they can calibrate their impressions of their own teaching
  against those of other people.
  Most people are harder on themselves than they ought to be,
  and it's important for them to realize this.

\end{itemize}

The announcement of this exercise is often greeted with groans and apprehension,
since few people enjoy seeing or hearing themselves.
However,
those same people consistently rate it as one of the most valuable parts of teaching workshops.
It's also good preparation for co-teaching (\secref{s:classroom-together}):\index{co-teaching}
teachers find it a lot easier to give each other informal feedback
if they have had some practice doing so
and have a shared rubric to set expectations.

And speaking of rubrics:
once the class has put all of their feedback on a shared grid,
pick a handful of positive and negative comments,
write them up as a checklist,
and have them do the exercise again.
Most people are more comfortable the second time around,
and being assessed on the things that they themselves have decided are important
increases their sense of self-determination (\chapref{s:motivation}).

\begin{aside}{Tells}
  We all have nervous habits:
  we talk more rapidly and in a higher-pitched voice than usual when we're on stage,
  play with our hair,
  or crack our knuckles.
  Gamblers call these ``tells,''
  and people often don't realize that they pace,
  look at their shoes,
  or rattle the change in their pocket
  when they don't actually know the answer to a question.

  You can't get rid of tells completely,
  and trying to do so can make you obsess about them.
  A better strategy is to try to displace them---for example,
  to train yourself to scrunch your toes inside your shoes when you're nervous
  instead of cleaning your ear with your pinky finger.
\end{aside}

\seclbl{Exercises}{s:performance-exercises}

\exercise{Give Feedback on Bad Teaching}{whole class}{20}

As a group,
watch \hreffoot{https://www.youtube.com/watch?v=-ApVt04rB4U}{this video of bad teaching}
and give feedback on two axes:
positive vs.\ negative and content vs.\ presentation.
Have each person in the class add one point to a 2x2 grid on a whiteboard or in the shared notes
without duplicating any points.
What did other people see that you missed?
What did they think that you strongly agree or disagree with?

\exercise{Practice Giving Feedback}{small groups}{45}

Use the process described in \secref{s:performance-practice}
to practice teaching in groups of three
and pool feedback.

\exercise{The Bad and the Good}{whole class}{20}

Watch the videos of \hreffoot{https://youtu.be/bXxBeNkKmJE}{live coding done poorly}
and \hreffoot{https://youtu.be/SkPmwe\_WjeY}{live coding done well}
and summarize your feedback on the differences using the usual 2x2 grid.
How is the second round of teaching better than the first?
Is there anything that was better in the first than in the second?

\exercise{See, Then Do}{pairs}{30}

Teach 3--4 minutes of a lesson using live coding to a classmate,
then swap and watch while that person live codes for you.
Don't bother trying to record these sessions---it's difficult to capture
both the person and the screen with a handheld device---but
give feedback the same way you have previously.
Explain in advance to your fellow trainee what you will be teaching
and what the learners you teach it to are expected to be familiar with.

\begin{itemize}

\item
  What felt different about live coding compared to standing up and lecturing?
  What was easier or harder?

\item
  Did you make any mistakes?
  If so, how did you handle them?

\item
  Did you talk and type at the same time, or alternate?

\item
  How often did you point at the screen?
  How often did you highlight with the mouse?

\item
  What will you try to keep doing next time?
  What will you try to do differently?

\end{itemize}

\exercise{Tells}{small groups}{15}

\begin{enumerate}

\item
  Make a note of what you think your tells are,
  but do not share them with other people.

\item
  Teach a short lesson (2--3 minutes long).

\item
  Ask your audience how they think you betray nervousness.
  Is their list the same as yours?

\end{enumerate}

\exercise{Teaching Tips}{small groups}{15}

The \hreffoot{http://csteachingtips.org/}{CS Teaching Tips} site
has a large number of practical tips on teaching computing,
as well as a collection of downloadable tip sheets.
Go through their tip sheets in small groups and classify each tip
according to whether you use it all the time,
use it occasionally,
or never use it.
Where do your practice and your peers' practice differ?
Are there any tips you strongly disagree with or think would be ineffective?

\section*{Review}

\figpdfhere{figures/conceptmap-feedback.pdf}{Concepts: Feedback}{f:performance-feedback}
