\chapter{Aprendizaje Individual}\label{s:individual}

Los capítulos previos han explorado como las/los docentes pueden ayudar a sus estudiantes.
Este capítulo se enfoca en cómo las/los estudiantes pueden ayudarse a si mismos 
al cambiar sus estrategias de estudio y descansando lo necesario.

La estrategia más efectiva es hacer el cambio de \gref{g:passive-learning}{aprendizaje pasivo} 
a \gref{g:active-learning}{aprendizaje activo}~\cite{Hpl2018},
ya que mejora la taza de rendimiento y reduce la taza de fracaso~\cite{Free2014}:

\begin{longtable}{ll}
\textbf{Pasivo}		& \textbf{Activo} \\
Leer sobre un tema		& Hacer ejercicios \\
Mirar un video			& Discutir un tema \\
Asistir a una clase		& Tratar de explicar un tema
\end{longtable}

\noindent
Haciendo referencia a nuestro modelo simplificado de la arquitectura cognitiva (\figref{f:arch-model}),\index{cognitive architecture}
el aprendizaje activo es más efectivo porque retiene la información nueva en la memoria a corto plazo por más tiempo,\index{short-term memory}
lo cual aumenta la chance que sea codificada con éxito y almacenada en la memoria a largo plazo.\index{long-term memory}
Y al usar la nueva información a medida que llega, 
tus estudiantes pueden construir y fortalecer los lazos entre la información nueva y la información que ya poseen, 
lo cual incrementa a la vez, las chances de que puedan recuperarla más tarde.

La otra clave para poder sacarle más provecho al aprendizaje es la \gref{g:metacognition}{metacognición},
o en otras palabras, pensar sobre lo que una/uno está pensando.
Así como las/los musicos escuchan lo que están tocando, 
y las/los buenas docentes reflexionan sobre su enseñanza (\chapref{s:performance}),
tus estudiantes aprenderán mejor y más rápido si hacen planes, 
fijan metas
y monitorean su progreso. 
Para tus estudiantes es difícil dominar estas habilidades en el abstracto---solo
diciéndoles que planeen no tiene ningún efecto---pero 
las lecciones pueden diseñarse para motivar buenas prácticas de estudio,
y al hacer referencia a estas prácticas en la clase 
ayudas a que tus estudiantes se den cuenta que aprender es una habilidad que pueden mejorar como cualquier otra~\cite{McGu2015,Miya2018}.

El gran premio es la \gref{g:transfer-of-learning}{transferencia del aprendizaje},
que ocurre cuando algo que hemos aprendido nos ayuda a aprender algo nuevo más rápido.
Investigadores distinguen entre \grefdex{g:near-transfer}{transferencia cercana}{transfer of learning!near transfer},
que ocurre entre áreas similares o relacionadas como las fracciones y decimales en matemáticas,
y \grefdex{g:far-transfer}{transferencia lejana}{transfer of learning!far transfer},
que ocurre entre dominios diferentes---por ejemplo,
la idea de que aprender ajedrez ayudará al razonamiento matemático y viceversa.

La transferencia cercana ocurre indudablemente---ningún tipo de aprendizaje
más allá de la simple memorización podría ocurrir si no fuera así---y
las/los docentes lo utilizan todo el tiempo
al dar a sus estudiantes ejercicios similares al material que acaba de ser presentado en la lección.
Sin embargo,
\cite{Sala2017} analizó varios estudios sobre la transferencia lejana
y concluye que:

\begin{quote}

  {\ldots}los resultados muestran un efecto pequeño a moderado.
  Sin embargo, el tamaño de efecto está inversamente relacionado a la calidad del diseño experimental.{\ldots}
  Concluimos que la transferencia lejana raramente sucede.

\end{quote}

Cuando la transferencia lejana sucede,
parece que sucede solamente cuando el tema ha sido dominado~\cite{Gick1987}.
En la práctica,
esto significa que aprender a programar no ayudará a jugar ajedrez y vice versa.

\seclbl{Seis estrategias}{s:individual-strategies}

Las/los psicólogos estudian el aprendizaje en una amplia variedad de formas,
pero han llegado a conclusiones similares sobre lo que realmente funciona~\cite{Mark2018}.
Los \hreffoot{http://www.learningscientists.org/}{Learning Scientists}
han catalogado seis de estas estrategias y
las resumieron en \hreffoot{http://www.learningscientists.org/downloadable-materials}{un set de afiches para descargar}.
Si enseñas estas estrategias a tus estudiantes,
y las mencionas por su nombre cuando las utilices en la clase,
las/los ayudas a aprender cómo aprender más rápido y mejor~\cite{Wein2018a,Wein2018b}.

\subsection*{Práctica Distribuida}
\index{spaced practice (learning strategy)}
\index{learning strategy!spaced practice}

Diez horas de estudio repartidas en cinco días
es más efectivo que dos días de cinco horas,
y mucho mejor que un día de diez horas.
Por lo tanto, deberías crear un horario de estudio donde distribuyas las actividades de estudio a lo largo del tiempo:
reserva al menos media hora para estudiar cada tema, cada día
en vez de amontonar todo para la noche antes del examen~\cite{Kang2016}.

También deberías revisar los materiales después de cada clase,
pero no inmediatamente después---toma al menos media hora de receso.
Cuando repases,
asegurate de incluir al menos un poquito del material más antiguo:
por ejemplo,
utiliza veinte minutos para revisar las notas de la clase de hoy
y luego cinco minutos para revisar material de los días anteriores
y de la semana pasada.
Esto te ayudará a identificar algun vacio o errores en tus apuntes previos
cuando todavía haya tiempo para corregirlos o hacer preguntas:
es doloroso darse cuenta la noche del examen
que no tienes idea por qué subrayaste ``Demodular!!'' tres veces.

Al repasar,
haz notas sobre las cosas que te hayas olvidado:
por ejemplo,
haz una tarjeta de memoria para cada concepto que no pudiste recordar
o que recordaste incorrectamente~\cite{Matt2019}.
Esto te ayudará a enfocarte en las cosas que necesitan más atención cuando vuelvas a estudiar.

\begin{aside}{El Valor de las Clases Magistrales}
  Según~\cite{Mill2016a},
  ``Las clases magistrales que predominan en los cursos presenciales son relativamente formas     |   ineficientes de enseñar,
  pero probablemente contribuyen a distribuir el material en el tiempo,
  porque se desenvuelven en un cronograma establecido en el tiempo.
  Al contrario,
  dependiendo de cómo se organicen los cursos,
  las/los estudiantes en línea a veces pueden evitar exponerse al material por completo hasta que una     tarea esté cerca.''
\end{aside}

\subsection*{Recordar lo Aprendido, del inglés Retrieval Practice}
\index{retrieval practice (learning strategy)}
\index{learning strategy!retrieval practice}

El factor limitante de la memoria a largo plazo no es retener (lo que se almacena)
pero recordar (lo que puede accederse).
Recordar una información específica mejora con la práctica,
asi que los resultados en situaciones reales pueden mejorar
al hacer examenes de práctica o resumiendo en detalle un tema de memoria
y luego revisando que fue y que no fue recordado.
Por ejemplo,
\cite{Karp2008} encontró que hacer exámenes de forma repetida mejora el recuerdo de listas de palabras, de un 35\% al 80\%.

La habilidad de recordar mejora cuando en la práctica se utilizan actividades similares a las que se evalúan.
Por ejemplo,
escribir entradas en un diario personal ayuda con los examenes de opción múltiple,
pero menos que hacer examenes de práctica~\cite{Mill2016a}.
Este fenómeno se llama
\gref{g:transfer-appropriate-processing}{transferencia apropiada de procesamiento}.

Una manera de ejercitar las habilidades de recuerdo es resolver problemas dos veces.
La primera vez,
hacerlo completamente de memoria, sin notas o discusiones con pares.
Después de evaluar tu propio trabajo con una rúbrica de respuestas distribuidas por la/el docente,
resuelve el problema de nuevo, utilizando cualquier material de apoyo que quieras.
La diferencia entre ambos te muestra que tan bien pudiste recordar y aplicar el conocimiento.

Otro método (mencionado previamente) es crear tarjetas de estudio.
Las tarjetas físicas tienen una pregunta en un lado y la respuesta en el otro,
y existen múchas aplicaciones para generarlas disponibles para teléfono móbil.
Si estas estudiando con un grupo de estudio,
intercambiar las tarjetas de estudio con tu colega
te ayudará a descubrir ideas importantes que tal vez habias obviado o malinterpretado.

\grefdex{g:read-cover-retrieve}{Leer-cubrir-recordar}{read-cover-retrieve}
es una alternativa rápida a las tarjetas de estudio.
Mientras lees algo,
cubre los términos clave o secciones con notas adhesivas pequeñas.
Cuando hayas terminado,
vuelve a leer y ve que tan bien puedes adivinar las palabras cubiertas por las notas adhesivas.
Independientemente del método que uses,
no sólo practiques recordar datos y definiciones:
asegurate de evaluar la comprensión de ideas grandes
y de las conexiones entre ellas.
Diagramar un mapa conceptual y compararlo con tus apuntes
o con un mapa conceptual dibujado previamente
es una forma rápida de hacer esto.

\begin{aside}{Hipercorrección}
    Un descubrimiento poderoso en la investigación del aprendizaje es
  el \gref{g:hypercorrection}{fenómeno de hipercorrección}~\cite{Metc2016}.
  A la mayoría de la gente no le gusta que le digan cuando dicen algo incorrecto,
  asi que sería razonable asumir que
  mientras mas confianza tenga una persona en la respuesta que dan en un examen,
  más dificíl es cambiarle de opinión si el resultado era incorrecto.
  Y resulta que lo opuesto es cierto:
  mientras mas confianza tenga una persona de que tiene la razón,
  más probable que no repitan el error si este es corregido.
\end{aside}

\subsection*{Práctica intercalada}
\index{interleaving (learning strategy)}
\index{learning strategy!interleaving}

Una forma de espaciar la práctica es intercalar el estudio de diferentes temas:
en vez de dominar un tema,
luego el segundo y el tercero,
alterna las sesiones de estudio.
Aún mejor,
cambia el orden:
A-B-C-B-A-C es mejor que A-B-C-A-B-C,
que a la vez es mejor que A-A-B-B-C-C~\cite{Rohr2015}.
Esto funciona porque intercalar permite la creación de más vínculos entre los diferentes temas,
lo cual mejora el aprendizaje.

Cuánto deberías tardar en cada ítem
depende del tema y que tan bien lo conozcas.
Entre 10 y 30 minutos es un tiempo suficiente para
entrar en tema(\secref{s:individual-time})
pero no para deambular.
Intercalar el estudio parecerá más difícil que enfocarse en un sólo tema al principio,
pero ese es un signo de que está funcionando.
Si usas tarjetas de estudio, o haces exámenes de práctica para medir tu progreso,
deberías ver una mejora después de un par de días.

\subsection*{Elaboración}
\index{elaboración (estrategia de aprendizaje)}
\index{estrategia de aprendizaje!elaboración}

Explicar los temas a uno mismo mientras se estudia
permite entenderlos y recordarlos.
Una forma de hacer esto es complementar la respuesta en un exámen práctico
con la explicación de por qué la respuesta es correcta,
o al contrario, con una explicación de por qué otras respuestas plausibles no lo serían.
Otra alternativa es decirse a uno mismo
cómo una nueva idea es similar o diferente a una que ya hayas visto previamente.

Hablarse a una misma o a uno mismo puede parecer una forma extraña de estudiar,
pero~\cite{Biel1995} encontró que
las personas entrenadas en auto-explicarse descatan cuando se comparan quienes no se entrenaron.
Similarmente,
\cite{Chi1989} encontró que estudiantes simplemente frenan cuando se encuentran con un paso que no entienden
al tratar de resolver problemas.
Otros pausan y generan una explicación de lo que está pasando,
y aprenden más rápido.
Un ejercicio para construir esta habilidad es revisar un ejercicio de programación línea por línea con una clase,
y que diferentes personas expliquen cada línea,
y digan por qué está ahí y qué es lo que produce.

\subsection*{Ejemplos concretos}
\index{concrete examples (learning strategy)}
\index{learning strategy!concrete examples}

Una forma particularmente útil de elaborar es el uso de ejemplos concretos.
Cuando uno tiene una definición de un principio general,
intenta proveer uno o más ejemplos de su uso,
o por el contrario, toma un problema en particular y enuncia los principios generales que representa.
\cite{Raws2014} encontró que intercalar ejemplos y definiciones de esta forma
permite que las/los estudiantes puedan recordar lo último correctamente.

Una forma estructurada de hacer esto es con
el \hreffoot{https://betterexplained.com/articles/adept-method/}{método ADEPT}:
\index{ADEPT (lesson pattern)}
\index{lesson pattern!ADEPT}
da una \textbf{A}nalogía,
dibuja un \textbf{D}iagrama,
presenta un \textbf{E}jemplo,
describe la idea en un lenguaje \textbf{S}ensillo (del inglés Plain Language),
y luego da los detalles \textbf{T}écnicos.
De nuevo,
si estás estudiando con alguien o en grupo,
puedes intercambiar y revisar el trabajo:
ve si estás de acuerdo si los ejemplos de otras personas representan el principio que se está discutiendo,
o qué principios se usan en un ejemplo que no hayan anotado.

Otra técnica útil es enseñar por contraste,
\index{teach by contrast (lesson pattern)}
\index{lesson pattern!teach by contrast}
p.ej.\ mostrar a tus estudiantes cuál \emph{no} es la solución
o cuál es técnica que \emph{no} resolverá un problema.
Por ejemplo,
al mostrar a las/los niños cómo simplificar fracciones,
es importante darles a menos un par de ejemplos como 5/7 que no pueden simplificarse,
para que no se frustren buscando respuestas que no existen.

\subsection*{Programación Dual}
\index{dual coding (learning strategy)}
\index{learning strategy!dual coding}

La última de las seis estrategias principales
descritas en \hreffoot{http://www.learningscientists.org/}{Learning Scientists}
es presentar palabras e imágenes juntas.
Cómo discutimos en \secref{s:architecture-brain},
diferentes subsistemas en nuestro cerebro manejan y almacenan la información linguistica y visual,
de tal manera que, si se presenta informacion complementaria por ambos canales,
se refuerzan mutuamente.
Sin embargo,
aprender es menos efectivo cuando la misma información se presenta de forma simultánea por dos canales diferentes,
porque el cerebro tiene que hacer un esfuerzo para comparar los canales entre si~\cite{Maye2003}.

Una forma de aprovechar la programación dual es dibujar o etiquetar lineas de tiempo,
mapas o árboles familiares,
o cualquier otro material que sea relevante.
(Personalmente, me gustan las imágenes que muestran qué funciones llaman a qué otras en un programa.)
Dibujar un diagrama \emph{sin} etiquetas,
y después volver atrás para etiquetarlo,
es una práctica excelente para recordar.

\seclbl{Gestión del tiempo}{s:individual-time}

Solía presumir sobre la cantidad de horas que trabajaba.
No en muchas palabras,
obviamente---tenía \emph{algunas} habilidades sociales---pero
me presentaba en clases alrededor del mediodía,
sin rasurar y bostezando,
y casualmente mencionaba a quien sea que pudiera escuchar
que estaba trabajando desde las 6:00 a.m.

Mirando atrás,
no puedo recordar a quién estaba tratando de impresionar.
Pero lo que si recuerdo es,
cuánto del trabajo que hice durante las trasnochadas tuve que tirar una vez que dormí un poco,
y cuanto daño le hizo a mis notas lo que no tiré.

Mi error fue confundir ``trabajar'' con ``ser productivo.''
No puedes producir software (o cualquier otra cosa) sin hacer algo de trabajo,
pero se puede hacer fácilmente mucho trabajo sin producir nada de valor.
Convencer a la gente de esto es difícil,
especialmente cuando son adolescentes o veinteañeras y veinteañeros,
pero paga tremendos dividendos.

El estudio científico en trabajo exceso y deprivación del sueño se remonta al ménos a la década de 1890s---vea
\cite{Robi2005} para un resumen breve y legible.\index{overwork}\index{sleep deprivation}
Los resultados más importantes para estudiantes son:

\begin{enumerate}

\item
  Trabajar más de 8 horas al día por un periodo extendido de tiempo
  disminuye la productividad total,
  no sólo la productividad por hora ---i.e.\ haces menos en total (no sólo por hora)
  cuando tienes trabajo acumulado y cerca de una fecha límite de entrega.

\item
  Trabajar durante 21 horas seguidas aumenta la chance de que tengas un error catastrófico
  tanto como estar legalmente en estado de ebriedad.

\item
  La productividad varía a lo largo de la jornada laboral,
  con la mayor productividad en las primeras 4 a 6 horas.
  Después de cierta cantidad de horas,
  la productividad disminuye a cero;
  y eventualmente se vuelve negativa.

\end{enumerate}

Estos hechos se han reproducido y verificado durante más de un siglo,
y los datos detrás de ellos son tan sólidos como los que relacionan el tabaquismo con el cáncer de pulmón.
El problema es que
\emph{las personas generalmente no notan que sus habilidades disminuyen}.
Como cuando personas en estado de ebriedad creen que todavía pueden conducir,
las personas que están deprivadas de sueño no se dan cuenta que
no están terminando sus oraciones (o pensamientos).
Se ha demostrado que cinco días de 8 horas por semana maximizan la producción total a largo plazo
en todas las industrias que se han estudiado;
estudiar o programar no es diferente.

Pero que pasa con las rachas que surgen de vez en cuando,
como trabajar toda la noche para cumplir con un plazo?
Eso también se ha estudiado,
Y los resultados no son agradables.
La habilidad de pensar dismuye en un 25\% por cada 24 horas sin dormir.
Puesto de otra forma,
el coeficiente intelectual de una persona promedio es sólo 75 después de una trasnochada,
lo que la desplaza al 5\% inferior de la población.
Si haces dos trasnochadas seguidas, tu coeficiente intelectual es de 50,
que es el nivel en el que las personas suelen ser consideradas incapaces de vivir de forma independiente.

``Pero---pero---tengo tantas tareas que hacer!'' dices tú.
``Y todas tienen que ser entregadas a la misma vez!
\emph{Tengo} que trabajar horas extra para completarlas!''
No:
las personas tienen que priorizar y enfocarse para ser productivas,
y para hacerlo,
deben ser ser enseñadas como.
Una técnica ampliamente utilizada es hacer una lista de tareas que deben hacerse,
organizadas por prioridad,
y luego desconectarse del correo electrónico u otras interrupciones por 30--60 minutos
y completar una de esas tareas.
Si una de esas tareas de la lista para-hacer lleva más de una hora,
sepárala en pedazos más pequeños y priorizalos de forma separada.

La parte más importante de esto es apagar las interrupciones.
A pesar de lo que mucha gente quiere creer,
los seres humanos no somos buenos en hacer múltiples tareas a la vez.\index{multi-tasking}
pero, en lo que podemos volvernos buenos \gref{g:automaticity}{automaticity},
es en la habilidad de hacer algo de forma rutinaria de fondo
mientras realizamos otra tarea~\cite{Mill2016a}.
La mayoría de las personas puede hablar mientras corta una cebolla,
o tomar café mientras lee;
con la práctica,
también podemos tomar notas mientras escuchamos,
pero no podemos estudiar de forma efectiva,
programar,
o hacer otra tarea mentalmente exigente mientras prestamos atención a algo más---sólo
creémos que podemos.

El punto de organizarse y prepararse es
para entrar en el estado mental más productivo posible.
Quienes estudiaron psicología lo llaman \gref{g:flow}{flujo}~\cite{Csik2008};
las personas que realizan atletismo lo llaman ``estar en la zona,''
y las/los músicos hablan de perderse en lo que están tocando.
CUalquier nombre que uses,
las personas producen mucho más por unidad de tiempo en este estado que en un estado normal.
La mala noticia es que
tarda aproximadamente 10 minutos volver a entrar en este estado después de tener una interrupción,
sin importar lo corta que haya sido la interrupción.
Lo que significa que si te interrumpieron seis veces por hora,
\emph{nunca} llegaste al máximo de tu productividad.

\newpage

\begin{aside}{Cómo lo supo?}

  En su breve historia en 1961 ``\hreffoot{https://en.wikipedia.org/wiki/Harrison\_Bergeron}{Harrison Bergeron},''
  Kurt Vonnegut describió un futuro en el que las personas están obligadas a ser iguales.
  Las personas atractivas tienen que usar máscaras,
  las personas atléticas tienen que cargar pesas---y las personas inteligentes
  estan obligadas a llevar radios que interrumpen sus pensamientos en intervalos aleatorios.
  A veces me pregunto si---oh, un momento, mi teléfono acaba de---perdón, de qué estábamos hablando?

\end{aside}

\seclbl{Evaluación de pares}{s:individual-peer}
\index{peer assessment}

Perdirle a las personas de un equipo que evalúen a sus pares es una práctica común en la industria.
\cite{Sond2012} revisó la literatura en evaluación de pares,
distinguiendo entre calificar y evaluar.
Descubrieron que la evaluación por pares aumentaba la cantidad, la diversidad y la puntualidad de la retroalimentación,
ayudó a las/los estudiantes a ejercitar el pensamiento de nivel superior,
fomenta la práctica reflexiva,
y apoya el desarrollo de habilidades sociales.
Las preocupaciones fueron predecibles:
validez y confiabilidad,
motivación y procrastinación,
trolls, colusión y plagio.

Sin embargo,
la evidencia muestra que estas preocupaciones no fueron significantes en la mayoría de las clases.
Por ejemplo,
\cite{Kauf2000} comparó las evaluaciones y calificaciones confidenciales de pares en varios ejes
para dos cursos en licenciatura en ingeniería,
y descubrió que la auto-calificación y la evaluación de pares tuvo una concordancia estadística,
que la colusión no fue significativa (i.e.\ no dieron irrespectivamente la nota más alta a todos sus pares),
que las/los estudiantes no inflaron sus auto-calificaciones,
y crucialmente,
que las calificaciones no estaban sesgadas por género o raza.

Una forma de implementar la evaluación de pares es \gref{g:contributing-student}{contribuyendo a la pedagogía estudiantil},
en la cual tus estudiantes producen artefactos para contribuir al aprendizaje de otros.
Esto puede hacerse desarrollando una lección corta y compartiéndola con la clase,
contribuir a un banco de preguntas,
o escribir notas de una lección en particular para una publicación durante la clase.
Por ejemplo,
\cite{Fran2018} evidenció que las/los estudiantes que realizan videos cortos para enseñar conceptos a sus pares
tuvieron un incremento significativo de su propio aprendizaje
comparado al de aquellos que sólo estudiaron el material o vieron los videos.
Yo he visto que pedir a mis estudiantes que muestren un error en su cóodigo y que muestren la solución con la clase cada día, 
ayuda en sus habilidades analíticas y disminuye el síndrome del impostor.

Otra alternativa es la \gref{g:calibrated-peer-review}{revisión por pares calibrada},
en la cual tus estudiantes revisan uno o más ejemplos utilizando una rúbrica
y comparan su evaluación con la evaluación del cuerpo docente~\cite{Kulk2013}.
Una vez que la evaluación de tu estudiante sea similar a la del cuerpo docente,
pueden empezar a evaluar el trabajo de sus pares.
Si se combinan muchas evaluaciones de pares,
estas pueden ser tan precisas como la evaluación del cuerpo docente ~\cite{Pare2008}.

Como todo lo demás,
la evaluación es ayudada por rúbricas.
La planilla de evaluación \secref{s:checklists-teameval} muestra un ejemplo para que puedas empezar.
Para usarla,
evalúate y a tus colegas,
luego calcula y compara las notas.
Una diferencia grande generalmente indica que la necesidad de una major conversación.

\seclbl{Ejercicios}{s:individual-exercises}

\exercise{Estrategias de aprendizaje}{individual}{20}

\begin{enumerate}

\item
  Cuál de la seis estrategias de aprendizaje generalmente usas?
  Cuáles generalmente no usas?

\item
  Escribe tres conceptos generales que quieras que tus estudiantes aprendan
  y da dos ejemplos específicos para cada caso
  (practica ejemplos concretos).
  Para cada uno de estos conceptos,
  trabaja al revés, parte de uno de los ejemplos y elabora sobre el concepto que lo explica
  (elaboración).

\end{enumerate}

\exercise{Conectando ideas}{en pares}{5}

Este ejercicio es un ejemplo de utilizar la elaboración para mejorar la retención.
Elige a una/un colega
que cada persona independientemente elija una idea,
luego anuncia tu idea y trata de encontrar una cadena de cuatro lazos
que conecte ambas ideas.
Por ejemplo,
si dos ideas son ``Saskatchewan'' y ``estadística,''
los lazos pueden ser:

\begin{itemize}

\item
  Saskatchewan es una provincia de Canadá;

\item
  Canada es un país;

\item
  países tienen gobiernos;

\item
  los gobiernos utilizan estadísticas para analizar la opinión pública.

\end{itemize}

\exercise{Evolución Convergente}{en pares}{15}

Una práctica que no hemos cubierto anteriormente son las \gref{g:guided-notes}{notas guiadas},
que son notas preparadas por la/el docente
que solicita a las/los estudiantes a responder preguntas respecto a la información clave de una lectura o discusión.
Estas preguntas pueden ser espacios en blancos dónde tus estudiantes agregan información,
asteriscos junto a términos que deben definir,
etcétera.

Crea dos a cuatro tarjetas con notas guidas para una lección que hayas enseñado recientemente
o que vas a enseñar.
Intercambia tarjetas con tu colega:
que tan fácil es entender lo que se está preguntando?
Cuánto tiempo te lleva llenar las respuestas?
Qué tan bien funciona esto para ejemplos de programación?

\exercise{Cambiando de opinión}{en pares}{10}

\cite{Kirs2013} argumenta que los mitos sobre nativos digitales,
estilos de aprendizaje,
y personas autodidactas es que todas son reflejos de creencias equivocadas que
las/los estudiantes saben lo que es mejor para ellos,
y advierte que podemos estar en una espiral descendente
en la que todos los intentos de las/los investigadores en educacion para refutar estos mitos
confirman la creencia de sus oponentes de que aprender ciencia es seudo-ciencia.
Elige una cosa que hayas aprendido hasta ahora en es este libro
que te haya sorprendido o contradecido a algo que creías previamente
y practica explicar esto a tu colega en 1--2 minutos.
Cuán convincente eres?

\exercise{Tarjetas de estudio}{individual}{15}

Utiliza las notas adhesivas o algo similar que tengas a mano
para hacer seis tarjetas de estudio
para un tema que recientemente hayas enseñado o aprendido.
Intercambia con tu colega y ve cuánto tiempo tardas en recordar 
al 100\% cada tarjeta
Deja las tarjetas a un lado cuando termines,
y vuelve media hora después para evaluar cuál es tu taza de retención.

\exercise{Utilizando ADEPT}{toda la clase}{15}

Elige un tema que recién hayas enseñado o aprendido,
y delínea una pequeña lección que utilice los cinco pasos del método ADEPT para introducir el tema.

\exercise{El costo de la multitarea}{en pares}{10}

\hreffoot{http://www.learningscientists.org/blog/2017/7/28-1}{The Learning Scientists blog}
describe un experimento simple que puedes hacer con un sólo un cronómetro, 
para demostrar el costo de la multitarea.
Trabajando en pares,
mide cuánto tarda cada persona en hacer cada una de estas tres tareas:

\begin{itemize}
\item
  Contar del 1 al 26 dos veces
\item
  Recitar el alfabeto de la A a la Z dos veces.
\item
  Intercalar números y letras,
  i.e.\ decir, ``1, A, 2, B, {\ldots}''
  y continuar.
\end{itemize}

Que cada pareja reporte sus números.
Sin práctica específica,
la tercera tarea siempre toma significativamente más tiempo que cada uno de los componentes de la tarea.

\exercise{Mitos en la educación de computación}{toda la clase}{20}

\cite{Guzd2015b} presenta una lista del top 10 de creencias erróneas sobre educación en computación,
que incluye:

\begin{enumerate}
\item
  La ausencia de mujeres en Ciencias de la Computación es similar en otras áreas de STEM.
\item
  Para tener más mujeres en las Ciencias de Computación necesitamos más docentes mujeres en el área.
\item
  Las evaluaciones de estudiantes son la mejor forma de evaluar la enseñanza.
\item
  Las/los buenos docentes personalizan la educación a los estilos de aprendizaje de sus estudiantes.
\item
  Una/un buen docente en ciencias de la computación debería modelar buenas practicas de desarrollo de software porque su trabajo es producir ingenieras e ingenieros de software excelentes.
\item
  Algunas personas son naturalmente mejores programadoras que otras.
\end{enumerate}

Invita a que cada persona vote +1 (de acuerdo), -1 (en desacuerdo), o 0 (neutro) para cada punto,
luego lee la explicación completa en
\hreffoot{https://cacm.acm.org/blogs/blog-cacm/189498-top-10-myths-about-teaching-computer-science/fulltext}{el artículo original}
y vuelve a hacer la votación.
En cuáles preguntas las personas cambiaron de parecer?
Cuáles todavía creen que son verdad, y por qué?

\exercise{Revisión por pares calibrada}{pares}{20}

\begin{enumerate}

\item
  Crea una rúbrica de 5--10 puntos
  con entradas como ``buenos nombres de variables,'' ``sin código redundante,'' y ``flujo de control propiamente anidado''
  para calificar el tipo de programa que esperarías que tus estudiantes escriban.

\item
  Elige o crea un pequeño programa que contenga 3--4 violaciones a estas entradas.

\item
  Califica el programa de acuerdo a la rúbrica.

\item
  Pide a tu colega que califique el mismo programa con la misma rúbrica.
  Qué aceptó que tu no aceptaste?
  Qué criticaron que tu criticaste?

\end{enumerate}

\section*{Revisión}

\figpdfhere{../figures/conceptmap-active-learning.pdf}{Concepts: Active learning}{f:individual-concept-map}
