\chapter{Arquitectura cognitiva}\label{s:architecture}

Hemos hablado acerca de modelos mentales como si fueran cosas reales, 
pero ¿qué es lo que realmente sucede en el cerebro de un aprendiz cuando está aprendiendo? 
La respuesta corta es que no lo sabemos, la respuesta larga es que sabemos mucho más que antes.
Este capítulo profundizará en lo que el cerebro hace mientras el aprendizaje sucede 
y cómo podemos aprovechar eso para diseñar y brindar lecciones de manera más efectiva.

\seclbl{¿Qué es lo que sucede allí?}{s:architecture-brain}

\figpdf{figures/cognitive-architecture.pdf}{Arquitectura Cognitiva}{f:arch-model}

La figura \figref{f:arch-model} es un modelo simplificado de la arquitectura cognitiva humana.\index{cognitive architecture}
El núcleo de este modelo es la separación entre la memoria a corto y a largo plazo vistas en \secref{s:memory-seven-plus-or-minus}.
La memoria a largo plazo es como tu sótano:\index{long-term memory}
almacena objetos de forma más o menos permanente 
pero tu conciencia no puede acceder a ella directamente. 
En cambio, 
confías en tu memoria a corto plazo,\index{short-term memory}
que es como el escritorio de tu mente.

Cuando necesitas algo, 
tu cerebro lo rescata de la memoria a largo plazo  
y lo coloca en la memoria a corto plazo. 
Por el contrario, la nueva información que llega a la memoria a corto plazo 
debe codificarse para poder ser almacenada en la memoria a largo plazo.
Si esa información no está codificada y almacenada, no se recuerda y esto significa que  
no se ha aprendido.

La información ingresa a la memoria a corto plazo principalmente 
a través de tu canal verbal (para el habla)\index{verbal channel}
y del canal visual\index{visual channel}
(para las imágenes)\footnote{
  Un modelo más completo
  también incluiría el sentido del tacto, del olfato y del gusto,
  pero por ahora, los ignoraremos.}.
La mayoría de las personas confía principalmente en su canal visual, 
pero cuando las imágenes y las palabras se complementan entre sí,
el cerebro hace un mejor trabajo al recordarlas a ambas: 
se codifican juntas,
de modo que el recuerdo de una más tarde ayude a activar el recuerdo de la otra.

Las entradas lingüísticas y visuales son procesadas por diferentes partes del cerebro humano, 
y a su vez los recuerdos lingüísticos y visuales son almacenados también de manera separada. 
Esto significa que correlacionar flujos de información lingüísticos y visuales requiere esfuerzo cognitivo: 
si alguien lee algo mientras lo escucha en voz alta, 
su cerebro no puede evitar comprobar que obtiene la misma información por ambos canales.

Por lo tanto, el aprendizaje aumenta cuando la información se presenta de manera simultánea por dos canales diferentes, 
pero se reduce cuando esa información es redundante, en lugar de ser complementaria, 
tal fenómeno es conocido como \gref{g:split-attention-effect}{efecto de atención dividida}~\cite{Maye2003}.
Por ejemplo, en general las personas encuentran difícil aprender de un video que tiene narración y 
al mismo tiempo capturas de pantalla más que de uno que solo tiene narración o capturas pero no ambos elementos, 
porque parte de su atención ha sido utilizada para chequear que la narración  
y las capturas se correspondan entre sí. Dos notables excepciones a esto, 
son las personas que aún no hablan bien un idioma y las que tienen algún impedimento auditivo u 
otras necesidades especiales, quienes quizás encuentren que el valor de la información redundante 
supera el esfuerzo de procesamiento adicional.

\begin{aside}{Pieza por pieza}
  El efecto de la atención dividida explica porque es más efectivo dibujar un diagrama 
  pieza por pieza mientras enseñas que presentar todo el gráfico de una sola vez. 
  Si las partes de un diagrama aparecen al mismo tiempo en que los gráficos son explicados, 
  ambos elementos serán correlacionados en la memoria del aprendiz. 
  Enfocarnos luego solo en una parte del diagrama es lo más parecido a activar la recuperación 
  de lo que fue dicho cuando esa parte fue dibujada. 
\end{aside}

El efecto de la atención dividida \emph{no} significa 
que los estudiantes no deberían intentar conciliar múltiples flujos de información entrantes, 
después de todo, esto es lo que ellos tienen que hacer en el mundo real~\cite{Atki2000}.
En cambio, significa que la instrucción no debería solicitar a las personas 
que lo hagan mientras ellos son los primeros en manejar las habilidades de la unidad, 
en lugar de usar múltiples fuentes  de información de manera simultánea debe tratarse como una tarea de aprendizaje separada. 

\begin{aside}{No todos los gráficos son creados iguales}
  \cite{Sung2012} presenta un elegante estudio que distingue los gráficos \emph{seductores}\index{graphics!seductive} 
  (los cuales son altamente interesantes pero no son directamente relevantes al objetivo de la enseñanza), 
  los gráficos\emph{decorativos}\index{graphics!decorative}
  (los cuales son neutros pero no son directamente relevantes al objetivo de la enseñanza), 
  y por último
  los gráficos \emph{instructivos}\index{graphics!instructive} 
  (los cuales si son directamente relevantes al objetivo de la enseñanza). 
  Los estudiantes que recibieron cualquier tipo de gráfico obtuvieron calificaciones de satisfacción 
  del material más altas que aquellos que no obtuvieron gráficos, 
  pero en realidad solo los estudiantes que obtuvieron gráficos instructivos obtuvieron mejores resultados.


  Del mismo modo,~\cite{Stam2013,Stam2014} descubrió que 
  tener más información, en realidad puede disminuir el rendimiento. 
  Les mostraron a los niños dibujos, dibujos y números, y simplemente números 
  para dos tareas. 
  Para algunos, tener imágenes o imágenes y números superó al tener solo números, 
  pero para otros, tener imágenes superó a las imágenes y números, 
  lo que superó solo tener números.
\end{aside}

\seclbl{Carga cognitiva}{s:architecture-load}

En~\cite{Kirs2006}, Kirschner, Sweller y Clark escribieron:

\begin{quote}
 
  Aunque los enfoques educativos no guiados o mínimamente guiados son muy    
  populares e intuitivamente atractivos{\ldots}estos enfoques ignoran tanto 
  las estructuras que constituyen la arquitectura cognitiva humana como 
  la evidencia de estudios empíricos de los últimos cincuenta años que indican 
  sistemáticamente que la instrucción guiada mínimamente es menos eficaz y 
  menos eficiente que los enfoques educacionales que hacen un fuerte énfasis 
  en la orientación del proceso de aprendizaje del estudiante.
  La ventaja de la orientación disminuye sólo cuando los estudiantes 
  tienen un conocimiento previo suficientemente elevado para proporcionar una orientación ``interna''.
\end{quote}

Debajo de la jerga, 
los autores afirmaban que el hecho de que los estudiante hagan sus propias preguntas, 
establezcan sus propias metas y 
encuentren su propio camino a través de un tema es menos efectivo que mostrarles 
cómo hacer las cosas paso a paso. El enfoque ``elige tu propia aventura'' se conoce como \gref{g:inquiry-based-learning}{aprendizaje basado en la indagación} 
y es intuitivamente atractivo: después de todo, 
¿quién se \emph{opondría} a tener estudiantes que utilicen su propia iniciativa 
para resolver problemas del mundo real de forma realista? 
Sin embargo, pedir a los estudiante que lo hagan en un nuevo dominio les sobrecarga 
al exigirles que dominen el contenido fáctico de un dominio y sus estrategias de resolución de problemas al mismo tiempo.
Más específicamente, 
\gref{g:cognitive-load}{la teoría de la carga cognitiva} propone que 
la gente tiene que lidiar con tres cosas cuando está aprendiendo:


\begin{description}

\item[\grefdex{g:intrinsic-load}{Carga Intrínseca}{cognitive load!intrinsic}]
  es lo que la gente tiene que tener en cuenta para aprender el material nuevo.

\item[\grefdex{g:germane-load}{Carga Pertinente}{cognitive load!germane}]
  es el esfuerzo mental (deseable) requerido para vincular la nueva información con la antigua, 
  que es una de las cosas que distinguen el aprendizaje de la memorización. 


\item[\grefdex{g:extraneous-load}{Carga Extrínseca}{cognitive load!extraneous}]
es cualquier cosa que distraiga del aprendizaje.

\end{description}

La teoría de la carga cognitiva sostiene que 
la gente tiene que dividir una cantidad fija de memoria de trabajo entre estas tres cosas. 
Nuestro objetivo como profesores es maximizar la memoria disponible para manejar la carga intrínseca, 
lo cual significa reducir la carga pertinente en cada paso y eliminar la carga extrínseca.


\subsection*{Problemas de Parsons} 

Un tipo de ejercicio que puede ser explicado en términos de carga cognitiva 
se utiliza a menudo en la enseñanza de idiomas.
Supongamos que le pides a alguien que traduzca la frase, 
``¿Cómo está tu rodilla hoy?'' a lengua frisona(frisón).
Para resolver el problema, necesitan recordar tanto el vocabulario 
como la gramática, que es una carga cognitiva doble.
Si les pides que pongan ``hoe'', ``har'', ``is'', ``hjoed'' y ``knie'' en el orden correcto, 
por otro lado, les permites que se centren únicamente en el aprendizaje de la gramática.
Si escribes estas palabras en cinco fuentes o colores diferentes, 
sin embargo, has aumentado la carga cognitiva externa, porque involuntariamente 
(y posiblemente de manera inconsciente) invertirán algo de esfuerzo tratando de averiguar 
si las diferencias son significativas (\figref{f:architecture-frisian}).

\figimg{figures/frisian.png}{Construyendo una oración}{f:architecture-frisian}

El equivalente de codificación de este
se llama \gref{g:parsons-problem}{Problema de Parsons}\footnote{Nombrado debido a uno de sus creadores.}~\cite{Pars2006}.

Cuando se enseña a la gente a programar,
puedes darles las líneas de código que necesitan para resolver un problema
y pedirles que las pongan en el orden correcto.
Esto les permite concentrarse en el flujo de control y las dependencias de datos
sin distraerse con la denominación de las variables o tratando de recordar qué funciones llamar.
Múltiples estudios han demostrado que los problemas de Parsons les toma a los estudiantes menos tiempo resolverlo
pero producen resultados educativos equivalentes~\cite{Eric2017}.


\subsection*{Ejemplos desvanecidos}

Otro tipo de ejercicio que se puede explicar en términos de carga cognitiva 
es dar a los estudiantes una serie de ejemplos desvanecidos \grefdex{g:faded-example}{faded examples}{faded example}.
El primer ejemplo de una serie presenta un uso completo de una estrategia 
particular de resolución de problemas. 
El siguiente problema es del mismo tipo, 
pero tiene algunas lagunas que el estudiante debe llenar. 
Cada problema sucesivo le da al estudiante menos \gref{g:scaffolding}{scaffolding},
hasta que se le pide que resuelva un problema completo desde cero. 
Al enseñar álgebra en la escuela secundaria, 
por ejemplo, 
podríamos comenzar con esto:


\begin{center}
\begin{tabular}{rcl}
  (4x + 8)/2	& = &	5	\\
  4x + 8	& = &	2 * 5	\\
  4x + 8	& = &	10	\\
  4x		& = &	10 - 8	\\
  4x		& = &	2	\\
  x		& = &	2 / 4	\\
  x		& = &	1 / 2
\end{tabular}
\end{center}

\noindent
y luego pide a los estudiantes que resuelvan esto:

\begin{center}
\begin{tabular}{rcl}
  (3x - 1)*3	& = &	12	\\
  3x - 1	& = &	\_ / \_	\\
  3x - 1	& = &	4	\\
  3x		& = &	\_	\\
  x		& = &	\_ / 3	\\
  x		& = &	\_
\end{tabular}
\end{center}

\noindent
y esto:

\begin{center}
\begin{tabular}{rcl}
  (5x + 1)*3	& = &	4	\\
  5x + 1	& = &	\_ 	\\
  5x		& = &	\_ 	\\
  x		& = &	\_
\end{tabular}
\end{center}

\noindent
y finalmente, esto:

\begin{center}
\begin{tabular}{rcl}
  (2x + 8)/4	& = &	1	\\
   x		& = &	\_
\end{tabular}
\end{center}

Un ejercicio similar para enseñar Python podría comenzar mostrando a los estudiantes\index{Python} 
cómo encontrar la longitud total de una lista de palabras:

\begin{minted}{text}
# total_length(["red", "green", "blue"]) => 12
define total_length(list_of_words):
    total = 0
    for word in list_of_words:
        total = total + length(word)
    return total
\end{minted}

\noindent

y luego pídales que llenen los espacios en blanco en esto 
(lo que centra su atención en las estructuras de control):


\begin{minted}{text}
# word_lengths(["red", "green", "blue"]) => [3, 5, 4]
define word_lengths(list_of_words):
    list_of_lengths = []
    for ____ in ____:
        append(list_of_lengths, ____)
    return list_of_lengths
\end{minted}

El siguiente problema podría ser este 
(que centra su atención en actualizar el resultado final):

\begin{minted}{text}
# join_all(["red", "green", "blue"]) => "redgreenblue"
define join_all(list_of_words):
    joined_words = ____
    for ____ in ____:
        ____
    return joined_words
\end{minted}

Finalmente, se pedirá a los estudiantes que escriban una función completa por su cuenta:

\begin{minted}{text}
# make_acronym(["red", "green", "blue"]) => "RGB"
define make_acronym(list_of_words):
    ____
\end{minted}

Los ejemplos difusos funcionan porque 
presentan la estrategia de resolución de problemas pieza por pieza: 
en cada paso, 
los estudiantes tienen un nuevo problema que abordar, 
que es menos intimidante que una pantalla en blanco o una hoja de papel en blanco (\secref{s:classroom-practices}). 
También anima a los estudiantes a pensar en las similitudes y diferencias entre varios enfoques, 
lo que ayuda a crear los vínculos en sus modelos mentales que ayudan a la recuperación de la información.

La clave para construir un buen ejemplo desvanecido es 
pensar en la estrategia de resolución de problemas que se pretende enseñar. 
Por ejemplo, 
los problemas de programación sobre todo utilizan el patrón de diseño del acumulador, 
en el que los resultados del procesamiento de elementos de una colección 
se agregan repetidamente a una sola variable de alguna manera para crear el resultado final.


\begin{aside}{Aprendizaje cognitivo}
  Un modelo alternativo de aprendizaje e instrucción que también usa andamiaje y desvanecimiento 
  es el \gref{g:cognitive-apprenticeship}{aprendizaje cognitivo}, 
  que enfatiza la forma en que un maestro transmite habilidades y conocimientos a un aprendiz. 
  El maestro proporciona modelos de desempeño y resultados, 
  luego entrena a los principiantes explicando qué están haciendo y por qué~\cite{Coll1991,Casp2007}.
  El aprendiz reflexiona sobre su propia resolución de problemas, 
  por ejemplo, pensando en voz alta o criticando su propio trabajo, 
  y finalmente explora problemas de su propia elección.

  Este modelo nos dice que 
  los profesores deben presentar varios ejemplos al explicar una nueva idea 
  para que los estudiantes puedan ver qué generalizar, 
  y que deben variar la forma del problema para dejar en claro 
  cuáles son y cuáles no son características superficiales features\footnote{Por mucho tiempo,
     creí que la variable que contenía el valor que una función iba a devolver 
     \emph{tenía} que llamarse \texttt{resultado}
     porque mi maestro siempre usaba ese nombre en los ejemplos.}.    
  Los problemas deben presentarse en contextos del mundo real, 
  y debemos fomentar la autoexplicación para ayudar a los estudiantes 
  a organizarse y dar sentido a lo que se les acaba de enseñar 
 (\secref{s:individual-strategies}).
\end{aside}


\subsection*{Subobjetivos etiquetados}
 
\grefdex{g:subgoal-labeling}{Labeling subgoals}{labeled subgoals} significa 
dar nombre a los pasos en una descripción paso a paso de un proceso de resolución de problemas. 
\cite{Marg2016,Morr2016} descubrieron que los estudiantes con subobjetivos etiquetados 
resolvían los problemas de Parsons mejor que los estudiantes sin estos, 
y se observa el mismo beneficio en otros dominios~\cite{Marg2012}. 
Volviendo al ejemplo de Python usado anteriormente, 
los objetivos secundarios para encontrar la longitud total de una lista de palabras o construir un acrónimo son:

\begin{enumerate}

\item
  Crea un valor vacío del tipo que se devolverá.

\item
  Obtiene el valor que se agregará al resultado de la variable de ciclo.

\item
  Actualiza el resultado con ese valor.

\end{enumerate}

Etiquetar subobjetivos funciona porque agrupar los pasos relacionados en fragmentos con nombre(\secref{s:memory-seven-plus-or-minus})\index{chunking} 
ayuda a los estudiantes a distinguir lo que es genérico de lo que es específico del problema en cuestión. 
También les ayuda a construir un modelo mental de ese tipo de problema 
para que puedan resolver otros problemas de ese tipo 
y les da una oportunidad natural para la autoexplicación (\secref{s:individual-strategies}).

\subsection*{Manuales mínimos}

La aplicación más pura de la teoría de la carga cognitiva puede ser el manual mínimo de 
John Carroll\index{Carroll, John} \gref{g:minimal-manual}{minimal manual}~\cite{Carr1987,Carr2014}. 
Su punto de partida es una cita de un usuario: ``Quiero hacer algo, no aprender a hacer todo''. 
Carroll y sus colegas rediseñaron la capacitación para presentar cada idea como una tarea autónoma de una sola página: 
un título que describa de qué trata la página, 
instrucciones paso a paso sobre cómo hacer una sola cosa 
(por ejemplo, cómo eliminar una línea en blanco en un editor de texto)
y luego varias notas sobre cómo reconocer y resolver problemas comunes. 
Descubrieron que reescribir los materiales de capacitación de esta manera los hacía más cortos en general 
y que las personas que los usaban aprendían más rápido. 
Estudios posteriores confirmaron que este enfoque superó al enfoque tradicional 
independientemente de la experiencia previa con computadoras~\cite{Lazo1993}.
\cite{Carr2014} resumieron este trabajo diciendo:

\begin{quote}

  Nuestros diseños ``minimalistas'' buscaban aprovechar la iniciativa del usuario y el conocimiento previo, 
  en lugar de controlarlo mediante advertencias y pasos ordenados. 
  Enfatizó que los usuarios generalmente aportan mucha experiencia y conocimiento a este aprendizaje, 
  por ejemplo, 
  conocimiento sobre el dominio de la tarea, 
  y que dicho conocimiento podría ser un recurso para los diseñadores instruccionales. 
  El minimalismo aprovechó los episodios de reconocimiento, diagnóstico y corrección de errores, 
  en lugar de intentar simplemente prevenirlos. 
  Enmarca la resolución de problemas y la corrección como oportunidades de aprendizaje en lugar de aberraciones.

\end{quote}


\seclbl{Otros modelos de aprendizaje}{s:architecture-theory}

Los críticos de la teoría de la carga cognitiva a veces han argumentado que\index{cognitive load!criticism of} 
cualquier resultado puede justificarse a posteriori al etiquetar las cosas que perjudican el rendimiento como cargas extrañas 
y las que no lo hacen como intrínsecas o pertinentes. 
Sin embargo, 
la instrucción basada en la teoría de la carga cognitiva es innegablemente efectiva. 
Por ejemplo, 
\cite{Maso2016} rediseñó un curso de base de datos para eliminar la atención dividida y los efectos de redundancia 
y para proporcionar ejemplos prácticos y subobjetivos. 
El nuevo curso redujo la tasa de reprobación del examen en un 34\% 
y aumentó la satisfacción del estudiante.


Una década después de la publicación de~\cite{Kirs2006}, 
un número creciente de personas cree que la teoría de la carga cognitiva y los enfoques basados en la investigación son compatibles 
si se ven de la manera correcta. 
\cite{Kaly2015} sostiene que la teoría de la carga cognitiva es básicamente una microgestión del aprendizaje 
dentro de un contexto más amplio que considera cosas como la motivación, 
mientras que~\cite{Kirs2018} extiende la teoría de la carga cognitiva para incluir aspectos colaborativos del aprendizaje. 
Al igual que con~\cite{Mark2018} (discutido en \secref{s:individual-strategies}), 
las perspectivas de los investigadores pueden diferir, 
pero la implementación práctica de sus teorías a menudo termina siendo la misma.

Uno de los desafíos en la investigación educativa es que 
lo que queremos decir con ``aprendizaje'' resulta complicado 
una vez que se mira más allá del aula occidental estandarizada. 
Dos perspectivas específicas de la \gref{g:educational-psychology}{psicología educativa} han influido en este libro. 
El que hemos utilizado hasta ahora es el \gref{g:cognitivism}{cognitivismo}, 
que se centra en cosas como el reconocimiento de patrones, la formación de la memoria y el recuerdo. 
Es bueno para responder preguntas de bajo nivel, 
pero generalmente ignora cuestiones más importantes como, 
``¿Qué queremos decir con 'aprendizaje'?'' y 
``¿Quién decide?'' 
El otro es el \gref{g:situated-learning}{aprendizaje situado}, 
que se centra en llevar a las personas a una comunidad 
y reconoce que 
la enseñanza y el aprendizaje siempre están arraigados en quiénes somos y quiénes aspiramos a ser. 
Lo discutiremos con más detalle en el \chapref{s:community}.


El sitio web de \hreffoot{http://www.learning-theories.com/}{Learning Theories, teorías de aprendizaje en inglés} 
y~\cite{Wibu2016} 
tienen buenos resúmenes de estas y otras perspectivas. 
Además del cognitivismo, los que se encuentran con mayor frecuencia incluyen el \gref{g:behaviorism}{conductismo} 
(que trata la educación como un condicionamiento de estímulo/respuesta), 
el \gref{g:constructivism}{constructivismo} 
(que considera el aprendizaje como un proceso activo durante el cual los estudiantes construyen conocimiento por sí mismos) 
y el \gref{g:connectivism}{conectivismo}.
(que sostiene que el conocimiento se distribuye, 
que el aprendizaje es el proceso de navegar, crecer y podar conexiones, 
y que enfatiza los aspectos sociales del aprendizaje que Internet hace posible). 
Estas perspectivas pueden ayudarnos a organizar nuestros pensamientos, 
pero en la práctica, 
siempre tenemos que probar nuevos métodos en la clase, 
con estudiantes reales, 
para descubrir qué tan bien equilibran las muchas fuerzas en juego.

\seclbl{Ejercicios}{s:architecture-exercises}

\exercise{Crear un ejemplo desvanecido}{pares}{30'}

Es muy común que los programas cuenten cuántas cosas caen en diferentes categorías: 
por ejemplo, 
cuántas veces aparecen colores diferentes en una imagen 
o cuántas veces aparecen palabras diferentes en un párrafo de texto.

\begin{enumerate}
\item
	Crea un ejemplo breve (no más de 10 líneas de código) que muestre a las personas cómo hacer esto, 
	y luego crea un segundo ejemplo que resuelva un problema similar de una manera similar 
	pero que tenga un par de espacios en blanco para que los estudiante los completen. 
	¿Decides qué desvanecer? 
	¿Cuál sería el siguiente ejemplo de la serie?

\item
	Define la audiencia de sus ejemplos. 
	Por ejemplo, 
	¿son estos principiantes que solo conocen algunos conceptos básicos de programación? 
	¿O estos estudiantes tienen alguna experiencia en programación?

\item
	Muestra tu ejemplo a un compañero, 
	pero no le digas para qué nivel crees que es. 
	Una vez que hayan llenado los espacios en blanco, 
	pídeles que adivinen el nivel deseado.

\end{enumerate}

Si hay personas entre los aprendices que no programan en absoluto, 
intenta ubicarlos en diferentes grupos 
y pídeles que hagan el papel de aprendices para esos grupos. 
Alternativamente, 
elija un dominio de problema diferente y desarrolle un ejemplo difuso para él.


\exercise{Clasificación de carga}{grupos pequeños}{15'}

\begin{enumerate}

\item
  Elige una lección corta que un miembro de tu grupo haya enseñado o tomado recientemente.

\item
  Haz una lista en forma de puntos de las ideas, instrucciones y explicaciones que contiene.

\item
  Clasifica cada uno como intrínseco, pertinente o extraño.
  ¿En qué partes estaban todos de acuerdo?
  ¿Dónde estuvo en desacuerdo y por qué?

\end{enumerate}

(El ejercicio ``Cómo darse cuenta de su punto ciego'' en \secref{s:memory-exercises}
le dará una idea de cuán detallada debe ser su lista de formularios de puntos).


\exercise{Crear un problema de Parsons}{en parejas}{20'}

Escribe cinco o seis líneas de código que hagan algo útil, 
mezclalas y pídele a tu compañero que las ponga en orden. 
Si estás utilizando un lenguaje basado en identación como Python, 
no utilices sangría en ninguna de las líneas; 
si estás utilizando un lenguaje de llaves como Java, 
no incluyas ninguna de las llaves. 
(Si tu grupo incluye personas que no son programadores, 
usa un dominio de problema diferente, 
como hacer budín/pan de plátano).


\exercise{Manuales mínimos}{individual}{20'}

Escribe una guía de una página para hacer algo que tus estudiantes puedan encontrar en una de tus clases, 
como centrar el texto horizontalmente 
o imprimir un número con un cierto número de dígitos después del punto decimal. 
Intentá enumerar al menos tres o cuatro resultados incorrectos que el estudiante pueda ver 
e incluí una explicación de una o dos líneas 
de por qué ocurre cada uno y cómo corregirlo.


\exercise{Aprendizaje cognitivo}{parejas}{15'}

Elige un problema de codificación que puedas resolver en dos o tres minutos 
y piensa en voz alta mientras lo resuelves, 
al mismo tiempo tu compañero te hace preguntas sobre lo que estás haciendo y por qué. 
No solo explica lo que estás haciendo, 
sino también por qué lo estás haciendo, 
cómo sabes que es lo correcto y qué alternativas has considerado pero descartado. 
Cuando hayas terminado, 
intercambia roles con tu compañero y repetí el ejercicio.


\exercise{Ejemplos resueltos}{parejas}{15'}

Ver ejemplos resueltos ayuda a las personas a aprender a programar más rápido que simplemente escribiendo mucho código~\cite{Skud2014}, 
y deconstruir/desagregar el código rastreándolo o depurándolo también aumenta el aprendizaje~\cite{Grif2016}.
Trabajando en parejas, revisa un fragmento de código de 10 a 15 líneas 
y explica qué hace cada declaración y por qué es necesaria. 
¿Cuánto tiempo te tardas? 
¿Cuántas cosas crees que necesitas explicar por línea de código?

\exercise{Gráficos críticos}{individual}{30'}

\cite{Maye2009,Mill2016a} presentan seis principios para una buena enseñanza de gráficos:

\begin{description}

\item[Señalización:]
  resalta visualmente los puntos más importantes 
  para que se destaquen del material menos crítico.

\item[Contigüidad espacial:]
  coloca los subtítulos lo más cerca posible de los gráficos para compensar el costo de cambiar entre los dos.

\item[Contigüidad temporal:]
  Presenta narraciones y gráficos hablados tan seguidos en el tiempo como sea práctico. 
  (Presentar ambos a la vez es mejor que presentarlos uno tras otro).

\item[Segmentación:]
  Cuando presentes una secuencia larga de material o 
  cuando los estudiantes no tengan experiencia con el tema, 
  divide la presentación en segmentos cortos y deja que los estudiantes controlen la rapidez con que avanzan al siguiente.

\item[Pre-entrenamiento:]
  Si los estudiantes no conocen los conceptos y la terminología principales utilizados en tu presentación, 
  enseña solo esos conceptos y términos de antemano.

\item[Modalidad:]
las personas aprenden mejor de las imágenes más la narración que de las imágenes más el texto, 
a menos que no sean hablantes nativos o haya palabras o símbolos técnicos.

\end{description}

Elige un video de una lección o charla en línea que utilice diapositivas u otras presentaciones estáticas 
y califique sus gráficos como ``deficientes'', ``promedio'' o ``buenos'' de acuerdo con estos seis criterios.

\section*{Revisión}

\figpdfhere{figures/conceptmap-cognitive-load.pdf}{Concepto: Carga congnitiva}{f:architecture-cognitive-load}
