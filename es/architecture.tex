\chapter{Arquitectura Cognitiva}\label{s:architecture}

Hemos hablado acerca de modelos mentales como si fueran cosas reales, 
pero ¿qué es lo que realmente sucede en el cerebro de un aprendiz cuando está aprendiendo? 
La respuesta corta es que no lo sabemos, la respuesta larga es que sabemos mucho más que antes.
Este capítulo profundizará en lo que el cerebro hace mientras el aprendizaje sucede 
y cómo podemos aprovechar eso para diseñar y brindar lecciones de manera más efectiva.

\seclbl{¿Qué es lo que sucede allí?}{s:architecture-brain}

\figpdf{../figures/cognitive-architecture.pdf}{Arquitectura Cognitiva}{f:arch-model}

La figura \figref{f:arch-model} es un modelo simplificado de la arquitectura cognitiva humana.\index{cognitive architecture}
El núcleo de este modelo es la separación entre la memoria a corto y a largo plazo vistas en \secref{s:memory-seven-plus-or-minus}.
La memoria a largo plazo es como tu sótano:\index{long-term memory}
almacena objetos de forma más o menos permanente 
pero tu conciencia no puede acceder a ella directamente. 
En cambio, 
confías en tu memoria a corto plazo,\index{short-term memory}
que es como el escritorio de tu mente.

Cuando necesitas algo, 
tu cerebro lo rescata de la memoria a largo plazo  
y lo coloca en la memoria a corto plazo. 
Por el contrario, la nueva información que llega a la memoria a corto plazo 
debe codificarse para poder ser almacenada en la memoria a largo plazo.
Si esa información no está codificada y almacenada, no se recuerda y esto significa que  
no se ha aprendido.

La información ingresa a la memoria a corto plazo principalmente 
a través de tu canal verbal (para el habla)\index{verbal channel}
y del canal visual\index{visual channel}
(para las imágenes)\footnote{
  Un modelo más completo
  también incluiría el sentido del tacto, del olfato y del gusto,
  pero por ahora, los ignoraremos.}.
La mayoría de las personas confía principalmente en su canal visual, 
pero cuando las imágenes y las palabras se complementan entre sí,
el cerebro hace un mejor trabajo al recordarlas a ambas: 
se codifican juntas,
de modo que el recuerdo de una más tarde ayude a activar el recuerdo de la otra.

Las entradas lingüísticas y visuales son procesadas por diferentes partes del cerebro humano, 
y a su vez los recuerdos lingüísticos y visuales son almacenados también de manera separada. 
Esto significa que correlacionar flujos de información lingüísticos y visuales requiere esfuerzo cognitivo: 
si alguien lee algo mientras lo escucha en voz alta, 
su cerebro no puede evitar comprobar que obtiene la misma información por ambos canales.

Por lo tanto, el aprendizaje aumenta cuando la información se presenta de manera simultánea por dos canales diferentes, 
pero se reduce cuando esa información es redundante, en lugar de ser complementaria, 
tal fenómeno es conocido como \gref{g:split-attention-effect}{efecto de atención dividida}~\cite{Maye2003}.
Por ejemplo, en general las personas encuentran difícil aprender de un video que tiene narración y 
al mismo tiempo capturas de pantalla más que de uno que solo tiene narración o capturas pero no ambos elementos, 
porque parte de su atención ha sido utilizada para chequear que la narración  
y las capturas se correspondan entre sí. Dos notables excepciones a esto, 
son las personas que aún no hablan bien un idioma y las que tienen algún impedimento auditivo u 
otras necesidades especiales, quienes quizás encuentren que el valor de la información redundante 
supera el esfuerzo de procesamiento adicional.

\begin{aside}{Pieza por pieza}
  El efecto de la atención dividida explica porque es más efectivo dibujar un diagrama 
  pieza por pieza mientras enseñas que presentar todo el gráfico de una sola vez. 
  Si las partes de un diagrama aparecen al mismo tiempo en que los gráficos son explicados, 
  ambos elementos serán correlacionados en la memoria del aprendiz. 
  Enfocarnos luego solo en una parte del diagrama es lo más parecido a activar la recuperación 
  de lo que fue dicho cuando esa parte fue dibujada. 
\end{aside}

El efecto de la atención dividida \emph{no} significa 
que los estudiantes no deberían intentar conciliar múltiples flujos de información entrantes, 
después de todo, esto es lo que ellos tienen que hacer en el mundo real~\cite{Atki2000}.
En cambio, significa que la instrucción no debería solicitar a las personas 
que lo hagan mientras ellos son los primeros en manejar las habilidades de la unidad, 
en lugar de usar múltiples fuentes  de información de manera simultánea debe tratarse como una tarea de aprendizaje separada. 

\begin{aside}{No todos los gráficos son creados iguales}
  \cite{Sung2012} presenta un elegante estudio que distingue los gráficos \emph{seductores}\index{graphics!seductive} 
  (los cuales son altamente interesantes pero no son directamente relevantes al objetivo de la enseñanza), 
  los gráficos\emph{decorativos}\index{graphics!decorative}
  (los cuales son neutros pero no son directamente relevantes al objetivo de la enseñanza), 
  y por último
  los gráficos \emph{instructivos}\index{graphics!instructive} 
  (los cuales si son directamente relevantes al objetivo de la enseñanza). 
  Los estudiantes que recibieron cualquier tipo de gráfico obtuvieron calificaciones de satisfacción 
  del material más altas que aquellos que no obtuvieron gráficos, 
  pero en realidad solo los estudiantes que obtuvieron gráficos instructivos obtuvieron mejores resultados.


  Del mismo modo,~\cite{Stam2013,Stam2014} descubrió que 
  tener más información, en realidad puede disminuir el rendimiento. 
  Les mostraron a los niños dibujos, dibujos y números, y simplemente números 
  para dos tareas. 
  Para algunos, tener imágenes o imágenes y números superó al tener solo números, 
  pero para otros, tener imágenes superó a las imágenes y números, 
  lo que superó solo tener números.
\end{aside}

\seclbl{Carga cognitiva}{s:architecture-load}

En~\cite{Kirs2006}, Kirschner, Sweller y Clark escribieron:

\begin{quote}
 
  Aunque los enfoques educativos no guiados o mínimamente guiados son muy    
  populares e intuitivamente atractivos{\ldots}estos enfoques ignoran tanto 
  las estructuras que constituyen la arquitectura cognitiva humana como 
  la evidencia de estudios empíricos de los últimos cincuenta años que indican 
  sistemáticamente que la instrucción guiada mínimamente es menos eficaz y 
  menos eficiente que los enfoques educacionales que hacen un fuerte énfasis 
  en la orientación del proceso de aprendizaje del estudiante.
  La ventaja de la orientación disminuye sólo cuando los estudiantes 
  tienen un conocimiento previo suficientemente elevado para proporcionar una orientación ``interna''.
\end{quote}

Debajo de la jerga, 
los autores afirmaban que el hecho de que los estudiante hagan sus propias preguntas, 
establezcan sus propias metas y 
encuentren su propio camino a través de un tema es menos efectivo que mostrarles 
cómo hacer las cosas paso a paso. El enfoque ``elige tu propia aventura'' se conoce como \gref{g:inquiry-based-learning}{aprendizaje basado en la indagación} 
y es intuitivamente atractivo: después de todo, 
¿quién se \emph{opondría} a tener estudiantes que utilicen su propia iniciativa 
para resolver problemas del mundo real de forma realista? 
Sin embargo, pedir a los estudiante que lo hagan en un nuevo dominio les sobrecarga 
al exigirles que dominen el contenido fáctico de un dominio y sus estrategias de resolución de problemas al mismo tiempo.
Más específicamente, 
\gref{g:cognitive-load}{la teoría de la carga cognitiva} propone que 
la gente tiene que lidiar con tres cosas cuando está aprendiendo:


\begin{description}

\item[\grefdex{g:intrinsic-load}{Carga Intrínseca}{cognitive load!intrinsic}]
  es lo que la gente tiene que tener en cuenta para aprender el material nuevo.

\item[\grefdex{g:germane-load}{Carga Pertinente}{cognitive load!germane}]
  es el esfuerzo mental (deseable) requerido para vincular la nueva información con la antigua, 
  que es una de las cosas que distinguen el aprendizaje de la memorización. 


\item[\grefdex{g:extraneous-load}{Carga Extrínseca}{cognitive load!extraneous}]
es cualquier cosa que distraiga del aprendizaje.

\end{description}

La teoría de la carga cognitiva sostiene que 
la gente tiene que dividir una cantidad fija de memoria de trabajo entre estas tres cosas. 
Nuestro objetivo como profesores es maximizar la memoria disponible para manejar la carga intrínseca, 
lo cual significa reducir la carga pertinente en cada paso y eliminar la carga extrínseca.


\subsection*{Problemas de Parsons} 

Un tipo de ejercicio que puede ser explicado en términos de carga cognitiva 
se utiliza a menudo en la enseñanza de idiomas.
Supongamos que le pides a alguien que traduzca la frase, 
``¿Cómo está tu rodilla hoy?'' a lengua frisona(frisón).
Para resolver el problema, necesitan recordar tanto el vocabulario 
como la gramática, que es una carga cognitiva doble.
Si les pides que pongan ``hoe'', ``har'', ``is'', ``hjoed'' y ``knie'' en el orden correcto, 
por otro lado, les permites que se centren únicamente en el aprendizaje de la gramática.
Si escribes estas palabras en cinco fuentes o colores diferentes, 
sin embargo, has aumentado la carga cognitiva externa, porque involuntariamente 
(y posiblemente de manera inconsciente) invertirán algo de esfuerzo tratando de averiguar 
si las diferencias son significativas (\figref{f:architecture-frisian}).

\figimg{../figures/frisian.png}{Construyendo una oración}{f:architecture-frisian}

El equivalente de codificación de este
se llama \gref{g:parsons-problem}{Problema de Parsons}\footnote{Nombrado debido a uno de sus creadores.}~\cite{Pars2006}.

Cuando se enseña a la gente a programar,
puedes darles las líneas de código que necesitan para resolver un problema
y pedirles que las pongan en el orden correcto.
Esto les permite concentrarse en el flujo de control y las dependencias de datos
sin distraerse con la denominación de las variables o tratando de recordar qué funciones llamar.
Múltiples estudios han demostrado que los problemas de Parsons les toma a los estudiantes menos tiempo resolverlo
pero producen resultados educativos equivalentes~\cite{Eric2017}.


\subsection*{Ejemplos desvanecidos}

Otro tipo de ejercicio que se puede explicar en términos de carga cognitiva 
es dar a los estudiantes una serie de ejemplos desvanecidos \grefdex{g:faded-example}{faded examples}{faded example}.
El primer ejemplo de una serie presenta un uso completo de una estrategia 
particular de resolución de problemas. 
El siguiente problema es del mismo tipo, 
pero tiene algunas lagunas que el estudiante debe llenar. 
Cada problema sucesivo le da al estudiante menos \gref{g:scaffolding}{scaffolding},
hasta que se le pide que resuelva un problema completo desde cero. 
Al enseñar álgebra en la escuela secundaria, 
por ejemplo, 
podríamos comenzar con esto:


\begin{center}
\begin{tabular}{rcl}
  (4x + 8)/2	& = &	5	\\
  4x + 8	& = &	2 * 5	\\
  4x + 8	& = &	10	\\
  4x		& = &	10 - 8	\\
  4x		& = &	2	\\
  x		& = &	2 / 4	\\
  x		& = &	1 / 2
\end{tabular}
\end{center}

\noindent
y luego pide a los estudiantes que resuelvan esto:

\begin{center}
\begin{tabular}{rcl}
  (3x - 1)*3	& = &	12	\\
  3x - 1	& = &	\_ / \_	\\
  3x - 1	& = &	4	\\
  3x		& = &	\_	\\
  x		& = &	\_ / 3	\\
  x		& = &	\_
\end{tabular}
\end{center}

\noindent
y esto:

\begin{center}
\begin{tabular}{rcl}
  (5x + 1)*3	& = &	4	\\
  5x + 1	& = &	\_ 	\\
  5x		& = &	\_ 	\\
  x		& = &	\_
\end{tabular}
\end{center}

\noindent
y finalmente, esto:

\begin{center}
\begin{tabular}{rcl}
  (2x + 8)/4	& = &	1	\\
   x		& = &	\_
\end{tabular}
\end{center}

Un ejercicio similar para enseñar Python podría comenzar mostrando a los estudiantes\index{Python} 
cómo encontrar la longitud total de una lista de palabras:

\begin{minted}{text}
# total_length(["red", "green", "blue"]) => 12
define total_length(list_of_words):
    total = 0
    for word in list_of_words:
        total = total + length(word)
    return total
\end{minted}

\noindent

y luego pídales que llenen los espacios en blanco en esto 
(lo que centra su atención en las estructuras de control):


\begin{minted}{text}
# word_lengths(["red", "green", "blue"]) => [3, 5, 4]
define word_lengths(list_of_words):
    list_of_lengths = []
    for ____ in ____:
        append(list_of_lengths, ____)
    return list_of_lengths
\end{minted}

El siguiente problema podría ser este 
(que centra su atención en actualizar el resultado final):

\begin{minted}{text}
# join_all(["red", "green", "blue"]) => "redgreenblue"
define join_all(list_of_words):
    joined_words = ____
    for ____ in ____:
        ____
    return joined_words
\end{minted}

Finalmente, se pedirá a los estudiantes que escriban una función completa por su cuenta:

\begin{minted}{text}
# make_acronym(["red", "green", "blue"]) => "RGB"
define make_acronym(list_of_words):
    ____
\end{minted}

Los ejemplos difusos funcionan porque 
presentan la estrategia de resolución de problemas pieza por pieza: 
en cada paso, 
los estudiantes tienen un nuevo problema que abordar, 
que es menos intimidante que una pantalla en blanco o una hoja de papel en blanco (\secref{s:classroom-practices}). 
También anima a los estudiantes a pensar en las similitudes y diferencias entre varios enfoques, 
lo que ayuda a crear los vínculos en sus modelos mentales que ayudan a la recuperación de la información.

La clave para construir un buen ejemplo desvanecido es 
pensar en la estrategia de resolución de problemas que se pretende enseñar. 
Por ejemplo, 
los problemas de programación sobre todo utilizan el patrón de diseño del acumulador, 
en el que los resultados del procesamiento de elementos de una colección 
se agregan repetidamente a una sola variable de alguna manera para crear el resultado final.


\begin{aside}{Cognitive Apprenticeship}
  An alternative model of learning and instruction that also uses scaffolding and fading
  is \gref{g:cognitive-apprenticeship}{cognitive apprenticeship},
  which emphasizes the way in which a master passes on skills and insights to an apprentice.
  The master provides models of performance and outcomes,
  then coaches novices by explaining what they are doing and why~\cite{Coll1991,Casp2007}.
  The apprentice reflects on their own problem solving,
  e.g.\ by thinking aloud or critiquing their own work,
  and eventually explores problems of their own choosing.

  This model tells us that
  teachers should present several examples when presenting a new idea
  so that learners can see what to generalize,
  and that we should vary the form of the problem to make it clear
  what are and aren't superficial features\footnote{For a long time,
    I believed that the variable holding the value a function was going to return
    \emph{had} to be called \texttt{result}
    because my teacher always used that name in examples.}.
  Problems should be presented in real-world contexts,
  and we should encourage self-explanation to help learners organize and make sense of what they have just been taught
  (\secref{s:individual-strategies}).
\end{aside}

\begin{aside}{Aprendizaje cognitivo}
Un modelo alternativo de aprendizaje e instrucción que también usa andamiaje y desvanecimiento es \gref{g:cognitive-apprenticeship}{aprendizaje cognitivo}, que enfatiza la forma en que un maestro transmite habilidades y conocimientos a un aprendiz. El maestro proporciona modelos de desempeño y resultados, luego entrena a los principiantes explicando qué están haciendo y por qué~\cite{Coll1991,Casp2007}.
El aprendiz reflexiona sobre su propia resolución de problemas, por ejemplo, pensando en voz alta o criticando su propio trabajo, y finalmente explora problemas de su propia elección.
Este modelo nos dice que los profesores deben presentar varios ejemplos al explicar una nueva idea para que los estudiantes puedan ver qué generalizar, y que deben variar la forma del problema para dejar en claro cuáles son y cuáles no son características superficiales features\footnote{For a long time,
I believed that the variable holding the value a function was going to return
    \emph{had} to be called \texttt{result}
. Los problemas deben presentarse en contextos del mundo real, y debemos fomentar la autoexplicación para ayudar a los estudiantes a organizarse y dar sentido a lo que se les acaba de enseñar 
 (\secref{s:individual-strategies}).
\end{aside}


\subsection*{Labeled Subgoals}

\grefdex{g:subgoal-labeling}{Labeling subgoals}{labeled subgoals} means
giving names to the steps in a step-by-step description of a problem-solving process.
\cite{Marg2016,Morr2016} found that learners with labeled subgoals
solved Parsons Problems better than learners without,
and the same benefit is seen in other domains~\cite{Marg2012}.
Returning to the Python example used earlier,
the subgoals in finding the total length of a list of words or constructing an acronym are:

\subsection*{Subobjetivos etiquetados}
 
\grefdex{g:subgoal-labeling}{Labeling subgoals}{labeled subgoals}  significa 
dar nombre a los pasos en una descripción paso a paso de un proceso de resolución de problemas. 
[ Marg2016 , Morr2016 ] descubrieron que los estudiantes con subobjetivos etiquetados 
resolvían los problemas de Parsons mejor que los estudiantes sin estos, 
y se observa el mismo beneficio en otros dominios [ Marg201]. 
Volviendo al ejemplo de Python usado anteriormente, 
los objetivos secundarios para encontrar la longitud total de una lista de palabras o construir un acrónimo son:

\begin{enumerate}

\item
  Crea un valor vacío del tipo que se devolverá.

\item
  Obtiene el valor que se agregará al resultado de la variable de ciclo.

\item
  Actualiza el resultado con ese valor.

\end{enumerate}

Labeling subgoals works because grouping related steps into named chunks (\secref{s:memory-seven-plus-or-minus})\index{chunking}
helps learners distinguish what's generic from what is specific to the problem at hand.
It also helps them build a mental model of that kind of problem
so that they can solve other problems of that kind,
and gives them a natural opportunity for self-explanation (\secref{s:individual-strategies}).

Etiquetar subobjetivos funciona porque agrupar los pasos relacionados en fragmentos con nombre (Sección  3.2 ) ayuda a los estudiantes a distinguir lo que es genérico de lo que es específico del problema en cuestión. También les ayuda a construir un modelo mental de ese tipo de problema para que puedan resolver otros problemas de ese tipo y les da una oportunidad natural para la autoexplicación (Sección  5.1 ).

\subsection*{Minimal Manuals}

The purest application of cognitive load theory may be John Carroll's\index{Carroll, John}
\gref{g:minimal-manual}{minimal manual}~\cite{Carr1987,Carr2014}.
Its starting point is a quote from a user:
``I want to do something, not learn how to do everything.''
Carroll and colleagues redesigned training to present every idea as a single-page self-contained task:
a title describing what the page was about,
step-by-step instructions of how to do just one thing
(e.g.\ how to delete a blank line in a text editor),
and then several notes on how to recognize and debug common problems.
They found that rewriting training materials this way made them shorter overall,
and that people using them learned faster.
Later studies confirmed that this approach outperformed the traditional approach
regardless of prior experience with computers~\cite{Lazo1993}.
\cite{Carr2014} summarized this work by saying:

Manuales mínimos

La aplicación más pura de la teoría de la carga cognitiva puede ser el manual mínimo de John Carroll  [ Carr1987 , Carr2014]. Su punto de partida es una cita de un usuario: "Quiero hacer algo, no aprender a hacer todo". Carroll y sus colegas rediseñaron la capacitación para presentar cada idea como una tarea autónoma de una sola página: un título que describa de qué trata la página, instrucciones paso a paso sobre cómo hacer una sola cosa (por ejemplo, cómo eliminar una línea en blanco en un editor de texto), y luego varias notas sobre cómo reconocer y resolver problemas comunes. Descubrieron que reescribir los materiales de capacitación de esta manera los hacía más cortos en general y que las personas que los usaban aprendían más rápido. Estudios posteriores confirmaron que este enfoque superó al enfoque tradicional independientemente de la experiencia previa con computadoras [ Lazo1993 ]. [ Carr2014 ] resumieron este trabajo diciendo:

\begin{quote}

  Our ``minimalist'' designs sought to leverage user initiative and prior knowledge,
  instead of controlling it through warnings and ordered steps.
  It emphasized that users typically bring much expertise and insight to this learning,
  for example,
  knowledge about the task domain,
  and that such knowledge could be a resource to instructional designers.
  Minimalism leveraged episodes of error recognition, diagnosis, and recovery,
  instead of attempting to merely forestall error.
  It framed troubleshooting and recovery as learning opportunities instead of as aberrations.

\end{quote}

Nuestros diseños “minimalistas” buscaban aprovechar la iniciativa del usuario y el conocimiento previo, en lugar de controlarlo mediante advertencias y pasos ordenados. Enfatizó que los usuarios generalmente aportan mucha experiencia y conocimiento a este aprendizaje, por ejemplo, conocimiento sobre el dominio de la tarea, y que dicho conocimiento podría ser un recurso para los diseñadores instruccionales. El minimalismo aprovechó los episodios de reconocimiento, diagnóstico y corrección de errores, en lugar de intentar simplemente prevenirlos. Enmarca la resolución de problemas y la corrección como oportunidades de aprendizaje en lugar de aberraciones.

\seclbl{Otros modelos de aprendizaje}{s:architecture-theory}

Critics of cognitive load theory have sometimes argued that\index{cognitive load!criticism of}
any result can be justified after the fact by labeling things that hurt performance as extraneous load
and things that don't as intrinsic or germane.
However,
instruction based on cognitive load theory is undeniably effective.
For example,
\cite{Maso2016} redesigned a database course to remove split attention and redundancy effects
and to provide worked examples and sub-goals.
The new course reduced the exam failure rate by 34\%
and increased learner satisfaction.

Los críticos de la teoría de la carga cognitiva a veces han argumentado que cualquier resultado puede justificarse a posteriori al etiquetar las cosas que perjudican el rendimiento como cargas extrañas y las que no lo hacen como intrínsecas o pertinentes. Sin embargo, la instrucción basada en la teoría de la carga cognitiva es innegablemente efectiva. Por ejemplo, [ Maso2016 ] rediseñó un curso de base de datos para eliminar la atención dividida y los efectos de redundancia y para proporcionar ejemplos prácticos y subobjetivos. El nuevo curso redujo la tasa de reprobación del examen en un 34% y aumentó la satisfacción del estudiante.


A decade after the publication of~\cite{Kirs2006},
a growing number of people believe that cognitive load theory and inquiry-based approaches are compatible
if viewed in the right way.
\cite{Kaly2015} argues that cognitive load theory is basically micro-management of learning
within a broader context that considers things like motivation,
while~\cite{Kirs2018} extends cognitive load theory to include collaborative aspects of learning.
As with~\cite{Mark2018} (discussed in \secref{s:individual-strategies}),
researchers' perspectives may differ,
but the practical implementation of their theories often wind up being the same.

Una década después de la publicación de [ Kirs2006 ], un número creciente de personas cree que la teoría de la carga cognitiva y los enfoques basados ​​en la investigación son compatibles si se ven de la manera correcta. [ Kaly2015 ] sostiene que la teoría de la carga cognitiva es básicamente una microgestión del aprendizaje dentro de un contexto más amplio que considera cosas como la motivación, mientras que [ Kirs2018 ] extiende la teoría de la carga cognitiva para incluir aspectos colaborativos del aprendizaje. Al igual que con [ Mark2018 ] (discutido en la Sección  5.1 ), las perspectivas de los investigadores pueden diferir, pero la implementación práctica de sus teorías a menudo termina siendo la misma.

One of the challenges in educational research is that
what we mean by ``learning'' turns out to be complicated
once you look beyond the standardized Western classroom.
Two specific perspectives from \gref{g:educational-psychology}{educational psychology} have influenced this book.
The one we have used so far is \gref{g:cognitivism}{cognitivism},
which focuses on things like pattern recognition, memory formation, and recall.
It is good at answering low-level questions,
but generally ignores larger issues like,
``What do we mean by `learning'?''
and, ``Who gets to decide?''
The other is \gref{g:situated-learning}{situated learning},
which focuses on bringing people into a community
and recognizes that
teaching and learning are always rooted in who we are and who we aspire to be.
We will discuss it in more detail in \chapref{s:community}.

Uno de los desafíos en la investigación educativa es que lo que queremos decir con “aprendizaje” resulta complicado una vez que se mira más allá del aula occidental estandarizada. Dos perspectivas específicas de la psicología educativa han influido en este libro. El que hemos utilizado hasta ahora es el cognitivismo , que se centra en cosas como el reconocimiento de patrones, la formación de la memoria y el recuerdo. Es bueno para responder preguntas de bajo nivel, pero generalmente ignora cuestiones más importantes como, "¿Qué queremos decir con 'aprendizaje'?" y, "¿Quién decide?" El otro es el aprendizaje situado , que se centra en llevar a las personas a una comunidad y reconoce que la enseñanza y el aprendizaje siempre están arraigados en quiénes somos y quiénes aspiramos a ser. Lo discutiremos con más detalle en el Capítulo  13.


The \hreffoot{http://www.learning-theories.com/}{Learning Theories website}
and~\cite{Wibu2016}
have good summaries of these and other perspectives.
Besides cognitivism,
those encountered most frequently include \gref{g:behaviorism}{behaviorism}
(which treats education as stimulus/response conditioning),
\gref{g:constructivism}{constructivism}
(which considers learning an active process during which learners construct knowledge for themselves),
and \gref{g:connectivism}{connectivism}
(which holds that knowledge is distributed,
that learning is the process of navigating, growing, and pruning connections,
and which emphasizes the social aspects of learning made possible by the internet).
These perspectives can help us organize our thoughts,
but in practice,
we always have to try new methods in the class,
with actual learners,
in order to find out how well they balance the many forces in play.

El sitio web de Learning Theories y [ Wibu 2016 ] tienen buenos resúmenes de estas y otras perspectivas. Además del cognitivismo, los que se encuentran con mayor frecuencia incluyen el conductismo (que trata la educación como un condicionamiento de estímulo / respuesta), el constructivismo (que considera el aprendizaje como un proceso activo durante el cual los estudiantes construyen conocimiento por sí mismos) y el conectivismo.(que sostiene que el conocimiento se distribuye, que el aprendizaje es el proceso de navegar, crecer y podar conexiones, y que enfatiza los aspectos sociales del aprendizaje que Internet hace posible). Estas perspectivas pueden ayudarnos a organizar nuestros pensamientos, pero en la práctica, siempre tenemos que probar nuevos métodos en la clase, con estudiantes reales, para descubrir qué tan bien equilibran las muchas fuerzas en juego.

\seclbl{Ejercicios}{s:architecture-exercises}

\exercise{Crear un ejemplo desvanecido}{pares}{30'}

It's very common for programs to count how many things fall into different categories:
for example,
how many times different colors appear in an image,
or how many times different words appear in a paragraph of text.

\begin{enumerate}
\item
  Create a short example (no more than 10 lines of code) that shows people how to do this,
  and then create a second example that solves a similar problem in a similar way
  but has a couple of blanks for learners to fill in.
  How did you decide what to fade out?
  What would the next example in the series be?

\item
  Define the audience for your examples.
  For example,
  are these beginners who only know some basics programming concepts?
  Or are these learners with some experience in programming?

\item
  Show your example to a partner,
  but do \emph{not} tell them what level you think it is for.
  Once they have filled in the blanks,
  ask them to guess the intended level.

\end{enumerate}

If there are people among the trainees who don't program at all,
try to place them in different groups
and have them play the part of learners for those groups.
Alternatively,
choose a different problem domain and develop a faded example for it.

Es muy común que los programas cuenten cuántas cosas caen en diferentes categorías: por ejemplo, cuántas veces aparecen colores diferentes en una imagen o cuántas veces aparecen palabras diferentes en un párrafo de texto.
Crea un ejemplo breve (no más de 10 líneas de código) que muestre a las personas cómo hacer esto, y luego crea un segundo ejemplo que resuelva un problema similar de una manera similar pero que tenga un par de espacios en blanco para que los estudiante los completen. ¿Decides qué desvanecer? ¿Cuál sería el siguiente ejemplo de la serie?
Define la audiencia de sus ejemplos. Por ejemplo, ¿son estos principiantes que solo conocen algunos conceptos básicos de programación? ¿O estos estudiantes tienen alguna experiencia en programación?
Muestra tu ejemplo a un compañero, pero no le digas para qué nivel crees que es. Una vez que hayan llenado los espacios en blanco, pídeles que adivinen el nivel deseado.
Si hay personas entre los aprendices que no programan en absoluto, intenta ubicarlos en diferentes grupos y pídeles que hagan el papel de aprendices para esos grupos. Alternativamente, elija un dominio de problema diferente y desarrolle un ejemplo difuso para él.


\exercise{Classifying Load}{small groups}{15}

\begin{enumerate}

\item
  Choose a short lesson that a member of your group has taught or taken recently.

\item
  Make a point-form list of the ideas, instructions, and explanations it contains.

\item
  Classify each as intrinsic, germane, or extraneous.
  What did you all agree on?
  Where did you disagree and why?

\end{enumerate}

(The exercise ``Noticing Your Blind Spot'' in \secref{s:memory-exercises}
will give you an idea of how detailed your point-form list should be.)

Clasificación de carga (grupos pequeños / 15)
Elige una lección corta que un miembro de tu grupo haya enseñado o tomado recientemente.
Haz una lista en forma de puntos de las ideas, instrucciones y explicaciones que contiene.
Clasifica cada uno como intrínseco, pertinente o extraño. ¿En qué partes estaban todos de acuerdo? ¿Dónde estuvo en desacuerdo y por qué?
(El ejercicio "Cómo darse cuenta de su punto ciego" en la Sección  3.4 le dará una idea de cuán detallada debe ser su lista de formularios de puntos).


\exercise{Create a Parsons Problem}{pairs}{20}

Write five or six lines of code that does something useful,
jumble them,
and ask your partner to put them in order.
If you are using an indentation-based language like Python,
do not indent any of the lines;
if you are using a curly-brace language like Java,
do not include any of the curly braces.
(If your group includes people who aren't programmers,
use a different problem domain,
such as making banana bread.)

Crear un problema de Parsons (pares / 20)
Escribe cinco o seis líneas de código que hagan algo útil, mezclalas y pídele a tu compañero que las ponga en orden. Si estás utilizando un lenguaje basado en sangrías como Python, no utilices sangría en ninguna de las líneas; si estás utilizando un lenguaje de llaves como Java, no incluyas ninguna de las llaves. (Si tu grupo incluye personas que no son programadores, usa un dominio de problema diferente, como hacer budín/pan de plátano).


\exercise{Minimal Manuals}{individual}{20}

Write a one-page guide to doing something that your learners might encounter in one of your classes,
such as centering text horizontally
or printing a number with a certain number of digits after the decimal point.
Try to list at least three or four incorrect behaviors or outcomes the learner might see
and include a one- or two-line explanation
of why each happens and how to correct it.

Manuales mínimos (individual / 20)
Escribe una guía de una página para hacer algo que tus estudiantes puedan encontrar en una de tus clases, como centrar el texto horizontalmente o imprimir un número con un cierto número de dígitos después del punto decimal. Intentá enumerar al menos tres o cuatro resultados incorrectos que el estudiante pueda ver e incluí una explicación de una o dos líneas de por qué ocurre cada uno y cómo corregirlo.


\exercise{Cognitive Apprenticeship}{pairs}{15}

Pick a coding problem that you can do in two or three minutes
and think aloud as you work through it
while your partner asks questions about what you're doing and why.
Do not just explain what you're doing,
but also why you're doing it,
how you know it's the right thing to do,
and what alternatives you've considered but discarded.
When you are done,
swap roles with your partner and repeat the exercise.

Aprendizaje cognitivo (parejas / 15)
Elige un problema de codificación que puedas resolver en dos o tres minutos y piensa en voz alta mientras lo resuelves, al mismo tiempo tu compañero te hace preguntas sobre lo que estás haciendo y por qué. No solo explica lo que estás haciendo, sino también por qué lo estás haciendo, cómo sabes que es lo correcto y qué alternativas has considerado pero descartado. Cuando hayas terminado, intercambia roles con tu compañero y repetí el ejercicio.


\exercise{Worked Examples}{pairs}{15}

Seeing worked examples helps people learn to program faster than just writing lots of code~\cite{Skud2014},
and deconstructing code by tracing it or debugging it also increases learning~\cite{Grif2016}.
Working in pairs,
go through a 10--15 line piece of code and explain what every statement does
and why it is necessary.
How long does it take?
How many things do you feel you need to explain per line of code?

Ejemplos resueltos (pares / 15)
Ver ejemplos resueltos ayuda a las personas a aprender a programar más rápido que simplemente escribiendo mucho código [ Skud2014 ], y deconstruir/desagregar el código rastreándolo o depurándolo también aumenta el aprendizaje [ Grif2016 ]. Trabajando en parejas, revisa un fragmento de código de 10 a 15 líneas y explica qué hace cada declaración y por qué es necesaria. ¿Cuánto tiempo te tardas? ¿Cuántas cosas crees que necesitas explicar por línea de código?

\exercise{Critiquing Graphics}{individual}{30}

\cite{Maye2009,Mill2016a} presents six principles for good teaching graphics:

\begin{description}

\item[Signalling:]
  visually highlight the most important points
  so that they stand out from less-critical material.

\item[Spatial contiguity:]
  place captions as close to the graphics as practical to offset the cost of shifting between the two.

\item[Temporal contiguity:]
  present spoken narration and graphics as close in time as practical.
  (Presenting both at once is better than presenting them one after another.)

\item[Segmenting:]
  when presenting a long sequence of material or when learners are inexperienced with the subject,
  break the presentation into short segments
  and let learners control how quickly they advance from to the next.

\item[Pre-training:]
  if learners don't know the major concepts and terminology used in your presentation,
  teach just those concepts and terms beforehand.

\item[Modality:]
  people learn better from pictures plus narration than from pictures plus text,
  unless they are non-native speakers
  or there are technical words or symbols.

\end{description}

Choose a video of a lesson or talk online that uses slides or other static presentations
and rate its graphics as ``poor,'' ``average,'' or ``good'' according to these six criteria.

\section*{Review}

\figpdfhere{figures/conceptmap-cognitive-load.pdf}{Concepts: Cognitive load}{f:architecture-cognitive-load}

Gráficos críticos (individual / 30)
[ Maye2009 , Mill2016a ] presentn seis principios para una buena enseñanza de gráficos:
Señalización:
resalta visualmente los puntos más importantes para que se destaquen del material menos crítico.
 
Contigüidad espacial:
coloca los subtítulos lo más cerca posible de los gráficos para compensar el costo de cambiar entre los dos.
Contigüidad temporal:
Presenta narraciones y gráficos hablados tan seguidos en el tiempo como sea práctico. (Presentar ambos a la vez es mejor que presentarlos uno tras otro).
Segmentación:
Cuando presentes una secuencia larga de material o cuando los estudiantes no tengan experiencia con el tema, divide la presentación en segmentos cortos y deja que los estudiantes controlen la rapidez con que avanzan al siguiente.
Pre-entrenamiento:
Si los estudiantes no conocen los conceptos y la terminología principales utilizados en tu presentación, enseña solo esos conceptos y términos de antemano.
Modalidad:
las personas aprenden mejor de las imágenes más la narración que de las imágenes más el texto, a menos que no sean hablantes nativos o haya palabras o símbolos técnicos.
Elige un video de una lección o charla en línea que utilice diapositivas u otras presentaciones estáticas y califique sus gráficos como "deficientes", "promedio" o "buenos" de acuerdo con estos seis criterios.
