\chapter{Reuniones, reuniones, reuniones}\label{s:meetings}

La mayoría de la gente no es muy buena al organizar reuniones:
no llevan una agenda,
no se toman unos minutos,
hablan vagamente o se desvían en irrelevancias,
dicen algo trivial o repiten lo que otros han dicho
sólo para decir algo,
y mantienen conversaciones paralelas 
(lo cual garantiza que la reunión será una pérdida de tiempo).
Saber cómo organizar una reunión de manera eficiente
es una habilidad central para cualquiera que desee terminar bien un trabajo;
saber cómo participar en la reunión de otra persona es igual de importante
(y aunque recibe mucha menos atención, como dijo una colega una vez:
todos ofrecen entrenamiento para ser líderes
pero nadie ofrece entrenamiento para seguidores/as).

Las reglas más importantes para hacer que las reuniones sean eficientes no son secretas,
pero rara vez se siguen:

\begin{description}

\item[Decide si realmente se necesita una reunión.]
  Si el único propósito es compartir información,
  envía un breve correo electrónico en su lugar.
  Recuerda,
  puedes leer más rápido que lo que cualquiera puede hablar:
  si alguien tiene datos para que el resto del equipo los asimile,
  la forma más educada de comunicarlos es escribirlos.

\item[Escribe una agenda.]
  Si a nadie le importa lo suficiente la reunión como para escribir una lista de puntos
  de lo que se discutirá,
  la reunión probablemente no se necesita.

\item[Incluye horarios en la agenda.]
  Incluir el tiempo que le dedicarás a cada punto en la agenda,
  puede ayudarte a evitar que los primeros puntos le roben tiempo a los últimos.
  Tus primeras estimaciones con cualquier grupo nuevo serán tremendamente optimistas,
  así que revísalas nuevamente para las siguientes reuniones.
  Sin embargo,
  no deberías planear una segunda o tercera reunión
  porque no alcanzó el tiempo:
  en cambio,
  trata de averiguar por qué ocupaste tiempo extra y arregla el problema que lo originó. 

\item[Prioriza.]
  Cada reunión es un microproyecto,
  por lo tanto el trabajo debería priorizarse de la misma manera que se hace para otros proyectos:
  aquello que tendrá alto impacto pero lleva poco tiempo debería realizarse primero,
  y aquello que tomará mucho tiempo pero tiene bajo impacto debería omitirse.

\item[Haz a una persona responsable de mantener las cosas en movimiento.]
  Una persona debe tener la tarea de mantener el tratamiento 
  de cada punto en la agenda a tiempo, por ejemplo,
  llamando la atención a la gente que esté revisando el correo electrónico 
  o de aquellas que están teniendo conversaciones paralelas;
  pidiendo a aquellos que están hablando mucho que lleguen al punto
  e invitando a las personas que no intervienen a expresar su opinión.
  Esta persona \emph{no} debería ser quien más hable;
  en realidad, 
  en una reunión bien armada la persona a cargo hablará menos 
  que el resto.

\item[Pide amabilidad.]
  Que nadie llegue a ser grosero/a,
  que nadie empiece a divagar,
  y si alguien se sale del tema
  es tanto el derecho como la responsabilidad del moderador/a decir
  ``Discutamos eso en otro momento''.

\item[Sin interrupciones.]
  Los/las participantes deben levantar la mano o poner una nota adhesiva
  si quieren hablar después.
  Si la persona que está hablando no los percibe,
  quien modera la reunión debería hacerlo.

\item[Sin tecnología.]
  A menos que sea necesario por razones de accesibilidad,
  insistir amablemente en que todas las personas
  guarden sus teléfonos, tabletas y computadoras.
  (p.ej.\ Por favor, cierren sus aparatos electrónicos).

\item[Registro de minutas.]
  Alguna otra persona que no sea quien modere 
  debería tomar notas de forma puntual sobre 
  los fragmentos más importantes de información compartida,
  todas las decisiones tomadas
  y todas las tareas que se asignaron a alguna persona.

\item[Toma notas.]
  Mientras otras personas están hablando,
  los/as participantes deberían tomar notas de preguntas 
  que quieran hacer o de observaciones que quieran realizar
  (te sorprenderás lo inteligente parecerás 
  cuando llegue tu turno para hablar).

\item[Termina temprano.]
  Si tu reunión está programada de 10:00 a 11:00,
  debes intentar terminar a las 10:50 para darle tiempo 
  a las personas de pasar por el baño 
  en su camino a donde vayan luego.

\end{description}

Tan pronto termina la reunión,
envía a todos un correo electrónico con la minuta o publícala en la web:

\begin{description}

\item[La gente que no estuvo en la reunión puede mantenerse al tanto de lo que ocurrió.]
  Una página web o un mensaje de correo electrónico es una forma mucho más eficiente de ponerse al día
  que preguntarle a un/a compañero/a de equipo qué te perdiste.

\item[Cualquiera puede comprobar lo que realmente se dijo o prometió.]
  Más de una vez
  he revisado la minuta de una reunión en la que estuve
  y pensé: ``Yo dije eso?''
  o ``Espera un minuto, yo no prometí tenerlo listo para entonces!''
  Accidentalmente o no,
  muchas veces la gente recordará las cosas de manera diferente;
  escribirlo da la oportunidad al equipo de corregir errores,
  lo que puede ahorrar futuros malos entendidos.

\item[Las personas pueden ser responsables en reuniones posteriores.]
  No tiene sentido hacer listas de preguntas y puntos de acción
  si después no los sigues.
  Si estás utilizando algún tipo de sistema de seguimiento de temas,
  crea un tema por cada pregunta o tarea justo después de la reunión
  y actualiza los que se cumplieron,
  luego comienza cada reunión repasando la lista de esos temas.

\end{description}

\cite{Brow2007,Broo2016,Roge2018} tienen muchos consejos para organizar reuniones.
Según mi experiencia,
una hora de entrenamiento en cómo ser moderador
es una de las mejores inversiones que harás.

\begin{aside}{Notas Adhesivas y Bingo para Interrupción}
  Algunas personas están tan acostumbradas al sonido de su propia voz
  que insistirán en hablar la mitad del tiempo
  sin importar cuántas personas haya en la habitación.
  Para evitar esto
  entrega a todos tres notas adhesivas al comienzo de la reunión.
  Cada vez que hablen
  tienen que sacar una nota adhesiva.
  Cuando se queden sin notas
  no se les permitirá hablar hasta que todos hayan usado al menos una.
  En ese momento todos recuperan sus tres notas adhesivas.
  Esto asegura que nadie hable más de tres veces que
  la persona más callada de la reunión,
  y cambia completamente la dinámica de la mayoría de los grupos:
  personas que dejan de intentar ser escuchadas porque siempre son tapadas
  de repente tienen espacio para contribuir,
  y aquellas que hablaban con demasiada frecuencia se dan cuenta lo injustos que han sido\footnote{
    Yo ciertamente lo hice cuando me hicieron vivir esto{\ldots}
  }.

  Otra técnica es un bingo de interrupción.
  Dibuja una tabla y etiqueta las filas y columnas con los nombres de los/as participantes.
  Agrega en la celda apropiada una marca para contar 
  cada vez que alguien interrumpa a otra persona,
  y toma un momento para compartir los resultados a la mitad de la reunión.
  En la mayoría de los casos
  verás que una o dos personas son las que interrumpen siempre,
  a menudo sin ser conscientes de ello.
  Eso solo muchas veces es suficiente para detenerlas.
  Nota que esta técnica está destinada a manejar las interrupciones,
  no el tiempo de conversación:
  puede ser apropiado que las personas con más conocimiento de un tema 
  sean las que más hablan de dicho tema en una reunión,
  pero nunca es apropiado interrumpir repetidamente a las personas.
\end{aside}

\seclbl{Las reglas de Martha}{s:meetings-marthas-rules}

Las organizaciones de todo el mundo realizan sus reuniones de acuerdo a 
\hreffoot{https://en.wikipedia.org/wiki/Robert\%27s\_Rules\_of\_Order}{Reglas de Orden de Roberto (\emph{Robert's Rules of Order} en Inglés)},
pero son mucho más formales que lo requerido para proyectos pequeños.
Una ligera alternativa conocida como ``Las reglas de Martha''
puede que sea mucho mejor para la toma de decisiones por consenso~\cite{Mina1986}:

\begin{enumerate}

\item
  Antes de cada reunión
  cualquiera que lo desee puede patrocinar una propuesta compartiéndola con el grupo.
  Las propuestas deben ser compartidas al menos 24 horas antes de una reunión para ser consideradas en esa reunión,
  y deben incluir:
  \begin{itemize}
  \item un resumen de una línea;
  \item el texto completo de la propuesta;
  \item cualquier información de antecedentes requerida;
  \item pros y contras; y
  \item posibles alternativas
  \end{itemize}
  Las propuestas deberían ser a lo sumo de 2 páginas.

\item
  Se establece un quórum en una reunión si la mitad o más de las personas
  que pueden votar están presentes.

\item
  Una vez que una persona patrocina una propuesta
  es responsable de ella.
  El grupo no puede discutir o votar sobre el tema a menos que quien patrocina o 
  a quien se lo haya delegado esté presente.
  La persona patrocinadora también es responsable de presentar el tema al grupo.

\item
  Después que la persona patrocinadora presente la propuesta
  se emite un voto preliminar para la propuesta antes de cualquier discusión:
  \begin{itemize}
  \item ¿A quién le gusta la propuesta?
  \item ¿A quién le parece razonable la propuesta?
  \item ¿Quién se siente incómodo con la propuesta?
  \end{itemize}
  Los votos preliminares se pueden hacer con el pulgar hacia arriba, 
  el pulgar hacia los lados o el pulgar hacia abajo (en persona)
  o escribiendo +1, 0 o -1 en el chat en línea (en reuniones virtuales).

\item
  Si a todos/as o a la mayoría del grupo le gusta o resulta razonable la propuesta,
  se pasa inmediatamente a una votación formal sin más discusión.

\item
  Si la mayoría del grupo está disconforme con la propuesta
  se pospone para que la persona patrocinadora pueda volver a trabajar sobre ella.

\item
  Si algunos miembros se sienten disconformes pueden expresar brevemente sus objeciones.
  Luego se establece un temporizador para una breve discusión moderada por una persona facilitadora.
  Después de diez minutos o cuando nadie más tenga algo que agregar (lo que ocurra primero),
  quien facilita llama a una votación sí-o-no sobre la pregunta:
  ``¿Deberíamos implementar esta decisión aun con las objeciones establecidas?''
  Si la mayoría vota ``sí'' la propuesta se implementa.
  De lo contrario, la propuesta se devuelve a la persona patrocinadora para trabajarla más.

\end{enumerate}

\seclbl{Reuniones en línea}{s:meetings-online}

La \hreffoot{https://chelseatroy.com/2018/03/29/why-do-remote-meetings-suck-so-much/}{discusión de Chelsea Troy}
sobre por qué las reuniones en línea son a menudo frustrantes e improductivas 
rescata un punto importante:
en la mayoría de las reuniones en línea
la primera persona en hablar durante una pausa toma la palabra.
¿El resultado?
``Si tienes algo que quieres decir,
tienes que dejar de escuchar a la persona que está hablando actualmente
y en lugar de eso, enfocarte en cuándo van a detenerse o terminar, 
para que puedas saltar sobre ese nanosegundo de silencio y ser el/la primero/a en decir algo.
El formato{\ldots} alienta a los/as participantes que deseen contribuir a decir más y escuchar menos.''

La solución es chatear (charla en texto) a la par de la videoconferencia
donde las personas pueden indicar que quieren hablar.
Quien modere entonces selecciona personas de la lista de espera.
Si la reunión es grande o argumentativa,
mantener a todos silenciados
y solo permitir a quien modere liberar el micrófono de quienes participan.

\seclbl{La autopsia}{s:meetings-post-mortem}

Cada proyecto debe terminar con una autopsia
en la que los/as participantes reflexionan sobre lo que acaban de lograr
y qué podrían mejorar la próxima vez.
Su objetivo es \emph{no} señalar con el dedo de la vergüenza a las personas,
aunque si eso tiene que suceder,
la autopsia es el mejor lugar para ello.

Una autopsia se realiza como cualquier otra reunión
con algunas pautas adicionales~\cite{Derb2006}:

\begin{description}

\item[Conseguir una persona que modere y que no sea parte del proyecto]
  y no tenga interés en serlo.

\item[Reservar una hora y solo una hora.]
  Según mi experiencia,
  nada útil se dice en los primeros diez minutos de la primera autopsia de alguien,
  dado que las personas son naturalmente un poco tímidas para alabar o condenar su propio trabajo.
  Igualmente,
  no se dice nada útil después de la primera hora:
  si aún sigues hablando,
  probablemente sea porque una o dos personas
  tienen cosas que quieren sacarse del pecho
  en lugar de dar sugerencias para poder mejorar.

\item[Requerir asistencia.]
  Todas las personas que formaron parte del proyecto deben estar en la sala para la autopsia.
  Esto es más importante de lo que piensas:
  las personas que tienen más que aprender de la autopsia
  en general son menos propensas a presentarse si la reunión es opcional.

\item[Confeccionar dos listas.]
  Cuando estoy moderando
  pongo los encabezados ``Hazlo otra vez'' y ``Hazlo diferente'' en la pizarra,
  luego pido a cada persona que me dé una respuesta para cada lista, en orden y 
  sin repetir nada que ya se haya dicho.

\item[Comentar sobre acciones en lugar de individuos.]
  Para cuando el proyecto esté terminado
  es posible que algunas personas ya no sean amigas.
  No dejes que esto desvíe la reunión:
  si alguien tiene una queja específica sobre otro/a integrante del equipo,
  pídeles que critiquen un evento o decisión en particular.
  ``Tiene una mala actitud'' \emph{no} ayuda a nadie a mejorar.

\item[Priorizar las recomendaciones.]
  Una vez que los pensamientos de todos/as estén al descubierto
  ordénalos según importancia para mantenerlos
  y para cambiarlos.
  Probablemente solo podrás abordar uno o dos de cada lista en tu próximo proyecto,
  pero si haces eso cada vez
  tu vida mejorará rápidamente.

\end{description}
