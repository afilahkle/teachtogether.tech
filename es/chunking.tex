\chapter{Solución del ejercicio de particionar}\label{s:chunking}

\begin{reviewer}
{Priscilla Minotti}
{Yanina Bellini Saibene y Natalia Morandeira}
\end{reviewer}

Mira el último ejercicio en el \chapref{s:memory} para la representación completa de estos símbolos.

\figpdfhere{figures/chunking-chunked.pdf}{Representación fragmentada}{f:chunking-chunked}{El diagrama presenta una cuadrícula de tres por tres sin las líneas externas. En el primer renglón tiene los números uno a tres, en el segundo cuatro a seis y en el tercero siete a nueve. De esta manera el número uno está rodeado por una línea vertical a su derecha y una línea horizontal debajo. El número dos está rodeado por dos líneas verticales (a su derecha y a su izquierda) y una línea horizontal en su parte inferior. El número tres está rodeado por una línea vertical a su izquierda y una línea horizontal en su parte inferior. Al número cuatro lo rodean dos líneas horizontales (arriba y abajo) y una línea vertical a su derecha. El número cinco está dentro de un cuadrado con todas las líneas marcadas. El número seis tiene las dos líneas horizontales (arriba y abajo) y una línea vertical a su izquierda. El número siete tiene una línea horizontal arriba y otra linea vertical a la derecha. El número ocho está rodeado por dos líneas verticales (una a cada lado) y una línea horizontal en la parte superior. Finalmente, el número nueve tiene una línea vertical a la izquierda y una línea horizontal arriba.}
