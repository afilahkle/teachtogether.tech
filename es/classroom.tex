\chapter{En el salón de clase}\label{s:classroom}

El capítulo anterior describió cómo practicar la entrega de lecciones
y describió un método ---programación en vivo--- que
permite a las/los docentes adaptarse al ritmo y los intereses de sus estudiantes.
Este capítulo describe otras prácticas que también son útiles en clases de programación.

Antes de describirlas,
vale la pena detenerse un momento para establecer expectativas.
El mejor método de enseñanza que conocemos es la tutoría individual:\index{individual tutoring (effectiveness of)}
\cite{Bloo1984} descubrió que las/los estudiantes a quienes se les enseñó uno a uno
hicieron dos desviaciones estándar mejor que los/las que aprendieron mediante una clase convencional,
es decir que los/las estudiantes con tutoría individual superaron al
98\% de estudiantes a los que se les dio clases.
Aunque,
si bien la tutoría y el aprendizaje fueron las formas más comunes de transmitir conocimientos
a lo largo de la mayor parte de la historia,
la industrialización de la educación formal la ha convertido en la excepción en la actualidad.
A pesar de la explosión en torno a la inteligencia artificial,\index{artificial intelligence (hype)}
esto no va a cerrar este círculo pronto,
por lo que cada método que se describe a continuación es esencialmente
un intento de abordar la efectividad de la tutoría individual a escala.

\seclbl{Hacer cumplir el código de conducta}{s:classroom-coc}

Lo más importante que he aprendido sobre la enseñanza en los últimos 30 años es
lo importante que es para todos/as tratar a los/las demás con respeto,
tanto dentro como fuera de clase.
Si usas este material de alguna manera,
por favor adopta un Código de Conducta como el de \appref{s:conduct}\index{Code of Conduct}
y exige a todos/as los/las que participan en tus clases que lo respeten.
No puedes evitar que las personas sean ofensivas,
como tampoco las leyes contra el robo evitan que las personas roben,
pero \emph{puedes} aclarar las expectativas y las consecuencias,
y señalar que estás tratando de que tu clase sea acogedora para todos/as.

Pero un Código de Conducta sólo es útil si se hace cumplir.
Si crees que alguien ha violado el tuyo,
puedes advertirles,
pedirles que se disculpen,
y / o expulsarlos/las,
dependiendo de la gravedad de la infracción y si crees o no que fue intencional.
Hagas lo que hagas:\index{Code of Conduct!enforcement}

\begin{description}

\item[Hazlo frente a testigos/as.]
  La mayoría de las personas bajarán el tono de su lenguaje y hostilidad frente a una audiencia,
  y tener a alguien más presente asegura que
  la discusión posterior no degenere en afirmaciones contradictorias sobre quién dijo qué.

\item[Si expulsas a alguien, comunícalo al resto de la clase y explica por qué.]
  Esto ayuda a evitar que los rumores se difundan
  y muestra que tu Código de Conducta realmente significa algo.

\item[Envía un correo electrónico a la persona infractora tan pronto como puedas]
  para resumir lo que sucedió y los pasos que tomaste,
  y copia el mensaje a los/las anfitriones/as del taller o a alguno de tus colegas educadores
  para que haya un registro contemporáneo de la conversación.
  Si la persona infractora responde,
  no participes en un debate largo:
  nunca es productivo.
 
\end{description}

Lo que sucede fuera de la clase importa al menos tanto como lo que sucede dentro de ella~\cite{Part2011},
por lo que debes proporcionar una forma para que los/las estudiantes informen sobre los problemas que tú mismo no puedes ver.
Un paso es pedirle a alguien fuera de tu grupo que sea el primer punto de contacto;
de esta manera,
si alguien quiere presentar una queja sobre ti o alguno de tus colegas educadores,
tiene cierta garantía de confidencialidad y acción independiente.
\cite{Auro2019} tiene muchos otros consejos
y es a la vez breve y práctico.

\seclbl{Instrucción de pares}{s:classroom-peer}

No importa lo buena que sea una persona que enseña,
solo puede decir una cosa a la vez.
Entonces, ¿cómo puede aclarar muchos conceptos erróneos diferentes en un tiempo razonable?
La mejor solución desarrollada hasta ahora es una técnica llamada \gref{g:peer-instruction}{instrucción de pares}.
Originalmente creada por Eric Mazur en Harvard~\cite{Mazu1996},\index{Mazur, Eric}
se ha estudiado extensamente en una amplia variedad de contextos,
incluida la programación~\cite{Crou2001,Port2013},
y~\cite{Port2016} descubrió que los/las estudiantes valoran la instrucción de sus pares incluso en el primer contacto.

La instrucción entre pares intenta proporcionar instrucción individualizada de manera escalable
intercalando la evaluación formativa con la discusión del estudiante:

\begin{enumerate}

\item
  Haz una breve introducción al tema.

\item
  Ofrece a los estudiantes una pregunta de opción múltiple que investigue sus conceptos erróneos.
  (en lugar de probar el simple conocimiento de hechos).

\item
  Haz que todos los estudiantes voten por sus respuestas a las preguntas de opción múltiple.

  \begin{itemize}

  \item
    Si todos los estudiantes tienen la respuesta correcta, continúa.

  \item
    Si todos tienen la misma respuesta incorrecta,
    aborda ese error específico.

  \item
    Si tienen una combinación de respuestas correctas e incorrectas,
    dales varios minutos para discutir entre ellos/as en grupos de 2 a 4,
    luego que vuelvan a votar.

  \end{itemize}

\end{enumerate}

Asi como
\hreffoot{https://www.youtube.com/watch?v=2LbuoxAy56o}{este video} muestra que
la discusión en grupo mejora significativamente la comprensión de los estudiantes
porque descubre lagunas en su razonamiento y los obliga a aclarar su pensamiento.
Volver a sondear a la clase le permite al/la docente saber si pueden seguir adelante
o si se necesitan más explicaciones.
Una ronda final de explicación adicional después de que se presenta la respuesta correcta
les da a los estudiantes una oportunidad más para solidificar su comprensión.

Pero, ¿podría ser esto un falso positivo?
¿Los resultados están mejorando debido a una mayor comprensión durante la discusión
o simplemente por un efecto de seguimiento del líder (``vota como Jane, ella siempre tiene la razón'')?
\cite{Smit2009} probó esto siguiendo la primera pregunta con una segunda
que los estudiantes respondieron individualmente.
Descubrieron que la discusión entre pares mejora la comprensión,
incluso cuando ninguno de los estudiantes de un grupo de discusión sabía originalmente la respuesta correcta.
Siempre que exista diversidad de opiniones dentro del grupo,
sus conceptos erróneos se anulan.

\newpage
\begin{aside}{Tomar una posición}
  Es importante que tus estudiantes voten públicamente
  para que no puedan cambiar de opinión después y racionalizarlo
  con excusas como: ``Acabo de interpretar mal la pregunta.''
  Gran parte del valor de la instrucción entre pares proviene de la hipercorrección
  de obtener la respuesta incorrecta
  y tener que pensar en las razones del porqué de esta respuesta
  (\secref{s:individual-strategies}).
\end{aside}

\seclbl{Enseñar en comunidadr}{s:classroom-together}

\grefdex{g:co-teaching}{Co-enseñar}{co-teaching} describe cualquier situación
en la que dos educadores trabajan juntos en la misma clase.
\cite{Frie2016} describe muchas formas de hacer esto:

\begin{description}

\item[Enseñar en equipo:]
  Ambos educadores entregan un único flujo de contenido en conjunto,
  turnándose como músicos haciendo solos.

\item[Enseñar y ayudar:]
  El educador A enseña mientras el educador B se mueve por el aula
  para ayudar a los estudiantes con dificultades.

\item[Enseñanza alternativa:]
  El educador A proporciona a un pequeño grupo de estudiantes
  una instrucción más intensiva o especializada
  mientras que el educador B ofrece una lección general al grupo principal.
 
\item[Enseñar y observar:]
  El educador A enseña mientras el educador B observa a los estudiantes
  y recopila datos sobre su comprensión para ayudar a planificar las lecciones futuras.

\item[Enseñanza paralela:]
  La clase se divide en dos
  y los educadores presentan el mismo material simultáneamente a cada grupo.

\item[Enseñanza por estaciones:]
  Los estudiantes se dividen en pequeños grupos
  que rotan de una estación o actividad a la siguiente
  mientras los educadores supervisan donde sea necesario.

\end{description}

Todos estos modelos crean más oportunidades para la transferencia involuntaria de conocimientos que la enseñanza por sí sola.
\index{unintended knowledge transfer}
Enseñar en equipo es particularmente beneficioso en talleres de un día:
le da a la voz de cada docente la oportunidad de descansar
y reduce el riesgo de que estén tan exhaustos/as al final del día
que comiencen a hablar bruscamente con sus estudiantes
o a manipularles el teclado.

\begin{aside}{Ayudar}
  Muchas personas que no se sienten cómodas enseñando
  están dispuestas y son capaces de brindar asistencia técnica en clase.
  Pueden ayudar a los estudiantes con la configuración e instalación,
  responder preguntas técnicas durante los ejercicios,
  supervisar la sala para detectar personas que puedan necesitar ayuda
  o poner atención a las notas compartidas (\secref{s:classroom-notetaking}),
  y responder preguntas
  o recordar al educador que lo haga durante los descansos.

  Los ayudantes a veces son personas que se están capacitando para convertirse en docentes
  (es decir, son el/la docente B en el modelo de enseñar y apoyar),
  pero también pueden ser parte del personal de apoyo técnico de la institución anfitriona,
  ex estudiantes de la clase
  o estudiantes avanzados que ya conocen bien el material.
  Usar a estos últimos como ayudantes es doblemente efectivo: no solo es más probable que comprendan los problemas que tienen sus compañeros/as,
  sino que también evita que se aburran.
  Esto ayuda a que toda la clase se mantenga comprometida porque el aburrimiento es contagioso:
  si un puñado de personas comienza a pagar,
  las personas que los rodean seguirán su ejemplo.
\end{aside}

Si tú y un compañero/a están co-enseñando:

\begin{itemize}

\item
  Toma de 2 a 3 minutos antes del comienzo de cada clase
  para confirmar quién está enseñando qué.
  Si tienen tiempo,
  intenten dibujar o revisar juntos un mapa conceptual.

\item
  Usa ese tiempo para hacer también un par de señales con las manos.
  ``Vas demasiado rápido'',
  ``habla'',
  ``ese estudiante necesita ayuda''
  y ``es hora de ir al baño'' son todos útiles.

\item
  Cada persona debe enseñar durante al menos 10 a 15 minutos seguidos,
  ya que los estudiantes se distraen con cambios más frecuentes.

\item
  La persona que no está enseñando no debe interrumpir,
  ofrecer correcciones o elaboraciones,
  o hacer cualquier otra cosa que distraiga de lo que está haciendo o diciendo la persona que enseña.
  La única excepción es hacer preguntas guía
  si los/as estudiantes parecen pasivos/as o inseguros/as de sí mismos.
 
\item
  Cada persona debería tomarse un par de minutos antes de empezar la clase
  para ver lo que su compañero/a va a enseñar después de su turno,
  y luego \emph{no} presentar nada de ese material.

\item
  La persona que no está enseñando debe mantenerse comprometida con la clase,
  no debe hacer otra cosa, como ponerse al día con su correo electrónico.
  Supervisar las notas compartidas (\secref{s:classroom-notetaking}),
  vigilar a los/as estudiantes para ver quién tiene dificultades,
  anotar algunos comentarios para dárselos a su co-docente en el próximo receso---cualquier
  cosa que contribuya a la lección es mejor que cualquier cosa que no lo haga.
 
\end{itemize}

Lo más importante es que,
cuando termine la clase, tomen unos minutos para felicitarse o compadecerse unos de otros:
tanto en la enseñanza como en la vida,
la pena compartida disminuye y la alegría compartida aumenta.

\seclbl{Evaluar conocimientos previos}{s:classroom-prior}

Cuanto más sepas sobre tus estudiantes antes de comenzar a enseñar,
más podrás ayudarles.
Dentro de un sistema escolar formal,
los pre-requisitos de tu curso te dirán algo sobre
lo que probablemente ya sepan.
Sin embargo,
en un entorno de campo libre,
tus estudiantes pueden ser mucho más diversos,
por lo que es posible que quieras darles una breve encuesta o cuestionario antes de tu clase
para averiguar qué conocimientos y habilidades tienen.

Pedir a las personas que se califiquen a sí mismas en una escala del 1 al 5 no tiene sentido
porque cuanto menos saben las personas sobre un tema, \index{self-assessment (perils of)}
con menor precisión pueden estimar sus conocimientos
(\figref{f:classroom-dunning-kruger},
de \hreffoot{Neurologica}{https://theness.com/neurologicablog/index.php/misunderstanding-dunning-kruger/}),
un fenómeno llamado  \gref{g:dunning-kruger-effect}{Dunning-Kruger effect}~\cite{Krug1999}.
Por el contrario,
las personas que son miembros de grupos subrepresentados a menudo subestiman sus habilidades.

\figimg{figures/dunning-kruger.png}{The Dunning-Kruger Effect}{f:classroom-dunning-kruger}

En lugar de pedirles a las personas que se autoevalúen,
puedes preguntarles con qué facilidad podrían completar algunas tareas específicas.
Sin embargo,
hacer esto es arriesgado
porque la escuela entrena a las personas
para que traten cualquier cosa que parezca un examen como algo que tienen que aprobar
en lugar de una oportunidad para dar forma a la instrucción.
Si alguien responde ``No sé'' incluso a un par de preguntas en su pre-evaluación,
podría concluir que tu clase es demasiado avanzada para ellos.
Por lo tanto, podrías asustar a muchas de las personas que más deseas ayudar.

\secref{s:checklists-preassess} presenta un breve cuestionario de preevaluación
que es poco probable que la mayoría de los potenciales estudiantes encuentren intimidante.
Si usas este formulario o una herramienta parecida,
trata de hacer un seguimiento a las personas que \emph{no} responden para averiguar por qué no lo hacen
y compara tu evaluación con la que ellos/as se hicieron a sí mismos
para mejorar tus preguntas.

\seclbl{Plan para habilidades mixtas}{s:classroom-mixed}
\index{mixed abilities (accommodating)}

Si tus estudiantes tienen niveles muy diversos de conocimientos previos,
puedes terminar fácilmente en una situación en la que un tercio de tu clase se pierde
y un tercio se aburre.
Eso es poco satisfactorio para todos,
pero hay algunas estrategias que puedes utilizar para manejar la situación:

\begin{itemize}

\item
  Antes de realizar un taller,
  comunica claramente su nivel a todos mostrando algunos ejemplos de ejercicios que se les pedirá que completen.
  Esto ayuda a los potenciales participantes a medir el nivel de la clase
  de manera mucho más efectiva que una lista de temas en forma de puntos.
 
\item
  Proporciona ejercicios adicionales a su propio ritmo
  para que los estudiantes más avanzados no terminen temprano y se aburran.

\item
  Pon atención a los estudiantes que se están quedando atrás
  y actúa temprano para que no se frustren ni se den por vencidos.

\item
  Pide a tus estudiantes más avanzados ayudar a las personas que están a su lado
  (vea \secref{s:classroom-pair} debajo).

\end{itemize}

Otra forma de adaptarse a las habilidades mixtas es
hacer que todos trabajen en el material por su cuenta a su propio ritmo,
como lo harían en un curso en línea,
pero hacerlo simultáneamente y lado a lado
con los ayudantes que deambulan por la sala para despejar las dudas de las personas.
Algunas personas llegarán tres o cuatro veces más lejos que otras cuando los talleres se realicen de esta manera,
pero todos habrán tenido un día gratificante y desafiante.

\begin{aside}{Falsos principiantes}
  Un \gref{g:false-beginner}{falso principiante} es alguien
  que ha estudiado un lenguaje antes pero lo está aprendiendo de nuevo.
  Pueden ser indistinguibles de \grefdex{g:absolute-beginner}{principiantes absolutos}{absolute beginner}
  en las pruebas de pre-evaluación,
  pero pueden moverse mucho más rápido una vez que comienza la clase
  porque están volviendo a aprender en lugar de aprender por primera vez.
 
  Ser un falso principiante es a menudo una señal de \gref{g:preparatory-privilege}{privilegio preparatorio}~\cite{Marg2010},
  y los falsos principiantes son comunes en las clases de programación de rango abierto.
  Por ejemplo,
  un niño cuya familia es lo suficientemente acomodada como para haberlos enviado a un campamento de verano de robótica
  puede tener un desempeño pobre en una pre-evaluación de conocimientos de programación
  porque el material no está fresco en su mente,
  pero aún tiene una ventaja sobre un niño de un trasfondo menos afortunado.
  Las estrategias descritas anteriormente pueden ayudar a nivelar el campo de juego en casos como este,
  pero nuevamente, la solución real es usar tu propio privilegio
  para abordar factores fuera de clase más importantes~\cite{Part2011}.
\end{aside}

Lo más importante es aceptar que
no puedes ayudar a todos todo el tiempo.
Si reduces la velocidad para acomodar a dos personas que les está costando,
estás fallando a las otras dieciocho.
Del mismo modo,
si dedicas unos minutos a hablar sobre un tema avanzado con un estudiante que está aburrido,
el resto de la clase se sentirá excluida.

\seclbl{Pair Programming}{s:classroom-pair}

\grefdex{g:pair-programming}{Pair programming}{pair programming} es una practica del desarrollo de software
en la que \hreffoot{https://www.youtube.com/watch?v=vgkahOzFH2Q}{dos programadores trabajan juntos en una computadora}.
Una persona (el conductor) escribe
mientras que la otra (el navegador) ofrece comentarios y sugerencias,
y los dos cambian de función varias veces por hora.

El Pair programming es una práctica eficaz en el trabajo profesional~\cite{Hann2009}
y también es una buena forma de enseñar:
los beneficios incluyen una mayor tasa de éxito en los cursos introductorios,
un mejor software
y una mayor confianza de los estudiantes en sus soluciones.
También hay evidencia de que estudiantes de grupos subrepresentados
se benefician incluso más que otros~\cite{McDo2006,Hank2011,Port2013,Cele2018}.
Los compañeros pueden ayudarse mutuamente durante los ejercicios prácticos,
aclarar los conceptos erróneos de los demás cuando se presenta la solución
y discutir intereses comunes durante los descansos.
Lo he encontrado particularmente útil con las clases de habilidades mixtas,
ya que las parejas son más homogéneas que los individuos.

Cuando utilices el pair programming,
coloca a \emph{todos} en parejas,
no solo a los estudiantes que tienen dificultades,
para que nadie se sienta señalado.
También es útil que las personas se sienten en lugares nuevos
(y, por lo tanto, se emparejen con diferentes personas)
de forma regular,
y que las personas cambien de rol dentro de cada pareja tres o cuatro veces por hora
para que la personalidad más fuerte de cada pareja no domine la sesión.

Si tus estudiantes son nuevos en el pair programming,
toma unos minutos para demostrar cómo se ve realmente
para que comprendan que
la persona que no tiene las manos en el teclado
no debe sentarse y mirar.
Finalmente,
diles que las personas que se enfocan en tratar de completar la tarea lo más rápido posible
son menos justas al compartir~\cite{Lewi2015}.

\begin{aside}{Cambio de parejas}
  Los/las docentes tienen opiniones encontradas sobre si se debería exigir a las personas que cambien de pareja a intervalos regulares.
  Por un lado, les da a todos la oportunidad de obtener nuevos conocimientos y hacer nuevos amigos.
  Por otro lado,
  trasladar las computadoras y los adaptadores de corriente a escritorios nuevos varias veces al día es desgastante
  y la combinación puede resultar incómoda para los introvertidos.
  Dicho esto,
  \cite{Hann2010} encontró una correlación débil entre los ``Cinco Grandes'' rasgos de personalidad 
  y el rendimiento en la programación por parejas,
  aunque un estudio anterior~\cite{Wall2009} encontró que
  las parejas cuyos miembros tenían diferentes niveles de rasgos de personalidad se comunicaban con más frecuencia.
\end{aside}

\seclbl{Tomar notas{\ldots}¿juntos/as?}{s:classroom-notetaking}
\index{note-taking}

La toma de notas es una forma de elaboración en tiempo real (\secref{s:individual-strategies}):
te obliga a organizar y reflexionar sobre el material a medida que se presenta,
lo que a su vez aumenta la probabilidad de que lo transfieras a la memoria a largo plazo.
Muchos estudios han demostrado que
tomar notas mientras se aprende mejora la retención~\cite{Aike1975,Boha2011}.
Si bien aún no se ha estudiado ampliamente~\cite{Ornd2015,Yang2015},
he descubierto que hacer que los/as estudiantes tomen notas juntos en una página en línea compartida también es eficaz:

\begin{itemize}

\item
  Permite a las personas comparar lo que creen que ellas están escuchando
  con lo que están escuchando las otras personas,
  lo que les ayuda a llenar los vacíos y corregir conceptos erróneos de inmediato.
 
\item
  Les da a los/as estudiantes más avanzados de la clase algo útil que hacer.
  En lugar de aburrirse y revisar Instagram durante la clase,
  pueden tomar la iniciativa para registrar lo que se dice,
  lo que los/as mantiene comprometidos
  y permite que los/as estudiantes menos avanzados concentren más su atención en el nuevo material.
 
\item
  Las notas que toman los/las estudiantes suelen ser más útiles para \emph{ellos(as)}
  que las que el/la docente prepararía de antemano,
  ya que es más probable que los/las estudiantes escriban lo que realmente encontraron nuevo
  en lugar de lo que el/la docente predijo que sería nuevo.

\item
  Mirar notas recientes mientras los/las estudiantes están trabajando en un ejercicio
  ayuda al docente a descubrir si la clase se perdió o si algo se entendió mal.
 
\end{itemize}

\begin{aside}{¿Es el lápiz más poderoso que el teclado?}
  \cite{Muel2014} informó que tomar notas en una computadora
  es generalmente menos efectivo que tomar notas con lápiz y papel.
  Si bien su resultado fue ampliamente compartido,
  \cite{More2019} no pudo replicarlo.
\end{aside}

Si los estudiantes toman notas juntos,
también puedes hacer que peguen fragmentos cortos de código
y respuestas en forma de puntos o de oraciones a las preguntas de evaluación formativa.
Para evitar que todos intenten editar el mismo par de líneas al mismo tiempo,
haz una lista con el nombre de todos y pégala en el documento
siempre que quieras que cada persona responda una pregunta.

Los estudiantes a menudo descubren que tomar notas juntos les distrae
la primera vez que lo intentan porque tienen que dividir su atención entre
lo que dice el/la docente
y lo que escriben sus compañeros(as) (\secref{s:architecture-brain}).
Si estás trabajando con un grupo en particular solo una vez,
debes prestar atención a los consejos en \secref{s:classroom-innovate}
y pedirles que tomen notas individualmente.

\begin{aside}{Puntos para Mejorar}
  Una forma de demostrarles a los estudiantes que están aprendiendo \emph{contigo},
  no solo \emph{de} ti,
  es permitirles que tomen notas editando (una copia de) su lección.
  En lugar de publicar archivos PDF para que los descarguen,
  crea copias editables de tus diapositivas, notas y ejercicios
  en una wiki,
  un documento de Google
  o cualquier otra cosa que te permita revisar y comentar los cambios.
  Dale crédito a las personas por corregir errores,
  aclarar explicaciones,
  agregar nuevos ejemplos
  y escribir nuevos ejercicios no reduce tu carga de trabajo,
  pero aumenta el compromiso y la duración de la lección.
  (\secref{s:process-maintainability}).
\end{aside}

\seclbl{Notas adhesivas}{s:classroom-sticky-notes}

Las notas adhesivas son una de mis herramientas de enseñanza favoritas,
y no soy el único que ama su versatilidad,
portabilidad, adherencia, capacidad de plegado
y aroma sutil pero atractivo~\cite{Ward2015}.

\subsection*{Como indicadores de estado}
\index{sticky notes!as status flags}

Dale a cada estudiante dos notas adhesivas de diferentes colores,
como naranja y verde.
Estos se pueden sostener para votar,
pero su uso real es como indicadores de estado.
Si alguien ha completado un ejercicio y quiere que lo revisen,
coloca la nota adhesiva verde en su laptop;
si tienen un problema y necesitan ayuda,
colocan el naranja.
Esto funciona mucho mejor a que la gente levante la mano:
es más discreto (lo que significa que es más probable que lo hagan),
pueden seguir trabajando mientras su bandera está levantada en lugar de intentar escribir con una sola mano,
y el/la docente puede ver rápidamente desde el frente del salón en qué estado se encuentra la clase.
Los indicadores de estado son particularmente útiles en clases donde las personas con habilidades mixtas
están trabajando en el material a su propio ritmo (\secref{s:classroom-mixed}).

Una vez que los/as estudiantes se sientan cómodos con dos notas adhesivas,
dales una tercera que puedan usar cuando tengan el cerebro lleno
o necesiten un descanso para ir al baño \footnote{Una colega me dijo una vez que
la unidad básica de enseñanza es la vejiga.
Cuando le contesté que yo nunca había pensado en eso,
dijo: ``Obviamente nunca has estado embarazado''}.
Nuevamente,
es más probable que los adultos muestren una nota adhesiva a que levanten la mano,
y una vez que una color azul es levantada,
generalmente le sigue una ráfaga de otras.

\subsection*{Para distribuir atención}
\index{sticky notes!to distribute attention}

También se pueden usar notas adhesivas para garantizar que la atención del docente se distribuya de manera justa.
Haz que cada estudiante escriba su nombre en una nota adhesiva
y que lo ponga en su computadora portátil.
Cada vez que el/la docente los llama o responde una de sus preguntas,
pone su nota adhesiva debajo.
Una vez que todas las notas adhesivas están abajo,
todos vuelven a poner las suyas arriba.

Esta técnica hace que sea fácil para el/la docente ver con quién no ha hablado recientemente,
lo que a su vez ayuda a evitar prejuicios inconscientes
e interactuar preferentemente con sus estudiantes más extrovertidos.
Sin una verificación como esta,
es muy fácil crear un ciclo de retroalimentación en el que los extrovertidos reciben más atención,
lo que los lleva a mejorar,
lo que a su vez los lleva a recibir más atención,
mientras que los/as estudiantes más introvertidos, menos seguros o marginados se quedan detrás~\cite{Alvi1999,Juss2005}.

También muestra a los/as estudiantes que la atención se distribuye de manera justa,
de modo que cuando se les llame,
no se sentirán como si los estuvieran molestando.
Cuando trabajo con un grupo nuevo,
permito que las personas tomen sus propias notas adhesivas
durante la primera o la segunda hora de clase
si prefieren que no los llamen.
Si continúan haciendo esto a medida que pasa el tiempo,
trato de tener una conversación tranquila con ellos para averiguar por qué
y para ver si hay algo que pueda hacer para que se sientan más cómodos/as.

\subsection*{Como tarjetas de actas}
\index{sticky notes!as minute cards}

También puedes usar notas adhesivas como \grefdex{g:minute-cards}{tarjetas de actas}{minute card}.
Antes de cada receso,
los/as estudiantes se toman un minuto para escribir una cosa en la nota adhesiva verde
que creen que será útil
y otra cosa en la nota naranja
que encontraron demasiado rápido,
demasiado lento,
confuso
o irrelevante.
Mientras disfrutan de su café o almuerzo,
revisa sus notas y busca patrones.
Se necesitan menos de cinco minutos para ver qué disfrutan los/as estudiantes de una clase de 40 personas,
en qué se sienten confundidos,
qué problemas tienen
y qué preguntas aún no has respondido.

Los/as estudiantes no deben firmar sus tarjetas de actas:
están pensadas como comentarios anónimos.
La técnica de uno arriba/uno abajo descrita en \secref{s:classroom-practices}
es una oportunidad para una retroalimentación colectiva.

\seclbl{Nunca una página en blanco}{s:classroom-blank}

Los talleres de programación y otros tipos de clases
se pueden construir en torno a un conjunto de ejercicios independientes,
desarrollar un solo ejemplo extendido en etapas
o utilizar una estrategia mixta.
Las dos ventajas principales de los ejercicios independientes son que
las personas que se retrasan pueden volver a sincronizarse fácilmente
y que los/las desarrolladores(as) de lecciones pueden agregar, eliminar y reorganizar el material a voluntad
(\secref{s:process-maintainability}).
Un solo ejemplo extendido,
por otro lado,
mostrará a los/las estudiantes cómo encajan las partes y piezas que están aprendiendo:
en el lenguaje educativo,
les brinda más oportunidades para integrar sus conocimientos.

Independientemente del enfoque que adoptes,
los/las principiantes nunca deben comenzar a hacer ejercicios con una página o pantalla en blanco,
ya que a menudo les resulta intimidante o desconcertante.
Si te han seguido mientras realizas la codificación en vivo,
pídeles que agreguen algunas líneas más
o que modifiquen el ejemplo que creaste.
Alternativamente, si están tomando notas juntos,
pega algunas líneas de código de inicio en el documento
para que lo amplíen o modifiquen.

Modificar el código existente en lugar de escribir código nuevo desde cero
no sólo proporciona a los estudiantes una estructura:
también está más cerca de lo que harán en la vida real.
Sin embargo,
ten en cuenta que los estudiantes pueden distraerse tratando de comprender todo el código de inicio
en lugar de hacer su propio trabajo.
El \texttt{public static void main()} de Java
o un conjunto de sentencias \texttt{import} al inicio de un programa en Python
podría tener sentido para ti,
pero es una carga extraña para ellos(as). (\chapref{s:architecture}).

\seclbl{Configuración de tus estudiantes}{s:classroom-setup}

Los/las estudiantes free-range a menudo quieren traer sus propias computadoras
y dejar la clase con esas máquinas configuradas para hacer un trabajo real.
Por lo tanto, los/as docentes free-range deberían prepararse para enseñar tanto en Windows como en MacOS \footnote{``¡Y Linux!'', grita alguien desde el fondo del salón.},
aunque sería más sencillo exigir a los/as estudiantes que utilicen solo uno.

\begin{aside}{Denominadores comunes}
  Si tus participantes utilizan diferentes sistemas operativos,
  trata de evitar el uso de funciones que sean específicas de uno solo
  y señala las que \emph{utilices}.
  Por ejemplo,
  los controles y el comportamiento de ''minimizar ventana'' en Windows son diferentes
  a los de MacOS.
\end{aside}

No importa cuántas plataformas tengas que manejar,
coloca instrucciones de configuración detalladas en el sitio web del curso
y envía un correo electrónico a los/las estudiantes un par de días antes de que comience el taller
para recordarles que realicen la configuración.
Algunas personas seguirán apareciendo sin el software requerido porque
tuvieron problemas,
no pudieron encontrar tiempo para completar todos los pasos
o simplemente son el tipo de persona que nunca sigue las instrucciones por adelantado.
Para detectar esto,
haz que todos ejecuten un comando simple tan pronto como lleguen
y muestren el resultado a los/as docentes,
luego busque ayudantes y otros(as) estudiantes
para ayudar a las personas que se han encontrado con problemas.

\begin{aside}{Máquinas Virtuales}
  Algunas personas usan herramientas como \hreffoot{http://docker.com}{Docker}
  para poner máquinas virtuales en las computadoras de sus estudiantes\index{virtual machines}
  para que todos trabajen exactamente con las mismas herramientas,
  pero esto presenta un nuevo conjunto de problemas.
  Las máquinas más antiguas o más pequeñas simplemente no son lo suficientemente rápidas para ejecutarlas,
  los estudiantes luchan por alternar
  entre dos conjuntos diferentes de atajos de teclado para cosas como copiar y pegar,
  e incluso los profesionales competentes se confundirán sobre qué está sucediendo exactamente y dónde.
\end{aside}

La configuración es tan complicada que
muchos/as docentes prefieren que los/las estudiantes usen herramientas basadas en el navegador.
Sin embargo,
esto hace que la clase dependa del WiFi\index{institutional WiFi (perils of)}
(que puede ser de calidad muy variable)
y no satisface el deseo de los/las estudiantes de irse con sus propias máquinas listas para su uso en el mundo real.
Mientras herramientas basadas en la nube como \hreffoot{https://glitch.com/}{Glitch}
y \hreffoot{http://rstudio.cloud}{RStudio Cloud} se vuelven más robustas,
esta última consideración se está volviendo menos importante.

Una última forma de abordar los problemas de configuración es dividir la clase en varios días
y hacer que las personas instalen lo que se requiere para cada día
antes de dejar la clase el día anterior.
Dividir el trabajo en partes hace que cada una sea menos intimidante,
es más probable que los estudiantes realmente lo hagan
y garantiza que puedas comenzar a tiempo para cada lección, excepto la primera.

\seclbl{Otras prácticas de enseñanza}{s:classroom-practices}

Ninguna de las prácticas más pequeñas que se describen a continuación son esenciales,
pero todas mejorarán la entrega de lecciones.
Como ocurre con el ajedrez y el matrimonio,
el éxito en la enseñanza suele ser una cuestión de progreso lento y constante.

\subsection*{Comience con introducciones}

Comienza tu clase presentándote.
Si eres un experto,
cuéntales un poco cómo llegaste a donde estás;
si solo estás dos pasos por delante de ellos,
enfatiza lo que usted y ellos tienen en común.
Digas lo que digas,
tus metas son hacerte más accesible
y fomentar la creencia de que pueden tener éxito.

Los estudiantes también deben presentarse entre sí.
En una clase de una docena,
pueden hacer esto verbalmente;
en una clase más grande o si son desconocidos entre sí,
creo que es mejor que cada uno escriba una o dos líneas sobre sí mismos en las notas compartidas (\secref{s:classroom-notetaking}).

\subsection*{Configura tu propio entorno}

Configurar tu entorno es tan importante como configurar el de tus estudiantes,
pero más involucrado.
Además de tener acceso a la red y todo el software que vas a utilizar,
también debes tomar un vaso de agua
o una taza de té o café.
Esto ayuda a mantener tu garganta lubricada,
pero su propósito real es darte una excusa para hacer una pausa y pensar durante un par de segundos
cuando alguien te hace una pregunta difícil
o cuando pierdes la noción de lo que ibas a decir a continuación.
Probablemente también quieras algunos marcadores de pizarra
y varias otras cosas que se describen en \secref{s:checklists-events}.

Una manera de evitar que tu trabajo diario se entrometa en tu manera de enseñar
es creando una cuenta separada en tu computadora para tu rol docente.
Usa valores predeterminados del sistema para todo lo referido a esta segunda cuenta, 
así como un tamaño de letra grande y un fondo de pantalla blanco,
y silencia las notificaciones de manera que tus lecciones no sean interrumpidas por 
ventanas emergentes.
\subsection*{Evite dejar tarea para la casa}

Los/las estudiantes que hayan pasado todo un día programando estarán cansados.
Si les das tarea para hacer fuera del horario de clase,
también empezarán cansados al día siguiente,
así que no lo hagas.

\subsection*{No toques el teclado de tu estudiante}

A menudo es tentador arreglar las cosas para los estudiantes,
pero incluso si narras cada paso,
es probable que los desmotives
al enfatizar la brecha entre sus conocimientos y los tuyos.
En su lugar,
mantén tus manos fuera del teclado y habla con los/las estudiantes sobre lo que tengan que hacer:
llevará más tiempo,
pero es más probable que se quede.

\subsection*{Repite la pregunta}

Siempre que alguien haga una pregunta en clase,
repítesela antes de responder
para comprobar que la ha entendido
y para que las personas que quizás no la hayan escuchado tengan la oportunidad de hacerlo.
Esto es particularmente importante cuando se graban o transmiten presentaciones,
ya que tu micrófono generalmente no captará lo que otras personas están diciendo. Repetir las preguntas también te da la oportunidad
de redirigir la pregunta a algo con lo que se sienta más cómodo(a) respondiendo{\ldots}

\subsection*{Uno arriba, uno abajo}

Un complemento de las tarjetas de actas es solicitar comentarios resumidos al final de cada día.
Los estudiantes dan alternativamente un punto positivo o negativo sobre el día
sin repetir nada de lo que ya se ha dicho.
La prohibición de las repeticiones obliga a las personas a decir cosas que de otro modo no harían:
una vez que se hayan dado todos los comentarios ``seguros'',
los participantes comenzarán a decir lo que realmente piensan.

\begin{aside}{Diferentes modos, diferentes respuestas}
  Las tarjetas de actas (\secref{s:classroom-sticky-notes}) son anónimas;
  la retroalimentación alterna de arriba a abajo no lo es.
  Debes usar los dos juntos
  porque el anonimato permite tanto la honestidad como la ofensa.
\end{aside}

\subsection*{Haz que los estudiantes hagan predicciones}

Las investigaciones han demostrado que las personas aprenden más de las demostraciones
si se les pide que predigan lo que sucederá~\cite{Mill2013}.
Hacer esto encaja naturalmente en la codificación en vivo:
después de agregar o cambiar algunas líneas de un programa,
pregunta a la clase qué sucederá cuando se ejecute.
Si el ejemplo es incluso moderadamente complejo,
la predicción puede servir como una pregunta motivadora para una ronda de instrucción entre pares.

\subsection*{Configuración de mesas}

Es posible que no tengas ningún control sobre la distribución de los escritorios o mesas
en la sala en la que enseñas,
pero si lo tienes,
creemos que es mejor tener asientos planos (estilo cena)
en lugar de asientos bancarios (estilo teatro),
así puedes llegar a los/las estudiantes que necesitan ayuda más fácilmente
y que sea más fácil para los/las estudiantes emparejarse entre sí (\secref{s:classroom-mixed}).
Los tomacorrientes en el piso para que no tengas que pasar cables de alimentación por ahí
hacen la vida más fácil y segura,
pero siguen siendo poco comunes.

Independientemente del diseño que tengas,
trata de asegurarte de que cada asiento tenga una vista sin obstáculos de la pantalla.
Un buen soporte para la espalda también es importante,
ya que las personas estarán en ellos durante un período prolongado.
Al igual que los tomacorrientes en el piso,
los buenos asientos en el salón de clases aún son infrecuentes.

\subsection*{Pastillas para la tos}

Si hablas todo el día en una habitación llena de gente,
se te irritará la garganta porque estarás irritando las células epiteliales de la laringe y la faringe.
Esto no solo te vuelve ronco, sino que también te hace más vulnerable a las infecciones
(que es parte de la razón por la que las personas a menudo se resfrían después de enseñar).

La mejor manera de protegerse contra esto es mantener la garganta alineada,
y una buena recomendación es usar pastillas para la tos pronto y con frecuencia.
Los buenos también disimularán la aparición del aliento a café,
por lo que tus estudiantes probablemente estarán agradecidos.

\subsection*{Piensa, empareja, comparte}

\grefdex{g:think-pair-share}{Piensa, empareja, comparte}{think-pair-share} es una técnica ligera
que ayuda a las personas a mejorar sus ideas
mediante la discusión con sus compañeros.
Cada persona comienza pensando individualmente sobre una pregunta o problema
y anotando algunas notas.
Luego, se explican las ideas por parejas,
fusionándolas o seleccionando las más prometedoras.
Finalmente,
algunas parejas presentan sus ideas a todo el grupo.
Piensa, empareja, comparte funciona porque obliga a las personas a exteriorizar su cognición
(\secref{s:memory-concept-maps}).
También les da la oportunidad de detectar y resolver brechas o contradicciones en sus ideas
\emph{antes} de exponerlas a un grupo más grande,
lo que puede hacer que tus estudiantes menos extrovertidos estén un poco menos nerviosos/as
por parecer tontos/as.

\subsection*{Mañana, mediodía y noche}

\cite{Smar2018} descubrió que
a los/las estudiantes les va peor
si sus clases y otros trabajos se programan en horarios que no coinciden con sus relojes corporales naturales,
es decir, que si una persona matutina toma clases nocturnas o viceversa,
sus calificaciones se ven afectadas.
Por lo general, no es posible acomodar esto en grupos pequeños,
pero los más grandes deben intentar escalonar las horas de inicio de las sesiones paralelas.
Esto también puede ayudar a las personas a hacer malabarismos con las responsabilidades del cuidado de los niños y otras limitaciones,
y reducir la duración de las filas en las pausas para el café y en los baños.

\subsection*{Humor}

El humor debe usarse con moderación al enseñar:
la mayoría de los chistes son menos divertidos cuando se escriben
y se vuelven aún menos divertidos con cada relectura.
Ser espontáneamente divertido mientras enseñas generalmente funciona mejor,
pero puede salir mal fácilmente:
lo que es una broma para tu círculo de amigos
puede convertirse en un problema político serio para tu audiencia.
Si haces bromas cuando enseñas,
no las hagas a expensas de ningún grupo
o de ningún individuo excepto posiblemente de ti mismo.

\seclbl{Limita la innovación}{s:classroom-innovate}

Cada una de las técnicas presentadas en este capítulo mejorará tus clases,
pero no debes intentar adoptarlas todas a la vez.
La razón es que cada nueva práctica aumenta \emph{tu} carga cognitiva, así como la de tus estudiantes,
ya que ahora todos están tratando de aprender una nueva forma de aprender,
así como el tema de la lección.
Si trabajas con un grupo repetidamente,
puedes introducir una técnica nueva cada pocas lecciones;
si solo los tienes para un taller de un día,
es mejor elegir solo un método que no hayan visto antes
y que se sientan cómodos con eso.

\seclbl{Ejercicios}{s:classroom-exercises}

\exercise{Crea un cuestionario}{individual}{20’}

Utiliza el cuestionario de \secref{s:checklists-preassess} como plantilla,
crea un breve cuestionario que puedas entregar a tus estudiantes antes de impartir una clase propia.
¿Qué es lo que más deseas saber sobre sus antecedentes
y cómo pueden ambas partes estar seguras de que están de acuerdo en qué nivel de comprensión se está preguntando?

\exercise{Uno de los tuyos}{Toda la clase}{15’}

Piensa en una práctica de enseñanza que no se haya descrito hasta ahora.
Presenta tu idea a un compañero(a),
escucha la de ellos(as)
y selecciona una para presentarla al grupo en general.
(Este ejercicio es un ejemplo de pensar-emparejar-compartir).

\exercise{¿Puedo conducir?}{parejas}{10’}

Intercambia computadoras con un compañero(a)
(preferiblemente alguien que use un sistema operativo diferente al tuyo)
y trabaja con un simple ejercicio de programación.
¿Qué tan frustrante es?
¿Cuánta información te da sobre lo que los(as) novatos(as) tienen que pasar todo el tiempo?

\exercise{Emparejar}{parejas}{15’}

Mira \hreffoot{https://www.youtube.com/watch?v=vgkahOzFH2Q}{este video} de programación de pares
y luego practica hacerlo con un compañero(a).
Recuerda cambiar los roles entre conductor y navegador cada pocos minutos.
¿Cuánto tiempo tardas en adoptar un ritmo de trabajo?

\exercise{Compara Notas}{grupos pequeños}{15’}

Forma grupos de 3 a 4 personas
y compara las notas que ha tomado en este capítulo.
¿Qué le pareció digno de mención que sus compañeros(as) perdieron de vista y viceversa?
¿Qué entendiste diferente?

\exercise{Credibilidad}{individual}{15’}

\cite{Fink2013} describe tres cosas
que hacen que los/las docentes sean creíbles a los ojos de sus estudiantes:

\begin{description}

\item[Competencia:]
  conocimiento del tema
  como lo demuestra la capacidad para explicar ideas complejas
  o hacer referencia al trabajo de otros.
 
\item[Integridad:]
  teniendo en cuenta los mejores intereses de los estudiantes.
  Esto se puede demostrar dando retroalimentación individualizada,
  ofreciendo una explicación racional para las decisiones de calificación
  y tratando a todos los/las estudiantes por igual.

\item[Dinamismo:]
  entusiasmo por el tema (\chapref{s:performance}).

\end{description}

Describe una cosa que haces al enseñar que se ajuste a cada categoría
y luego describe una cosa que no haces pero que deberías hacer.

\exercise{Medir la eficacia}{individual}{15’}

\cite{Kirk1994} define cuatro niveles en los cuales evaluar la formación:

\begin{description}

\item[Reacción:]
  ¿Cómo se sintieron los estudiantes con respecto a la formación?

\item[Aprendizaje:]
  ¿Cuánto aprendieron realmente?

\item[Comportamiento:]
  ¿Cuánto han cambiado su comportamiento como resultado?

\item[Resultados:]
  ¿Cómo han afectado esos cambios de comportamiento su resultado
  o el resultado de su grupo?

\end{description}

¿Qué estás haciendo en cada nivel para evaluar qué y cómo enseñas?
¿Qué podrías hacer que no estés haciendo?

\exercise{Objeciones y contraobjeciones}{piensa-empareja-comparte}{15’}

Has decidido no preguntarles a tus estudiantes si su clase fue útil
porque sabes que no existe una correlación entre sus respuestas
y cuánto aprenden realmente (\secref{s:pck-now}).
En cambio,
has presentado cuatro propuestas,
cada una de las cuales tus colegas han rechazado:

\begin{description}

\item[Ver si recomiendan la clase a sus amigos(as).]
  ¿Por qué esto sería más significativo
  que preguntarles cómo se sienten acerca de la clase?
 
\item[Hazles un examen al final.]
  Pero cuánto saben los estudiantes al final del día
  es un mal predictor de cuánto recordarán dos o tres meses después,
  y cualquier tipo de examen final hará que la clase sea mucho más estresante.
 
\item[Hazles un examen dos o tres meses después.]
  Eso es prácticamente imposible con los estudiantes de rango abierto,
  y las personas que no obtuvieron nada del taller
  probablemente tengan menos probabilidades de participar en el seguimiento,
  por lo que los comentarios recopilados de esta manera serán sesgados.
 
\item[Revisa si siguen usando lo que aprendieron.]
  La instalación de software espía en las computadoras de los/las estudiantes está mal vista,
  entonces, ¿cómo se implementará?

\end{description}

Trabajando por tu cuenta,
encuentra respuestas a estas objeciones,
luego comparte tus respuestas con un compañero(a)
y discute los enfoques que ha ideado.
Cuando hayas terminado,
comparte tu enfoque favorito con la clase.
