\chapter{Pericia y Memoria}\label{s:memory}

\begin{quote}

  La memoria es el remanente del pensamiento. \\
  --- Daniel Willingham\index{Willingham, Daniel}, \emph{Por qué a los estudiantes no les gusta la escuela(Why Students Don't Like School)}

\end{quote}

El capítulo anterior explicaba las diferencias entre practicantes novatos y competentes.
En  éste se observa la pericia:
qué es,
cómo se puede adquirir,
y cómo puede ser perjudicial o también de ayuda.
Luego introduciremos uno de los límites más importantes en el aprendizaje
y miraremos cómo crear dibujos de modelos mentales puede ayudarnos a convertir el conocimiento en lecciones.

Para empezar,
¿a qué nos referimos cuando decimos que alguien es una persona experta?\index{expert}
La respuesta habitual es que puede resolver problemas mucho más rápido que la persona que es “simplemente competente'',
o que puede reconocer y entender casos donde las reglas normales no se pueden aplicar.
Es más, de alguna manera hace que parezca que no requiere esfuerzo alguno:
en muchos casos,
parece saber la respuesta correcta de un vistazo~\cite{Parn2017}.

Pericia es más que solo conocer más hechos:
los practicantes competentes pueden memorizar una gran cantidad de trivialidades sin  mejorar notablemente sus desempeños.
En cambio,
imagina por un momento que almacenamos conocimiento como una red o grafo en la cual los hechos son nodos
y las relaciones son arcos\footnote{Esto definitivamente \emph{no} es como nuestro cerebro trabaja, pero es una metáfora útil.}.
La diferencia clave entre personas expertas y practicantes competentes es que
los modelos mentales de las  personas expertas están mucho más densamente conectados, 
p.ej.\ es más probable que conozcan una conexión entre dos hechos cualesquiera.

La metáfora del grafo explica por qué  ayudar a los estudiantes a hacer conexiones es tan importante como presentarles los hechos:
sin esas conexiones,
la gente no puede recordar y usar aquello que sabe.
También explica varios aspectos observados del comportamiento experto:

\begin{itemize}

\item
  Las personas expertas pueden saltar directamente de un problema a una solución
  porque realmente existe una conexión directa entre ambos en sus mentes.
  Mientras un practicante competente debería razonar
  $A{\rightarrow}B{\rightarrow}C{\rightarrow}D{\rightarrow}E$,
  una persona experta puede ir de $A$ a $E$ en un solo paso.
  Esto lo llamamos \gref{g:intuición}{intuición}:
  en vez de razonar su camino a una solución,
  la persona experta reconoce una solución de la misma manera que reconocería una cara familiar.

\item
  Los grafos densamente conectados son también la base para la 
  \grefdex{g:fluid-representation}{representación fluida}{representación fluida} de las personas expertas,
 p.ej.\ sus habilidades para cambiar de una a otra entre distintas vistas de un problema~\cite{Petr2016}.
  Por ejemplo,
  tratando de resolver un problema en matemáticas,
  una persona experta puede cambiar entre abordarlo de manera geométrica 
  y representarlo como un conjunto de ecuaciones.

\item
  Esta metáfora también explica por qué las personas expertas son mejores en diagnósticos que los practicantes competentes:
 mayor cantidad de conexiones entre hechos hace más fácil razonar hacia atrás, de síntomas a causas.
  (Esta es la razón del por qué es preferible pedirle a programadores depurar un programa durante una entrevista de trabajo a pedirles que programen: da una impresión más precisa de su habilidad)

\item
  Finalmente,
  las personas expertas están muchas veces tan familiarizadas con su tema que
  no pueden imaginarse cómo puede ser \emph{no} ver el mundo de esa manera.
  Esto significa que muchas veces están menos capacitadas para enseñar un tema que  personas con menor experiencia, 
  que aún recuerda cómo lo ha aprendido.

\end{itemize}

El último de estos puntos se llama \gref{g:expert-blind-spot}{punto ciego de las personas expertas}.
Como se definió originalmente en~\cite{Nath2003},
es la tendencia de las personas expertas a organizar una explicación de acuerdo a los principios principales del tema
en lugar de guiarse por aquello que los aprendices ya conocen.
Se puede superar con entrenamiento,
pero es parte de la razón por la que no hay correlación entre
lo bueno que es alguien para investigar en un área
y lo bueno que es para enseñarlo~\cite{Mars2002}.

\begin{aside}{La letra S}
 Las personas expertas a menudo caen en sus puntos ciegos usando la palabra ``solo,''
  como en,
  ``Oh, es fácil, solo enciendes una nueva máquina virtual
  y luego solo instalas estos cuatro parches a Ubuntu
  y luego solo reescribes todo tu programa en un lenguaje funcional puro.''
  Como discutimos en \chapref{s:motivation},
  hacer ésto indica que quien habla piensa que el problema es trivial
  y por lo tanto la persona que lucha con esto debe ser estúpida,
  entonces no lo hagas.


\end{aside}

\seclbl{Mapas Conceptuales}{s:memory-concept-maps}

La herramienta que elegimos para representar el modelo mental de alguien es un \gref{g:concept-map}{mapa conceptual},
en el cual los hechos son burbujas y las conexiones son relaciones etiquetadas.
Como ejemplos,
\figref{f:memory-seasons} muestra por qué la Tierra tiene estaciones (de \hreffoot{https://cmap.ihmc.us/}{IHMC}),
y \appref{s:conceptmaps} presenta mapas conceptuales de librerías desde tres puntos de vista distintos.

\figpdf{figures/seasons.pdf}{Mapa conceptual para Estaciones}{f:memory-seasons}

Para mostrar cómo pueden ser usados los mapas conceptuales para enseñar programación,
considere este \texttt{for} bucle en Python:\index{Python}

\begin{minted}{text}
for letter in "abc":
    print(letter)
\end{minted}

\noindent
cuya salida es:

\begin{minted}{text}
a
b
c
\end{minted}

Los tres ``cosas'' clave en este bucle se muestra al principio de  \figref{f:memory-loop},
pero son solo la mitad de la historia.
La versión ampliada en la parte inferior muestra las relaciones entre esas cosas,
las cuales son tan importantes para la comprensión como los conceptos en sí mismos.

\figpdf{figures/for-loop.pdf}{Mapa conceptual para un \texttt{for} loop}{f:memory-loop}

\newpage
Los mapas conceptuales pueden ser usados de varias maneras:

\begin{description}

\item[Ayudando a docentes para descubrir que están tratando de enseñar.]
  Un mapa conceptual separa el contenido del orden:
  en nuestra experiencia,
  las personas rara vez terminan enseñando las cosas en el orden que las dibujaron por primera vez.

\item[Ayudando a la comunicación entre diseñadores de lecciones]
  Los docentes con ideas muy diferentes de aquello que están tratando de enseñar es probable que arrastren a sus estudiantes en diferentes direcciones.
  Dibujar y compartir mapas conceptuales puede ayudar a prevenirlo.
  Y sí,
  personas diferentes pueden tener mapas conceptuales diferentes para el mismo tema,
  pero el mapeo conceptual hace explícitas estas diferencias.

\item[Ayudando a la  comunicación con estudiantes.]
  Si bien es posible dar a los estudiantes un mapa pre-dibujado al inicio de la lección para   que puedan anotar,
  es mejor dibujarlo parte por parte mientras se está enseñando,
  para reforzar la relación entre lo que muestra el mapa y lo que dice el docente.
  Volveremos a esta idea en \secref{s:architecture-brain}.

\item[Para evaluación.]
  Hacer que los estudiantes dibujen lo que creen que acaban de aprender
  muestra al enseñante lo que se perdieron y lo que se comunicó mal.
  Revisar los mapas conceptuales de estudiantes insume demasiado tiempo para utilizarlo como una evaluación formativa durante las clases,
  pero es muy útil en clases semanales \emph{una vez que el estudiantado está familiarizado con la técnica}.
  La calificación es necesaria porque
  cualquier manera nueva de hacer algo inicialmente ralentiza a la gente---si un estudiante está tratando de encontrarle el sentido a la programación básica,
  pedirle que se imagine cómo esquematizar sus pensamientos al mismo tiempo, es una carga no conveniente.

\end{description}

Algunos enseñantes son escépticos a que las personas novatas puedan mapear efectivamente lo que entendieron,
dado que la introspección y la explicación de lo entendido son generalmente habilidades más avanzadas que la comprensión misma.
Por ejemplo,
\cite{Kepp2008} observó el uso del mapeo conceptual en la enseñanza de computación.
Uno de los hallazgos fue que,
``{\ldots} el mapeo conceptual es problemático para muchos estudiantes porque
evalúa la comprensión personal en lugar del conocimiento que simplemente se aprendió de memoria.''
Yo, como alguien que valora la comprensión sobre el conocimiento de memoria,
lo considero un beneficio.

\begin{aside}{Comienza por cualquier lugar}
  Cuando se pide por primera vez dibujar un mapa conceptual , muchas personas no saben por dónde empezar.
  Cuando esto ocurre,
  escribe dos palabras asociadas con el tema que estás tratando de mapear,
 luego dibuja una linea entre ellas y agrega una etiqueta explicando cómo estas dos ideas están relacionadas.
  Puedes entonces preguntar qué otras cosas están relacionadas en el mismo sentido,
  qué partes tienen esas cosas,
  o qué sucede antes o después con los conceptos que ya están en la hoja
  a fin de descubrir más nodos y arcos.
  Después de eso, casi siempre la parte más difícil está terminanda.
\end{aside}

Los mapas conceptuales son solo una forma de representar nuestro conocimiento de un tema~\cite{Eppl2006};
otros incluyen diagramas de Venn, diagramas de flujo, y árboles de decisión~\cite{Abel2009}.
Todos ellos \grefdex{g:externalized-cognition}{externalizaron la comprensión}{externalized cognition},
p.ej.\ hacen visibles los modelos mentales de manera que pueden ser comparados y combinados\footnote{Parafraseando a
 Lady Windermere, obra de Oscar Wilde,\index{Wilde, Oscar}
las personas a menudo no saben lo que piensan hasta que se escuchan a sí mismas decirlo}.

\begin{aside}{Trabajo crudo y Honestidad}
  Muchos diseñadores de interfaces de usuario creen que es mejor mostrar a la gente bocetos de sus ideas en lugar de maquetas pulidas
  porque estiman que  las personas dan una opinión más honesta sobre algo que 
  consideran solo ha requirido unos pocos minutos crear:
  si parece que lo que están criticando tardó horas en hacerse, 
  la mayoría lanzará sus golpes de puño.
  Al dibujar mapas de concepto para motivar un intercambio de ideas,
  deberías entonces usar lápices y papel borrador (o bolígrafos y una pizarra)
  en lugar de sofisticadas herramientas de dibujo por computadora.
\end{aside}

\seclbl{Siete Más o Menos Dos}{s:memory-seven-plus-or-minus}

Mientras el modelo gráfico de conocimiento es incorrecto pero útil,
otro modelo simple tiene bases fisiológicas profundas.
Como una aproximación rápida,
la memoria humana se puede dividir en dos capas distintas.
La primera,
llamada \grefdex{g:long-term-memory}{long-term}{memoria a largo plazo}
or \gref{g:persistent-memory}{memoria persistente},
es donde almacenamos cosas como los nombres de nuestros amigos,
nuestra dirección,
y lo que hizo el payaso en nuestra fiesta de cumpleaños de 8 que nos asustó mucho.
Su capacidad es esencialmente ilimitada,
pero es de acceso lento---demasiado lenta para ayudarnos a lidiar con leones hambrientos  y familiares descontentos.

La evolución entonces nos ha dado un segundo sistema
llamado \grefdex{g:short-term-memory}{short-term}{memoria a corto plazo}
or \gref{g:working-memory}{memoria de trabajo}.
Es mucho más rápida,
pero también más pequeña:~\cite{Mill1956} estimó que la memoria de trabajo del adulto promedio sólo podía contener 7 ± 2 elementos a la vez.
Esta es la razón por la cual \hreffoot{https://www.quora.com/Why-did-Bell-Labs-create-phone-numbers-of-7-digits-10-digits-Is-there-a-reason-that-dashes-and-brackets-are-used}{los números de teléfono}
son de 7 u 8 dígitos de longitud:
antes cuando los teléfonos tenían dial en vez de teclado,
esa era la cadena de números más larga que la mayoría de los adultos podía recordar con precisión durante el tiempo que tardaba el dial en girar varias veces.

\begin{aside}{Participación}
  El tamaño de la memoria de trabajo a veces se usa para explicar por qué los equipos 
  deportivos tienden a formarse con aproximadamente media docena de miembros o
  se separan en sub-grupos como los delanteros y tres cuartos de rugby.
  También se usa para explicar por qué las reuniones sólo son productivas hasta un cierto número de participantes:
si veinte personas tratan de discutir algo,
o bien se arman tres reuniones al mismo tiempo
o media docena de personas hablan mientras los demás escuchan.
El argumento es que la habilidad de las personas para llevar registro de sus  pares está limitada al tamaño de la memoria de trabajo,
pero hasta donde sé,
la relación jamás fue probada.
\end{aside}

7±2 es simplemente el número más importante al enseñar.
Un docente no puede colocar información directamente en la memoria a largo plazo de un estudiante.
En cambio,
cualquier cosa que presente se almacena primero en la memoria a corto plazo del estudiante,
y sólo se transfiere a la memoria a largo plazo después que ha sido mantenida ahí y ensayada (\secref{s:individual-strategies}).
Si el docente presenta demasiada información y muy rápidamente,
esa nueva información desplaza la vieja antes que esta última se transfiera.

Esta es una de las razones de usar mapas conceptuales cuando se diseña una lección:
sirve para asegurarse que la memoria a corto plazo de los estudiantes no estará sobrecargada.
una vez que se dibuja el mapa,
el docente eligirá un fragmento que se ajuste para la memoria a corto plazo y continuara con una evaluación formativa (\figref{f:memory-photosynthesis}),\index{evaluación formativa}
luego agregará otro fragmento para la próxima lección y así sucesivamente.

\figpdf{figures/photosynthesis.pdf}{Usando mapas conceptuales en el diseño de la lección}{f:memory-photosynthesis}

\begin{aside}{Construyendo juntos mapas conceptuales}
 La próxima vez que tengas una reunión de equipo,
 entrega a todos una hoja de papel
 y que pasen unos minutos dibujando sus propios mapas conceptuales del proyecto en el que  están trabajando.
  A la cuenta de tres,
  haz que todos revelen sus mapas conceptuales a su grupo.
  La discusión que sigue puede ayudar a las personas a comprender
  por qué se han estado tropezando.
\end{aside}

Ten en cuenta que el modelo simple de memoria presentado aquí ha sido reemplazado en gran medida por uno más sofisticado
en el que la memoria a corto plazo se divide en varios almacenamientos
(p. ej.\ para memoria visual versus linguistica),
cada uno de los cuales realiza un preprocesamiento involuntario~\cite{Mill2016a}.
Nuestra presentación es entonces un ejemplo de un modelo mental que ayuda al aprendizaje y al trabajo diario.

\subsection*{Reconocimiento de Patrones}

Investigaciones recientes sugieren que el tamaño real de la memoria a corto plazo 
podría ser tan bajo como 4±1 elementos~\cite{Dida2016}.
Para manejar conjuntos de información más grandes,
nuestras mentes crean \grefdex{g:chunking}{fragmentos}{fragmentación}.
Por ejemplo,
la mayoría de nosotros recordamos palabras como elementos simples más que como secuencia de letras.
Del mismo modo,
el patrón formado por cinco puntos en cartas o dados se recuerda como un todo 
en lugar de cinco piezas de información separadas.

Las personas expertas tienen más fragmentos y de mayor tamaño que las no-expertas,
p.ej.\``ven'' patrones más grandes y tienen más patrones para contrastar cosas.
Esto les permite razonar a un nivel superior
y para buscar información de manera más rápida y precisa.
Sin embargo,
la fragmentación también puede engañarnos si identificamos mal las cosas:
quienes recién llegan a veces pueden ver cosas que personas expertas han visto y perdido.

Dada la importancia de la fragmentación para pensar,
es tentador identificar \hreffoot{https://en.wikipedia.org/wiki/Software\_design\_pattern}{design patterns}\index{patrones de diseño}
y enseñarlos directamente.
Estos patrones ayudan a los practicantes competentes a pensar y dialogar en varios dominios (incluída la enseñanza~\cite{Berg2012}),
pero los catálogos de patrones son demasiado duros y abstractos para personas novatas para que ellas mismas les encuentren sentido.
Dicho esto,
dar nombres a un pequeño número de patrones parece ayudar con la enseñanza,
principalmente dando a los alumnos un vocabulario más rico para pensar y comunicarse \cite{Kuit2004,Byck2005,Saja2006}.
Volveremos a este tema en \secref{s:pck-programación}.

\seclbl{Convirtiéndose en Experto}{s:memory-becoming-expert}

Entonces cómo se convierte alguien en experto?
La idea de que diez mil horas de práctica lo harán es ampliamente citada
pero \hreffoot{http://www.goodlifeproject.com/podcast/anders-ericsson/}{probablemente no sea verdad}:
hacer lo mismo una y otra vez es más probable que fortalezca los malos hábitos a que mejore la actuación.
Lo que realmente funciona es hacer cosas similares pero sutilmente diferentes,
poniendo atención en qué funciona y qué no,
y luego cambiar el comportamiento como respuesta a las devoluciones para mejorar de forma acumulativa.
Esto se llama \grefdex{g:deliberate-practice}{deliberate}{práctica deliberada}
or \gref{g:reflective-practice}{práctica reflectiva},
y una progresión común es que las personas pasen por tres etapas:

\begin{description}

\item[Actuar según las devoluciones de otros.]
  Los estudiantes pueden escribir un ensayo sobre qué hicieron en sus vacaciones de verano
  y recibir devoluciones de un enseñante que les diga cómo mejorarlo.

\item[Dar devoluciones sobre el trabajo de otros.]
  Los estudiantes pueden realizar críticas de la evolución de un personaje en una novela de  Harry Potter
  y recibir una devolución de un enseñante sobre esas críticas.

\item[Darse devoluciones a sí mismos.]
  En algún punto,
  los estudiantes empiezan a criticar sus propios trabajos como lo hacían 
  usando las habilidades que ahora han construído.
  Hacer esto es mucho más rápido que esperar los comentarios de otros 
  esta competencia de pronto empieza a despegar.

\end{description}

\begin{aside}{¿Qué cuenta como Práctica Deliberada?}
  \cite{Macn2014} descubrió que,
  ``{\ldots}la práctica deliberada explicaba el 26\% de la varianza en el rendimiento de los juegos,
  21\% para música,
  18\% para deportes,
  4\% para educación,
  y menos del 1\% para profesiones.''
  Sin embargo,~\cite{Eric2016} criticó este hallazgo diciendo,
  ``Resumir cada hora de cualquier tipo de práctica durante la carrera de un individuo
 implica que el impacto de todos los tipos de actividad práctica respecto a  rendimiento es igual ------una suposición
  que{\ldots}es inconsistente con la evidencia.''
  Para ser efectivo,
  la práctica deliberada requiere tanto un objetivo de rendimiento claro
  como una devolución informativa inmediata,
  ambas son cosas que los enseñantes, de cualquier manera, deberían esforzarse en conseguir.
\end{aside}

\seclbl{Ejercicios}{s:memory-exercises}

\exercise{Mapear Conceptos}{de a pares}{30}

Dibuja un mapa conceptual sobre algo que puedas enseñar en cinco minutos.
Discutan con tu colega y critiquen el mapa de cada uno.
¿Presentan conceptos o detalles de superficie?
¿Cuáles de las relaciones en el mapa de tu colega consideras conceptos y viceversa?

\exercise{ Mapeo de Conceptos (Nuevamente)}{grupos pequeños}{20}

Trabajar en grupos de 3--4,
cada persona debe dibujar independientemente del resto un mapa conceptual mostrando su modelo mental de qué sucede en un aula.
cuando todos hayan terminado,
comparen los mapas conceptuales.
¿Dónde coinciden y difieren sus modelos mentales?

\exercise{Mejora de la memoria a corto plazo}{individual}{5 minutos}

\cite{Cher2007} sugiere que
La razón principal por la que las personas dibujan diagramas cuando discuten cosas
es para ampliar su memoria a corto plazo:
Señalar una burbuja dibujada hace unos minutos provoca el recuerdo de varios minutos de debate.
Cuando intercambiaste mapas conceptuales en el ejercicio anterior,
¿Qué tan fácil fue para otras personas entender lo que significaba tu mapa?
¿Qué tan fácil sería para ti si lo dejas de lado por un día o dos y luego lo miras de nuevo?

\exercise{Eso es un poco autorreferencial, ¿no??}{toda la clase}{30}

Trabajando independientemente,
dibuja un mapa conceptual para mapas conceptuales.
Compara tu mapa conceptual con los dibujados por los demás.
Qué incluyeron la mayoría de las personas?
Cuáles fueron las diferencias más significativas?

\exercise{Notar tus puntos ciegos}{grupos pequeños}{10}

Elizabeth Wickes listó
\hreffoot{https://twitter.com/elliewix/status/981285432922202113}{todo aquello que necesitas para entender} 
Para leer esta línea de Python:

\begin{minted}{text}
answers = ['tuatara', 'tuataras', 'bus', "lick"]
\end{minted}

\begin{itemize}

\item
  Los corchetes rodeando el contenido, significa que estamos trabajando con una lista
  (lo opuesto a corchetes inmediatamente a la derecha de algo,
  que es la notación utilizada para una extracción de datos).

\item
  Los elementos se separan por comas fuera y entre comillas
  (en vez de adentro, como sería para un texto citado).

\item
  Cada elemento es una cadena de caracteres,
  y lo sabemos por las comillas.
  Aquí podríamos tener números u otro tipo de datos si quisiéramos;
  necesitamos comillas porque estamos trabajando con  cadenas.

\item
  Estamos mezclando el uso de comilla  simples y dobles;
  A Python no le importa eso siempre que estén balanceadas alrededor de las cadenas individuales (para cada comilla que abre haya una que cierre).

\item
  A cada coma le sigue un espacio,
  que no es obligatorio para Python,
  pero que preferimos para una lectura más clara.

\end{itemize}

Cada uno de estos detalles un experto ni los vería.
Trabajando en grupos de  3--4 personas,
Selecciona algo igualmente corto de una lección que hayas enseñado o aprendido
y divídelo a este nivel de detalle.

\exercise{Qué enseñar a continuación}{individual}{5}

Vuelve al mapa conceptual para la fotosíntesis en \figref{f:memory-photosynthesis}.
Cuántos conceptos y relaciones hay en los fragmentos seleccionados?
¿Qué incluirías en el próximo fragmento de la lección y por qué?

\exercise{El poder de fragmentación}{individual}{5}

Mira  \figref{f:memory-unchunked} por 10 segundos,
luego mira hacia otro lado e intenta escribir tu número de teléfono con estos símbolos\footnote{
  Mi agradecimiento a Warren Code\index{Code, Warren} por presentarme este ejemplo.
}.
(Usa un espacio para  '0'.)
Cuando hayas terminado,
Mira la representación alternativa en \appref{s:chunking}.
¿Cuánto más fáciles de recordar son los símbolos cuando el patrón se hace explícito?

\figpdfhere{figures/chunking-unchunked.pdf}{Unchunked representation}{f:memory-unchunked}
