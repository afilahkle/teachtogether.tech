\chapter{Unirse a nuestra comunidad}\label{s:joining}

\begin{reviewer}
{Yanina Bellini Saibene}
{Natalia Morandeira y Juliana Benitez}
\end{reviewer}

Esperamos que elijas ayudarnos a mejorar este libro.
Si esta forma de trabajo colaborativa es nueva para ti,
consulta en el \appref{s:conduct} nuestro código de conducta,
y luego:

\begin{description}

\item[Empieza pequeño.]
  Arregla un error tipográfico,
  aclara la redacción de un ejercicio,
  corrige o actualiza una cita,
  o sugiere un mejor ejemplo o analogía para ilustrar algún punto.

\item[Únete a la conversación.]
  Mira los \emph{issues} y los cambios propuestos por otras personas
  y añádeles tus comentarios.
  A menudo es posible mejorar las mejoras,
  y es una buena manera de presentarte a la comunidad y hacer nuevas amistades.
  Las etiquetas \emph{``Beginner-Friendly''} (apto para principiantes en inglés), 
  \emph{``Work in Progress''} (trabajo en progreso en inglés) y \emph{``Help Wanted''} (se busca ayuda en inglés) 
  son buenos \emph{issues} para unirte a la conversación. 

\item[Discute, luego edita.]
  Si quieres proponer un gran cambio,
  como reorganizar o dividir un capítulo completo,
  por favor, completa un \emph{issue} que describa tu propuesta y tu razonamiento y etiquétalo como  \emph{``Addition''} (agregado en inglés), \emph{``Correction''} (corrección en inglés) ó \emph{``Discussion''} (discusión en inglés).
  Te alentamos a que agregues comentarios a estos \emph{issues}
  para que toda la discusión sobre \emph{qué} y \emph{por qué} esté abierta y se pueda archivar.
  Si se acepta la propuesta,
  el trabajo real puede dividirse en varios problemas o cambios más pequeños
  que se pueden abordar de forma independiente.
\end{description}

\seclbl{Usando este material}{s:joining-using}

Como se declaró en el \chapref{s:intro},
todo este material puede distribuirse y reutilizarse libremente
bajo la licencia Creative Commons Atribución -- No Comercial 4.0
(\appref{s:license}).
Puedes usar la versión en línea en \url{http://teachtogether.tech/} en cualquier clase (gratuita o paga),
y puedes citar extractos breves bajo las disposiciones de \hreffoot{https://es.wikipedia.org/wiki/Uso\_justo}{uso justo},
pero no puedes volver a publicar fragmentos grandes en obras comerciales sin permiso previo.

Este material ha sido usado de muchas maneras,
desde una clase en línea de varias semanas hasta un taller intensivo en persona.
Por lo general, es posible cubrir gran parte de los capítulos \chapref{s:models} a \chapref{s:process},
\chapref{s:performance},
y \chapref{s:motivation} en dos días de jornada completa.

\subsection*{En persona}

Esta es la forma más efectiva de impartir esta capacitación,
pero también la más exigente.
Las personas que participan están físicamente en el mismo lugar.
Cuando necesitan practicar cómo enseñar en pequeños grupos,
parte de la clase o toda la clase va a habitaciones cercanas.
Cada participante usa su propia tableta o computadora portátil para ver material en línea durante la clase
y para tomar notas compartidas (\secref{s:classroom-notetaking}),
y usa lápiz y papel o pizarras para otros ejercicios.
Las preguntas y la discusión se hacen en voz alta.

Si estás enseñando en este formato,
debes usar notas adhesivas como indicadores de estado
para que puedas ver quién necesita ayuda,
quién tiene preguntas
y quién está listo/a para seguir adelante (\secref{s:classroom-sticky-notes}).
También debes usarlos para distribuir la atención,
para que todo el curso obtenga tu atención y tiempo de forma justa,
como tarjetas de actas para alentar a tus estudiantes a reflexionar sobre lo que acaban de aprender
y para darte retroalimentación procesable mientras todavía tienes tiempo para actuar en consecuencia.

\subsection*{En línea en grupos}

En este formato,
entre 10 a 40 estudiantes se juntan en 2 a 6 grupos de 4 a 12 personas,
pero esos grupos están distribuidos geográficamente.
Cada grupo usa una cámara y un micrófono para conectarse a la videollamada,
en lugar de que cada persona esté en la llamada por separado.
Un buen audio es más importante que un buen video en ambas direcciones:
una voz sin imágenes (como la radio)
es mucho más fácil de entender que las imágenes sin narrativa,
y los/las docentes no necesitan ver a las personas para responder preguntas,
siempre y cuando esas preguntas se puedan escuchar con claridad.
Dicho esto,
si una lección no es accesible, entonces no es útil (\secref{s:motivation-accessibility}):
proporcionar texto descriptivo es una ayuda cuando la calidad del audio es deficiente,
e incluso si el audio es bueno resulta importante para aquellas personas con dificultades auditivas. 

Toda la clase toma notas compartidas,
y también usa las notas compartidas para hacer y responder preguntas.
Tener varias decenas de personas tratando de hablar en una llamada no funciona bien,
así que en la mayoría de las sesiones
el/la docente habla y sus estudiantes responden a través del chat de la herramienta para tomar notas.

\subsection*{En línea de forma individual}

La extensión natural de estar en línea en grupos es estar en línea en forma individual.
Al igual que con los grupos en línea,
el/la docente hablará la mayoría de las veces y los/las estudiantes participarán principalmente a través del chat de texto.
También en este caso, y siempre teniendo en cuenta los aspectos de accesibilidad, un buen audio es más importante que un buen video,
y quienes participan deberían usar el chat de texto para indicar que quieren hablar (\appref{s:meetings}).

Tener participantes en línea individualmente hace que sea más difícil dibujar y compartir mapas conceptuales (\secref{s:memory-exercises})
o dar retroalimentación sobre la enseñanza (\secref{s:performance-exercises}).
Por lo tanto, quienes enseñen deberán confiar más en el uso de ejercicios con resultados escritos que se puedan poner en las notas compartidas,
como por ejemplo, dar una devolución sobre videos de personas enseñando.

\subsection*{Multi-semana en línea}

La clase se reúne todas las semanas durante una hora a través de videoconferencia.
Cada reunión puede realizarse dos veces para acomodar las zonas horarias y los horarios de los/las estudiantes.
Los/las participantes toman notas compartidas como se describió anteriormente para las clases grupales en línea,
para publicar tareas en línea entre clases
y para comentar sobre el trabajo de los demás.
En la práctica,
los comentarios son relativamente raros:
la gente prefiere discutir el material en las reuniones semanales.

Este fue el primer formato que utilicé
y ya no lo recomiendo:
mientras que extender la clase les da a las personas tiempo para reflexionar y abordar ejercicios más extensos,
también aumenta en gran medida las probabilidades de que tengan que abandonar debido a otras demandas de su tiempo.

\seclbl{Contribuir y mantener}{s:joining-contributing}

Las contribuciones de todo tipo son bienvenidas,
desde sugerencias para mejoras hasta erratas y nuevo material.
Todas las personas que contribuyan deben cumplir con nuestro Código de Conducta (\appref{s:conduct});
al enviar tu trabajo,
aceptas que pueda incorporarse tanto en forma original como editada
y que pueda ser publicado bajo la misma licencia que el resto de este material (\appref{s:license}).

Si tu material es incorporado,
te agregaremos a los agradecimientos (\secref{s:intro-acknowledgments}) a menos que solicites lo contrario.

La fuente de la versión original de este libro se almacena en GitHub en:

\begin{center}
  \url{https://github.com/gvwilson/teachtogether.tech/}
\end{center}

y la versión en español se encuentra dentro de la carpeta \emph{es}.
Para conocer más sobre el proceso de traducción y los acuerdos generados durante nuestro trabajo colaborativo, 
puedes consultar el \chapref{s:traduccion} y el \emph{Readme.md} en el repositorio \emph{es}.


\noindent
Si sabes cómo usar \emph{Git} y \emph{GitHub} y deseas cambiar, arreglar o agregar algo,
por favor envía un \gref{g:pull-request}{\emph{pull request}} que modifique la fuente del \emph{LaTeX}.
Si deseas obtener una vista previa de tus cambios,
por favor ejecuta \texttt{make~pdf} o \texttt{make~html} en la línea de comandos de tu computadora.

Si quieres reportar un error,
hacer una pregunta
o hacer una sugerencia,
presenta un \emph{issue} en el repositorio.
Necesitas tener una cuenta de \emph{GitHub} para hacer esto,
pero no necesitas saber cómo usar \emph{Git}.

Siempre etiqueta tus \emph{issues} y \emph{pull requests} 
como \emph{``Español''}.

Si no deseas crear una cuenta de GitHub,
envía tu contribución por correo electrónico a \texttt{gvwilson@third-bit.com}
con ``T3'' o \emph{``Teaching Tech Together''} en algún lugar del asunto.
Intentaremos responder en una semana.

Finalmente,
siempre nos gusta escuchar cómo se ha usado este material,
y estamos \emph{siempre} agradecidos/as por el aporte de más mapas conceptuales.
